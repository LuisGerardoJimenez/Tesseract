\chapter{Metodología} \label{cap:siete}

En este capítulo se describen las actividades realizadas durante cada uno de los sprints planteados para la entrega del trabajo terminal.\\

De acuerdo a las necesidades del proyecto, se determinó trabajar con base en la metodología scrum empleando específicamente el uso de sprints.\\

El núcleo central de la metodología de trabajo ‘scrum’ es el ‘sprint’. Se trata de un miniproyecto de no más de un mes (ciclos de ejecución muy cortos -entre una y cuatro semanas), cuyo objetivo es conseguir un incremento de valor en el producto que estamos construyendo. Todo ‘sprint’ cuenta con una definición y una planificación que ayudará a lograr las metas marcadas. \hyperlink{b21}{[21]}\\ 

Scrum, que se emplea cada vez con más frecuencia para desarrollar productos y servicios digitales, es un marco de trabajo donde los miembros de un equipo colaboran en la construcción de un producto de manera que sea valioso desde sus primeras etapas. Para lograr esta entrega de valor tan rápida y continua, los equipos ‘scrum’ trabajan en ciclos de ejecución muy cortos -entre una y cuatro semanas- que se denominan ‘sprints’ y tienen un objetivo muy claro. \hyperlink{b21}{[21]}\\ 

El primer paso para alcanzar este objetivo -o hito del proyecto- es la reunión de planificación, una sesión en la que debe participar todo el equipo ‘scrum’ y que supone el pistoletazo de salida del ‘sprint’. Esta reunión se divide en dos partes que tratan de dar respuesta a dos preguntas fundamentales: ¿Qué se va a entregar? y ¿cómo se va a realizar el trabajo?, donde cada miembro inspecciona el trabajo de los otros para poder hacer las adaptaciones necesarias, comunica cuales son los confictos en los que se encuentra, actualiza el estado de la lista de tareas de la iteración (Sprint Backlog) y los gráficos de trabajo pendiente (Burndown charts). \hyperlink{b22}{[22]}\\ 

\section{Sprint 0: Configuración del ambiente de trabajo}
\subsection{Selección de herramientas}

A continuación se detallan las herramientas consideradas para el desarrollo del proyecto, así como la justificación de las alternativas seleccionadas.\\

En primer lugar se describe en el Cuadro \ref{tbl:compotex} las herramientas de composición de textos seleccionadas que pueden coadyuvarnos en la edición de los documentos entregables.

\begin{table}[H]
	\centering
	\begin{tabular}{|p{3cm}|p{12cm}|}
		\hline
		 \rowcolor{black} \textcolor{white}{\textbf{Herramienta}} & \textcolor{white}{\textbf{Características}}  \\ \hline
		Latex & Es un sistema de software libre y de composición tipográfica de alta calidad, esta herramienta incluye características diseñadas para la producción de la documentación técnica en donde se establecen las características del proyecto.\hyperlink{b54}{[54]}\\
		\hline
		Word & Es un programa informático orientado al procesamiento de textos. Fue creado por la empresa Microsoft, y viene integrado de manera predeterminada en el paquete ofimático denominado Microsoft Office. \hyperlink{b53}{[53]}\\
		\hline
		Google Docs & Procesador de texto es una aplicación informática que permite crear y editar documentos de texto en una computadora. Se trata de un software de múltiples funcionalidades para la redacción, con diferentes tipografías, tamaños de letra, colores, tipos de párrafos, efectos artísticos y otras opciones. \hyperlink{b55}{[55]}\\
		\hline
	\end{tabular}
\caption{Cuadro de características de las herramientas para el sistema de composición de textos.}
\label{tbl:compotex}
\end{table}

\textbf {HERRAMIENTA SELECCIONADA: LATEX}\\

\textbf {JUSTIFICACIÓN:} Para la elaboración de nuestros entregables, Latex nos facilita la generación de todo tipo de índices, listados de bibliografía citada, permite centrarnos en el contenido, no en la forma. Los documentos tendrán un contenido extenso por lo que necesitamos un reducido tiempo de generación del documento, Latex es rápido, limpio y seguro, sin olvidar que al realizar todos los cambios de formato no se altera el estilo a comparación de Word.\\
\newpage
En el Cuadro \ref{tbl:lenguaje} se exponen las características de los lenguajes de programación para elegir el que mejor se adapta a nuestro proyecto.

\begin{table}[H]
	\centering
	\begin{tabular}{|p{3cm}|p{12cm}|}
		\hline
		\rowcolor{black}  \textcolor{white}{\textbf{Herramienta}} & \textcolor{white}{\textbf{Características}}  \\ \hline
		Java &  Es una tecnología que se usa para el desarrollo de aplicaciones que convierten a la Web en un elemento más interesante y útil. Java es la base para prácticamente todos los tipos de aplicaciones de red, además del estándar global para desarrollar y distribuir aplicaciones móviles y embebidas, juegos, contenido basado en web y software de empresa.\hyperlink{b56}{[56]} \\
		\hline
		C Sharp & C\# es un lenguaje de tipos y orientado a objetos que permite a los desarrolladores crear una gran variedad de aplicaciones seguras y sólidas que se ejecutan en .NET Framework. Puede usar C\# para crear aplicaciones cliente de Windows, servicios web XML, componentes distribuidos, aplicaciones cliente-servidor y aplicaciones de base de datos.\hyperlink{b57}{[57]} \\
		\hline
		Python & Python es un lenguaje de programación web de alto nivel basado en un código compacto, pero con una sintaxis fácil de entender. Es un lenguaje de programación multiparadigma. Esto significa que más que forzar a los programadores a adoptar un estilo particular de programación, permite varios estilos: programación orientada a objetos, programación imperativa y programación funcional. Otros paradigmas están soportados mediante el uso de extensiones.\hyperlink{b58}{[58]} \\
		\hline

	\end{tabular}
\caption{Cuadro de características de las herramientas de leguajes de programación.}
\label{tbl:lenguaje}
\end{table}

\textbf {HERRAMIENTA SELECCIONADA: JAVA}\\

\textbf {JUSTIFICACIÓN:} La idea principal de TESSERACT es que opere como una plataforma Web, java al ser un lenguaje multiplataforma nos permite construir un producto con las características deseadas. C\# no es un lenguaje multiplataforma y Phyton no nos proporciona muchos recursos. Otras de las ventajas que tiene java son sus librerias estandar, la facilidad de programación y su paradigma orientado a objetos. 
\newpage
Después de haber seleccionado el lenguaje de programación hay que adaptar el uso de un framework para del desarrollo. En el Cuadro \ref{tbl:frame} se exponen las características de los frameworks para elegir la que mejor se adapta a nuestro proyecto.

\begin{table}[H]
	\centering
	\begin{tabular}{|p{3cm}|p{12cm}|}
		\hline
		\rowcolor{black} \textcolor{white} {\textbf{Herramienta}} & \textcolor{white}{\textbf{Características}}  \\ \hline
		Struts 2 & Apache Struts 2 es un marco de aplicación web de código abierto para desarrollar aplicaciones web Java EE . Utiliza y amplía la API de Java Servlet para alentar a los desarrolladores a adoptar una arquitectura modelo-vista-controlador (MVC). El marco de trabajo WebWork surgió de Apache Struts con el objetivo de ofrecer mejoras y refinamientos al tiempo que conserva la misma arquitectura general del marco original de Struts. \hyperlink{b59}{[59]} \\
		\hline
		Spring & Spring es un framework para el desarrollo de aplicaciones y contenedor de inversión de control, de código abierto para la plataforma Java. Nos ofrece la posibilidad de crear código de alto rendimiento, liviano y reutilizable. Ya que su finalidad es estandarizar, agilizar, manejar y resolver los problemas que puedan ir surgiendo en el trayecto de la programación. \hyperlink{b59}{[59]} \\
		\hline
		Server Faces & Java Server Faces es un framework MVC (Modelo-Vista-Controlador) basado en el API de Servlets que proporciona un conjunto de componentes en forma de etiquetas definidas en páginas XHTML mediante el framework Facelets. Facelets se define en la especificación 2 de JSF como un elemento fundamental de JSF que proporciona características de plantillas y de creación de componentes compuestos. Antes de la especificación actual se utilizaba JSP para componer las páginas JSF. \hyperlink{b60}{[60]}\\
		\hline
	\end{tabular}
\caption{Cuadro de características de los frameworks estudiados.}
\label{tbl:frame}
\end{table}

\textbf {HERRAMIENTA SELECCIONADA: STRUTS 2}\\

\textbf {JUSTIFICACIÓN:}
Al usar un framework y seguir su convención, estamos programando de una manera aceptada, ya que esta es la idea que constituye la base de los patrones de diseño.  A pear de que como lo vimos en la comparativa, struts no es el único framework MVC existente en J2EE si es el más extendido y por lo tanto, usando Struts dispondremos de una gran cantidad de recursos: documentación (tutoriales, artículos, libros). El código para configurar repeticiones ya no es necesario porque los interceptores se ocupan de ello . La aportación principal de JSF no es MVC sino los componentes gráficos de usuario (GUI) de ``alto nivel" para la web. Por último con base en la experiencia obtenido Struts es el lenguaje más estudiado.\\

\newpage

 En el Cuadro \ref{tbl:sgbd} se exponen las características de los gestores de bases de datos para elegir el que mejor se adapte a nuestro proyecto.

\begin{table}[H]
	\centering
	\begin{tabular}{|p{3cm}|p{12cm}|}
		\hline
		\rowcolor{black} \textcolor{white} {\textbf{Herramienta}} & \textcolor{white}{\textbf{Características}}  \\ \hline
		MySQL & Es el sistema gestor de bases de datos relacional por excelencia.
		Es un SGBD multihilo y multiusuario utilizado en la gran parte de las páginas web actuales. Además es el más usado en aplicaciones creadas como software libre. 
		Las principales ventajas de este Sistema Gestor de Bases de datos son: Facilidad de uso y gran rendimiento, facilidad para instalar y configurar, soporte multiplataforma y soporte SSL. \hyperlink{b61}{[61]}\\
		\hline
		MariaDB & Este SGBD es una derivación de MySQL que cuenta con la mayoría de características de este e incluye varias extensiones. Entre las principales características de este Sistema Gestor de Bases de datos se encuentran: Aumento de motores de almacenamiento, gran escalabilidad, seguridad y rapidez en transacciones, extensiones y nuevas características relacionadas con su aplicación para Bases de datos NoSQL. \hyperlink{b61}{[61]} \\
		\hline
		PostgreSQL & Sus principales características son: control de Concurrencias multiversión (MVCC), flexibilidad en cuanto a lenguajes de programación multiplataforma, pispone de una herramienta (pgAdmin, https://www.pgadmin.org/) muy fácil e intuitiva para la administración de las bases de datos. Robustez, Eficiencia y Estabilidad.
		La principal desventaja es la lentitud para la administración de bases de datos pequeñas ya que está optimizado para gestionar grandes volúmenes de datos. \hyperlink{b61}{[61]}\\
		\hline
	\end{tabular}
	\caption{Cuadro de características de los SGBD (Sistema Gestor de Bases de Datos) estudiados.}
	\label{tbl:sgbd}
\end{table}

\textbf {HERRAMIENTA SELECCIONADA: MySQL}\\

\textbf {JUSTIFICACIÓN:}
MySQL es el gestor más utilizado en los proyectos escolares, razón por la cual es la herramienta con la que más experiencia se cuenta, sin olvidar que la velocidad al realizar las operaciones es alta, es uno de los gestores con mejor rendimiento, tiene un bajo consumo de recursos, es fácil de configurar e instalar. Soporta gran variedad de Sistemas Operativos y se adapta perfectamente a las necesidades del proyecto.
\newpage
\section{Sprint 1: Gestión de colaboradores}

\subsection{Iniciar sesión}
\subsubsection {Análisis}
Se requiere un proceso mediante el cual se controle el acceso individual al sistema mediante la identificación de usuario utilizando credenciales provistas por el registro de personal.\\

\subsection{Gestionar colaborador}
El actor puede visualizar aquellas personas de la organización que se encuentran involucradas en uno o más proyectos, del  mismo modo el actor puede realizar diferentes acciones, como un nuevo registro o consultar, modificar y eliminar personas previamente registradas.

\subsubsection {Análisis}
Es importante tener un mecanismo de control sobre aquellas entidades humanas que participan en las activades, toma de desiciones y otras funciones dentro de un proyecto de software. Estas personas pueden estar involucradas en uno o más proyectos.
\bigskip
El administrador, tiene la facultad y poder de:\\

\textbf {Registrar colaborador:} Para el registro fué de suma importancia establecer los datos de entrada más relevantes para el sistema, esto con el fin de no saturar la base de datos.\\

La información solicitada para el registro de personal es la siguiente:

\begin{itemize}
	\item \textbf{CURP:} Con el fin de tener unicidad de registros en la base de datos.
	\item \textbf{Nombre, Primer Apellido, Segundo Apellido:} Con el fin de identificar a las personas.
	\item \textbf{Correo electrónico:} Para utilizarse como nombre de usuario en el inicio de sesión.
	\item \textbf{Contraseña:} Con el fin de tener una autenticación segura y comprobada en el sistema.
\end{itemize}

Una vez que se realiza el registro de una persona de manera exitosa, esta podrá participar en algún proyecto.\\

\textbf {Modificar colaborador:} Dentro del sistema existe la posibilidad de que los datos de una persona previamente registrada sean modificados, a excepción de la CURP.

\textbf {Eliminar colaborador:} El sistema debe permitir al administrador eliminar el registro de una persona siempre y cuando esta no lidere ningún proyecto.

\section{Sprint 2: Gestión de proyectos}
\subsection{Gestionar proyectos de administrador}
\subsubsection {Análisis}
El administrador puede visualizar todos los proyectos registrados en el sistema, así como registrar, modificar o eliminar un proyecto.\\

El administrador, tiene la facultad y poder de:\\

\textbf {Registrar Proyecto:}
Para registrar los proyectos se le solicita al administrador ingresar la siguiente información (atributos).

\begin{itemize}
	\item \textbf{Clave:} Con el fin de tener unicidad de registros en la base de datos.
	\item \textbf{Nombre:} Con el fin de identificar los módulos.
	\item \textbf{Fecha de inicio:} Que identifica la fecha de inicio del proyecto.
	\item \textbf{Fecha de término:} Que identifica la fecha fin del proyecto prevista.
	\item \textbf{Fecha de inicio programada:} Que identifica la fecha de inicio programada del proyecto.
	\item \textbf{Fecha de término programada:} Que identifica la fecha fin del proyecto programada.
	\item \textbf{Líder del Proyecto:} Para la creación de un proyecto debe de haber al menos un colaborador registrado.
	\item \textbf{Descripción} Para especificar las características y contenido del proyecto.
	\item \textbf{Contraparte:} Para identificar al cliente o solicitante del proyecto.
	\item \textbf{Presupuesto:} Con el fin de identificar el presupuesto destinado al proyecto.
	\item \textbf{Estado del Proyecto:} Con el fin de identificar en que estado se encuentra el proyecto.
\end{itemize}

Se puede registrar un proyecto siempre y cuando exista al menos un colaborador registrado, así como la iformaciónn referente a los estados del proyecto.\\

Una vez registrado el proyecto se podrán gestionar los Términos del glosario, Entidades, Reglas de negocio, Mensajes y Actores.\\

\textbf {Modificar Proyecto:}
Dentro del sistema existe la posibilidad de modificar la información de un proyecto previamente registrado, siempre y cuando el proyecto se encuentre en estado "En negociación" o "Iniciado"\\

\textbf {Eliminar Proyecto:}
El sistema debe permitir al administrador eliminar el registro de un proyecto.\\



\section{Sprint 3: Gestión de Proyectos de Colaborador}
\subsection{Gestionar proyectos de colaborador}
\subsubsection {Análisis}
El actor puede visualizar los proyectos en los que participa, este es el punto de acceso para gestionar: módulos, términos del glosario, entidades, reglas de negocio, mensajes y actores, así como para descargar el documento de análisis y en caso de ser líder, elegir a los colaboradores.\\

\textbf {Elegir Colaboradores:} El líder de análisis tiene la facultad de seleccionar a aquellas personas que colaborarán en el proyecto. Para que una persona puedas ser colaboradora tiene que haberse registrado previamente en el módulo de registro de personal.\\

\section{Sprint 4: Gestión de Módulos}
\subsection{Gestionar módulos}

\subsubsection {Análisis}
Para facilitar la construcción de sistemas de software, la estrategia de Divide y Vencerás es una técnica imprescindible, la cual se basa en la descomposición de un problema en subproblemas de su mismo tipo, lo que permite disminuir la complejidad de los mismos. La manera en la que se descompone un sistema es através de sus módulos, es por eso que se le proporcionar al actor un mecanismo para llevar el control de los módulos de un proyecto. \\

\textbf {Registrar Módulo:}
Para registrar los módulos del proyecto se le solicita al actor ingresar la siguiente información (atributos).

\begin{itemize}
	\item \textbf{Clave:} Con el fin de tener unicidad de registros en la base de datos.
	\item \textbf{Nombre:} Con el fin de identificar los módulos.
	\item \textbf{Descripción} Para especificar las características y contenido del módulo.
\end{itemize}

\textbf {Modificar Módulo:}
Dentro del sistema existe la posibilidad de modificar los datos de un módulo previamente registrado, a excepción de su clave.\\

\textbf {Eliminar Módulo:} 
El sistema debe permitir al actor eliminar el registro de un módulo siempre y cuando no existan referencias al contenido del módulo a eliminar desde elementos de algún otro módulo.

\section{Sprint 5: Gestión de Términos de glosario}
\subsection{Gestionar términos del glosario}

\subsubsection {Análisis}
En un proyecto es de suma importancia que se tenga una lista de palabras y expresiones clasificadas de un texto con su respectivo significado. La correcta definición conceptual de los términos de negocio en el sistema se hace imprescindible para comprenderlo.

El colaborador, tiene la facultad y poder de:\\

\textbf {Registrar Término:}
Para registrar la información de un término se le solicita al actor ingresar la siguiente información (atributos).
\begin{itemize}
	\item \textbf{Nombre:} Con el fin de identificar los términos.
	\item \textbf{Descripción:} Para especificar las características y contenido del término.
\end{itemize}

\textbf {Modificar Término:}
Dentro del sistema existe la posibilidad de modificar los datos de un término previamente registrado.\\

\textbf {Eliminar Término:} 
El sistema debe permitir al actor eliminar el registro de un término existente.\\

\textbf {Consultar Término:} 
El sistema debe permitir al actor consultar la información de un término existente.

\newpage

\section{Sprint 6: Gestión de Entidades}
\subsection{Gestionar Entidades}

\subsubsection {Análisis}
Cuando se habla sobre la construcción de un proyecto de software, se habla también de la interacción de diversas entidades para lograr un fin. Una entidad es un objeto exclusivo único en el mundo real que se está controlando. Una entidad puede referirse a una persona, organización, tipo de objeto o concepto sobre los que se almacena información. Al ser un elemento clave participe dentro del desarrollo de software es importante contar con un catálogo de estos elementos.\\

El colaborador, tiene la facultad y poder de:\\

\textbf {Registrar Entidad:}
Para registrar la información de una entidad se le solicita al colaborador ingresar la siguiente información (atributos).
\begin{itemize}
	\item \textbf{Clave:} Que permitirá distinguir que el Elemento es una Entidad en una palabra corta.
	\item \textbf{Número: (Generado Automáticamente)} Para llevar un control interno de la cantidad de elementos en el proyecto.
	\item \textbf{Nombre:} Nombre que identificará a la Entidad en una frase o enunciado.
\end{itemize}

\textbf {Modificar Entidad:}
Dentro del sistema existe la posibilidad de modificar los datos de una entidad previamente registrada.\\

\textbf {Eliminar Entidad:} 
El sistema debe permitir al colaborador eliminar el registro de una entidad existente.\\

\textbf {Consultar Entidad:} 
El sistema debe permitir al colaborador consultar la información de una entidad existente.


\section{Sprint 7: Gestión de Atributos}
\subsection{Gestionar Atributos}

\subsubsection {Análisis}
Al considerar la inclusión de entidades como elementos participes dentro del Proyecto, no se deben perder de vista los atributos propios de cada Entidad. Un Atributo es una característica o rasgo de una Entidad que la describe, por ejemplo, el tipo de entidad Persona tiene el atributo Fecha de nacimiento. \\

El colaborador, tiene la facultad y poder de:\\

\textbf {Registrar Atributo:}
Para registrar la información de un atributo se le solicita al colaborador ingresar la siguiente información (atributos).
\begin{itemize}
	\item \textbf{Nombre:} Que identificará al Atributo en la Entidad en una frase o enunciado.
	\item \textbf{Descripción:} Texto que describirá al Atributo  en uno o más párrafos.
	\item \textbf{Obligatorio:} Bandeara que indicará si el Atributo es obligatorio o no lo es.
	\item \textbf{Longitud:} Número que describirá la longitud máxima del Atributo.
\end{itemize}

\textbf {Modificar Atributo:}
Dentro del sistema existe la posibilidad de modificar los datos de un atributo previamente registrado.\\

\textbf {Eliminar Atributo:} 
El sistema debe permitir al colaborador eliminar el registro de un atributo existente.\\


\section{Sprint 8: Gestión de Reglas de Negocio}

\subsection{Gestionar Reglas de negocio}
\subsubsection {Análisis}
Para poder establecer las condiciones del proyecto que se está trabajando, es importante que se tenga un catálogo de reglas de negocio, estas sirven para definir o restringir alguna acción en los procesos de lo que se está desarrollando. Gracias a este elemeno sabemos cómo se deben realizar ciertas operaciones y si hay algún límite que se debe aplicar en las trayectorias de los casos de uso.\\

A través de la gestión de este elemento el colaborador podrá:\\

\textbf {Registrar Regla de Negocio:}
Para registrar la información de la regla de negocio se le solicita al colaborador ingresar la siguiente información (atributos).
\begin{itemize}
	\item \textbf{Nombre:} Que identificará a la regla de negocio en una frase o enunciado.
	\item \textbf{Descripción:} Texto que describirá a la regla de negocio  en uno o más párrafos.
	\item \textbf{Redacción:} Texo que contiene el enunciado de la regla de negocio.
	\item \textbf{Tipo:} La clasificación en donde se encuentra la regla de negocio.\\
	En caso de que la regla de negocio sea de tipo comparación de atributos:
	\item \textbf{Entidad 1 y 2:} Se seleccionará la entidad de las previamente registradas.
	\item \textbf{Atributo 1 y 2:} Se seleccionará con base en las entidades seleccionadas.
	\item \textbf{Operador:} Se seleccionará el operador que comparará los atributos.\\
	En caso de que la regla de negocio sea de tipo unicidad de parámetros o formato correcto:
	\item \textbf{Entidad:} Se seleccionará la entidad de las previamente registradas.
	\item \textbf{Atributo:} Se seleccionará de una lista con base en la entidad previamete seleccionada.
\end{itemize}

\textbf {Modificar Regla de Negocio:}
Dentro del sistema existe la posibilidad de modificar los datos de una regla de negocio previamente registrada.\\

\textbf {Eliminar Regla de Negocio:} 
El sistema debe permitir al colaborador eliminar el registro de una regla de negocio existente.\\

\textbf {Consultar Regla de Negocio:} 
El sistema debe permitir al colaborador consultar y visualizar la información de una regla de negocio existente.\\

\section{Sprint 9: Gestión de Mensajes}

\subsection{Gestionar Mensajes}
\subsubsection {Análisis}
Los mensajes son un elemento fundamental que debe considerarse antes, durante y después del desarrollo de un sistema, ya que determina la forma de comunicación que se entabla con el usuario de una manera visual a través de mensajes que se van mostrando a cada acción realizada. Para la construcción del documento de casos de uso es importante tener ya definidos estos mensajes, por lo que se incluye este catálogo y así poder referenciar los mensajes al momento de editar el caso de uso.\\

A través de la gestión de este elemento el colaborador podrá:\\

\textbf {Registrar Mensaje:}
Para registrar la información de un mensaje se le solicita al colaborador ingresar la siguiente información (atributos).
\begin{itemize}
	\item \textbf{Nombre:} Que identificará el mensaje en una frase o enunciado.
	\item \textbf{Descripción:} Texto que describirá el propósito del mensaje en uno o más párrafos.
	\item \textbf{Redacción:} Texo que contendrá el enunciado del mensaje tal como se mostrará en el sistema.
	\item \textbf{Parametrizado:} Para el uso de tokens en los mensajes.\\
\end{itemize}

\textbf {Modificar Mensaje:}
Dentro del sistema existe la posibilidad de modificar los datos de un mensaje previamente registrado.\\

\textbf {Eliminar Mensaje:} 
El sistema debe permitir al colaborador eliminar el registro de un mensaje existente.\\

\textbf {Consultar Mensaje:} 
El sistema debe permitir al colaborador consultar y visualizar la información de un mensaje existente.\\


\section{Sprint 10: Gestión de Actores}

\subsection{Gestionar Actores}
\subsubsection {Análisis}

Los actores son uno de los elementos fundamentales dentro del sistema; los actores son cualquier individuo, grupo, entidad, organización, máquina o sistema de información externos; con los que el negocio interactúa; operaran el sistema con base en un rol asignado, por lo cual es importante tener un catálogo con los actores ya definidos para referenciarlos al momento de crear el caso de uso.

A través de la gestión de este elemento el colaborador podrá:\\

\textbf {Registrar Actor:}
Para registrar la información de un actor se le solicita al colaborador ingresar la siguiente información (atributos).
\begin{itemize}
	\item \textbf{Nombre:} Que identificará al actor en una frase o enunciado.
	\item \textbf{Descripción:} Texto que describirá el propósito del actor y sus funcionalidades en uno o más párrafos.
	\item \textbf{Cardinalidad:} Es el número de actores que participarán o serán requeridos en el sistema. Es un tipo de dato para el sistema y puede tomar alguno de los siguientes valores: Uno, Muchos u Otro.
\end{itemize}

\textbf {Modificar Actor:}
Dentro del sistema existe la posibilidad de modificar los datos de un actor previamente registrado.\\

\textbf {Eliminar Actor:} 
El sistema debe permitir al colaborador eliminar el registro de un actor existente.\\

\textbf {Consultar Actor:} 
El sistema debe permitir al colaborador consultar y visualizar la información de un actor existente.\\

\section{Sprint 11: Gestión de Pantallas}

\subsection{Gestionar Pantallas}
\subsubsection {Análisis}

Cuando se está construyendo el caso de uso es de gran ayuda tener el diseño o maqueta del modelo de las pantallas o interfaces que el sistema tendrá para la demostración, evaluación del diseño, o correción de las mismas. De ahí radica la importancia de tener un catálogo de pantallas, de esta manera el colaborador podrá mostrar visualmente cual es el objetivo de sus casos de uso.

A través de la gestión de este elemento el colaborador podrá:\\

\textbf {Registrar Pantalla:}
Para registrar la información de una pantalla se le solicita al colaborador ingresar la siguiente información (atributos).
\begin{itemize}
	\item \textbf{Nombre:} Que identificará a la pantalla en una frase o enunciado.
	\item \textbf{Descripción:} Texto que describirá el propósito de la pantalla y sus funcionalidades en uno o más párrafos.
	\item \textbf{Imagen:} Es la representación visual de la pantalla en una imagen.
\end{itemize}

\textbf {Modificar Pantalla:}
Dentro del sistema existe la posibilidad de modificar los datos de una pantalla previamente registrada.\\

\textbf {Eliminar Pantalla:} 
El sistema debe permitir al colaborador eliminar el registro de una pantalla existente.\\

\textbf {Consultar Pantalla:} 
El sistema debe permitir al colaborador consultar y visualizar la información de una pantalla existente.\\

\section{Sprint 12: Gestión de Casos de Uso}

\subsection{Gestionar Casos de Uso}
\subsubsection {Análisis}

La parte central y la razón de ser de todos los elementos que se crearon en sprints anteriores tienen que ver con la creación de casos de uso. En este punto del proyecto es en donde se integran todos los elementos creados en catálogos, del mismo modo en esta gestión se crean mas elementos que mas que formar parte del caso de uso son parte de.

A través de esta gestión el colaborador podrá:\\

\textbf {Registrar Caso de Uso:}
Para registrar la información de un caso de uso se le solicita al colaborador ingresar la siguiente información (atributos).
\begin{itemize}
	
	\item \textbf{Nombre:} Que identificará al caso de uso en una frase o enunciado.
	\item \textbf{Resúmen:} Texto que describirá el propósito del caso de uso, así como su contenido descrito en uno o más párrafos.\\
	De la sección Descripción del caso de uso:
	\item \textbf{Actores:} En donde se referenciarán los actores involucrados en el caso de uso y que ya fueron previamente registrados en la gestión correspondiente.
	\item \textbf{Entradas:} Se referencian los atributos o campos donde se recibe la entrada de información para su procesamiento.
	\item \textbf{Salidas:} Se referencian los atributos de salida que dan como resultado el procesamiento de la información.
	\item \textbf{Reglas de negocio:} Se referencian las reglas de negocio que se ven involucradas en el caso de uso y que ya fueron previamente registrados en la gestión correspondiente.
	
\end{itemize}

\textbf {Modificar Caso de Uso:}
Dentro del sistema existe la posibilidad de modificar los datos de un caso de uso previamente registrado.\\

\textbf {Eliminar Caso de Uso:} 
El sistema debe permitir al colaborador eliminar el registro de un caso de uso existente.\\

\textbf {Consultar Caso de Uso:} 
El sistema debe permitir al colaborador consultar y visualizar la información de un caso de uso existente.\\

\section{Sprint 13: Gestión de Acciones}

\subsection{Gestionar Acciones}
\subsubsection {Análisis}
Al considerar las pantallas como catálogos también se tiene que tomar en cuenta que hay elementos presentes en ellas como las acciones. En este catálogo se tendrá una relación de aquellos componentes como botones, hipervínculos y opciones del menú que cumplen una funcionalidad específica dentro de la pantalla.

\textbf {Registrar Acción:}
Para registrar la información de las acciones se le solicita al colaborador ingresar la siguiente información (atributos).
\begin{itemize}
	
	\item \textbf{Nombre:} Que identificará a la acción en una frase o enunciado.
	\item \textbf{Descripción:} Descripción de la acción explicando su propósito.
	\item \textbf{Tipo:} A que categoria corresponde la acción
	\item \textbf{Pantalla Destino: Con cuál de las pantallas previamente registradas tendrá comunicación}
	\item \textbf{Imágen: El icono o imagen que la representa visualmente.}
\end{itemize}

\textbf {Modificar Acción:}
Dentro del sistema existe la posibilidad de modificar los datos de una acción previamente registrada.\\

\textbf {Eliminar Acción:} 
El sistema debe permitir al colaborador eliminar el registro de una acción existente.\\


\section{Sprint 14: Gestión de Trayectorias del Caso de Uso}
\subsection{Gestionar Trayectorias}
\subsubsection {Análisis}

Las trayectorias de un caso de uso agrupan un conjunto de pasos en específico que un sistema realiza en comunicación con el actor. Las trayectorias son un elemento que forma parte del caso de uso; por defecto se debe tener una trayectoria principal la cual describirá el comportamiento ideal del caso de uso, el otro tipo de trayectorias son alternativas y como su nombre lo indica son rutas alternas al comportamiento del caso de uso.

\textbf {Registrar Trayectoria:}
Para registrar la información de una trayectoria se le solicita al colaborador ingresar la siguiente información (atributos).
\begin{itemize}
	
	\item \textbf{Clave:} Nombre o clave que identificará a la trayectoria en una frase o enunciado.
	\item \textbf{Tipo:} Bandera que especifica si la trayectoria es alternativa o principal.
	\item \textbf{Condición:} Texto que determina la circunstancia con la que se ejecuta la trayectoria en caso de ser alternativa. Es una frase o enunciado
	\item \textbf{Fin del caso de uso:} Bandera que especifica si la trayectoria concluye el Caso de uso. Indica ”si” o ”no”.
	
\end{itemize}

\textbf {Modificar Trayectoria:}
Dentro del sistema existe la posibilidad de modificar los datos de una trayectoria previamente registrada.\\

\textbf {Eliminar Trayectoria:} 
El sistema debe permitir al colaborador eliminar el registro de una trayectoria existente.\\

\section{Sprint 15: Gestión de Pasos en las Trayectorias del Caso de Uso}

\subsection{Gestionar Pasos}
\subsubsection {Análisis}
Dentro de las trayectorias del caso de uso se encuentran agrupados los pasos que describen la interacción del usuario con el sistema. Cada paso tendrá una acción realizada por el usuario y posteriormente la acción que realiza el sistema como respuesta a la acción anterior.\\

\textbf {Registrar Pasos:}
Para registrar la información de los pasos se le solicita al colaborador ingresar la siguiente información (atributos).
\begin{itemize}
	
	\item \textbf{Número:} Número del paso en la trayectoria en un valor numérico entero.
	\item \textbf{Redacción:} Redacción del paso en una frase o enunciado explicando la acciónn a realizar.
	\item \textbf{Realiza:} Badera que especifica quién realiza el paso, si el actor o el sistema.
	
\end{itemize}

\textbf {Modificar Paso:}
Dentro del sistema existe la posibilidad de modificar los datos de un paso previamente registrado.\\

\textbf {Eliminar Paso:} 
El sistema debe permitir al colaborador eliminar el registro de un paso existente.\\

\section{Sprint 16 y Sprint 17: Gestión de Precondiciones y Postcondiciones del caso de uso}
\subsection{Gestionar Precondiciones/Postcondiciones}
\subsubsection {Análisis}
Las precondiciones son aquellas que disparan la ejecución del Caso de Uso, detro del caso de uso se requiere tener el registro de las condiciones que se tienen que cumplir para el caso de uso antes de inciarse.

\textbf {Registrar Precondiciones/Postcondiciones:}
Para registrar la información de las precondiciones se le solicita al colaborador ingresar la siguiente información (atributos).
\begin{itemize}
	\item \textbf{Tipo de condición:} Se selecciona de una lista si es una precondición o una postcondición.
	\item \textbf{Redacción de la precondición/postcondición:} Texto que describe desde el teclado la redacción.
\end{itemize}

\section{Sprint 18: Gestión de Puntos de Extensión del Caso de Uso}
\subsection{Gestionar Puntos de Extensión}
\subsubsection {Análisis}
Los puntos de extensión describen una región de la trayectoria en la que se puede extender el funcionamiento a través de otro caso de uso. Esta relación significa que un caso de uso puede estar basado en otro caso de uso más básico.

\textbf {Registrar Puntos de extensión:}
Para registrar la información de los puntos de extensión se le solicita al colaborador ingresar la siguiente información (atributos).
\begin{itemize}
	\item \textbf{Causa:} Texto que describe la razón por la que se extiende a otro Caso de Uso.
	\item \textbf{Región de la trayectoria:} Texto que describe la región de la rtayectoria dónde se hace el punto de extensión.
	\item \textbf{Caso de uso al que extiende:} Se selecciona de los casos de uso previamente registrados.
\end{itemize}

\textbf {Modificar Punto de extensión:}
Dentro del sistema existe la posibilidad de modificar los datos de un punto de extensión previamente registrado.\\

\textbf {Eliminar Punto de extensión:} 
El sistema debe permitir al colaborador eliminar el registro de un punto de extensión existente.\\
