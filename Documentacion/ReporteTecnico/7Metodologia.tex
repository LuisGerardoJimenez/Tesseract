\chapter{Metodología} \label{cap:siete}

En este capítulo se describen las actividades realizadas para la entrega de sprints planteados para la entrega del trabajo terminal.\\

En Scrum, la metodología bajo la cual se está desarrollando el proyecto, se ejecuta en bloques temporales cortos y fijos (iteraciones de un mes). Cada iteración proporciona un resultado completo, un incremento de producto que sea potencialmente entregable. Para ello, durante la iteración el equipo colabora estrechamente y se llevan a cabo las siguientes dinámicas:\\

Cada día el equipo realiza una reunión de sincronización, donde cada miembro inspecciona el trabajo de los otros para poder hacer las adaptaciones necesarias, comunica cuales son los impedimconfictos en los que se encuentra, actualiza el estado de la lista de tareas de la iteración (Sprint Backlog) y los gráficos de trabajo pendiente (Burndown charts).\\

El Facilitador, se encarga de que el equipo pueda mantener el foco para cumplir con sus objetivos.
Elimina los obstáculos que el equipo no puede resolver por sí mismo, protege al equipo de interrupciones externas que puedan afectar el objetivo de la iteración o su productividad.
\newpage

\section{Sprint 0: Configuración del ambiente de trabajo}
\subsection{Selección de herramientas}

\begin{table}[H]
	\centering
	\begin{tabular}{|p{3cm}|p{3cm}|p{6cm}|}
		\hline
		\rowcolor{black} \textcolor{white} {\textbf{Elemento}} & \textcolor{white}{\textbf{Herramienta}} & \textcolor{white}{\textbf{Justificación}}  \\ \hline
		Sistema de composición de textos & Latex & Es un sistema de software libre y de composición tipográfica de alta calidad, esta herramienta incluye características diseñadas para la producción de la documentación técnica en donde se establecen las características del proyecto. \\
		\hline
		Lenguaje de programación & Java & La idea principal de TESSERACT es que opere como un sistema web, java al ser un lenguaje multiplataforma nos permite construir un producto con las características deseadas. Otras de las ventajas que tiene java son sus librerias estandar, la facilidad de programación y su paradigma orientado a objetos. \\
		\hline
		Framework & Struts & Struts permite reducir el tiempo de desarrollo. Su carácter de "software libre" y su compatibilidad con todas las plataformas en las que Java Entreprise esté disponible lo convierten en una herramienta altamente disponible. \\
		\hline
		SGBD & Mysql & MySQL es un sistema gestor de base de datos muy popular, es usado por su simplicidad y rendimiento. Es la opción para algunas aplicaciones comerciales gracias a su facilidad de uso y el poco tiempo que requiere para ponerse en marcha.  \\
		\hline
	\end{tabular}
\end{table}
\newpage
%Inicio de sesión
%Proyectos
%Personal
%Colaboradores de proyecto
%Módulos
%Glosario

\section{Sprint 1: Gestión de colaboradores}

\subsection{Iniciar sesión}
\subsubsection {Análisis}
Se requiere un proceso mediante el cual se controle el acceso individual al sistema mediante la identificación de usuario utilizando credenciales provistas por el registro de personal.\\

\subsection{Gestionar colaborador}
El actor puede visualizar aquellas personas de la organización que se encuentran involucradas en uno o más proyectos, del  mismo modo el actor puede realizar diferentes acciones, como un nuevo registro o consultar, modificar y eliminar personas previamente registradas.

\subsubsection {Análisis}
Es importante tener un mecanismo de control sobre aquellas entidades humanas que participan en las activades, toma de desiciones y otras funciones dentro de un proyecto de software. Estas personas pueden estar involucradas en uno o más proyectos.
\bigskip
El administrador, tiene la facultad y poder de:\\

\textbf {Registrar colaborador:} Para el registro fué de suma importancia establecer los datos de entrada más relevantes para el sistema, esto con el fin de no saturar la base de datos.\\

La información solicitada para el registro de personal es la siguiente:

\begin{itemize}
	\item \textbf{CURP:} Con el fin de tener unicidad de registros en la base de datos.
	\item \textbf{Nombre, Primer Apellido, Segundo Apellido:} Con el fin de identificar a las personas.
	\item \textbf{Correo electrónico:} Para utilizarse como nombre de usuario en el inicio de sesión.
	\item \textbf{Contraseña:} Con el fin de tener una autenticación segura y comprobada en el sistema.
\end{itemize}

Una vez que se realiza el registro de una persona de manera exitosa, esta podrá participar en algún proyecto.\\

\textbf {Modificar colaborador:} Dentro del sistema existe la posibilidad de que los datos de una persona previamente registrada sean modificados, a excepción de la CURP.

\textbf {Eliminar colaborador:} El sistema debe permitir al administrador eliminar el registro de una persona siempre y cuando esta no lidere ningún proyecto.

\section{Sprint 2: Gestión de proyectos}
\subsection{Gestionar proyectos de administrador}
\subsubsection {Análisis}
El administrador puede visualizar todos los proyectos registrados en el sistema, así como registrar, modificar o eliminar un proyecto.\\

El administrador, tiene la facultad y poder de:\\

\textbf {Registrar Proyecto:}
Para registrar los proyectos se le solicita al administrador ingresar la siguiente información (atributos).

\begin{itemize}
	\item \textbf{Clave:} Con el fin de tener unicidad de registros en la base de datos.
	\item \textbf{Nombre:} Con el fin de identificar los módulos.
	\item \textbf{Fecha de inicio:} Que identifica la fecha de inicio del proyecto.
	\item \textbf{Fecha de término:} Que identifica la fecha fin del proyecto prevista.
	\item \textbf{Fecha de inicio programada:} Que identifica la fecha de inicio programada del proyecto.
	\item \textbf{Fecha de término programada:} Que identifica la fecha fin del proyecto programada.
	\item \textbf{Líder del Proyecto:} Para la creación de un proyecto debe de haber al menos un colaborador registrado.
	\item \textbf{Descripción} Para especificar las características y contenido del proyecto.
	\item \textbf{Contraparte:} Para identificar al cliente o solicitante del proyecto.
	\item \textbf{Presupuesto:} Con el fin de identificar el presupuesto destinado al proyecto.
	\item \textbf{Estado del Proyecto:} Con el fin de identificar en que estado se encuentra el proyecto.
\end{itemize}

Se puede registrar un proyecto siempre y cuando exista al menos un colaborador registrado, así como la iformaciónn referente a los estados del proyecto.\\

Una vez registrado el proyecto se podrán gestionar los Términos del glosario, Entidades, Reglas de negocio, Mensajes y Actores.\\

\textbf {Modificar Proyecto:}
Dentro del sistema existe la posibilidad de modificar la información de un proyecto previamente registrado, siempre y cuando el proyecto se encuentre en estado "En negociación" o "Iniciado"\\

\textbf {Eliminar Proyecto:}
El sistema debe permitir al administrador eliminar el registro de un proyecto.\\



\section{Sprint 3: Gestión de Proyectos de Colaborador}
\subsection{Gestionar proyectos de colaborador}
\subsubsection {Análisis}
El actor puede visualizar los proyectos en los que participa, este es el punto de acceso para gestionar: módulos, términos del glosario, entidades, reglas de negocio, mensajes y actores, así como para descargar el documento de análisis y en caso de ser líder, elegir a los colaboradores.\\

\textbf {Elegir Colaboradores:} El líder de análisis tiene la facultad de seleccionar a aquellas personas que colaborarán en el proyecto. Para que una persona puedas ser colaboradora tiene que haberse registrado previamente en el módulo de registro de personal.\\

\section{Sprint 4: Gestión de Módulos}
\subsection{Gestionar módulos}

\subsubsection {Análisis}
Para facilitar la construcción de sistemas de software, la estrategia de Divide y Vencerás es una técnica imprescindible, la cual se basa en la descomposición de un problema en subproblemas de su mismo tipo, lo que permite disminuir la complejidad de los mismos. La manera en la que se descompone un sistema es através de sus módulos, es por eso que se le proporcionar al actor un mecanismo para llevar el control de los módulos de un proyecto. \\

\textbf {Registrar Módulo:}
Para registrar los módulos del proyecto se le solicita al actor ingresar la siguiente información (atributos).

\begin{itemize}
	\item \textbf{Clave:} Con el fin de tener unicidad de registros en la base de datos.
	\item \textbf{Nombre:} Con el fin de identificar los módulos.
	\item \textbf{Descripción} Para especificar las características y contenido del módulo.
\end{itemize}

\textbf {Modificar Módulo:}
Dentro del sistema existe la posibilidad de modificar los datos de un módulo previamente registrado, a excepción de su clave.\\

\textbf {Eliminar Módulo:} 
El sistema debe permitir al actor eliminar el registro de un módulo siempre y cuando no existan referencias al contenido del módulo a eliminar desde elementos de algún otro módulo.

\section{Sprint 5: Gestión de Términos de glosario}
\subsection{Gestionar términos del glosario}

\subsubsection {Análisis}
En un proyecto es de suma importancia que se tenga una lista de palabras y expresiones clasificadas de un texto con su respectivo significado. La correcta definición conceptual de los términos de negocio en el sistema se hace imprescindible para comprenderlo.

El colaborador, tiene la facultad y poder de:\\

\textbf {Registrar Término:}
Para registrar la información de un término se le solicita al actor ingresar la siguiente información (atributos).
\begin{itemize}
	\item \textbf{Nombre:} Con el fin de identificar los términos.
	\item \textbf{Descripción:} Para especificar las características y contenido del término.
\end{itemize}

\textbf {Modificar Término:}
Dentro del sistema existe la posibilidad de modificar los datos de un término previamente registrado.\\

\textbf {Eliminar Término:} 
El sistema debe permitir al actor eliminar el registro de un término existente.\\

\textbf {Consultar Término:} 
El sistema debe permitir al actor consultar la información de un término existente.

\newpage

\section{Sprint 6: Gestión de Entidades}
\subsection{Gestionar Entidades}

\subsubsection {Análisis}
Cuando se habla sobre la construcción de un proyecto de software, se habla también de la interacción de diversas entidades para lograr un fin. Una entidad es un objeto exclusivo único en el mundo real que se está controlando. Una entidad puede referirse a una persona, organización, tipo de objeto o concepto sobre los que se almacena información. Al ser un elemento clave participe dentro del desarrollo de software es importante contar con un catálogo de estos elementos.\\

El colaborador, tiene la facultad y poder de:\\

\textbf {Registrar Entidad:}
Para registrar la información de una entidad se le solicita al colaborador ingresar la siguiente información (atributos).
\begin{itemize}
	\item \textbf{Clave:} Que permitirá distinguir que el Elemento es una Entidad en una palabra corta.
	\item \textbf{Número: (Generado Automáticamente)} Para llevar un control interno de la cantidad de elementos en el proyecto.
	\item \textbf{Nombre:} Nombre que identificará a la Entidad en una frase o enunciado.
\end{itemize}

\textbf {Modificar Entidad:}
Dentro del sistema existe la posibilidad de modificar los datos de una entidad previamente registrada.\\

\textbf {Eliminar Entidad:} 
El sistema debe permitir al colaborador eliminar el registro de una entidad existente.\\

\textbf {Consultar Entidad:} 
El sistema debe permitir al colaborador consultar la información de una entidad existente.


\section{Sprint 7: Gestión de Atributos}
\subsection{Gestionar Atributos}

\subsubsection {Análisis}
Al considerar la inclusión de entidades como elementos participes dentro del Proyecto, no se deben perder de vista los atributos propios de cada Entidad. Un Atributo es una característica o rasgo de una Entidad que la describe, por ejemplo, el tipo de entidad Persona tiene el atributo Fecha de nacimiento. \\

El colaborador, tiene la facultad y poder de:\\

\textbf {Registrar Atributo:}
Para registrar la información de un atributo se le solicita al colaborador ingresar la siguiente información (atributos).
\begin{itemize}
	\item \textbf{Nombre:} Que identificará al Atributo en la Entidad en una frase o enunciado.
	\item \textbf{Descripción:} Texto que describirá al Atributo  en uno o más párrafos.
	\item \textbf{Obligatorio:} Bandeara que indicará si el Atributo es obligatorio o no lo es.
	\item \textbf{Longitud:} Número que describirá la longitud máxima del Atributo.
\end{itemize}

\textbf {Modificar Atributo:}
Dentro del sistema existe la posibilidad de modificar los datos de un atributo previamente registrado.\\

\textbf {Eliminar Atributo:} 
El sistema debe permitir al colaborador eliminar el registro de un atributo existente.\\


