\chapter{Resultados Obtenidos} \label{cap:nueve}

\section{Gestor de proyectos y colaboradores}

A través del cual el administrador registra los proyectos existentes así como la información personal de todo los involucrados en el desarrollo de los sistemas registrados. Posteriormente el administrador se encarga de asignar un líder a cada proyecto (de la relación de colaboradores). Una vez asignado, este actor podrá gestionar los proyectos como analista desde su sesión.

\section{Concentración de la información de los elementos de casos de uso}

Un caso de uso está compuesto por elementos que forman parte del proyecto y que intervienen en la redacción del caso de uso, estos elementos son: Actores, Términos del Glosario, Reglas de negocio, Entidades, Módulos, Mensajes y Pantallas. Por otro lado tenemos elementos que son propios del caso de uso, estos elementos son: Trayectorias, Pasos, Precondiciones, Poscondiciones y puntos de extensión.
Toda esta información se encuentra organizada y centralizada dentro de un repositorio de datos.

\section{Generador de tokens para la construcción de los casos de uso}

Los elementos que componen la redacción del caso de uso pero que no son propios del este son gestionados de forma externa, sin embargo gracias a los valores parametrizados es posible referenciar en el editor del caso de uso cada uno de estos elementos para ser utilizados dentro de la gestión del caso de uso.

\section{Revisión y libreación de los casos de uso generados}

Al finalizar la elaboración de un caso de uso debidamente redactado y referenciado, los colaboradores de los proyectos indican al líder que culminaron su caso de uso, el líder procede a revisarlo y realizar las observaciones correspondientes dentro de la misma plataforma. Si hay detalles en el caso de uso registra una serie de comentarios, mismos que el analista podrá visualizar para corregirlos. Un caso de uso podrá ser liberado hasta que no haya observaciones por parte del líder y lo libere.

\section{Generador del documento de análisis}

El documento de análisis no solo contiene los casos de uso elaborados, también presenta el catálogo de todos los elementos del proyecto para facilitar la lectura de los casos de uso .

