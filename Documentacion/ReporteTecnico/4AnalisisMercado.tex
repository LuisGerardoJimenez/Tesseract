\chapter{Análisis de mercado} \label{cap:cuatro}

En este apartado se demuestra la viabilidad comercial del trabajo terminal TESSERACT en México, así mismo se realiza un estudio en donde se determina el campo en donde un sistema con las características del generador de documento de casos de uso podría generar un mayor impacto y aceptación por parte de los equipos de desarrollo de software. Cabe resaltar que TESSERACT no pretende ser comercializado por el momento.

%---------------------------------------------------------
\section{Situación actual  y evolución del mercado}

El software es un elemento consustancial a la economía moderna, es uno de los sectores tecnológicos más competitivos, se usa en en una gran cantidad de productos manufacturados y servicios, por lo que  la elaboración de programas de cómputo figura en casi todas las industrias y es, de hecho, factor de éxito de todos los sectores de la economía. Esta industra ha tenido una evolución constante en lo que se refiere a las metodologías o bien, las formas en las cuales se realiza la planeación para el diseño del software, básicamente con el objetivo de mejorar, optimizar procesos y ofrecer una mejor calidad. \hyperlink{b22}{[22]}.\\

En el campo del desarrollo de software, existen dos grupos de metodologías, las denominadas tradicionales (formales) y las ágiles. Las primeras son un tanto rígidas, exigen una documentación exhaustiva y se centran en cumplir con el plan del proyecto definido totalmente en la fase inicial del desarrollo del mismo; mientras que la segunda enfátiza el esfuerzo en la capacidad de respuesta a los cambios, las habilidades del equipo y mantener una buena relación con el usuario. La metodología que sea seleccionada, debe ser adaptada al contexto del proyecto, teniendo en cuenta los recursos técnicos y humanos; tiempo de desarrollo y tipo de sistema. \hyperlink{b24}{[24]}.\\

Dean Leffingwell, autor de Scaling Software Agility, menciona que los Casos de Uso son una herramienta valiosa para modelar requerimientos en metodologías Lean/Ágiles de gran envergadura, si embargo, no es común encontrar casos de uso en los proyectos ágiles (especialmente en XP y Scrum), en donde se suele utilizar historias de usuario para recolectar los requerimientos \hyperlink{b12}{[12]}.\\

Ahora bien, de acuerdo a la teoría expuesta, TESSERACT al ser una herramienta que asiste a la generación del documento de casos de uso, se convierte en un instrumento que puede contribuir en cualquier metodología, ya sea formal o ágil, sin embargo el beneficio e impacto incrementa cuando se utiliza en la construcción de sistemas con metodologias formales y de gran escala, los casos de uso son una herramienta muy poderosa para explorar las interacciones entre los usuarios, los sistemas, y los sub-sistemas. Más aún, la técnica de casos de uso es la mejor forma para identificar todos los escenarios alternativos que se nos aparecen, fundamentales para asegurar la calidad de los sistemas. En los desarrollos ágiles, los casos de uso no reemplazan a las historias de usuario pero pueden resultar sumamente útiles para analizar, elaborar y comprender el funcionamiento deseado de sistemas complejos.

\subsection{Industria Mexicana del Software}

Para conocer el nivel de oportunidad que tiene TESSERACT dentro de la industria en México, es importante conocer de manera cuantitativa cual es el perfil de las empresas desarrolladoras de software en México.\\

\textbf{Localización Geográfica de las Empresas Participantes}\\
 
Las empresas participantes en el estudio se localizan en 11 de los 32 estados de la República Mexicana, presentando la siguiente distribución: 2.9\% Chihuahua, 1.5\%  en Coahuila, 44.1\%  en la Ciudad de México, 11.8\%  en Durango, 2.9\%  en el Estado de México, 1.5\%  en Guanajuato, 2.9\%  en Jalisco, 2.9\%  en Michoacán, 2.9\%  en Morelos, 23.5\%  en Nuevo León y 2.9\%  en Querétaro. Esta concentración es similar a la de otros estudios realizados para este sector en México.\hyperlink{b13}{[13]}\\

\textbf{Número de Empresas Desarrolladoras de Software en México}\\

La respuesta a esta pregunta no tiene una cifra exacta. De acuerdo con estimaciones realizadas por la empresa denominada ``ESANE consultores" sobre el número total de empleados y empresas de la Industria del Software en México, el número aproximado de empresas de la industria mexicana del software podría ser del orden de 1,500 empresas.\hyperlink{b13}{[13]}.\\
 
\textbf{Tamaño de las Empresas}\\

El estudio revela que el 85.29\% de las empresas del sector de la Industria Mexicana del Software son de tamaño micro (54.41\%) y pequeño (30.88\%), el 5.8\% mediana, y tan sólo el 8.82\% son de tamaño grande (con un número de empleados mayor a 100) \hyperlink{b13}{[13]}.\\

Las oportunidades que se tienen de posicionar TESSERACT en el mercado son amplias, la industria del software muestra un constante crecimiento y en México hay posibilidades de colocar nuestra herramienta en diversos sectores para contribuir en el desarrollo y construcción de software. 