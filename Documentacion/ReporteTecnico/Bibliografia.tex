%=========================================================
\chapter{Bibliografía}

\begin{description}
	
		\item[\hypertarget{b01}{[1]}] Rui, K. Butler, G. (2003, April 21). Refactoring use case models: the metamodel [Online]. Available: https://dl.acm.org/citation.cfm?id=783140
		
		\item[\hypertarget{b02}{[2]}] Shuang, L. Sun, L. (2014, September 19). Automatic early defects detection in use case documents [Online]. Available: https://dl.acm.org/citation.cfm?id=2642969
		
		\item[\hypertarget{b03}{[3]}] 
		J. Lee. (1999, August). Analyzing user requirements use cases a goal driven approach. [Online]. Avaible: https://ieeexplore.ieee.org/document/776956
		
		\item[\hypertarget{b04}{[4]}]  L. Julijana. (2007, August). “Information Systems Modeling with Use Cases” IEEE Computer [Online]. Available:
		https://ieeexplore.ieee.org/document/4283759
		
		\item[\hypertarget{b05}{[5]}] Jason Gorman, J. G. (2007, 9 marzo). 10 Common Use Case Pitfalls. Recuperado 23 abril, 2018,  [Online]. Available: http://codemanship.co.uk/parlezuml/blog/?postid=364
		
		\item[\hypertarget{b06}{[6]}] Susan Lilly [2002, August]. “Use Case Pitfalls: Top 10 Problems from Real Projects Using Use Cases” [Online]. Avaible: https://ieeexplore.ieee.org/document/787547
		
		\item[\hypertarget{b07}{[7]}] Universidad Michoacana de San Nicolás de Hidalgo. (2014): “Competitividad y factores de éxito en empresas desarrolladoras de software”.[Online]. \\ Avaible: https://dialnet.unirioja.es/servlet/articulo?codigo=5101928
		
		\item[\hypertarget{b08}{[8]}] Mr. Vikas S. Chomal, Dr. Jatinderkumar R. Saini, (2014). “Significance of Software Documentation in Software
		Development Process“
 / (7 páginas). International Journal of Engineering Innovation and Research. 
 
 		\item[\hypertarget{b62}{[62]}] Enya Valeria Martínez Paredes, Gerardo Ismael Solís Garrido, Er-ramani Velasco Rosete "Prototipo generador de guiones de prueba", ESCOM-IPN
 	
		\item[\hypertarget{b09}{[9]}] Pressman, Roger. (2010). Ingenieria de Software. Un enfoque práctico / 7 ED.(777 páginas). USA: Mcgraw-Hill Interamericana.
		
		\item[\hypertarget{b40}{[40]}] J. Lee, International biographical dictionary of computer pioneers. London: Fitzroy Dearborn Publ, 1996.
		
		\item[\hypertarget{b41}{[41]}] "What is IEEE?", Supportcenter.ieee.org, 2019. [Online].\\
		Available: https://supportcenter.ieee.org/app/answers/detail/a\_id/190/~/what-is-ieee
		
		\item[\hypertarget{b38}{[38]}] BAUER, Fritz tomada de NAUR, P y RANDELL, B (editores). Software Engineering: A report on a Conference sponsored by the NATO Science Comittee/NATO. 1969 citada en MARTIN, James y McCLURE, Carma. Structured
		Techniques for Computing. Prentice-Hall. Englewood Cliffs, NJ, EE.UU. 1985 
		
		\item[\hypertarget{b39}{[39]}] "729-1983 - IEEE Standard Glossary of Software Engineering Terminology - IEEE Standard", Ieeexplore.ieee.org, 1983. [Online]. Available: https://ieeexplore.ieee.org/document/7435207. [Accessed: 28- Oct- 2019].
		
		\item[\hypertarget{b42}{[42]}] Jacobson, I., Object-Oriented Software Engineering, Addison-Wesley, 1992
		
		\item[\hypertarget{b43}{[43]}] A. Cockburn, Writing effective use cases by Alistair Cockburn. Addison-Wesley: Pearson Professional Education, 2001.
		
		\item[\hypertarget{b44}{[44]}] "IBM Knowledge Center", Ibm.com, 2019. [Online].\\ Available: https://www.ibm.com/support/knowledgecenter/es/SSWSR9\_11.6.0/com.ibm.\\mdmhs.overview.doc/entityconcepts.html.
		
		\item[\hypertarget{b45}{[45]}] "Federal Standard 1037C: Glossary of Telecommunications Terms", Its.bldrdoc.gov, 1996. [Online]. Available: https://www.its.bldrdoc.gov/fs-1037/fs-1037c.htm
		
		\item[\hypertarget{b46}{[46]}] S. Zorraquino Comunicación, "Interfaz gráfica de usuario | Zorraquino", Zorraquino, 2019. [Online]. Available: https://www.zorraquino.com/diccionario/marketing-digital/que-es-interfaz-grafica-de-usuario.html.
		
		\item[\hypertarget{b47}{[47]}]J. JUNOY, "Mensajes del sistema", Alzado.org, 2005. [Online]. \\ Available: https://www.alzado.org/articulo.php?id\_art=429. [Accessed: 29- Oct- 2019].
		
		\item[\hypertarget{b48}{[48]}]"Las precondiciones y postcondiciones en los casos de uso", Jummp, 2012. [Online]. Available: https://jummp.wordpress.com/2011/07/22/las-precondiciones-y-postcondiciones-en-los-casos-de-uso/.
		
		\item[\hypertarget{b49}{[49]}]J. Barquinero, “Tipos de relaciones en diagramas de casos de uso. UML. | Blog SEAS", Blog de SEAS, 2013. [Online]. Available: https://www.seas.es/blog/informatica/tipos-de-relaciones-en-diagramas-de-casos-de-uso-uml/. 
		
		\item[\hypertarget{b50}{[50]}] Fuggetta, Alfonso. “A classification of CASE technology.” Computer 26 (1993).
		
		\item[\hypertarget{b51}{[51]}] H. Peress, "MVC (Model, View, Controller) explicado.", CódigoFacilito, 2015. [Online]. Available: https://codigofacilito.com/articulos/mvc-model-view-controller-explicado. \\
		
		\item[\hypertarget{b52}{[52]}] J. Uria Tejero and J. Camps Riva, "Diseño e implementación de un framework de persistencia", Openaccess.uoc.edu, 2009. [Online]. Available: http://openaccess.uoc.edu/webapps/o2/bitstream/10609/761/1/00883tfc.pdf.
		
		\item[\hypertarget{b25}{[25]}] (https://docs.spring.io/spring-framework/docs/current/spring-framework-reference/data-access.html\#transaction)
		
		\item[\hypertarget{b26}{[26]}] (https://docs.spring.io/spring-framework/docs/current/spring-framework-reference/data-access.htmlt\#ransaction-declarative)
		
		\item[\hypertarget{b27}{[27]}] https://struts.apache.org/birdseye.html 
		
		
		\item[\hypertarget{b29}{[29]}]
		https://struts.apache.org/core-developers/interceptors.html\#order-of-interceptor-execution
		
		\item[\hypertarget{b10}{[10]}] B. Imran S. and H. Irfan, "UCD-generator - a LESSA application for use case design - IEEE Conference Publication", Ieeexplore.ieee.org, 2007. [Online]. \\Available: https://ieeexplore.ieee.org/document/4381333.
		
		\item[\hypertarget{b11}{[11]}] G. Booch, J. Rumbaugh, I. Jacobson, J. García Molina and J. Saez Martínez, El lenguaje unificado de modelado, 2nd ed. Madrid: Pearson Educación, 2010.
		
		\item[\hypertarget{b12}{[12]}] D. West, "Use Cases Considered Valuable (but Optional) For Lean/Agile Requirements Capture", InfoQ, 2010. [Online]. Available: https://www.infoq.com/news/2009/02/Use-Cases-Valuable-But-Optional. 
		
		\item[\hypertarget{b13}{[13]}] D. González, "Industria Mexicana del Software. Un estudio en cifras.", SG Buzz, 2005. [Online]. Available: https://sg.com.mx/revista/9/industria-mexicana-cifras. [Accessed: 20- Apr- 2019].
		
		\item[\hypertarget{b14}{[14]}] J. Gómez, "Métodos de Medición en Puntos Función (I): IFPUG FPA", El Laboratorio de las TI, 2014. [Online]. Available: https://www.laboratorioti.com/2013/01/16/metodos-de-medicion-en-puntos-funcion-i/. 
		
		\item[\hypertarget{b15}{[15]}] Ganesh Krishnamurthy, "CASE Tools 
		Adoption and Relevance" University of Missouri–St. Louis [Online]. Available: http://www.umsl.edu/~sauterv/analysis/F08papers/View.html
		
		\item[\hypertarget{b16}{[16]}] Annette L. du Plessis, A method for CASE tool evaluation, Information and Management, Volume 25, Issue 2, August 1993, Pages 93-102
		
		\item[\hypertarget{b17}{[17]}]	Erich Gamma, "Patrones de diseño: elementos de software orientado a objetos reutilizable", Pearson Educación, 2002. Addison-Wesley professional computing series.
		
		\item[\hypertarget{b18}{[18]}] IBM Knowledge Center, "Patrón de diseño de modelo-vista-controlador" [Online].\\ Available: https://www.ibm.com/support/knowledgecenter/es
		
		\item[\hypertarget{b19}{[19]}] Christopher Alexander, Sara Ishikawa, MurraySilverstein, Max Jacobson,
		Ingrid Fiksdahl-King, and Shlomo Angel.A Pattern Language. Oxford University
		Press, NewYork, 1977.
		
		\item[\hypertarget{b20}{[20]}] Glenn E. Krasner and Stephen T. Pope. A cookbook for using the model-view
		controller user interface paradigm in Smalltalk-80. Journal of Object-Oriented Programming, August/September 1988.
		
		\item[\hypertarget{b70}{[70]}] Rose-hulman.edu, 2014. [Online]. Available: https://www.rose-hulman.edu/class/csse/csse374-201020-02/SlidePDFs/session25.pdf. 
		
		\item[\hypertarget{b72}{[72]}] W. Sanders and C. Cumaranatunge, ActionScript 3.0 design patterns. Beijing: O'Reilly, 2007.
		
		\item[\hypertarget{b71}{[71]}] O. Blancarte Iturralde, Introducción a los patrones de diseño. 2017.
		
		\item[\hypertarget{b21}{[21]}] BBVA-TRANSFORMACIÓN DIGITAL, "Metodología 'scrum': ¿Qué es un 'sprint'?", [Online].\\ Available: https://www.bbva.com/es/metodologia-scrum-que-es-un-sprint/
		
		\item[\hypertarget{b22}{[22]}] Proyectos Agiles, "Lista de tareas de la iteración (Sprint Backlog)", [Online].\\ Available: https://proyectosagiles.org/lista-tareas-iteracion-sprint-backlog/
		
		\item[\hypertarget{b23}{[23]}] Clemente Ruiz Durán, Michael Piore
		Andrew Schrkarn, "Los retos para el desarrollo
		de la industria del software", [Online].\\ Available: http://revistas.bancomext.gob.mx/rce/magazines/87/1/Ruiz-Schrank.pdf

		\item[\hypertarget{b24}{[24]}] Universidad Católica de los Angeles Chimbote-PERÚ, "Metodología de Desarrollo de Software"  [Online].\\ Available: https://www.uladech.edu.pe/images/stories/universidad/documentos/2018/metodologia-desarrollo-software-v001.pdf
		
		\item[\hypertarget{b34}{[34]}] ISTQB (International Software Testing Qualifications Board), "Foundations of Software Testing" [Online].\\ Available: 
		https://www.istqb.org/downloads/send/51-ctfl2018/208-ctfl-2018-syllabus.html
		
		\item[\hypertarget{b35}{[35]}] ISTQB (International Software Testing Qualifications Board), "Certified Tester
		Foundation Level Syllabus", 2018 Version 
		
		\item[\hypertarget{b36}{[36]}] Campell / Papapetrou, Ann / Patroklos (2013). Sonar (SonarQube) en acción . Greenwich, Connecticut, EE. UU .: Manning Publications. pags. 350. ISBN 978-1617290954.
		
		\item[\hypertarget{b37}{[37]}] Buijze, Allard (26 de febrero de 2010). "Medición de la calidad del código con sonda" . Consultado el 29/08/2017 .
				
		\item[\hypertarget{b30}{[30]}] 	https://www.researchgate.net/profile/Praveen\_Gupta25/publication/49619227\_MVC\_Design\_Pattern\_for\_the\_multi\_framework\_distributed\_applications\_using\_XML\_spring\_and\_struts\_framework/links/5672564e08ae54b5e462aac5.pdf
		
		\item[\hypertarget{b31}{[31]}] (https://link.springer.com/chapter/10.1007/978-1-4302-2370-2\_20)
		
		\item[\hypertarget{b28}{[28]}] https://struts.apache.org/primer.html\#mvc
		
		\item[\hypertarget{b32}{[32]}] (https://docs.microsoft.com/es-es/azure/architecture/guide/architecture-styles/n-tier)
		\item[\hypertarget{b33}{[33]}](https://docs.spring.io/spring/docs/3.0.0.M3/reference/html/ch04s04.html\#beans-factory-scopes-singleton)
		
		\item[\hypertarget{b53}{[53]}]"Microsoft Word - Wordad", Sites.google.com, 2019. [Online]. Available: https://sites.google.com/site/wordadexel/microsoft-word. 
		
		\item[\hypertarget{b54}{[54]}]L. Lamport, LATEX. Boston [u.a.]: Addison-Wesley, 2003.
		
		\item[\hypertarget{b55}{[55]}] "Definición de procesador de texto — Definicion.de", Definición.de, 2019. [Online]. Available: https://definicion.de/procesador-de-texto/. 
		
		\item[\hypertarget{b56}{[56]}] "¿Qué es Java?", Java.com, 2014. [Online]. Available: https://www.java.com/es/about/whatis\_java.jsp. 
		
		\item[\hypertarget{b57}{[57]}] "Introducción al lenguaje C\# y .NET Framework", Docs.microsoft.com, 2011. [Online]. Available: https://docs.microsoft.com/es-es/dotnet/csharp/getting-started/introduction-to-the-csharp-language-and-the-net-framework.
		
		\item[\hypertarget{b58}{[58]}] "BeginnersGuide - Python Wiki", Wiki.python.org, 2016. [Online]. Available: https://wiki.python.org/moin/BeginnersGuide. 
		
		\item[\hypertarget{b59}{[59]}] "Home - DEPRECATED: Apache Struts 2 Documentation - Apache Software Foundation", Cwiki.apache.org, 2015. [Online]. Available: https://cwiki.apache.org/confluence/display/WW/Home. 
		
		\item[\hypertarget{b60}{[60]}] "Introducción a JavaServer Faces", Jtech.ua.es, 2010. [Online]. Available: http://www.jtech.ua.es/j2ee/publico/jsf-2012-13/sesion01-apuntes.html. 
		
		\item[\hypertarget{b61}{[61]}] R. Marín, "Los gestores de bases de datos (SGBD) más usados", Canal Informática y TICS, 2017. [Online]. Available: https://revistadigital.inesem.es/informatica-y-tics/los-gestores-de-bases-de-datos-mas-usados/.
		
		\item[\hypertarget{b63}{[63]}] C. Fontela, "Modularidad, cohesión y acoplamiento (Carlos Fontela)", CyS Ingeniería de Software, 2016. [Online]. Available: https://cysingsoft.wordpress.com/2009/06/23/modularidad-cohesion-y-acoplamiento-carlos-fontela/.
		
		\item[\hypertarget{b64}{[64]}] "TEMA 38: Conceptos en seguridad de los sistemas de información: Confidencialidad, integridad, disponibilidad y trazabilidad. - PDF", Docplayer.es, 2015. [Online]. Available: https://docplayer.es/15917327-Tema-38-conceptos-en-seguridad-de-los-sistemas-de-informacion-confidencialidad-integridad-disponibilidad-y-trazabilidad.html. 
		
		\item[\hypertarget{b65}{[65]}] R. Toro, "Confidencialidad, integridad y disponibilidad en los SG-SSI", PMG SSI - ISO 27001, 2017. [Online]. Available: https://www.pmg-ssi.com/2018/02/confidencialidad-integridad-y-disponibilidad/. 
		
		\item[\hypertarget{b66}{[66]}] "¿Qué es un requerimiento? | Informática | Wikiteka, apuntes, resúmenes, trabajos y exámenes de Secundaria, Bachiller, Universidad y Selectividad", Wikiteka.com, 2014. [Online]. Available: https://www.wikiteka.com/apuntes/que-es-un-requerimiento/. 
		
		\item[\hypertarget{b67}{[67]}] "¿Necesitas implantar la trazabilidad en tus productos?", AECOC, 2014. [Online]. Available: https://www.aecoc.es/servicios/implantacion/trazabilidad/. 
		
		\item[\hypertarget{b68}{[68]}] "Definición de estándar — Definicion.de", Definición.de, 2014. [Online]. Available: https://definicion.de/estandar/. 
		
		\item[\hypertarget{b69}{[69]}] "Consistencia Informática", Juan Estuardo Wyss, 2011. [Online]. Available: https://estuardowyss.wordpress.com/consistencia-informatica/. 
\end{description}