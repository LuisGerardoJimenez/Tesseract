\chapter{Arquitectura} \label{cap:seis}

La arquitectura implementada tiene como finalidad un sistema web estable, modular y escalable; para su realización se utilizaron las siguientes tecnologías:

\section{Frameworks}
\subsection {Spring}
Spring nos brinda ventajas como: la administración de objetos, garantizando que los recursos se creen, reutilicen y limpien correctamente (Managed Objects) \hyperlink{b25}{[25]}; El Manejo de transacciones declarativas, por ejemplo Rollback de una transacción \hyperlink{b26}{[26]} y un modelo de programación consistente en diferentes API de transacciones, tal como Java Persistence API.
La utilización de Spring junto con sus características nos brinda un sistema con alta disponibilidad.

\subsection {Struts}
Se implementó Struts2 porque es un framework de código abierto, extensible mediante plugins (REST, AJAX, etc) y utiliza el patrón de diseño MVC \hyperlink{b27}{[27]}.

\section{Arquitectura n capas}
La arquitectura n capas está implementada dentro del patrón MVC en el Modelo, ayuda a la modularidad de la capa de negocio, la capa de reglas de negocio y la capa de persistencia, de tal manera que se consigue un bajo acoplamiento.


\section{Patrones de Diseño}
\subsection{Interface}
El patrón interface se utilizó para denotar aquellas clases que pueden ser utilizadas por el DAO genérico y realizar las operaciones elementales a la base de datos


\subsection{Singleton}
El patrón de diseño singleton se implementa mediante anotaciones de Spring con la finalidad de \hyperlink{b28}{[28]}

\section{Logger log4j}



