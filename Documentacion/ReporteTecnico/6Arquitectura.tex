\chapter{Arquitectura} \label{cap:seis}

La arquitectura implementada tiene como finalidad que el sistema web que se desarrolle sea estable, modular y escalable; para su realización se utilizaron las siguientes tecnologías:

\section{Frameworks}
\subsection {Spring}
Spring nos brinda ventajas como: la administración de objetos, garantizando que los recursos se creen, reutilicen y limpien correctamente (Managed Objects) \hyperlink{b25}{[25]}; El Manejo de transacciones declarativas, por ejemplo Rollback de una transacción \hyperlink{b26}{[26]} y un modelo de programación consistente en diferentes API de transacciones, tal como Java Persistence API.
La utilización de Spring junto con sus características nos brinda un sistema con alta disponibilidad.

\subsection {Struts}
Se implementó Struts2 porque es un framework de código abierto, extensible mediante plugins (REST, AJAX, etc) y utiliza el patrón de diseño MVC. \hyperlink{b27}{[27]} \\ 
Las ventajas ofrecidas por este framework son: 

\begin{itemize}
	\item Patron Observador (Observer pattern), las vistas se actualizan por sí mismas del modelo. \hyperlink{b28}{[28]}
	\item Provee un bajo acoplamiento entre la vista y el modelo haciendo aplicaciones significativamente más fáciles de crear y mantener. \hyperlink{b28}{[28]}
	\item Uso de Interceptores, ejecutan código antes y después de que un Action es invocado, por ejemplo al subir un archivo .\hyperlink{b29}{[29]}
\end{itemize}

\section{Modelo MVC}
El patrón de diseño MVC (Modelo Vista Controlador) es fundamental para la separación entre
lógica de interfaz de usuario y lógica de negocio \hyperlink{b30}{[30]}; de esta manera la estructura del proyecto tiene las siguientes características \hyperlink{b31}{[31]}:

\begin{itemize}
	\item Los objetos del modelo de datos encapsulan información.
	\item Los objetos de vista muestran información al usuario.
	\item Los objetos del controlador implementan acciones.
	\item Los objetos de vista observan los objetos del modelo de datos y actualizan su \item visualización cada vez que cambian los objetos del modelo.
	\item Los objetos de vista recopilan la entrada del usuario y la pasan a un objeto controlador que realiza la acción. 
\end{itemize}


\section{Arquitectura N Capas}
La arquitectura n capas está implementada dentro del patrón MVC en el Modelo, ayuda a la modularidad de la capa de negocio, la capa de reglas de negocio y la capa de persistencia, de tal manera que se consigue un bajo acoplamiento.


\section{Patrones de Diseño}
\subsection{Interface}
El patrón interface se utilizó para denotar aquellas clases que pueden ser utilizadas por el DAO genérico y realizar las operaciones elementales a la base de datos


\subsection{Singleton}
El patrón de diseño singleton se implementó mediante anotaciones de Spring, cuando un bean es anotado con el alcance Singleton, solo se administrará una instancia compartida del bean, y todas las solicitudes de beans con un id o ids que coincidan con esa definición de bean harán que el contenedor Spring devuelva una instancia de bean específica \hyperlink{b28}{[28]}, de esta manera se convierte en un objeto administrado por Spring.\\

La ventaja de esta arquitectura es la capacidad de adaptación, si en algún momento se debe modificar alguna capa, por ejemplo la capa de persistencia, ésta modificación no afectaría a las otras capas ya que la comunicación entre ellas es transparente.

\section{Logger log4j}



