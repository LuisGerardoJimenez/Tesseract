\chapter{Trabajo a Futuro} \label{cap:once}
Durante la elaboración de este proyecto de trabajo terminal se encontrarón varias oportunidades de mejora en algunos de los módulos desarrollados, a continuación de describe en que áreas se pueden implementar mejoras en un trabajo a futuro.

\section{Funcionalidad}
\begin{itemize}
	\item A partir de la gestión de componentes de casos de uso generar las matrices correspondientes para la ejecución de pruebas dinámicas funcionales de caja negra:
	
	Con base en las siguientes aseveraciones:
	\begin{itemize}
	\item La herramienta diseñada para la ejecución de pruebas (Matriz de Pruebas) se basa en la  técnica de especificación de requerimientos.
	\item Para elaborar la matriz la base de prueba más importante fué la documentación de análisis basada en casos de uso.
	\item El sistema se encarga de gestionar todos los elementos para generar el documento de casos de uso.
	\end{itemize}

	La idea radica en usar esos elementos que el sistema ya gestiona para generar las matrices de prueba y coadyuvar al tester del sistema en la generación  de su herramienta de trabajo para que pueda realizar sus pruebas manuales.
	
	\item Integrar otros roles al sistema de tal manera que los desarrolladores y probadores puedan acceder a la plataforma y realizar sus tareas correspondientes.
	
	Este trabajo terminal nace con la idea de generar un instrumento de soporte para el equipo de análisis, sin embargo la plataforma podría escalarse a todos los integrantes del equipo de desarrollo del software. Es decir, a los desarrolladores para que consulten la información del caso de uso, hagan observaciones, cambien el estado del caso de uso y tenga una comunicación con el equipo de análisis. Para el probador podría corregir los documentos.
	
\end{itemize}
 