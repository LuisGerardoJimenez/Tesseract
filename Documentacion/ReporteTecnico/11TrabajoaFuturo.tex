\chapter{Trabajo a Futuro} \label{cap:once}
Durante la elaboración de este proyecto de trabajo terminal se encontrarón varias oportunidades de mejora en algunos de los módulos desarrollados, a continuación de describe en que áreas se pueden implementar mejoras en un trabajo a futuro.

\section{Funcionalidad}
\begin{itemize}
	
    \item \textbf {Ligar en el docuemento de análisis la especificación de un elemento con su referenciación en la especificación del caso de uso.}
    
    Navegar a través de los elementos que se están integrando en la redacción de un casos de uso es un recurso muy importante ya que le permitirá al desarrollador, cliente o cualquier interesado en el funcionamiento del sistema; dirigirse a la sección dentro del documento de análisis en donde se describen con mayor detalle cuales son las características y propiedades del elemento vinculado.
    
    \item \textbf {Implementar un control de versiones en las revisiones de los casos de uso.}
    
    Para fomentar el trabajo colaborativo a distancia (Trabajo remoto) se puede implementar la revisión de los casos de uso con un mecanismo de control de versiones. Si se considera que las revisiones se harán a través de la herramienta se debe trabajar en un módulo que le indique al líder el historial de las observaciones que el registró y posteriormente a través del versionador pueda verificar puntualmente que sus correcciones fueron atendidas por el analista que elaboró el caso de uso.
    
    \item \textbf {Implementar un módulo de diseño que permita personalizar el formato del documento de análisis.}
   
    El documento de análisis es un entregable de suma importancia que se le hace llegar al cliente para informarle del funcionamiento del sistema. El formato en el que se le debe presentar este documento puede variar dependiendo de las necesidades y condiciones que estipule el cliente. Algunas de estas condiciones pueden basarse en la inclusión de logos, tamaño de margenes, tipo de fuente, etc. Es por esto que se propone como trabajo a futuro darle la posibilidad al colaborador diseñar de manera personalizada la plantilla con la opción de modificar los parámetros propios del template y entregarlo con el formato que mejor se convenga.
    
    \item \textbf {Integrar otros roles al sistema de tal manera que los desarrolladores y probadores puedan acceder a la plataforma y realizar sus tareas correspondientes.}
	
	Este trabajo terminal nace con la idea de generar un instrumento de soporte para el equipo de análisis, sin embargo la plataforma podría escalarse a todos los integrantes del equipo de desarrollo del software. Es decir, a los desarrolladores para que consulten la información del caso de uso, hagan observaciones, cambien el estado del caso de uso y tenga una comunicación con el equipo de análisis. Para el probador podría corregir los documentos.
	
	\item \textbf {Incluir el modelado UML.}
	
  	La idea es generar un documento de análisis completo por lo que se plantea agregar secciones que contengan el modelado UML del proyecto sobre el cual se esté operando. 
\end{itemize}
 