\chapter{Pruebas Realizadas} \label{cap:ocho}
\section{Pruebas Dinámicas}
Static and dynamic testing are
both needed to cover the range of products we need to test. 

\subsection{Diseño de Pruebas}

\subsubsection{MATRICES DE PRUEBA}

Las matrices de pruebas son una herramienta que detalla el resultado de la ejecución de pruebas en comparación con cada una de las características y funcionalidades de los módulos del sistema. Están basadas en el documento de análisis, especifícamente en el documento de casos de uso.\\

La ejecución de pruebas se lleva a cabo una vez terminado el desarrollo de cada uno de los sprints estructurados en la metodología.\\

El objetivo de las matrices de prueba es validar y verificar a través de ciclos de pruebas que el producto cumpla con lo espeficicado en las etapas precedentes, en caso de que no se cumpla con las funcionalidades estipuladas o existan mejoras en su implementación se reporta una serie de defectos y sugerencias para que desarrollo se encargue de corregirlos; una vez que se solucionaron los problemas se ejecuta un ciclo de confirmación para comprobar que efectivamente se han corregido los defectos y que no se ha alterado ningúna otra parte del sistema.\\

A continuación se ilustra la estructura de las matrices de prueba que se diseñaron para registrar el resultado de su ejecución en los ciclos correspondientes.\\

\begin{figure}[H]
	\begin{center}
		\includegraphics[width=.95\textwidth]{images/pruebas/diseno/encabezado}
		\caption{Ejemplo del encabezado de la matriz de pruebas}
		\label{fig:encabezado}
	\end{center}
\end{figure}

\begin{figure}[H]
	\begin{center}
		\includegraphics[width=.95\textwidth]{images/pruebas/diseno/tabla}
		\caption{Ejemplo de la estructura de la matriz de pruebas}
		\label{fig:estructura}
	\end{center}
\end{figure}

\subsubsection{REPORTE DE PRUEBAS SPRINT 1 - CICLO 1}

Las pruebas contempladas para el primer ciclo de pruebas abarcan los Casos de Uso del Sprint 1, Los cuales comprenden las siguientes gestiones:

\begin{itemize}
	\item CU1 Iniciar sesión.
	\item CU2 Gestionar proyectos de Administrador.
	\item CU3 Gestionar Colaboradores.
	\item CU4 Gestionar Proyectos de Colaborador.
\end{itemize}

Se realizaron pruebas dinámicas de sistema, con técnicas de caja negra.\\

Base de prueba:
\begin{itemize}
	\item Casos de Uso
	\item Especificaciónes de requisitos del sistema y software.
	\item Sistema y manual de usuario.
\end{itemize}

Objeto de prueba:
\begin{itemize}
	\item Sistema de software
\end{itemize}

\newpage

Los resultados finales del primer ciclo de prueba arrojaron los siguientes datos:

\begin{figure}[H]
	\begin{center}
		\includegraphics[width=.95\textwidth]{images/pruebas/s1c1}
		\caption{Informe de defectos Sprint 1 Ciclo 1}
		\label{fig:infos1c1}
	\end{center}
\end{figure}

\newpage

\begin{figure}[H]
	\begin{center}
		\includegraphics[width=.85\textwidth]{images/pruebas/s1c1-1}
		\caption{Gráfica de defectos por severidad Sprint 1 Ciclo 1}
		\label{fig:infos1c1-1}
	\end{center}
\end{figure}

\begin{figure}[H]
	\begin{center}
		\includegraphics[width=.75\textwidth]{images/pruebas/s1c1-2}
		\caption{Gráfica de defectos por tipo de defecto Sprint 1 Ciclo 1}
		\label{fig:infos1c1-2}
	\end{center}
\end{figure}

\newpage

\begin{figure}[H]
	\begin{center}
		\includegraphics[width=.95\textwidth]{images/pruebas/s1c1-3}
		\caption{Gráfica de tipo de defectos por severidad Sprint 1 Ciclo 1}
		\label{fig:infos1c1-3}
	\end{center}
\end{figure}

Se corrigieron los defectos encontrados en el primer ciclo.
