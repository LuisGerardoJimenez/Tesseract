\chapter{Conclusiones} \label{cap:diez}

La etapa de mantenimiento de software representa un costo muy alto comparado con las fases precedentes en el ciclo de desarrollo de software (análisis, diseño e implementación), es por eso que se ha desarrollado una herramienta que coadyuvará a la etapa de análisis a disminuir los costos y tiempo futuros.

La documentación de los casos de uso suele ser escrita por más de un analista, lo que lleva a que,
la mayoría de veces, la redacción pueda ser distinta, lo cual complica el análisis y la creación del documento de casos de uno.

Con los sprints desarrollados durante este trabajo, se demostró que es posible gestionar y controlar los elementos de un caso de uso, dicha información debe estar organizada y centralizada dentro de un repositorio de datos, sin necesidad de que, los mismos analistas, tengan que registrar la información para cada uno de los atributos de las entidades a probar, a menos que estos tengan algún formato que la computadora no pueda entender. Con el fin de disminuir el tiempo y costo de esta etapa.

