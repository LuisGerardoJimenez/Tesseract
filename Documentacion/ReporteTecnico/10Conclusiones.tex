\chapter{Conclusiones} \label{cap:diez}

La construcción del proyecto de trabajo terminal representó un gran reto en donde se pusieron a prueba los conocimientos adquiridos en la carrera. Empezando por la unidad de Ingenieria de Software en donde aplicamos la teoría y actividades que involucrán todas las etapas de desarrollo de software, desde el análisis hasta las pruebas que fueron debidamente diseñadas para entregar un producto de alta calidad. Las unidades de aprendizaje de programación web fueron indispensables para el desarrollo de la plataforma, y las de bases de datos para centralizar la información de los elementos del caso de uso.

La etapa en donde se especifican los requisitos del sistema es de suma importancia ya que en esta se definen los elementos que compondrán la especificación de los casos de uso. Una vez definidos los actores, mensajes, reglas de negocio, pantallas, glosario, términos del glosario y entidades del sistema será más sencillo construir los casos de uso.\\

La estructura propuesta para la especificación de casos de uso muestra un panorama completo integrando todos los elementos que participan en un proceso o tarea en particular en el sistema. Para cualquier integrante del equipo o para culquier actor ajeno, le será más fácil leer la espeificación de un caso de uso y conocer con que elementos interactua en un punto determinado.\\

La etapa de mantenimiento de software representa un costo muy alto comparado con sus etapas precedentes en el ciclo de desarrollo de software, la solución a este problema radica en agilizar las actividades presentes en el proceso inicial. Es por eso que se ha desarrollado una plataforma que coadyuva en la etapa de análisis agilizando las tareas en la construcción de un documento de análisis de calidad basado en casos de uso.\\

La documentación de los casos de uso suele ser elaborada y redactada por más de un analista, esta situación provoca que el estilo de escritura pueda variar de un analista a otro, lo cual complica la creación del documento de casos de uso. Nuestro sistema proporciona a los analistas un estándar de escritura, redacción y composición de los casos de uso para obtener un documento final uniforme y fácil de comprender.\\

Durante los sprints desarrollados en este trabajo, se demostró que es posible gestionar y controlar todos y cada uno de los elementos de un caso de uso, para posteriormente construir el documento de análisis referenciando en el la información de cada elemento en un editor basado en tokens. Dicha información debe estar organizada y centralizada dentro de un repositorio de datos.\\

Gracias a los cambios realizados en arquitectura y en análisis fue posible reducir considerablemente el tiempo en el que se generaba el documento de análisis final.

La arquitectura utilizada en este sistema se puede dividir en dos, una Macro-arquitectura que es Cliente-Servidor y una arquitectura interna (en la capa de Modelo del patrón MVC) de tipo N capas; la utilización de esta arquitectura, nos ofrece muchas ventajas entre las cuales se pueden destacar las siguientes:\\

En la arquitectura Cliente-Servidor los recursos y la integridad de los datos son controladas por el servidor, lo que permite que personas sin autorización realicen modificaciones a la funcionalidad.\\

La arquitectura cliente servidor y N capas nos ofrecen escalabilidad en el sistema, permitiendo agregar nuevas funcionalidades sin afectar a las demás capas.\\

Ambas arquitecturas nos ofrecen mantenibilidad, de esta manera el tiempo de vida del sistema se puede extender ya que el mantenimiento es menos complicado si éste se encuentra dividido.\\

La arquitectura N capas nos ofrece un bajo acoplamiento, permitiendo cambiar de tecnologías en las capas sin afectar el funcionamiento de la aplicación, además mediante el patrón de diseño Facade solo se manda a llamar un método sin saber la complejidad que éste oculta.\\

En la arquitectura N capas el flujo de los datos es secuencial, las capas superiores conocen a las capas inferiores, pero las inferiores no conocen las superiores.\\

