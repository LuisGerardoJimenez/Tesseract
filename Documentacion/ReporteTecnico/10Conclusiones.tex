\chapter{Conclusiones} \label{cap:diez}

La etapa de mantenimiento de software representa un costo muy alto comparado con sus etapas precedentes en el ciclo de desarrollo de software, la solución a este problema radica en agilizar las actividades presentes en el proceso inicial. Es por eso que se ha desarrollado una plataforma que coadyuva en la etapa de análisis agilizando las tareas en la construcción de un documento de análisis de calidad basado en casos de uso.\\

La documentación de los casos de uso suele ser elaborada y redactada por más de un analista, esta situación provoca que el estilo de escritura pueda variar de un analista a otro, lo cual complica la creación del documento de casos de uso. Nuestro sistema proporciona a los analistas un estándar de escritura, redacción y composición de los casos de uso para obtener un documento final uniforme y fácil de comprender.\\

Durante los sprints desarrollados en este trabajo, se demostró que es posible gestionar y controlar todos y cada uno de los elementos de un caso de uso, para posteriormente construir el documento de análisis referenciando en el la información de cada elemento en un editor basado en tokens. Dicha información debe estar organizada y centralizada dentro de un repositorio de datos.

