
\chapter{Glosario}

\textbf {Acoplamiento:} 
El acoplamiento mide el grado de relacionamiento de un módulo con los demás. A menor acoplamiento, mejor: el módulo en cuestión será más sencillo de diseñar, programar, probar y mantener.
En el diseño estructurado, se logra bajo acoplamiento reduciendo las interacciones entre procedimientos y funciones, reduciendo la cantidad y complejidad de los parámetros y disminuyendo al mínimo los parámetros por referencia y los efectos colaterales. \hyperlink{b63}{[63]}\\

\textbf {Cohesión:}
La cohesión tiene que ver con que cada módulo del sistema se refiera a un único proceso o entidad. A mayor cohesión, mejor: el módulo en cuestión será más sencillo de diseñar, programar, probar y mantener.
En el diseño estructurado, se logra alta cohesión cuando cada módulo (función o procedimiento) realiza una única tarea trabajando sobre una sola estructura de datos. \hyperlink{b63}{[63]}\\

\textbf {Consistencia:}
Significa, asegurarse que la información está completa, que los datos se mantienen idénticos durante cualquier operación, como transferencia, almacenamiento y recuperación. Es la seguridad que la información es consistente y correcta en cualquier momento.

Está enfocada con la precisión, exactitud y validez de la información dentro de una Base de Datos.\hyperlink{b69}{[69]}\\

\textbf {Disponibilidad:} La disponibilidad supone que el sistema informático se mantenga trabajando sin sufrir ninguna degradación en cuanto a accesos. Es necesario que se ofrezcan los recursos que requieran los usuarios autorizados cuando se necesiten. La información deberá permanecer accesible a elementos autorizados.El objetivo es necesario prevenir interrupciones no autorizadas de los recursos informáticos. \hyperlink{b65}{[65]}\\

\textbf {Estándar:} El concepto se utiliza para nombrar a aquello que puede tomarse como referencia, patrón o modelo. Se conoce como estandarización o normalización al proceso que apunta a la creación y la aplicación de normas que son utilizadas a nivel general en un determinado ámbito. \hyperlink{b68}{[68]}\\

\textbf {Integridad:} 
La finalidad de la integridad es garantizar que los datos mantienen correctos y completos durante su almacenamiento, procesamiento y transmisión.Las   integridad   de   la   información   que   almacena   un   sistema   informático   puede   versecomprometida por las mismas amenazas que afectan a la disponibilidad del mismo: averías en elhardware, virus, accidentes de tipo catastrófico: incendios, inundaciones... Estas amenazas se pueden clasificar en dos tipos:\\
1.Amenazas que pueden provocar la perdida de datos. Por ejemplo: Un virus que eliminaunos ficheros.
2.La modificación de los datos que haga que pierdan su correctitud y, por tanto, su valor.Ejemplo: La introducción en un registro de una base de datos,  un campo que identifiqueun producto inexistente. \hyperlink{b64}{[64]}\\

\textbf {Modularidad:} Para resolver un problema complejo de desarrollo de software, conviene separarlo en partes más pequeñas, que se puedan diseñar, desarrollar, probar y modificar, de manera sencilla y lo más independientemente posible del resto de la aplicación.
Esas partes, cuando se quiere usar un nombre genérico, habitualmente se denominan módulos. De allí que otro nombre para la programación estructurada, luego caído en desuso, fue “programación modular”. \hyperlink{b63}{[63]}\\

\textbf {Requerimiento:} Un requerimiento puede definirse como un atributo necesario dentro de un sistema, que puede representar una capacidad, una característica o un factor de calidad del sistema de tal manera que le sea útil a los clientes o a los usuarios finales.\\

Un requerimiento es una descripción de una condición o capacidad que debe cumplir un sistema, ya sea derivada de una necesidad de usuario identificada, o bien, estipulada en un contrato, estándar, especificación u otro documento formalmente impuesto al inicio del proceso. \hyperlink{b66}{[66]}\\

\textbf {Trazabilidad:} Según el Comité de Seguridad Alimentaria de AECOC:\\

“Se entiende trazabilidad como el conjunto de aquellos procedimientos preestablecidos y autosuficientes que permiten conocer el histórico, la ubicación y la trayectoria de un producto o lote de productos a lo largo de la cadena de suministros en un momento dado, a través de unas herramientas determinadas.” \hyperlink{b67}{[67]}\\


