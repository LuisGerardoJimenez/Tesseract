%--------------------------------------
\subsection{IU 6.4 Revisar caso de uso}

\subsubsection{Objetivo}
	Esta pantalla permite al actor revisar la información de un caso de uso y determinar si es correcto o no.
\subsubsection{Diseño}
	En la figura \IUref{IU6.4}{Revisar caso de uso} se muestra la pantalla ''Revisar Caso de uso'' en la cual el actor podrá revisar el caso de uso y decidir si es correcto o no. Cada uno de los \hyperlink{tElemento}{elementos} utilizados contará con un enlace a su respectiva consulta. En caso de que alguna de las secciones no cuente con información se mostrará la leyenda ''Sin información''.
	
	Una vez que el actor ha consultado la información mostrada, deberá seleccionar para cada sección si esta es correcta o no. Para las secciones marcadas como incorrectas el sistema solicitará que se ingresen las correspondientes observaciones. Finalmente el actor deberá oprimir el botón \IUbutton{Aceptar} , el sistema realizará las validaciones correspondientes y determinará el nuevo estado del caso de uso.

\IUfig[.6]{interfaces/IU6-4revisarCU}{IU6.4}{Revisar Caso de uso}
\subsubsection{Comandos}
\begin{itemize}
	\item \IUbutton{Aceptar}: Permite al actor concluir la revisión, dirige a la pantalla \IUref{IU6}{Gestionar casos de uso}
\end{itemize}

\subsubsection{Mensajes}

\begin{Citemize}
	\item \cdtIdRef{MSG1}{Operación exitosa}: Se muestra en la pantalla \IUref{IU6}{Gestionar Casos de uso} para indicar que la revisión se ha realizado exitosamente.
	\item \cdtIdRef{MSG4}{Dato obligatorio}: Se muestra en la pantalla \IUref{IU6.4}{Revisar Caso de uso} cuando no se ha ingresado un dato marcado como obligatorio..
	\item \cdtIdRef{MSG6}{Longitud inválida}: Se muestra en la pantalla \IUref{IU6.4}{Revisar Caso de uso} cuando se ha excedido la longitud de alguno de los campos.
	\item \cdtIdRef{MSG12}{Ha ocurrido un error}: Se muestra en la pantalla donde se solicitó la operación cuando el caso de uso que se desea revisar no se encuentra en estado ''Revisión''.
\end{Citemize}
