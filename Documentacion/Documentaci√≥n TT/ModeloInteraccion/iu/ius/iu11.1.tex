%--------------------------------------
\subsection{IU 11.1 Registrar Término}

\subsubsection{Objetivo}
	Esta pantalla permite al actor registrar la información de un término nuevo.
\subsubsection{Diseño}
	En la figura \IUref{IU11.1}{Registrar Término} se muestra la pantalla ''Registrar Término'' que permite registrar un término.
	Una vez ingresada la información solicitada, el actor deberá oprimir el botón \IUbutton{Aceptar} . El sistema validará y registrará la información solo si se han cumplido todas las reglas de negocio establecidas.
	
	Finalmente se mostrará el mensaje \cdtIdRef{MSG1}{Operación Exitosa} en la pantalla \IUref{IU11}{Gestionar Términos}, para indicar que la información del término se ha registrado correctamente.
	En la parte superior derecha, el sistema muestra el proyecto en el que se encuentra trabajando.

\IUfig[1]{interfaces/IU11-1registrarTermino}{IU11.1}{Registrar Término}
\subsubsection{Comandos}
\begin{itemize}
	\item \IUbutton{Aceptar}: Permite al actor guardar el registro de un término, dirige a la pantalla \IUref{IU11}{Gestionar Términos}.
	\item \IUbutton{Cancelar}: Permite al actor cancelar el registro de un término, dirige a la pantalla \IUref{IU11}{Gestionar Términos}
\end{itemize}

\subsubsection{Mensajes}

\begin{Citemize}
	\item \cdtIdRef{MSG1}{Operación exitosa}: Se muestra en la pantalla \IUref{IU11}{Gestionar Términos de Glosario} para indicar que el registro fue exitoso.
	\item \cdtIdRef{MSG4}{Dato obligatorio}: Se muestra en la pantalla \IUref{IU11.1}{Registrar Término} cuando no se ha ingresado un dato marcado como obligatorio.
	\item \cdtIdRef{MSG5}{Dato incorrecto}: Se muestra en la pantalla \IUref{IU11.1}{Registrar Término} cuando el tipo de dato ingresado no cumple con el tipo de dato solicitado en el campo.
	\item \cdtIdRef{MSG6}{Longitud inválida}: Se muestra en la pantalla \IUref{IU11.1}{Registrar Término} cuando se ha excedido la longitud de alguno de los campos.
	\item \cdtIdRef{MSG7}{Registro repetido}: Se muestra en la pantalla \IUref{IU11.1}{Registrar Término} cuando se registre un término con un nombre que ya se encuentre registrado.
\end{Citemize}
