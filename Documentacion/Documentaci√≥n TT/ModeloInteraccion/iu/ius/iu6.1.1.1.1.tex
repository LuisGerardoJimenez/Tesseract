%--------------------------------------
\subsection{IU 6.1.1.1.1 Registrar Paso}

\subsubsection{Objetivo}
	Esta pantalla permite al actor registrar la información de un paso.
\subsubsection{Diseño}
	En la figura \IUref{IU6.1.1.1.1}{Registrar Paso} se muestra la pantalla ''Registrar Paso'' que permite registrar un paso. El actor deberá ingresar la información solicitada, esto incluye indicar quién realiza la acción, el verbo y la redacción del paso.
	Cuando el actor requiera un verbo que no está en la lista podrá seleccionar la opción ''Otro'' y el sistema mostrará un campo para que especifique el verbo que requiera como se muestra en la figura \IUref{IU6.1.1.1.1b}{Registrar Paso: Otro verbo}
	Una vez ingresada la información solicitada, el actor deberá oprimir el botón \IUbutton{Aceptar} . El sistema validará y agregará el paso a la trayectoria solo si se han cumplido todas las reglas de negocio establecidas.

\IUfig[1]{interfaces/IU6-1-1-1-1registrarPaso}{IU6.1.1.1.1}{Registrar Paso}
\IUfig[1]{interfaces/IU6-1-1-1-1bregistrarPaso}{IU6.1.1.1.1b}{Registrar Paso: Otro verbo}
\subsubsection{Comandos}
\begin{itemize}
	\item \IUbutton{Aceptar}: Permite al actor guardar el registro del paso, dirige a la pantalla \IUref{IU6.1.1.1.1.1}{Gestionar Pasos}.
	\item \IUbutton{Cancelar}: Permite al actor cancelar el registro del paso, dirige a la pantalla \IUref{IU6.1.1.1.1.1}{Gestionar Pasos}.
\end{itemize}

\subsubsection{Mensajes}

\begin{Citemize}
	\item \cdtIdRef{MSG4}{Dato obligatorio}: Se muestra en la pantalla \IUref{IU6.1.1.1.1}{Registrar Paso} cuando no se ha ingresado un dato marcado como obligatorio.
	\item \cdtIdRef{MSG6}{Longitud inválida}: Se muestra en la pantalla \IUref{IU6.1.1.1.1}{Registrar Paso} cuando se ha excedido la longitud de alguno de los campos.
\end{Citemize}
