%--------------------------------------
\subsection{IU 6.1.2.1 Registrar Pre/Postcondición}

\subsubsection{Objetivo}
	Esta pantalla permite al actor registrar ya sea una precondición o postcondición.
\subsubsection{Diseño}
	En la figura \IUref{IU6.1.2.1}{Registrar Pre/Postcondición} se muestra la pantalla ''Registrar Pre/Postcondición'' que permite registrar una condición. Una vez ingresada la redacción solicitada en la pantalla, el actor deberá oprimir el botón \IUbutton{Aceptar} . El sistema validará y agregará la condición a la tabla de ''Pre/Postcondiciones'' solo si se han cumplido todas las reglas de negocio establecidas.

\IUfig[1]{interfaces/IU6-1-2-1registrarCondicion}{IU6.1.2.1}{Registrar Pre/Postcondición}
\subsubsection{Comandos}
\begin{itemize}
	\item \IUbutton{Aceptar}: Permite al actor guardar el registro de la condición, dirige a la pantalla \IUref{IU6.1.2}{Gestionar Pre/Postcondiciones}.
	\item \IUbutton{Cancelar}: Permite al actor cancelar el registro de la condición, dirige a la pantalla \IUref{IU6.1.2}{Gestionar Pre/Postcondiciones}.
\end{itemize}

\subsubsection{Mensajes}

\begin{Citemize}
	\item \cdtIdRef{MSG4}{Dato obligatorio}: Se muestra en la pantalla \IUref{IU6.1.2.1}{Registrar Pre/Postcondición} cuando no se ha ingresado un dato marcado como obligatorio.
	\item \cdtIdRef{MSG6}{Longitud inválida}: Se muestra en la pantalla \IUref{IU6.1.2.1}{Registrar Pre/Postcondición} cuando se ha excedido la longitud de alguno de los campos.
	\item \cdtIdRef{MSG5}{Formato incorrecto}: Se muestra en la pantalla \IUref{IU6.1.2.1}{Registrar Pre/Postcondición} cuando el tipo de dato ingresado no cumple con el tipo de dato solicitado en el campo.
\end{Citemize}
