%--------------------------------------
\subsection{IU 6.1.2.2 Modificar Pre/Postcondición}

\subsubsection{Objetivo}
	Esta pantalla permite al actor editar la información de una condición.
\subsubsection{Diseño}
	En la figura \IUref{IU6.1.3}{Registrar Postcondición} se muestra la pantalla ''Registrar Postcondición'' que permite registrar una postcondición. Una vez ingresada la redacción solicitada en la pantalla, el actor deberá oprimir el botón \IUbutton{Aceptar} . El sistema validará y agregará la postcondición a la tabla de ''Postcondiciones'' solo si se han cumplido todas las reglas de negocio establecidas.

\IUfig[1]{interfaces/IU6-1-3registrarPostcondicion}{IU6.1.3}{Registrar Postcondición}
\subsubsection{Comandos}
\begin{itemize}
	\item \IUbutton{Aceptar}: Permite al actor guardar el registro de la postcondición, dirige a la pantalla \IUref{IU6.1}{Registrar Caso de uso} o a la pantalla \IUref{IU6.2}{Modificar Caso de uso}, según corresponda.
	\item \IUbutton{Cancelar}: Permite al actor cancelar el registro de la postcondición, dirige a la pantalla \IUref{IU6.1}{Registrar Caso de uso} o a la pantalla \IUref{IU6.2}{Modificar Caso de uso}, según corresponda
\end{itemize}

\subsubsection{Mensajes}

\begin{Citemize}
	\item \cdtIdRef{MSG4}{Dato obligatorio}: Se muestra en la pantalla \IUref{IU6.1.3}{Registrar postcondición} cuando no se ha ingresado un dato marcado como obligatorio.
	\item \cdtIdRef{MSG6}{Longitud inválida}: Se muestra en la pantalla \IUref{IU6.1.3}{Registrar postcondición} cuando se ha excedido la longitud de alguno de los campos.
\end{Citemize}
