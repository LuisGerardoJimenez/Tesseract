\section{Modelado de Reglas de negocio}


\begin{BussinesRule}{RN1}{Unicidad de número de elementos}
	\BRitem[Tipo:] Restricción de operación. 
	\BRitem[Clase:] Habilitadora. 
	\BRitem[Nivel:] Controla la operación. % Otras opciones para nivel: Control, Influencia.
	\BRitem[Descripción:] Un número de \hyperlink{tElemento}{elemento} no se puede duplicar en el ámbito donde es utilizado ni registrarse en más de una ocasión.
	\BRitem[Ejemplo positivo] Cumplen la regla:
	\begin{itemize}
		\item Dos \hyperlink{casoUso}{casos de uso} con diferentes números.
		\item Dos \hyperlink{EntidadPantalla}{pantallas} con diferentes números.
	\end{itemize}
	\BRitem[Ejemplo negativo] No cumplen con la regla:
	\begin{itemize}
		\item Dos \hyperlink{casoUso}{Casos de uso} con el mismo número.
		\item Dos \hyperlink{EntidadPantalla}{pantallas} con el mismo número.
	\end{itemize}
	\BRitem[Referenciado por:] \UCref{CU11.1}{Registrar Pantalla}, \UCref{CU11.2}{Modificar Pantalla}, \UCref{CU12.1}{Registrar Caso de uso}, \UCref{CU12.2}{Modificar Caso de uso}.
\end{BussinesRule}

%\begin{BussinesRule}{RN5}{Modificación de elementos}
%	\BRitem[Tipo:] Restricción de operación. 
%	\BRitem[Clase:] Habilitadora. 
%	\BRitem[Nivel:] Controla la operación. % Otras opciones para nivel: Control, Influencia.
%	\BRitem[Descripción:] Un \hyperlink{tElemento}{elemento} no puede ser modificado si tiene asociaciones con casos de uso en estado ''Liberado''.
%	\BRitem[Ejemplo positivo] Cumplen la regla:
%	\begin{itemize}
%		\item Modificar un \hyperlink{MSGEntidad}{mensaje} que no tiene asociados casos de uso en estado ''Liberado''.
%		\item Modificar una \hyperlink{EntidadPantalla}{pantalla} que no tiene asociados casos de uso en estado ''Liberado''.
%		\item Modificar una \hyperlink{BREntidad}{regla de negocio} que no tiene asociados casos de uso en estado ''Liberado''.
%	\end{itemize}
%	\BRitem[Ejemplo negativo] No cumplen con la regla:
%	\begin{itemize}
%		\item Modificar un \hyperlink{atributoEntidad}{atributo} perteneciente a una \hyperlink{entidadEntidad}{entidad} que se encuentra asociada a un caso de uso en estado ''Liberado''.
%		\item Modificar una \hyperlink{BREntidad}{regla de negocio} asociada con tres casos de uso en estado ''Liberado''.
%		\item Modificar un \hyperlink{actorEntidad}{actor} asociado con un caso de uso en estado ''Liberado''.
%	\end{itemize}
%	\BRitem[Referenciado por:] \UCref{CU6.2}{Modificar Término de Glosario}, \UCref{CU7.2}{Modificar Entidad}, \UCref{7.1.2}{Modificar Atributo}, \UCref{CU8.2}{Modificar Regla de Negocio}, \UCref{CU9.2}{Modificar Mensaje}, \UCref{CU10.2}{Modificar Actor}, \UCref{CU11.2}{Modificar Pantalla}, \UCref{CU11.1.2}{Modificar Acción}.
%\end{BussinesRule}



\begin{BussinesRule}{RN6}{Unicidad de nombres}
	\BRitem[Tipo:] Restricción de operación. 
	\BRitem[Clase:] Habilitadora. 
	\BRitem[Nivel:] Controla la operación. % Otras opciones para nivel: Control, Influencia.
	\BRitem[Descripción:] Un nombre de \hyperlink{tElemento}{elemento} no se puede duplicar en el ámbito donde es utilizado ni registrarse en más de una ocasión.
	\BRitem[Ejemplo positivo] Cumplen la regla:
	\begin{itemize}
		\item Dos \hyperlink{BREntidad}{reglas de negocio} con diferentes nombres.
		\item Dos \hyperlink{EntidadPantalla}{pantallas} con diferentes nombres.
		\item Dos \hyperlink{actorEntidad}{actores} con diferentes nombres.
	\end{itemize}
	\BRitem[Ejemplo negativo] No cumplen con la regla:
	\begin{itemize}
		\item Dos \hyperlink{BREntidad}{reglas de negocio} con el mismo nombre.
		\item Dos \hyperlink{EntidadPantalla}{pantallas} con el mismo nombre.
		\item Dos \hyperlink{actorEntidad}{actores} con el mismo nombre.
	\end{itemize}
	\BRitem[Referenciado por:] \UCref{CU2.1}{Registrar Proyecto de Administrador}, \UCref{CU2.2}{Modificar Proyecto de Administrador}, \UCref{CU5.1}{Registrar Módulo}, \UCref{CU5.2}{Modificar Módulo}, \UCref{CU6.1}{Registrar Término de Glosario}, \UCref{CU6.2}{Modificar Término de Glosario}, \UCref{CU7.1}{Registrar Entidad}, \UCref{CU7.2}{Modificar Entidad}, \UCref{CU7.1.1}{Registrar Atributo}, \UCref{CU7.1.2}{Modificar Atributo}, \UCref{CU8.1}{Registrar Regla de Negocio}, \UCref{CU8.2}{Modificar Regla de Negocio}, \UCref{CU9.1}{Registrar Mensaje}, \UCref{CU9.2}{Modificar Mensaje}, \UCref{CU10.1}{Registrar Actor}, \UCref{CU10.2}{Modificar Actor}, \UCref{CU11.1}{Registrar Pantalla}, \UCref{CU11.2}{Modificar Pantalla}, \UCref{CU11.1.1}{Registrar Acción}, \UCref{CU11.1.2}{Modificar Acción}, \UCref{CU11.1.2}{Modificar Acción}, \UCref{CU12.1}{Registrar Caso de uso}, \UCref{CU12.2}{Modificar Caso de uso}.
\end{BussinesRule}

\begin{BussinesRule}{RN7}{Información correcta}
	\BRitem[Tipo:] Restricción de operación. 
	\BRitem[Clase:] Habilitadora. 
	\BRitem[Nivel:] Controla la operación. % Otras opciones para nivel: Control, Influencia.
	\BRitem[Descripción:] Todos los datos proporcionados al sistema deben respetar el formato establecido en los modelos conceptuales.
	\BRitem[Ejemplo positivo] Cumplen la regla:
	\begin{itemize}
		\item El actor introduce el \cdtRef{proyectoEntidad:nombreProyecto}{nombre} de un \hyperlink{proyectoEntidad}{proyecto} que solamente contiene caracteres alfanuméricos.
		\item El actor introduce un \cdtRef{MSGEntidad:numeroMSG}{número} de \hyperlink{MSGEntidad}{mensaje} que contiene sólo caracteres numéricos.
		\item El actor introduce un \cdtRef{colaboradorEntidad:correoColaborador}{correo electrónico} que contiene un símbolo '@' y cuya terminación es un dominio de correo electrónico.
	\end{itemize}
	\BRitem[Ejemplo negativo] No cumplen con la regla:
	\begin{itemize}
		\item El actor introduce un nombre de un \hyperlink{terminoGLSEntidad}{término} que contiene el símbolo '@'.
		\item El actor introduce un número de \hyperlink{BREntidad}{regla de negocio} que contiene caracteres alfabéticos.
		\item El actor introduce un \cdtRef{colaboradorEntidad:correoColaborador}{correo electrónico} que no contiene el carácter '@'.
	\end{itemize}
	\BRitem[Referenciado por:] \UCref{CU1}{Iniciar Sesión}, \UCref{CU2.1}{Registrar Proyecto de Administrador}, \UCref{CU2.2}{Modificar Proyecto de Administrador}, \UCref{CU3.1}{Registrar Colaborador}, \UCref{CU3.2}{Modificar Colaborador}, \UCref{CU5.1}{Registrar Módulo}, \UCref{CU5.2}{Modificar Módulo}, \UCref{CU6.1}{Registrar Término de Glosario}, \UCref{CU6.2}{Modificar Término de Glosario}, \UCref{CU7.1}{Registrar Entidad}, \UCref{CU7.2}{Modificar Entidad}, \UCref{CU7.1.1}{Registrar Atributo}, \UCref{CU7.1.2}{Modificar Atributo}, \UCref{CU8.1}{Registrar Regla de Negocio}, \UCref{CU8.2}{Modificar Regla de Negocio}, \UCref{CU9.1}{Registrar Mensaje}, \UCref{CU9.2}{Modificar Mensaje}, \UCref{CU10.1}{Registrar Actor}, \UCref{CU10.2}{Modificar Actor}, \UCref{CU11.1}{Registrar Pantalla}, \UCref{CU11.2}{Modificar Pantalla}, \UCref{CU11.1.1}{Registrar Acción}, \UCref{CU11.1.2}{Modificar Acción}, \UCref{CU11.1.2}{Modificar Acción}, \UCref{CU12.1}{Registrar Caso de uso}, \UCref{CU12.2}{Modificar Caso de uso}, \UCref{CU12.1.1.1}{Registrar Trayectoria}, \UCref{CU12.1.1.2}{Modificar Trayectoria}, \UCref{CU12.1.1.1.1.1}{Registrar Paso}, \UCref{CU12.1.1.1.1.2}{Modificar Paso}, \UCref{CU12.1.2.1}{Registrar Precondición/Postcondicón}, \UCref{CU12.1.2.2}{Modificar Precondición/Postcondicón}, \UCref{CU12.1.3.1}{Registrar Punto de extensión}, \UCref{CU12.1.3.2}{Modificar Punto de extensión}, \UCref{CU12.5}{Revisar Caso de uso}.
\end{BussinesRule}

\begin{BussinesRule}{RN8}{Datos Obligatorios} 
	\BRitem[Tipo:] Restricción de operación. 
	\BRitem[Clase:] Habilitadora. 
	\BRitem[Nivel:] Control. % Otras opciones para nivel: Control, Influencia.
	\BRitem[Descripción:] Los campos proporcionados al sistema marcados como obligatorios no se deben omitir.
	\BRitem[Ejemplo positivo] Cumplen la regla:
	\begin{itemize}
		\item Para la entidad \hyperlink{colaboradorEntidad}{colaborador}, el actor introduce todos los atributos solicitados.
		\item Para la entidad \hyperlink{proyectoEntidad}{proyecto}, el actor introduce todos los datos con excepción de \cdtRef{proyectoEntidad:fechaIPProyecto}{fecha de inicio programada} y \cdtRef{proyectoEntidad:fechaFinProyecto}{fecha de termino programada}
		\item Para la entidad \hyperlink{MSGEntidad}{mensaje}, el actor introduce todos los atributos solicitados.
	\end{itemize}
	\BRitem[Ejemplo negativo] No cumplen con la regla:
	\begin{itemize}
		\item El actor no proporciona el \cdtRef{MSGEntidad:nombreMSG}{nombre} para la entidad \hyperlink{MSGEntidad}{mensaje}.
		\item El actor no proporciona la  \cdtRef{colaboradorEntidad:curpColaborador}{CURP} para la entidad \hyperlink{colaboradorEntidad}{colaborador}.
		\item El actor no proporciona la \cdtRef{proyectoEntidad:contraparteProyecto}{contraparte} para la entidad \hyperlink{proyectoEntidad}{proyecto}.
	\end{itemize}
	\BRitem[Referenciado por:] \UCref{CU1}{Iniciar Sesión}, \UCref{CU2.1}{Registrar Proyecto de Administrador}, \UCref{CU2.2}{Modificar Proyecto de Administrador}, \UCref{CU3.1}{Registrar Colaborador}, \UCref{CU3.2}{Modificar Colaborador}, \UCref{CU5.1}{Registrar Módulo}, \UCref{CU5.2}{Modificar Módulo}, \UCref{CU6.1}{Registrar Término de Glosario}, \UCref{CU6.2}{Modificar Término de Glosario}, \UCref{CU7.1}{Registrar Entidad}, \UCref{CU7.2}{Modificar Entidad}, \UCref{CU7.1.1}{Registrar Atributo}, \UCref{CU7.1.2}{Modificar Atributo}, \UCref{CU8.1}{Registrar Regla de Negocio}, \UCref{CU8.2}{Modificar Regla de Negocio}, \UCref{CU9.1}{Registrar Mensaje}, \UCref{CU9.2}{Modificar Mensaje}, \UCref{CU10.1}{Registrar Actor}, \UCref{CU10.2}{Modificar Actor}, \UCref{CU11.1}{Registrar Pantalla}, \UCref{CU11.2}{Modificar Pantalla}, \UCref{CU11.1.1}{Registrar Acción}, \UCref{CU11.1.2}{Modificar Acción}, \UCref{CU11.1.2}{Modificar Acción}, \UCref{CU12.1}{Registrar Caso de uso}, \UCref{CU12.2}{Modificar Caso de uso}, \UCref{CU12.1.1.1}{Registrar Trayectoria}, \UCref{CU12.1.1.2}{Modificar Trayectoria}, \UCref{CU12.1.1.1.1.1}{Registrar Paso}, \UCref{CU12.1.1.1.1.2}{Modificar Paso}, \UCref{CU12.1.2.1}{Registrar Precondición/Postcondicón}, \UCref{CU12.1.2.2}{Modificar Precondición/Postcondicón}, \UCref{CU12.1.3.1}{Registrar Punto de extensión}, \UCref{CU12.1.3.2}{Modificar Punto de extensión}, \UCref{CU12.5}{Revisar Caso de uso}. 
\end{BussinesRule}



\begin{BussinesRule}{RN9}{Operaciones disponibles de casos de uso} 
	\BRitem[Tipo:] Restricción de operación. 
	\BRitem[Clase:] Habilitadora. 
	\BRitem[Nivel:] Control. % Otras opciones para nivel: Control, Influencia.
	\BRitem[Descripción:] Las operaciones que pueden solicitarse sobre un caso de uso dependerán del estado en el que se encuentre y el rol del actor:
	
	\begin{table}[H]
		\centering
		\begin{tabular}{|p{5cm}| p{5cm}| p{5cm}|}
			\hline
			\rowcolor{blue} \textcolor{white}{\textbf{Estado}} & \textcolor{white}{\textbf{Operaciones Analista}} & \textcolor{white}{\textbf{Operaciones Líder de Análisis}} \\
			\hline
			Edición & Consultar, editar, gestionar trayectorias, gestionar puntos de extensión, terminar y eliminar & Consultar, editar, gestionar trayectorias, gestionar puntos de extensión, terminar y eliminar \\
			\hline
			Revisión & Consultar y revisar & Consultar y revisar\\
			\hline
			Por liberar & Consultar & Consultar y Liberar\\
			\hline
			Pendiente de correción & Consultar, editar, gestionar trayectorias, gestionar puntos de extensión, terminar y eliminar & Consultar, editar, gestionar trayectorias, gestionar puntos de extensión, terminar y eliminar\\
			\hline
			Liberado & Consultar & Consultar, solicitar correciones\\
			\hline
		\end{tabular}
	\end{table}

	\BRitem[Referenciado por:] \UCref{CU12}{Gestionar Casos de uso}.
	
\end{BussinesRule}

	\begin{BussinesRule}{RN11}{Registro de trayectorias} 
		\BRitem[Tipo:] Restricción de operación. 
		\BRitem[Clase:] Habilitadora. 
		\BRitem[Nivel:] Control. % Otras opciones para nivel: Control, Influencia.
		\BRitem[Descripción:] Al menos una de las trayectorias registradas debe ser marcada como principal.
		\BRitem[Referenciado por:] \UCref{}{}. 
	\end{BussinesRule}

\begin{BussinesRule}{RN12}{Identificador de elemento} 
	\BRitem[Tipo:] Restricción de operación. 
	\BRitem[Clase:] Habilitadora. 
	\BRitem[Nivel:] Control. % Otras opciones para nivel: Control, Influencia.
	\BRitem[Descripción:] El identificador de cada elemento se compone de una clave. Donde el nombre es el que le asigna el usuario, el número es secuencial y la clave define el tipo de elemento: ''ENT'' para las entidades, ''ACT'' para los actores, ''CU'' para los casos de uso, ''IU'' para las pantallas, ''MSG'' para los mensajes, ''RN'' para las reglas de negocio y ''GLS'' para los términos del glosario.
	\BRitem[Referenciado por:] \UCref{CU8.1}{Registrar Regla de Negocio}, \UCref{CU8.2}{Modificar Regla de Negocio}, \UCref{CU9.1}{Registrar Mensaje}, \UCref{CU9.2}{Modificar Mensaje}, \UCref{CU11.1}{Registrar Pantalla}, \UCref{CU11.2}{Modificar Pantalla}, \UCref{CU12.1}{Registrar Caso de uso}, \UCref{CU12.2}{Modificar Caso de uso}. 
\end{BussinesRule}

\begin{BussinesRule}{RN13}{Modificación del identificador} 
	\BRitem[Tipo:] Restricción de operación. 
	\BRitem[Clase:] Habilitadora. 
	\BRitem[Nivel:] Control. % Otras opciones para nivel: Control, Influencia.
	\BRitem[Descripción:] Una vez registrado un elemento no se podrá modificar el nombre, el número o la clave del identificador.
	\BRitem[Referenciado por:] \UCref{CU8.2}{Modificar Regla de Negocio}, \UCref{CU9.2}{Modificar Mensaje}, \UCref{CU11.2}{Modificar Pantalla}, \UCref{CU12.2}{Modificar Caso de uso}. 
\end{BussinesRule}

\begin{BussinesRule}{RN14}{Salidas del casos de uso} 
	\BRitem[Tipo:] Restricción de operación. 
	\BRitem[Clase:] Habilitadora. 
	\BRitem[Nivel:] Control. % Otras opciones para nivel: Control, Influencia.
	\BRitem[Descripción:] En las salidas del caso de uso podrán enlistarse mensajes y atributos de las entidades.
	\BRitem[Referenciado por:] \UCref{}{}. 
\end{BussinesRule}

\begin{BussinesRule}{RN15}{Operaciones disponibles} 
	\BRitem[Tipo:] Restricción de operación. 
	\BRitem[Clase:] Habilitadora. 
	\BRitem[Nivel:] Control. % Otras opciones para nivel: Control, Influencia.
	\BRitem[Descripción:] Cuando una entidad, regla de negocio, actor, término del glosario, pantalla y/o mensaje se encuentre asociado a un caso de uso con estado ''Liberado'' solamente estará disponible la operación de consulta.
	\BRitem[Referenciado por:] \UCref{CU6}{Gestionar Términos de Glosario}, \UCref{CU7}{Gestionar Entidades}, \UCref{CU8}{Gestionar Reglas de Negocio}, \UCref{CU9}{Gestionar Mensajes}, \UCref{CU10}{Gestionar Actores}, \UCref{CU11}{Gestionar Pantallas}. 
\end{BussinesRule}

%\begin{BussinesRule}{RN16}{Nombres de las trayectorias} 
%	\BRitem[Tipo:] Restricción de operación. 
%	\BRitem[Clase:] Habilitadora. 
%	\BRitem[Nivel:] Control. % Otras opciones para nivel: Control, Influencia.
%	\BRitem[Descripción:] Los nombres de las trayectorias no pueden contener espacio, coma, punto, punto medio, dos puntos o guión bajo.
%	\BRitem[Referenciado por:] \UCref{}{}. 
%\end{BussinesRule}


\begin{BussinesRule}{RN17}{Unicidad de puntos de extensión} 
	\BRitem[Tipo:] Restricción de operación. 
	\BRitem[Clase:] Habilitadora. 
	\BRitem[Nivel:] Control. % Otras opciones para nivel: Control, Influencia.
	\BRitem[Descripción:] No puede existir más de un punto de extensión con el mismo caso de uso origen y el mismo caso de uso destino.
	\BRitem[Referenciado por:] \UCref{CU12.1.4.1}{Registrar punto de extensión}, \UCref{CU12.1.4.122}{Modificar punto de extensión}. 
\end{BussinesRule}


%\begin{BussinesRule}{RN18}{Eliminación de elemento} 
%	\BRitem[Tipo:] Restricción de operación. 
%	\BRitem[Clase:] Habilitadora. 
%	\BRitem[Nivel:] Control. % Otras opciones para nivel: Control, Influencia.
%	\BRitem[Descripción:] Un \hyperlink{tElemento}{elemento} no puede ser eliminado si tiene asociaciones con casos de uso en estado ''Liberado''.
%	\BRitem[Ejemplo positivo] Cumplen la regla:
%	\begin{itemize}
%		\item Eliminar un mensaje que no tiene asociados casos de uso en estado ''Liberado''.
%		\item Eliminar una pantalla que no tiene asociados casos de uso en estado ''Liberado''.
%		\item Eliminar una regla de negocio que no tiene asociados casos de uso en estado ''Liberado''.
%	\end{itemize}
%	\BRitem[Ejemplo negativo] No cumplen con la regla:
%	\begin{itemize}
%		\item Eliminar un atributo perteneciente a una entidad que se encuentra asociada a un caso de uso en estado ''Liberado''.
%		\item Eliminar una regla de negocio asociada con tres casos de uso en estado ''Liberado''.
%		\item Eliminar un actor asociado con un caso de uso en estado ''Liberado''.
%	\end{itemize}
%	\BRitem[Referenciado por:] \UCref{CU6.3}{Eliminar Término de Glosario}, \UCref{CU7.3}{Eliminar Entidad}, \UCref{CU7.1.3}{Eliminar Atributo}, \UCref{CU8.3}{Eliminar Regla de Negocio}, \UCref{CU9.3}{Eliminar Mensaje}, \UCref{CU10.3}{Eliminar Actor}, \UCref{CU11.3}{Eliminar Pantalla}. 
%\end{BussinesRule}

%\begin{BussinesRule}{RN19}{Formato de correo electrónico} 
%	\BRitem[Tipo:] Restricción de operación. 
%	\BRitem[Clase:] Habilitadora. 
%	\BRitem[Nivel:] Control. % Otras opciones para nivel: Control, Influencia.
%	\BRitem[Descripción:] El correo electrónico debe ser una cadena de caracteres con la siguiente estructura ordenada:
%	\begin{enumerate}
%		\item Cadena de caracteres
%		\item ''@''
%		\item Cadena de caracteres
%		\item ''.''
%		\item Cadena de caracteres
%	\end{enumerate}
%	\BRitem[Ejemplo]: cadena1@cadena2.cadena3
%	\BRitem[Referenciado por:] \UCref{}{}. 
%\end{BussinesRule}

\begin{BussinesRule}{RN20}{Verificación de catálogos} 
	\BRitem[Tipo:] Restricción de operación. 
	\BRitem[Clase:] Habilitadora. 
	\BRitem[Nivel:] Control. % Otras opciones para nivel: Control, Influencia.
	\BRitem[Descripción:] Es necesario que exista información registrada en los catálogos al momento de solicitar una operación que requiera de estos.
	\BRitem[Referenciado por:] \UCref{CU2.1}{Registrar Proyecto de Administrador}, \UCref{CU2.2}{Modificar Proyecto de Administrador}, \UCref{CU7.1.1}{Registrar Atributo}, \UCref{CU7.1.2}{Modificar Atributo}, \UCref{CU8.1}{Registrar Regla de Negocio}, \UCref{CU8.2}{Modificar Regla de Negocio}, \UCref{CU10.1}{Registrar Actor}, \UCref{CU10.2}{Modificar Actor}. 
\end{BussinesRule}


%\begin{BussinesRule}{RN21}{Estados para iniciar un proyecto} 
%	\BRitem[Tipo:] Restricción de operación. 
%	\BRitem[Clase:] Habilitadora. 
%	\BRitem[Nivel:] Control. % Otras opciones para nivel: Control, Influencia.
%	\BRitem[Descripción:] Se podrán registrar proyectos con estado ''En Negociación'' o ''Iniciado''.
%	\BRitem[Referenciado por:] \UCref{}{}. 
%\end{BussinesRule}

\begin{BussinesRule}{RN22}{Unicidad de la clave del proyecto} 
	\BRitem[Tipo:] Restricción de operación. 
	\BRitem[Clase:] Habilitadora. 
	\BRitem[Nivel:] Control. % Otras opciones para nivel: Control, Influencia.
	\BRitem[Descripción:] La clave de los proyectos debe ser única en todo el sistema.
	\BRitem[Referenciado por:] \UCref{CU2.1}{Registrar Proyecto de Administrador}. 
\end{BussinesRule}

\begin{BussinesRule}{RN23}{Unicidad de la clave del módulo} 
	\BRitem[Tipo:] Restricción de operación. 
	\BRitem[Clase:] Habilitadora. 
	\BRitem[Nivel:] Control. % Otras opciones para nivel: Control, Influencia.
	\BRitem[Descripción:] La clave de los módulos debe ser única en un proyecto.
	\BRitem[Referenciado por:] \UCref{CU5.1}{Registrar Módulo}. 
\end{BussinesRule}

\begin{BussinesRule}{RN24}{Unicidad de la clave de la trayectoria} 
	\BRitem[Tipo:] Restricción de operación. 
	\BRitem[Clase:] Habilitadora. 
	\BRitem[Nivel:] Control. % Otras opciones para nivel: Control, Influencia.
	\BRitem[Descripción:] La clave de las trayectorias debe ser única en un caso de uso.
	\BRitem[Referenciado por:] \UCref{}{}. 
\end{BussinesRule}

%\begin{BussinesRule}{RN25}{Relación entre fechas del proyecto} 
%	\BRitem[Tipo:] Restricción de operación. 
%	\BRitem[Clase:] Habilitadora. 
%	\BRitem[Nivel:] Control. % Otras opciones para nivel: Control, Influencia.
%	\BRitem[Descripción:] La fecha de término del proyecto debe ser posterior a la fecha de inicio.
%	\BRitem[Referenciado por:] \UCref{}{}. 
%\end{BussinesRule}

%\begin{BussinesRule}{RN26}{Relación entre fechas programadas del proyecto} 
%	\BRitem[Tipo:] Restricción de operación. 
%	\BRitem[Clase:] Habilitadora. 
%	\BRitem[Nivel:] Control. % Otras opciones para nivel: Control, Influencia.
%	\BRitem[Descripción:] La fecha de término programada del proyecto debe ser posterior a la fecha de inicio programada.
%	\BRitem[Referenciado por:] \UCref{}{}. 
%\end{BussinesRule}

\begin{BussinesRule}{RN27}{Eliminación de Colaboradores} 
	\BRitem[Tipo:] Restricción de operación. 
	\BRitem[Clase:] Habilitadora. 
	\BRitem[Nivel:] Control. % Otras opciones para nivel: Control, Influencia.
	\BRitem[Descripción:] No es posible eliminar un Colaborador, si esta es líder de al menos un proyecto.
	\BRitem[Referenciado por:] \UCref{CU3.3}{Eliminar Colaborador}. 
\end{BussinesRule}

\begin{BussinesRule}{RN28}{Eliminación de módulos} 
	\BRitem[Tipo:] Restricción de operación. 
	\BRitem[Clase:] Habilitadora. 
	\BRitem[Nivel:] Control. % Otras opciones para nivel: Control, Influencia.
	\BRitem[Descripción:] No es posible eliminar un módulo, si algún elemento de otro módulo, tiene referencias a al menos un elemento del módulo que desea eliminarse.
	\BRitem[Referenciado por:] \UCref{CU5.3}{Eliminar Módulo}. 
\end{BussinesRule}

\begin{BussinesRule}{RN29}{Unicidad de casos de uso} 
	\BRitem[Tipo:] Restricción de operación. 
	\BRitem[Clase:] Habilitadora. 
	\BRitem[Nivel:] Control. % Otras opciones para nivel: Control, Influencia.
	\BRitem[Descripción:] Diferentes casos de uso pueden tener el mismo nombre y/o número, únicamente si cada uno de estos pertenecen a diferentes módulos.
	\BRitem[Referenciado por:] \UCref{CU12.1}{Registrar Caso de Uso}, \UCref{CU12.2}{Modificar Caso de Uso}. 
\end{BussinesRule}

\begin{BussinesRule}{RN30}{Unicidad de pantallas} 
	\BRitem[Tipo:] Restricción de operación. 
	\BRitem[Clase:] Habilitadora. 
	\BRitem[Nivel:] Control. % Otras opciones para nivel: Control, Influencia.
	\BRitem[Descripción:] Diferentes pantallas pueden tener el mismo nombre y/o número, únicamente si cada una de estas pertenecen a diferentes módulos.
	\BRitem[Referenciado por:] \UCref{}{}. 
\end{BussinesRule}

\begin{BussinesRule}{RN31}{Estructura de tokens} 
	\BRitem[Tipo:] Restricción de operación. 
	\BRitem[Clase:] Habilitadora. 
	\BRitem[Nivel:] Control. % Otras opciones para nivel: Control, Influencia.
	\BRitem[Descripción:] Los tokens utilizados para referenciar \hyperlink{tElemento}{elementos} deben mantener una estructura determinada de acuerdo al tipo de elemento referenciado:
	\begin{itemize}
		\item Regla de negocio: {\em RN·Número:Nombre}
		\begin{itemize}
			\item ''RN'': cadena que indica que el tipo de elemento referenciado es una regla de negocio.
			\item ''·'': símbolo para separar las partes del token (punto medio).
			\item ''Número'': número de la regla de negocio referenciada.
			\item '':'': símbolo para separar las partes del token (dos puntos).
			\item ''Nombre'': nombre de la regla de negocio referenciada.
		\end{itemize}
		\item Entidad: {\em ENT·Nombre}
		\begin{itemize}
			\item ''ENT'': cadena que indica que el tipo de elemento referenciado es una entidad.
			\item ''·'': símbolo para separar las partes del token (punto medio).
			\item ''Nombre'': nombre de la entidad referenciada.
		\end{itemize}
		\item Caso de uso: {\em CU·ClaveMódulo·Número:Nombre}
		\begin{itemize}
			\item ''CU'': cadena que indica que el tipo de elemento referenciado es un caso de uso.
			\item ''·'': símbolo para separar las partes del token (punto medio).
			\item ''ClaveMódulo'': clave del módulo a la que pertenece el caso de uso referenciado.
			\item ''Número'': número del caso de uso referenciado.
			\item '':'': símbolo para separar las partes del token (dos puntos).
			\item ''Nombre'': nombre del caso de uso referenciado.
		\end{itemize}
		\item Pantalla: {\em IU·ClaveMódulo·Número:Nombre}
		\begin{itemize}
			\item ''IU'': cadena que indica que el tipo de elemento referenciado es una pantalla.
			\item ''·'': símbolo para separar las partes del token (punto medio).
			\item ''ClaveMódulo'': clave del módulo a la que pertenece la pantalla referenciada.
			\item ''Número'': número de la pantalla referenciada.
			\item '':'': símbolo para separar las partes del token (dos puntos).
			\item ''Nombre'': nombre de la pantalla referenciada.
		\end{itemize}
		\item Mensaje: {\em MSG·Número:Nombre}
		\begin{itemize}
			\item ''MSG'': cadena que indica que el tipo de elemento referenciado es un mensaje.
			\item ''·'':  símbolo para separar las partes del token (punto medio).
			\item ''Número'': número del mensaje referenciado.
			\item '':'': símbolo para separar las partes del token (dos puntos).
			\item ''Nombre'': nombre del mensaje referenciado.
		\end{itemize}
	\item Actor: {\em ACT·Nombre}
		\begin{itemize}
			\item ''ACT'': cadena que indica que el tipo de elemento referenciado es un actor.
			\item ''·'':  símbolo para separar las partes del token (punto medio).
			\item ''Nombre'': nombre del actor referenciado.
		\end{itemize}
	\item Término de glosario: {\em GLS·Nombre}
		\begin{itemize}
			\item ''GLS'': cadena que indica que el tipo de elemento referenciado es un témino del glosario.
			\item ''·'':  símbolo para separar las partes del token (punto medio).
			\item ''Nombre'': nombre del término del glosario referenciado.
		\end{itemize}
	\item Atributo: {\em ATR·Entidad:Nombre}
		\begin{itemize}
			\item ''ATR'': cadena que indica que el tipo de elemento referenciado es un atributo.
			\item ''·'':  símbolo para separar las partes del token (punto medio).
			\item ''Entidad'': nombre de la entidad a la que pertenece el atributo referenciado.
			\item '':'': símbolo para separar las partes del token (dos puntos).
			\item ''Nombre'': nombre del atributo referenciado.
		\end{itemize}
	\item Trayectoria: {\em TRAY·ClaveCasoUso·NúmeroCasoUso:NombreCasoUso:Clave}
		\begin{itemize}
			\item ''TRAY'': cadena que indica que el tipo de elemento referenciado es una trayectoria.
			\item ''·'':  símbolo para separar las partes del token (punto medio).
			\item ''ClaveCasoUso'': clave del caso de uso al que pertenece la trayectoria referenciada.
			\item ''NúmeroCasoUso'': número del caso de uso al que pertenece la trayectoria referenciada.
			\item '':'': símbolo para separar las partes del token (dos puntos).
			\item ''NombreCasoUso'': nombre del caso de uso al que pertenece la trayectoria referenciada.
			\item ''Clave'': clave de la trayectoria referenciada.
		\end{itemize}
	\item Paso: {\em P·ClaveCasoUso·NúmeroCasoUso:NombreCasoUso:ClaveTrayectoria·Número}
		\begin{itemize}
			\item ''P'': cadena que indica que el tipo de elemento referenciado es un paso.
			\item ''·'':  símbolo para separar las partes del token (punto medio).
			\item ''ClaveCasoUso'': clave del caso de uso al que pertenece el paso referenciado.
			\item ''NúmeroCasoUso'': número del caso de uso al que pertenece el paso referenciado.
			\item '':'': símbolo para separar las partes del token (dos puntos).
			\item ''NombreCasoUso'': nombre del caso de uso al que pertenece el paso referenciado.
			\item ''ClaveTrayectoria'': clave de la trayectoria a la que pertenece el paso referenciado.
			\item ''Número'': número del paso referenciado.
		\end{itemize}
	\item Acción: {\em ACC·ClavePantalla·NúmeroPantalla:NombrePantalla:Nombre}
		\begin{itemize}
			\item ''ACC'': cadena que indica que el tipo de elemento referenciado es una acción.
			\item ''·'':  símbolo para separar las partes del token (punto medio).
			\item ''ClavePantalla'': clave de la pantalla a la que pertenece la acción referenciada.
			\item ''NúmeroPantalla'': número de la pantalla a la que pertenece la acción referenciada.
			\item '':'': símbolo para separar las partes del token (dos puntos).
			\item ''NombrePantalla'': nombre de la pantalla a la que pertenece la acción referenciada.
			\item ''Nombre'': nombre de la acción referenciada.
		\end{itemize}
	\item Parámetro (Mensajes): {\em PARAM·Nombre}
		\begin{itemize}
			\item ''PARAM'': cadena que indica que se está realizando una referencia a un parámetro en un mensaje.
			\item ''·'':  símbolo para separar las partes del token (punto medio).
			\item ''Nombre'': nombre del parámentro referenciado.
		\end{itemize}
	\end{itemize}
	\BRitem[Referenciado por:] \UCref{CU12.1}{Registrar Caso de Uso}. 
\end{BussinesRule}


\begin{BussinesRule}{RN32}{Pasos en la trayectoria} 
	\BRitem[Tipo:] Restricción de operación. 
	\BRitem[Clase:] Habilitadora. 
	\BRitem[Nivel:] Control. % Otras opciones para nivel: Control, Influencia.
	\BRitem[Descripción:] Una trayectoria debe debe contar al menos con un paso.
	\BRitem[Referenciado por:] \UCref{}{}. 
\end{BussinesRule}

\begin{BussinesRule}{RN33}{Unicidad de la CURP} 
	\BRitem[Tipo:] Restricción de operación. 
	\BRitem[Clase:] Habilitadora. 
	\BRitem[Nivel:] Control. % Otras opciones para nivel: Control, Influencia.
	\BRitem[Descripción:] La CURP de un colaborador debe ser única en todo el sistema.
	\BRitem[Referenciado por:] \UCref{CU3.1}{Registrar Colaborador}. 
\end{BussinesRule}


\begin{BussinesRule}{RN34}{Eliminación de proyectos} 
	\BRitem[Tipo:] Restricción de operación. 
	\BRitem[Clase:] Habilitadora. 
	\BRitem[Nivel:] Control. % Otras opciones para nivel: Control, Influencia.
	\BRitem[Descripción:] No es posible eliminar un proyecto, si este tiene caso de uso asociados asociados.
	\BRitem[Referenciado por:] \UCref{CU2.3}{Eliminar Proyecto de Administrador}. 
\end{BussinesRule}


\begin{BussinesRule}{RN35}{Validar Fecha} 
	\BRitem[Tipo:] Restricción de operación. 
	\BRitem[Clase:] Habilitadora. 
	\BRitem[Nivel:] Control. % Otras opciones para nivel: Control, Influencia.
	\BRitem[Descripción:] Para registrar la duración de un proyecto, la fecha de termino programada debe ser posterior a la fecha de inicio programada.
	\BRitem[Referenciado por:] \UCref{CU2.1}{Registrar Proyecto de Administrador}, \UCref{CU2.2}{Modificar Proyecto de Administrador}. 
\end{BussinesRule}


\begin{BussinesRule}{RN36}{Unicidad de correos} 
	\BRitem[Tipo:] Restricción de operación. 
	\BRitem[Clase:] Habilitadora. 
	\BRitem[Nivel:] Control. % Otras opciones para nivel: Control, Influencia.
	\BRitem[Descripción:] El \cdtRef{colaboradorEntidad:correoColaborador}{correo electrónico} de un \hyperlink{colaboradorEntidad}{colaborador} no se puede duplicar en el ámbito donde es utilizado ni registrarse en más de una ocasión.
	\BRitem[Ejemplo positivo] Cumplen la regla:
	\begin{itemize}
		\item Dos \hyperlink{colaboradorEntidad}{colaboradores} con diferentes \cdtRef{colaboradorEntidad:correoColaborador}{correos electrónicos}.
	\end{itemize}
	\BRitem[Ejemplo negativo] No cumplen con la regla:
	\begin{itemize}
		\item Dos \hyperlink{colaboradorEntidad}{colaboradores} con el mismo \cdtRef{colaboradorEntidad:correoColaborador}{correo electrónico}.
	\end{itemize}
	\BRitem[Referenciado por:] \UCref{CU3.1}{Registrar Colaborador}, \UCref{CU3.2}{Modificar Colaborador}. 
\end{BussinesRule}


\begin{BussinesRule}{RN37}{Longitud/Valor de datos} 
	\BRitem[Tipo:] Restricción de operación. 
	\BRitem[Clase:] Habilitadora. 
	\BRitem[Nivel:] Control. % Otras opciones para nivel: Control, Influencia.
	\BRitem[Descripción:] Los datos proporcionados al sistema deben tener el número y el valor de caracteres especificados en el modelo conceptual.
	\BRitem[Referenciado por:] \UCref{CU1}{Iniciar Sesión}, \UCref{CU2.1}{Registrar Proyecto de Administrador}, \UCref{CU2.2}{Modificar Proyecto de Administrador}, \UCref{CU3.1}{Registrar Colaborador}, \UCref{CU3.2}{Modificar Colaborador}, \UCref{CU5.1}{Registrar Módulo}, \UCref{CU5.2}{Modificar Módulo}, \UCref{CU6.1}{Registrar Término de Glosario}, \UCref{CU6.2}{Modificar Término de Glosario}, \UCref{CU7.1}{Registrar Entidad}, \UCref{CU7.2}{Modificar Entidad}, \UCref{CU7.1.1}{Registrar Atributo}, \UCref{CU7.1.2}{Modificar Atributo}, \UCref{CU8.1}{Registrar Regla de Negocio}, \UCref{CU8.2}{Modificar Regla de Negocio}, \UCref{CU9.1}{Registrar Mensaje}, \UCref{CU9.2}{Modificar Mensaje}, \UCref{CU10.1}{Registrar Actor}, \UCref{CU10.2}{Modificar Actor}, \UCref{CU11.1}{Registrar Pantalla}, \UCref{CU11.2}{Modificar Pantalla}, \UCref{CU11.1.1}{Registrar Acción}, \UCref{CU11.1.2}{Modificar Acción}, \UCref{CU11.1.2}{Modificar Acción}, \UCref{CU12.1}{Registrar Caso de uso}. 
\end{BussinesRule}

\begin{BussinesRule}{RN38}{Funciones de administrador} 
	\BRitem[Tipo:] Restricción de operación. 
	\BRitem[Clase:] Habilitadora. 
	\BRitem[Nivel:] Control. % Otras opciones para nivel: Control, Influencia.
	\BRitem[Descripción:] El administrador no puede participar en ningún proyecto como \hyperlink{jefe}{Líder de proyecto} ni como \hyperlink{analista}{Analista}.
	\BRitem[Referenciado por:] \UCref{CU4.1}{Elegir Colaboradores}.
\end{BussinesRule}

\begin{BussinesRule}{RN40}{Peso de archivos de imagen}
	\BRitem[Tipo:] Restricción de operación. 
	\BRitem[Clase:] Habilitadora. 
	\BRitem[Nivel:] Controla la operación. % Otras opciones para nivel: Control, Influencia.
	\BRitem[Descripción:] Todos los archivos proporcionados al sistema deben tener un peso de 10 mb como máximo.
	\BRitem[Referenciado por:] \UCref{CU11.1}{Registrar Pantalla}, \UCref{CU11.2}{Modificar Pantalla}, \UCref{CU11.1.1}{Registrar Acción}, \UCref{CU11.1.2}{Modificar Acción}, \UCref{CU11.1.2}{Modificar Acción}.
\end{BussinesRule}