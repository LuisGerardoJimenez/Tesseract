\begin{BusinessEntity}{entidadPrecondicion}{Precondición}
	
	\Battr{redaccionPrecondicion}{Redacción: }{Bandera que indica si es precondición o postcondición. Descrita en uno o más párrafos y este dato es \hyperlink{tRequerido}{requerido} {\em (no se puede omitir)}.}
		
	\Battr{redaccionPrecondicion}{Redacción: }{Redacción de la Precondición. Descrita en uno o más párrafos y este dato es \hyperlink{tRequerido}{requerido} {\em (no se puede omitir)}. Este atributo debe de contener como mínimo 1 y máximo 999 caracteres. Caracteres admitidos: [A-Z] $|$ [a-z] $|$ [ñ-á-é-í-ó-ú] $|$ [Ñ-Á-É-Í-Ó-Ú] $|$ [0-9] $|$ [  ] $|$ [\_] $|$ [-] $|$ [$!$, $?$, \&, @, \%, \#, $($, $)$, ., :, >, <, *, =, $,$, $/$, $"$]}
	
\end{BusinessEntity}

\subsubsection{Relaciones}

\begin{BusinessFact}{elementoRelPrecondicion}{Elemento}
	\BRitem{\textbf{Descripción: }}{La Precondición se encuentra constituida por diferentes elementos.}
	\BRitem{\textbf{Tipo: }}{\relAsociacion}
	\BRitem{\textbf{Cardinalidad: }}{Muchos a Muchos}
\end{BusinessFact}