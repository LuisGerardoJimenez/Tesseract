\begin{BusinessEntity}{casoUso}{Caso de Uso}
	\Battr{claveCU}{Clave: }{Clave que permitirá distinguir que el elemento es un Caso de Uso. Es una palabra corta y este dato es requerido {\em (no se puede omitir)}.  Este atributo debe de contener como mínimo 1 y máximo 10 caracteres. Caracteres admitidos: [A-Z] $|$ [0-9] $|$ [Ñ-Á-É-Í-Ó-Ú]}
		
	\Battr{numeroCU}{Número: }{Número del Caso de Uso. Es un valor numérico entero y este dato es requerido {\em (no se puede omitir)}. Este atributo debe de contener como máximo 20 dígitos. Caracteres admitidos: [0-9] $|$ [.]}
		
	\Battr{nombreCU}{Nombre: }{Nombre que identificará al Caso de Uso. Es una frase o enunciado y este dato es requerido {\em (no se puede omitir)}. Este atributo debe de contener como mínimo 1 y máximo 200 caracteres. Caracteres admitidos: [A-Z] $|$ [a-z] $|$ [ñ-á-é-í-ó-ú] $|$ [Ñ-Á-É-Í-Ó-Ú] $|$ [0-9] $|$ [ ] $|$ [-] $|$ [$,$, .] $|$ [$($, $)$] $|$ [$/$, $"$]}

	\Battr{resumenCU}{Resumen: }{Breve descripción del contenido del caso de uso. Descrita en uno o más párrafos y este dato es requerido {\em (no se puede omitir)}. Este atributo debe de contener como mínimo 1 y máximo 999 caracteres. Caracteres admitidos: [A-Z] $|$ [a-z] $|$ [ñ-á-é-í-ó-ú] $|$ [Ñ-Á-É-Í-Ó-Ú] $|$ [0-9] $|$ [  ] $|$ [\_] $|$ [-] $|$ [$!$, $?$, \&, @, \%, \#, $($, $)$, ., :, >, <, *, =, $,$, $/$, $"$]}
\end{BusinessEntity}

\subsubsection{Relaciones}
\begin{BusinessFact}{pantallaRelCU}{Pantalla}
	\BRitem{\textbf{Descripción: }}{Un Caso de Uso utiliza diferentes pantallas.}
	\BRitem{\textbf{Tipo: }}{\relAgregacion}
	\BRitem{\textbf{Cardinalidad: }}{Muchos a muchos}
\end{BusinessFact}

\begin{BusinessFact}{actorRelCU}{Actor}
	\BRitem{\textbf{Descripción: }}{Un Caso de Uso puede ser realizado por diferentes actores.}
	\BRitem{\textbf{Tipo: }}{\relAgregacion}
	\BRitem{\textbf{Cardinalidad: }}{Muchos a muchos}
\end{BusinessFact}

\begin{BusinessFact}{BrRelCU}{Reglas de Negocio}
	\BRitem{\textbf{Descripción: }}{Un Caso de Uso puede utilizar diferentes relgas de negocio.}
	\BRitem{\textbf{Tipo: }}{\relAgregacion}
	\BRitem{\textbf{Cardinalidad: }}{Muchos a muchos}
\end{BusinessFact}

\begin{BusinessFact}{salidaRelCU}{Salida}
	\BRitem{\textbf{Descripción: }}{Un caso de uso tiene diferentes salidas.}
	\BRitem{\textbf{Tipo: }}{\relComposicion}
	\BRitem{\textbf{Cardinalidad: }}{Uno a muchos}
\end{BusinessFact}

\begin{BusinessFact}{entradaRelCU}{Entrada}
	\BRitem{\textbf{Descripción: }}{Un caso de uso tiene diferentes entradas.}
	\BRitem{\textbf{Tipo: }}{\relComposicion}
	\BRitem{\textbf{Cardinalidad: }}{Uno a muchos}
\end{BusinessFact}

\begin{BusinessFact}{trayRelCU}{Trayectoria}
	\BRitem{\textbf{Descripción: }}{Un caso de uso se compone de trayectorias.}
	\BRitem{\textbf{Tipo: }}{\relComposicion}
	\BRitem{\textbf{Cardinalidad: }}{Uno a muchos}
\end{BusinessFact}

\begin{BusinessFact}{revRelCU}{Revisión}
	\BRitem{\textbf{Descripción: }}{Un caso de uso puede ser revisado.}
	\BRitem{\textbf{Tipo: }}{\relComposicion}
	\BRitem{\textbf{Cardinalidad: }}{Uno a muchos}
\end{BusinessFact}

\begin{BusinessFact}{extRelCU}{Extensión}
	\BRitem{\textbf{Descripción: }}{Un caso de uso puede tener puntos de extensión.}
	\BRitem{\textbf{Tipo: }}{\relComposicion}
	\BRitem{\textbf{Cardinalidad: }}{Uno a muchos}
\end{BusinessFact}

\begin{BusinessFact}{incluRelCU}{Inclusión}
	\BRitem{\textbf{Descripción: }}{Un caso de uso puede incluir otros casos de uso.}
	\BRitem{\textbf{Tipo: }}{\relComposicion}
	\BRitem{\textbf{Cardinalidad: }}{Uno a muchos}
\end{BusinessFact}

\begin{BusinessFact}{precRelCU}{Precondición}
	\BRitem{\textbf{Descripción: }}{Un caso de uso puede tener diferentes precondiciones.}
	\BRitem{\textbf{Tipo: }}{\relComposicion}
	\BRitem{\textbf{Cardinalidad: }}{Uno a muchos}
\end{BusinessFact}

\begin{BusinessFact}{postcRelCU}{Postcondición}
	\BRitem{\textbf{Descripción: }}{Un caso de uso puede tener diferentes postcondiciones.}
	\BRitem{\textbf{Tipo: }}{\relComposicion}
	\BRitem{\textbf{Cardinalidad: }}{Uno a muchos}
\end{BusinessFact}