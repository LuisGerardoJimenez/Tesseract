\begin{BusinessEntity}{proyectoEntidad}{Proyecto}
	\Battr{claveProyecto}{Clave: }{Palabra que permitirá distinguir al proyecto, regularmente es la sigla del nombre. Es una palabra corta y este dato es \hyperlink{tRequerido}{requerido} {\em (no se puede omitir)}. Este atributo debe de contener como mínimo 1 y máximo 10 caracteres. Caracteres admitidos: [A-Z] $|$ [0-9] $|$ [Ñ-Á-É-Í-Ó-Ú]}
		
	\Battr{nombreProyecto}{Nombre: }{Nombre que identificará al proyecto. Es una frase o enunciado y este dato es \hyperlink{tRequerido}{requerido} {\em (no se puede omitir)}. Este atributo debe de contener como mínimo 1 y máximo 50 caracteres. Caracteres admitidos: [A-Z] $|$ [a-z] $|$ [ñ-á-é-í-ó-ú] $|$ [Ñ-Á-É-Í-Ó-Ú] $|$ [0-9] $|$ [ ]}
	
	\Battr{fechaIProyecto}{Fecha de Inicio: }{Fecha en que se arranca el proyecto. Especifica una \hyperlink{tFechaCorta}{Fecha Corta} y este dato es \hyperlink{tOpcional}{opcional} {\em (se puede omitir)}.}
	
	\Battr{fechaFinProyecto}{Fecha de termino: }{Fecha en la que concluye el proyecto. Especifica una \hyperlink{tFechaCorta}{Fecha Corta} y este dato es \hyperlink{tOpcional}{opcional} {\em (se puede omitir)}.}
	
	\Battr{fechaIPProyecto}{Fecha de inicio programada: }{Fecha en que se desea arrancar el proyecto. Especifica una \hyperlink{tFechaCorta}{Fecha Corta:} y este dato es \hyperlink{tRequerido}{requerido} {\em (no se puede omitir)}.}
	
	\Battr{fechaFinPProyecto}{Fecha de termino programada: }{Fecha en que se desea finalizar el proyecto. Especifica una \hyperlink{tFechaCorta}{Fecha Corta:} y este dato es \hyperlink{tRequerido}{requerido} {\em (no se puede omitir)}.}
	
	\Battr{descripcionProyecto}{Descripción: }{Párrafo que contiene las características generales del proyecto que se comenzará. Descrita en uno o más párrafos y este dato es \hyperlink{tRequerido}{requerido} {\em (no se puede omitir)}. Este atributo debe de contener como mínimo 1 y máximo 999 caracteres. Caracteres admitidos: [A-Z] $|$ [a-z] $|$ [ñ-á-é-í-ó-ú] $|$ [Ñ-Á-É-Í-Ó-Ú] $|$ [0-9] $|$ [  ] $|$ [\_] $|$ [-] $|$ [$!$, $?$, \&, @, \%, \#, $($, $)$, ., :, >, <, *, =, $,$]}
	
	\Battr{liderProyecto}{Líder: }{Nombre y apellidos del líder del proyecto. Es una frase o enunciado y este dato es \hyperlink{tRequerido}{requerido} {\em (no se puede omitir)}.}
	
	\Battr{contraparteProyecto}{Contraparte: }{Es el cliente del proyecto. Es una frase o enunciado y este dato es \hyperlink{tOpcional}{Opcional} {\em (se puede omitir)}.Este atributo debe de contener como mínimo 1 y máximo 45 caracteres. Caracteres admitidos: [A-Z] $|$ [a-z] $|$ [ñ-á-é-í-ó-ú] $|$ [Ñ-Á-É-Í-Ó-Ú] $|$ [0-9] $|$ [  ] $|$ [\_] $|$ [-] $|$ [$!$, $?$, \&, @, \%, \#, $($, $)$, ., :, >, <, *, =, $,$}
	
	\Battr{presupuestoProyecto}{Presupuesto: }{Es el monto calculado del costo del proyecto. Es un valor numérico flotante y este dato es \hyperlink{tOpcional}{opcional} {\em (se puede omitir)}. Este atributo debe contener como valor máximo 999999999.99. Caracteres admitidos: [0-9] $|$ [.]}
	
		\Battr{estadoProyecto}{Estado: }{Es la situación actual en la que se encuentra el proyecto y se define bajo condiciones específicas. Los valores que puede tomar un proyecto son: Iniciado, En negociación y Terminado, este dato es \hyperlink{tRequerido}{requerido} {\em (no se puede omitir)}.}
\end{BusinessEntity}

\subsubsection{Relaciones}
\begin{BusinessFact}{elementoRelProyecto}{Elemento}
	\BRitem{\textbf{Descripción: }}{Un proyecto puede tener varios Elementos asociados como reglas de negocio, mensajes, entidades y actores.}
	\BRitem{\textbf{Tipo: }}{\relComposicion}
	\BRitem{\textbf{Cardinalidad: }}{Uno a muchos}
\end{BusinessFact}

\begin{BusinessFact}{estadoRelProyecto}{Estado del Proyecto}
	\BRitem{\textbf{Descripción: }}{Un proyecto tiene un Estado del Proyecto.}
	\BRitem{\textbf{Tipo: }}{\relAsociacion}
	\BRitem{\textbf{Cardinalidad: }}{Muchos a uno}
\end{BusinessFact}

\begin{BusinessFact}{colaboradorRelProyecto}{Colaborador del Proyecto}
	\BRitem{\textbf{Descripción: }}{Un proyecto tiene uno o varios Colaboradores del Proyecto.}
	\BRitem{\textbf{Tipo: }}{\relComposicion}
	\BRitem{\textbf{Cardinalidad: }}{Uno a muchos}
\end{BusinessFact}

\begin{BusinessFact}{moduloRelProyecto}{Módulo}
	\BRitem{\textbf{Descripción: }}{Un proyecto tiene uno o varios Módulos donde se organizarán los casos de uso y las pantallas.}
	\BRitem{\textbf{Tipo: }}{\relComposicion}
	\BRitem{\textbf{Cardinalidad: }}{Uno a muchos}
\end{BusinessFact}