	\begin{UseCase}{CU9.1}{Registrar Mensaje}{
			
			Este caso de uso permite al colaborador (\hyperlink{jefe}{Líder de Análisis} o \hyperlink{analista}{Analista}) registrar la información de un \hyperlink{MSGEntidad}{Mensaje} en el \hyperlink{proyectoEntidad}{Proyecto}.\\
			
			Los datos de un mensaje están comprendidos por un identificador, una descripción y su redacción. Al definir el contenido de su redacción, el sistema posibilita la utilización de parámetros a través de un tokens. Estos parámetros pueden ser creados al momento o utilizados en caso de que hayan sido creados para mensajes previamente registrados.\\
			
			El propósito de almacenar su datos es definir la comunicación actor-sistema a través de textos que informan errores, avisos, o notificaciones.
			Una vez registrado el mensaje, el colaborador podrá hacer uso de la mismo en el editor.\\
	}
		\UCitem{Versión}{\color{Gray}0.1}
		\UCitem{Actor}{\hyperlink{jefe}{Líder de Análisis}, \hyperlink{analista}{Analista}}
		\UCitem{Propósito}{Registrar la información de mensaje.}
		\UCitem{Entradas}{
		\begin{itemize}
			\item \cdtRef{MSGEntidad:numeroMSG}{Número del mensaje:} Se escribe desde el teclado.
			\item \cdtRef{MSGEntidad:nombreMSG}{Nombre del mensaje:} Se escribe desde el teclado.
			\item \cdtRef{MSGEntidad:descripcionMSG}{Descripción del mensaje:} Se escribe desde el teclado.
			\item \cdtRef{MSGEntidad:redaccionMSG}{Redacción del mensaje:} Se escribe desde el teclado.
			\item \cdtRef{MSGEntidad:paramtrizadoMSG}{Parametrizado:} Se escribe desde el teclado.
			\item Parámetros: Se escribe desde el teclado.
		\end{itemize}	
		}
		\UCitem{Salidas}{\begin{itemize}
				\item \cdtRef{proyectoEntidad:claveProyecto}{Clave del proyecto:} Lo obtiene el sistema.
				\item \cdtRef{proyectoEntidad:nombreProyecto}{Nombre del proyecto:} Lo obtiene el sistema.
				\item \cdtRef{MSGEntidad:claveMSG}{Clave:} Lo calcula el sistema mediante la regla de negocio \BRref{RN12}{Identifcador de elemento}.
				\item \cdtIdRef{MSG1}{Operación exitosa}: Se muestra en la pantalla \IUref{IU10}{Gestionar Mensajes} para indicar que el registro fue exitoso.
		\end{itemize}}
		\UCitem{Precondiciones}{Ninguna}
		\UCitem{Postcondiciones}{
		\begin{itemize}
			\item Se registrará un mensaje en el sistema.
			\item El mensaje podrá ser referenciado en casos de uso.
		\end{itemize}
		}
		\UCitem{Errores}{\begin{itemize}
		\item \cdtIdRef{MSG4}{Dato obligatorio}: Se muestra en la pantalla \IUref{IU10.1}{Registrar Mensaje} cuando no se ha ingresado un dato marcado como obligatorio.
		\item \cdtIdRef{MSG5}{Formato incorrecto}: Se muestra en la pantalla \IUref{IU10.1}{Registrar Mensaje} cuando el tipo de dato ingresado no cumple con el tipo de dato solicitado en el campo.
		\item \cdtIdRef{MSG5}{Formato de campo incorrecto}: Se muestra en la pantalla \IUref{IU10.1}{Registrar Mensaje} cuando el número de mensaje contiene un carácter no válido.
		\item \cdtIdRef{MSG6}{Longitud inválida}: Se muestra en la pantalla \IUref{IU10.1}{Registrar Mensaje} cuando se ha excedido la longitud de alguno de los campos.
		\item \cdtIdRef{MSG7}{Registro repetido}: Se muestra en la pantalla \IUref{IU10.1}{Registrar Mensaje} cuando se registre un mensaje con un nombre o número que ya se encuentre registrado en el sistema.
		\end{itemize}.
		}
		\UCitem{Tipo}{Secundario, extiende del caso de uso \UCref{CU9}{Gestionar Mensajes}.}
	\end{UseCase}
%--------------------------------------
	\begin{UCtrayectoria}
		\UCpaso[\UCactor] Solicita registrar una mensaje oprimiendo el botón \IUbutton{Registrar} de la pantalla \IUref{IU10}{Gestionar Mensajes}.
		\UCpaso[\UCsist] Muestra la pantalla \IUref{IU10.1}{Registrar Mensaje: No parametrizado}.
		\UCpaso[\UCactor] Ingresa la información solicitada. \label{CU9.1-P3} \hyperlink{CU9-1:TAA}{[Trayectoria A]}
		\UCpaso[\UCactor] Oprime el botón \IUbutton{Aceptar} . \label{CU9.1-P4} \hyperlink{CU9-1:TAB}{[Trayectoria B]} 
		\UCpaso[\UCsist] Verifica que el actor ingrese todos los campos obligatorios con base en la regla de negocio \BRref{RN8}{Datos obligatorios}. \hyperlink{CU9-1:TAC}{[Trayectoria C]}
		\UCpaso[\UCsist] Verifica que los datos ingresados cumpla con la longitud correcta, con base en la regla de negocio \BRref{RN37}{Longitud de datos}. \hyperlink{CU9-1:TAD}{[Trayectoria D]}
		\UCpaso[\UCsist] Verifica que los datos ingresados cumplan con el formato requerido, con base en la regla de negocio \BRref{RN7}{Información correcta}. \hyperlink{CU9-1:TAE}{[Trayectoria E]}
		\UCpaso[\UCsist] Verifica que el número del mensaje no se encuentre registrado en el sistema con base en la regla de negocio \BRref{RN1}{Unicidad de números}. \hyperlink{CU9-1:TAF}{[Trayectoria F]}
		\UCpaso[\UCsist] Verifica que el nombre de la mensaje no se encuentre registrado en el sistema con base en la regla de negocio \BRref{RN6}{Unicidad de nombres}. \hyperlink{CU9-1:TAG}{[Trayectoria G]} 
		\UCpaso[\UCsist] Registra la información del mensaje en el sistema.
		\UCpaso[\UCsist] Muestra el mensaje \cdtIdRef{MSG1}{Operación exitosa} en la pantalla \IUref{IU10}{Gestionar Mensajes} para indicar al actor que el registro se ha realizado exitosamente.
	\end{UCtrayectoria}		
%--------------------------------------
\hypertarget{CU9-1:TAA}{\textbf{Trayectoria alternativa A}}\\
\noindent \textbf{Condición:} El actor desea ingresar un parámetro.
\begin{enumerate}
	\UCpaso[\UCactor] Ingresa el token ''PARAM·''.
	\UCpaso[\UCsist] Muestra la pantalla \IUref{IU10.1A}{Registrar Mensaje: Parametrizado}.
	\UCpaso[\UCactor] Ingresa la descripción de cada parámetro. \label{CU9.1-TA2}
	\UCpaso[\UCsist] Continúa en el paso \ref{CU9.1-P4} de la trayectoria principal.
	\item[- -] - - {\em {Fin de la trayectoria}}.%
\end{enumerate}
%--------------------------------------
\hypertarget{CU9-1:TAB}{\textbf{Trayectoria alternativa B}}\\
\noindent \textbf{Condición:} El actor desea cancelar la operación.
\begin{enumerate}
	\UCpaso[\UCactor] Solicita cancelar la operación oprimiendo el botón \IUbutton{Cancelar} de la pantalla \IUref{IU10.1}{Registrar mensaje}.
	\UCpaso[\UCsist] Muestra la pantalla \IUref{IU10}{Gestionar Mensajes}.
	\item[- -] - - {\em {Fin del caso de uso}}.%
\end{enumerate}
%--------------------------------------
\hypertarget{CU9-1:TAC}{\textbf{Trayectoria alternativa C}}\\
\noindent \textbf{Condición:} El actor no ingresó algún dato marcado como obligatorio.
\begin{enumerate}
	\UCpaso[\UCsist] Muestra el mensaje \cdtIdRef{MSG4}{Dato obligatorio} señalando el campo que presenta el error en la pantalla \IUref{IU10.1}{Registrar Mensaje}.
	\UCpaso Regresa al paso \ref{CU9.1-P3} de la trayectoria principal o al paso \ref{CU9.1-TA2} de la trayectoria Alternativa A.
	\item[- -] - - {\em {Fin de la trayectoria}}.%
\end{enumerate}
%--------------------------------------
\hypertarget{CU9-1:TAD}{\textbf{Trayectoria alternativa D}}\\
\noindent \textbf{Condición:} El actor ingresó un dato con un número de caracteres fuera del rango permitido.
\begin{enumerate}
	\UCpaso[\UCsist] Muestra el mensaje \cdtIdRef{MSG6}{Longitud inválida} señalando el campo que presenta el error en la pantalla \IUref{IU10.1}{Registrar Mensaje}.
	\UCpaso Regresa al paso \ref{CU9.1-P3} de la trayectoria principal al paso \ref{CU9.1-TA2} de la trayectoria Alternativa A.
	\item[- -] - - {\em {Fin de la trayectoria}}.%
\end{enumerate}
%--------------------------------------
\hypertarget{CU9-1:TAE}{\textbf{Trayectoria alternativa E}}\\
\noindent \textbf{Condición:} El actor ingresó un dato con un formato incorrecto.
\begin{enumerate}
	\UCpaso[\UCsist] Muestra el mensaje \cdtIdRef{MSG5}{Formato incorrecto} señalando el campo que presenta el error en la pantalla \IUref{IU10.1}{Registrar Mensaje}.
	\UCpaso Regresa al paso \ref{CU9.1-P3} de la trayectoria principal o al paso \ref{CU9.1-TA2} de la trayectoria Alternativa A.
	\item[- -] - - {\em {Fin de la trayectoria}}.
\end{enumerate}
%--------------------------------------
\hypertarget{CU9-1:TAF}{\textbf{Trayectoria alternativa F}}\\
\noindent \textbf{Condición:} El actor ingresó un número de mensaje repetido.
\begin{enumerate}
	\UCpaso[\UCsist] Muestra el mensaje \cdtIdRef{MSG7}{Registro repetido} señalando el campo que presenta la duplicidad en la pantalla \IUref{IU10.1}{Registrar Mensaje}.
	\UCpaso Regresa al paso \ref{CU9.1-P3} de la trayectoria principal.
	\item[- -] - - {\em {Fin de la trayectoria}}.
\end{enumerate}
%--------------------------------------
\hypertarget{CU9-1:TAG}{\textbf{Trayectoria alternativa G}}\\
\noindent \textbf{Condición:} El actor ingresó un nombre de un mensaje repetido.
\begin{enumerate}
	\UCpaso[\UCsist] Muestra el mensaje \cdtIdRef{MSG7}{Registro repetido} señalando el campo que presenta la duplicidad en la pantalla \IUref{IU10.1}{Registrar Mensaje}.
	\UCpaso Regresa al paso \ref{CU9.1-P3} de la trayectoria principal.
	\item[- -] - - {\em {Fin de la trayectoria}}.
\end{enumerate}
