	\begin{UseCase}{CU9}{Gestionar Mensajes}{
	Este caso de uso permite al analista visualizar los registros de los mensajes del sistema. También permite al actor acceder a las operaciones de registro, modificación y eliminación de un mensaje.
	}
	\UCitem{Actor}{\hyperlink{jefe}{Líder de análisis}, \hyperlink{analista}{Analista}}
	\UCitem{Propósito}{Proporcionar al actor un mecanismo para llevar el control de los mensajes de un proyecto.}
	\UCitem{Entradas}{Ninguna}
	\UCitem{Salidas}{\begin{itemize}
			\item \cdtRef{proyectoEntidad:claveProyecto}{Clave del proyecto}: Lo obtiene el sistema.
			\item \cdtRef{proyectoEntidad:nombreProyecto}{Nombre del proyecto}: Lo obtiene el sistema.
			\item \cdtRef{MSGEntidad}{Mensaje}: Tabla que muestra \cdtRef{MSGEntidad:nombreMSG}{nombre} de todos los mensajes registrados de un proyecto.
			\item \cdtIdRef{MSG2}{No existe información}: Se muestra en la pantalla \IUref{IU10A}{Gestionar Mensajes} cuando no existen mensajes registradas.
	\end{itemize}}
	
	\UCitem{Precondiciones}{\begin{itemize}
			\item Que exista al menos un proyecto registrado
			\item Que exista al menos un mensaje registrado en el proyecto.
	\end{itemize}}
	\UCitem{Postcondiciones}{Ninguna}
	\UCitem{Errores}{Ninguno}
	\UCitem{Tipo}{Primario}
\end{UseCase}
%--------------------------------------
\begin{UCtrayectoria}
	\UCpaso[\UCactor] Solicita gestionar los mensajes seleccionando la opción ''Mensajes'' del menú \IUref{MN3}{Menú de Proyecto}.
	\UCpaso[\UCsist] Obtiene la información de los mensajes registrados del proyecto seleccionado. \hyperlink{CU9:TAA}{[Trayectoria A]}
	\UCpaso[\UCsist] Ordena los mensajes alfabéticamente basándose en la clave de los mismos.
	\UCpaso[\UCsist] Muestra la información de los mensajes en la pantalla \IUref{IU10}{Gestionar Mensajes} y las operaciones disponibles de acuerdo a la regla de negocio \BRref{RN15}{Operaciones disponibles}.
	\UCpaso[\UCactor] Gestiona los mensajes a través de los botones: \IUbutton{Registrar}, \editar y \eliminar. \label{CU9-P4}
\end{UCtrayectoria}		
%--------------------------------------
\hypertarget{CU9:TAA}{\textbf{Trayectoria alternativa A}}\\
\noindent \textbf{Condición:} No existen registros de Mensajes.
\begin{enumerate}
	\UCpaso[\UCsist] Muestra el mensaje \cdtIdRef{MSG2}{No existe información} en la pantalla \IUref{IU10A}{Gestionar Mensajes} para indicar que no hay registros de mensajes para mostrar. \label{CU9-TA1}
	\UCpaso[\UCactor] Gestiona los mensajes a través del botón: \IUbutton{Registrar}. 
	\item[- -] - - {\em {Fin del caso de uso}}.%
\end{enumerate}

%--------------------------------------

\subsubsection{Puntos de extensión}

\UCExtenssionPoint{El actor requiere registrar un mensaje.}{Paso \ref{CU9-P4} de la trayectoria principal o del paso \ref{CU9-TA1} de la trayectoria alternativa A.}{\UCref{CU9.1}{Registrar Mensaje}}
\UCExtenssionPoint{El actor requiere modificar un mensaje.}{Paso \ref{CU9-P4} de la trayectoria principal.}{\UCref{CU9.2}{Modificar Mensaje}}
\UCExtenssionPoint{El actor requiere eliminar un mensae.}{Paso \ref{CU9-P4} de la trayectoria principal.}{\UCref{CU9.3}{Eliminar Mensaje}}
\UCExtenssionPoint{El actor requiere consultar un mensaje.}{Paso \ref{CU9-P4} de la trayectoria principal.}{\UCref{CU9.4}{Consultar Mensaje}}