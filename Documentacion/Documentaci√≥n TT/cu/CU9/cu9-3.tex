	\begin{UseCase}{CU9.3}{Eliminar Mensaje}{
			
				Cuando ya no se hará uso de algún \hyperlink{MSGEntidad}{Mensaje} que estaba considerado o simplemente no tiene una razón de ser dentro del \hyperlink{proyectoEntidad}{Proyecto}, Tesseract permitirá al colaborador (\hyperlink{jefe}{Líder} o \hyperlink{analista}{Analista}) eliminar en su totalidad el registro del mensaje. \\
			   
			    Una Mensaje podrá ser eliminado siempre y cuando no se encuentre asociado con algún \hyperlink{casoUso}{Caso de Uso} en estado ”Liberado”.	
			
	}
		\UCitem{Versión}{\color{Gray}0.1}
		\UCitem{Actor}{\hyperlink{jefe}{Líder de Análisis}, \hyperlink{analista}{Analista}}
		\UCitem{Propósito}{Eliminar la información de un mensaje.}
		\UCitem{Entradas}{Ninguna}
		\UCitem{Salidas}{\begin{itemize}
				\item \cdtIdRef{MSG1}{Operación exitosa}: Se muestra en la pantalla \IUref{IU10}{Gestionar Mensajes} para indicar que el mensaje fue eliminado correctamente.
				\item \cdtIdRef{MSG10}{Confirmar eliminación}: Se muestra en la pantalla \IUref{IU10}{Gestionar Mensajes} preguntando al actor si desea continuar con la eliminación del mensaje.
		\end{itemize}}
		\UCitem{Destino}{Pantalla}
		\UCitem{Precondiciones}{\begin{itemize}
				\item Que el mensaje no se encuentre asociado a un caso de uso.
				\item Que el mensaje no se encuentre asociado a un caso de uso liberado.
		\end{itemize}}
		\UCitem{Postcondiciones}{
		\begin{itemize}
			\item Se eliminará el mensaje de un proyecto del sistema.
		\end{itemize}
		}
		\UCitem{Errores}{\begin{itemize}
		\item \cdtIdRef{MSG13}{Eliminación no permitida}: Se muestra en la pantalla \IUref{IU10}{Gestionar Mensajes} cuando el mensaje está siendo referenciado en algún caso de uso
		\end{itemize}
		}
		\UCitem{Tipo}{Secundario, extiende del caso de uso \UCref{CU9}{Gestionar Mensajes}.}
	\end{UseCase}
%--------------------------------------
	\begin{UCtrayectoria}
		\UCpaso[\UCactor] Da clic en el icono \eliminar del registro que desea eliminar de la pantalla \IUref{IU10}{Gestionar Mensajes}.
		\UCpaso[\UCsist] Muestra el mensaje emergente \cdtIdRef{MSG10}{Confirmar eliminación} con los botones \IUbutton{Aceptar} y \IUbutton{Cancelar} en la pantalla \IUref{IU10}{Gestionar Mensajes}.
		\UCpaso[\UCactor] Confirma la eliminación de la mensaje oprimiendo el botón \IUbutton{Aceptar}. \hyperlink{CU9-3:TAA}{[Trayectoria A]}
		\UCpaso[\UCsist] Verifica que ningún caso de uso se encuentre asociado al mensaje. \hyperlink{CU9-3:TAB}{[Trayectoria B]}
		\UCpaso[\UCsist] Elimina la información referente al mensaje.
		\UCpaso[\UCsist] Muestra el mensaje \cdtIdRef{MSG1}{Operación exitosa} en la pantalla \IUref{IU10}{Gestionar Mensajes} para indicar al actor que el registro se ha eliminado exitosamente.
	\end{UCtrayectoria}

%--------------------------------------
\hypertarget{CU9-3:TAA}{\textbf{Trayectoria alternativa A}}\\
\noindent \textbf{Condición:} El actor desea cancelar la operación.
\begin{enumerate}
	\UCpaso[\UCactor] Oprime el botón \IUbutton{Cancelar} de la pantalla emergente.
	\UCpaso[\UCsist] Muestra la pantalla \IUref{IU10}{Gestionar Mensajes}.
	\item[- -] - - {\em {Fin del caso de uso}}.%
\end{enumerate}
%--------------------------------------
\hypertarget{CU9-3:TAB}{\textbf{Trayectoria alternativa B}}\\
\noindent \textbf{Condición:} El mensaje está siendo referenciado en un caso de uso.
\begin{enumerate}
	\UCpaso[\UCsist] Muestra el mensaje \cdtIdRef{MSG13}{Eliminación no permitida} en la pantalla \IUref{IU10}{Gestionar Mensajes}.
	\item[- -] - - {\em {Fin del caso de uso}}.%
\end{enumerate}	

