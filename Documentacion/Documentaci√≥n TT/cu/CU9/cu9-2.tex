	\begin{UseCase}{CU9.2}{Modificar Mensaje}{
			
			Después de haber registrado un \hyperlink{MSGEntidad}{Mensaje} en el \hyperlink{proyectoEntidad}{Proyecto} y el colaborador (\hyperlink{jefe}{Líder de Análisis} o \hyperlink{analista}{Analista}) requiera modificar su información, el sistema le permitirá editar cualquiera de los datos previamente registrados (a excepción de la clave) mediante un formulario, este formulario contendrá los datos precargados de la última actualización para modificarlos y posteriormente guardarlos.\\
			
			La única condición para modificar un mensaje exitosamente es que este no se encuentre asociado a algún \hyperlink{casoUso}{Caso de Uso} con estado ”Liberado”.
			
	}
		\UCitem{Versión}{\color{Gray}0.1}
		\UCitem{Actor}{\hyperlink{jefe}{Líder de Análisis}, \hyperlink{analista}{Analista}}
		\UCitem{Propósito}{Modificar la información de mensaje.}
		\UCitem{Entradas}{
		\begin{itemize}
			\item \cdtRef{MSGEntidad:nombreMSG}{Nombre del mensaje:} Se escribe desde el teclado.
			\item \cdtRef{MSGEntidad:descripcionMSG}{Descripción del mensaje:} Se escribe desde el teclado.
			\item \cdtRef{MSGEntidad:redaccionMSG}{Redacción del mensaje:} Se escribe desde el teclado.
			\item \cdtRef{MSGEntidad:paramtrizadoMSG}{Parametrizado:} Se escribe desde el teclado.
			\item Parámetros: Se escribe desde el teclado.
		\end{itemize}	
		}
		\UCitem{Salidas}{\begin{itemize}
				\item \cdtRef{proyectoEntidad:claveProyecto}{Clave del proyecto:} Lo obtiene el sistema.
				\item \cdtRef{proyectoEntidad:nombreProyecto}{Nombre del proyecto:} Lo obtiene el sistema.
				\item \cdtRef{MSGEntidad:claveMSG}{Clave:} Lo calcula el sistema mediante la regla de negocio \BRref{RN12}{Identifcador de elemento}.
				\item \cdtRef{MSGEntidad:nombreMSG}{Nombre del mensaje:} Lo obtiene el sistema.
				\item \cdtRef{MSGEntidad:descripcionMSG}{Descripción del mensaje:} Lo obtiene el sistema.
				\item \cdtRef{MSGEntidad:redaccionMSG}{Redacción del mensaje:} Lo obtiene el sistema.
				\item \cdtRef{MSGEntidad:paramtrizadoMSG}{Parametrizado:} Lo obtiene el sistema.
				\item Parámetros: Lo obtiene el sistema.
				\item \cdtIdRef{MSG1}{Operación exitosa}: Se muestra en la pantalla \IUref{IU10}{Gestionar Mensajes} para indicar que la modificación exitosa.
		\end{itemize}}
		\UCitem{Destino}{Pantalla}
		\UCitem{Precondiciones}{\begin{itemize}
				\item Que exista al menos un mensaje registrado.
				\item Que el mensaje no se encuentre asociado a un caso de uso en estado ''Liberado''.
		\end{itemize}}
		\UCitem{Postcondiciones}{Se actualizará un mensaje de un proyecto}
		\UCitem{Errores}{\begin{itemize}
		\item \cdtIdRef{MSG4}{Dato obligatorio}: Se muestra en la pantalla \IUref{IU10.2}{Modificar Mensaje} cuando no se ha ingresado un dato marcado como obligatorio.
		\item \cdtIdRef{MSG5}{Formato incorrecto}: Se muestra en la pantalla \IUref{IU10.2}{Modificar Mensaje} cuando el tipo de dato ingresado no cumple con el tipo de dato solicitado en el campo.
		\item \cdtIdRef{MSG6}{Longitud inválida}: Se muestra en la pantalla \IUref{IU10.2}{Modificar Mensaje} cuando se ha excedido la longitud de alguno de los campos.
		\item \cdtIdRef{MSG7}{Registro repetido}: Se muestra en la pantalla \IUref{IU10.2}{Modificar Mensaje} cuando se registre un mensaje con un nombre que ya se encuentre registrado en el sistema.
		\end{itemize}.
		}
		\UCitem{Tipo}{Secundario, extiende del caso de uso \UCref{CU9}{Gestionar Mensajes}.}
	\end{UseCase}
%--------------------------------------
	\begin{UCtrayectoria}
		\UCpaso[\UCactor] Da clic en el icono \editar de la pantalla \IUref{IU10}{Gestionar Mensajes}.
		\UCpaso[\UCsist] Obtiene la información del mensaje seleccionado.	
		\UCpaso[\UCsist] Verifica que el mensaje sea parametrizado. \hyperlink{CU9-2:TAA}{[Trayectoria A]}
		\UCpaso[\UCsist] Muestra la pantalla \IUref{IU10.2A}{Modificar Mensaje: Parametrizado}.
		\UCpaso[\UCsist] Marca como etiqueta el campo ''Clave'' de acuerdo a la regla de negocio \BRref{RN13}{Modificación del identificador}.
		\UCpaso[\UCactor] Modifica la información del mensaje. \label{CU9.2-P6}
		\UCpaso[\UCactor] Modifica la descripción de los parámetros.
		\UCpaso[\UCactor]Oprime el botón \IUbutton{Aceptar} de la pantalla \IUref{IU10.2}{Modificar Mensaje}. \label{CU9.2-P8} \hyperlink{CU9-2:TAB}{[Trayectoria B]}
		\UCpaso[\UCsist] Verifica que el actor ingrese todos los campos obligatorios con base en la regla de negocio \BRref{RN8}{Datos obligatorios}. \hyperlink{CU9-2:TAC}{[Trayectoria C]}
		\UCpaso[\UCsist] Verifica que los datos ingresados cumpla con la longitud correcta, con base en la regla de negocio \BRref{RN37}{Longitud de datos}. \hyperlink{CU9-2:TAD}{[Trayectoria D]}
		\UCpaso[\UCsist] Verifica que los datos ingresados cumplan con el formato requerido, con base en la regla de negocio \BRref{RN7}{Información correcta}. \hyperlink{CU9-2:TAE}{[Trayectoria E]}
		\UCpaso[\UCsist] Verifica que el número del mensaje sea proporcionado correctamente con base en la regla de negocio \BRref{RN7}{Información correcta}. \hyperlink{CU9-2:TAF}{[Trayectoria F]}
		\UCpaso[\UCsist] Verifica que el nombre del mensaje no se encuentre registrado en el sistema con base en la regla de negocio \BRref{RN6}{Unicidad de nombres}. \hyperlink{CU9-2:TAG}{[Trayectoria G]} 
		\UCpaso[\UCsist] Actualiza la información del mensaje en el sistema.
		\UCpaso[\UCsist] Muestra el mensaje \cdtIdRef{MSG1}{Operación exitosa} en la pantalla \IUref{IU10}{Gestionar Mensajes} para indicar al actor que el la modificación se ha realizado exitosamente.
	\end{UCtrayectoria}		
%--------------------------------------
\hypertarget{CU9-2:TAA}{\textbf{Trayectoria alternativa A}}\\
\noindent \textbf{Condición:} El mensaje no es parametrizado.
\begin{enumerate}
	\UCpaso[\UCsist] Muestra la pantalla \IUref{IU10.2}{Modificar Mensaje: No Parametrizado}.
	\UCpaso[\UCsist] Marca como etiqueta el campo ''Clave'' de acuerdo a la regla de negocio \BRref{RN13}{Modificación del identificador}.
	\UCpaso[\UCactor] Modifica la información del mensaje. \label{CU9.2-AP-2}
	\UCpaso[\UCsist] Continúa en el paso \ref{CU9.2-P8} de la trayectoria principal.
	\item[- -] - - {\em {Fin de la trayectoria}}.%
\end{enumerate}
%--------------------------------------
\hypertarget{CU9-2:TAB}{\textbf{Trayectoria alternativa B}}\\
\noindent \textbf{Condición:} El actor desea cancelar la operación.
\begin{enumerate}
	\UCpaso[\UCactor] Solicita cancelar la operación oprimiendo el botón \IUbutton{Cancelar} de la pantalla \IUref{IU10.2}{Modificar mensaje}.
	\UCpaso[\UCsist] Muestra la pantalla \IUref{IU10}{Gestionar Mensajes}.
	\item[- -] - - {\em {Fin del caso de uso}}.%
\end{enumerate}
%--------------------------------------
\hypertarget{CU9-2:TAC}{\textbf{Trayectoria alternativa C}}\\
\noindent \textbf{Condición:} El actor no ingresó algún dato marcado como obligatorio.
\begin{enumerate}
	\UCpaso[\UCsist] Muestra el mensaje \cdtIdRef{MSG4}{Dato obligatorio} señalando el campo que presenta el error en la pantalla \IUref{IU10.2}{Modificar mensaje}.
	\UCpaso Regresa al paso \ref{CU9.2-P6} de la trayectoria principal o al paso \ref{CU9.2-AP-2} de la trayectoria Alternativa A.
	\item[- -] - - {\em {Fin de la trayectoria}}.%
\end{enumerate}
%--------------------------------------
\hypertarget{CU9-2:TAD}{\textbf{Trayectoria alternativa D}}\\
\noindent \textbf{Condición:} El actor ingresó un dato con un número de caracteres fuera del rango permitido.
\begin{enumerate}
	\UCpaso[\UCsist] Muestra el mensaje \cdtIdRef{MSG6}{Longitud inválida} señalando el campo que presenta el error en la pantalla \IUref{IU10.2}{Modificar mensaje}.
	\UCpaso Regresa al paso \ref{CU9.2-P6} de la trayectoria principal o al paso \ref{CU9.2-AP-2} de la trayectoria Alternativa A.
	\item[- -] - - {\em {Fin de la trayectoria}}.%
\end{enumerate}
%--------------------------------------
\hypertarget{CU9-2:TAE}{\textbf{Trayectoria alternativa E}}\\
\noindent \textbf{Condición:} El actor ingresó un dato con un formato incorrecto.
\begin{enumerate}
	\UCpaso[\UCsist] Muestra el mensaje \cdtIdRef{MSG5}{Formato incorrecto} señalando el campo que presenta el error en la pantalla \IUref{IU10.2}{Modificar mensaje}.
	\UCpaso Regresa al paso \ref{CU9.2-P6} de la trayectoria principal o al paso \ref{CU9.2-AP-2} de la trayectoria Alternativa A.
	\item[- -] - - {\em {Fin de la trayectoria}}.
\end{enumerate}
%--------------------------------------
\hypertarget{CU9-2:TAF}{\textbf{Trayectoria alternativa F}}\\
\noindent \textbf{Condición:} El actor ingresó un número de mensaje repetido.
\begin{enumerate}
	\UCpaso[\UCsist] Muestra el mensaje \cdtIdRef{MSG7}{Registro repetido} señalando el campo que presenta la duplicidad en la pantalla \IUref{IU10.2}{Modificar mensaje}.
	\UCpaso Regresa al paso \ref{CU9.2-P6} de la trayectoria principal o al paso \ref{CU9.2-AP-2} de la trayectoria Alternativa A.
	\item[- -] - - {\em {Fin de la trayectoria}}.
\end{enumerate}
%--------------------------------------
\hypertarget{CU9-2:TAG}{\textbf{Trayectoria alternativa G}}\\
\noindent \textbf{Condición:} El actor ingresó un nombre de un mensaje repetido.
\begin{enumerate}
	\UCpaso[\UCsist] Muestra el mensaje \cdtIdRef{MSG7}{Registro repetido} señalando el campo que presenta la duplicidad en la pantalla \IUref{IU10.2}{Modificar mensaje}.
	\UCpaso Regresa al paso \ref{CU9.2-P6} de la trayectoria principal o al paso \ref{CU9.2-AP-2} de la trayectoria Alternativa A.
	\item[- -] - - {\em {Fin de la trayectoria}}.
\end{enumerate}