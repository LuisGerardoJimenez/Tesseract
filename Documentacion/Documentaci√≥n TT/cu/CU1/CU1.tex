	\begin{UseCase}{CU1}{Iniciar sesión}{
			
        En el inicio de sesión se controla el acceso a las distintas funciones del sistema de acuerdo con el rol asignado a un actor y mediante la entrada de datos como lo son el correo electrónico y contraseña. Tesseract define los siguientes roles con las siguientes características:
			
		 \begin{enumerate}
         \item\hyperlink{admin}{Administrador}: Actor único, encargado de  registrar los proyectos y al personal de la organización. 
         \item\hyperlink{jefe}{Líder de proyecto}: Encargado de dirigir y coordinar el proyecto en desarrollo. Asignado desde el registro del proyecto. 
         \item\hyperlink{analista}{Analista} Encargado de documentar los elementos que componen el proyecto y validar los requerimientos del software. Asignado desde el registro del proyecto. \\
         \end{enumerate}

	}
		\UCitem{Versión}{\color{Gray}0.1}
		\UCitem{Actor}{\hyperlink{jefe}{Líder de proyecto}, \hyperlink{analista}{Analista}, \hyperlink{admin}{Administrador}}
		\UCitem{Propósito}{Iniciar sesión en el sistema.}
		\UCitem{Entradas}{\begin{itemize}
				\item \cdtRef{colaboradorEntidad:correoColaborador}{Correo}: Se escribe desde el teclado
				\item \cdtRef{colaboradorEntidad:passColaborador}{Contraseña}: Se escribe desde el teclado
		\end{itemize}}
		\UCitem{Salidas}{Ninguna}
		\UCitem{Precondiciones}{Interna: Que el actor se encuentre registrado en el sistema.}
		\UCitem{Postcondiciones}{El actor podrá hacer uso del sistema dependiendo del rol que tenga.}
		\UCitem{Errores}{\begin{itemize}
		\item \cdtIdRef{MSG4}{Dato Obligatorio}: Se muestra en la pantalla \IUref{IU1}{Iniciar Sesión} cuando no se ha ingresado un dato marcado como obligatorio.
		\item \cdtIdRef{MSG6}{Longitud inválida}: Se muestra en la pantalla \IUref{IU1}{Iniciar Sesión} cuando se ha excedido la longitud de alguno de los campos.
		\item \cdtIdRef{MSG29}{Formato incorrecto}: Se muestra en la pantalla \IUref{IU1}{Iniciar Sesión} cuando el tipo de dato ingresado no cumple con el tipo de dato solicitado en el campo.
		\item \cdtIdRef{MSG23}{Correo electrónico y/o contraseña incorrectos}: Se muestra en la pantalla \IUref{IU1}{Iniciar Sesión} cuando el correo electrónico y/o la contraseña ingresada son incorrectos.
		\end{itemize}
		}
		\UCitem{Tipo}{Caso de uso primario}
	\end{UseCase}
%--------------------------------------
	\begin{UCtrayectoria}
		\UCpaso[\UCactor] Solicita ingresar al sistema a través de la URL correspondiente al proyecto.
		\UCpaso[\UCsist] Muestra la pantalla \IUref{IU1}{Iniciar Sesión}.
		\UCpaso[\UCactor] Ingresa los datos solicitados. \label{P3}
		\UCpaso[\UCactor] Oprime el botón \IUbutton{Aceptar}.
		\UCpaso[\UCsist] Verifica que no se haya omitido ningún campo marcado como obligatorio con base en la regla de negocio \BRref{RN8}{Datos Obligatorios}. \hyperlink{CU1:TAA}{[Trayectoria A]}
		\UCpaso[\UCsist] Verificar que los datos ingresados cumpla con la longitud correcta, con base en la regla de negocio \BRref{RN37}{Longitud de datos}. \hyperlink{CU1:TAB}{[Trayectoria B]}
		\UCpaso[\UCsist] Verifica que los datos cumplan con el formato y el tipo de dato requerido, con base en la regla de negocio \BRref{RN7}{Información correcta}. \hyperlink{CU1:TAC}{[Trayectoria C]}
		\UCpaso[\UCsist] Verifica que el actor se encuentre registrado en el sistema. \hyperlink{CU1:TAD}{[Trayectoria D]}
		\UCpaso[\UCsist] Verifica que la contraseña ingresada corresponda con la del usuario. \hyperlink{CU1:TAE}{[Trayectoria E]}
		\UCpaso[\UCsist] Obtiene todos los roles que tiene asociado el usuario con base a la regla de negocio \BRref{RN39}{Roles del sistema}.
		\UCpaso[\UCsist] Muestra la pantalla emergente \IUref{IU1.1}{Elegir Rol} con los roles asociados al usuario.
		\UCpaso[\UCactor] Selecciona el rol con el que iniciar sesión.
		\UCpaso[\UCactor] Presiona el botón aceptar.
		\UCpaso[\UCsist] Inicia sesión con el rol seleccionado.
		\UCpaso[\UCsist] Muestra la pantalla del rol correspondiente.
	\end{UCtrayectoria}		
%--------------------------------------
		\hypertarget{CU1:TAA}{\textbf{Trayectoria alternativa A}}\\
		\noindent \textbf{Condición:} El actor no ingresó uno o más campos obligatorios.
		\begin{enumerate}
			\UCpaso[\UCsist] Muestra el mensaje \cdtIdRef{MSG4}{Dato Obligatorio} y señala el campo que presenta el error en la pantalla \IUref{IU1}{Iniciar Sesión}.
			\UCpaso[\UCactor] Regresa al paso \ref{P3} de la Trayectoria Principal.
			\item[- -] - - {\em {Fin de la trayectoria}}.%
		\end{enumerate}

%--------------------------------------

		\hypertarget{CU1:TAB}{\textbf{Trayectoria alternativa B}}\\
		\noindent \textbf{Condición:} El actor ingresó un dato con un número de caracteres fuera del rango permitido.
		\begin{enumerate}
	\UCpaso[\UCsist] Muestra el \cdtIdRef{MSG6}{Longitud inválida} señalando el campo que presenta el error en la pantalla \IUref{IU1}{Iniciar Sesión}.
	\UCpaso[\UCactor] Regresa al paso \ref{P3} de la Trayectoria Principal.
	\item[- -] - - {\em {Fin de la trayectoria}}.
\end{enumerate}
%--------------------------------------
		\hypertarget{CU1:TAC}{\textbf{Trayectoria alternativa C}}\\
		\noindent \textbf{Condición:} El actor ingresó un dato con un formato o tipo de dato incorrecto.
	\begin{enumerate}
		\UCpaso[\UCsist] Muestra el mensaje \cdtIdRef{MSG29}{Formato incorrecto} señalando el campo que presenta el error en la pantalla \IUref{IU1}{Iniciar sesión}.
		\UCpaso Regresa al paso \ref{P3} de la trayectoria principal.
		\item[- -] - - {\em {Fin de la trayectoria}}.
	\end{enumerate}

%--------------------------------------
		\hypertarget{CU1:TAD}{\textbf{Trayectoria alternativa D}}\\
		\noindent \textbf{Condición:} El actor un correo que no se encuentra registrado en el sistema.
		\begin{enumerate}
	\UCpaso[\UCsist] Muestra el mensaje \cdtIdRef{MSG23}{Correo electrónico y/o contraseña incorrectos} en la pantalla \IUref{IU1}{Iniciar Sesión} notificando que los datos ingresados son incorrectos.
	\UCpaso[\UCactor] Regresa al paso \ref{P3} de la Trayectoria Principal.
	\item[- -] - - {\em {Fin de la trayectoria}}.
\end{enumerate}

%--------------------------------------
\hypertarget{CU1:TAE}{\textbf{Trayectoria alternativa E}}\\
\noindent \textbf{Condición:} El actor ingresó una contraseña que no corresponde al correo ingresado.
\begin{enumerate}
	\UCpaso[\UCsist] Muestra el mensaje \cdtIdRef{MSG23}{Correo electrónico y/o contraseña incorrectos} en la pantalla \IUref{IU1}{Iniciar Sesión} notificando que los datos ingresados son incorrectos.
	\UCpaso[\UCactor] Regresa al paso \ref{P3} de la Trayectoria Principal.
	\item[- -] - - {\em {Fin de la trayectoria}}.
\end{enumerate}
