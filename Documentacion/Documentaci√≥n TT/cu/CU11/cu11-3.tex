	\begin{UseCase}{CU11.3}{Eliminar Pantalla}{
		Cuando el registro de alguna \hyperlink{pantalla}{Pantalla} ya no tiene una razón de ser dentro del hyperlink{moduloEntidad}{Módulo}, Tesseract permitirá al colaborador (\hyperlink{jefe}{Líder} o \hyperlink{analista}{Analista}) eliminar en su totalidad el registro de la pantalla. \\
		
		Una pantalla podrá ser eliminada siempre y cuando no se encuentre asociada a algún \hyperlink{casoUso}{Caso de Uso}.
	}
		\UCitem{Actor}{\hyperlink{jefe}{Líder de Análisis}, \hyperlink{analista}{Analista}}
		\UCitem{Propósito}{Eliminar la información de una pantalla.}
		\UCitem{Entradas}{Ninguna}
		\UCitem{Salidas}{\begin{itemize}
				\item \cdtIdRef{MSG1}{Operación exitosa}: Se muestra en la pantalla \IUref{IU7}{Gestionar Pantallas} para indicar que la pantalla fue eliminada correctamente.
				\item \cdtIdRef{MSG10}{Confirmar eliminación}: Se muestra en la pantalla \IUref{IU7}{Gestionar Pantallas} preguntando al actor si desea continuar con la eliminación de la pantalla.
		\end{itemize}}
		\UCitem{Precondiciones}{\begin{itemize}
				\item Que la pantalla no se encuentre asociada a un caso de uso.
				\item Que la pantalla no se encuentre asociada a un caso de uso liberado
		\end{itemize}}
		\UCitem{Postcondiciones}{
		\begin{itemize}
			\item Se eliminará una pantalla de un módulo perteneciente a un proyecto del sistema.
		\end{itemize}
		}
		\UCitem{Errores}{\begin{itemize}
		\item \cdtIdRef{MSG13}{Eliminación no permitida}: Se muestra en la pantalla \IUref{IU7}{Gestionar Pantallas} cuando la pantalla está siendo referenciado en algún caso de uso.
		\end{itemize}
		}
		\UCitem{Tipo}{Secundario, extiende del caso de uso \UCref{CU11}{Gestionar Pantallas}.}
	\end{UseCase}
%--------------------------------------
	\begin{UCtrayectoria}
		\UCpaso[\UCactor] Da clic en el icono \eliminar del registro que desea eliminar de la pantalla \IUref{IU7}{Gestionar Pantallas}.
		\UCpaso[\UCsist] Muestra el mensaje emergente \cdtIdRef{MSG10}{Confirmar eliminación} con los botones \IUbutton{Aceptar} y \IUbutton{Cancelar} en la pantalla \IUref{IU7}{Gestionar Pantallas}.
		\UCpaso[\UCactor] Confirma la eliminación de la pantalla oprimiendo el botón \IUbutton{Aceptar}. \hyperlink{CU11-3:TAA}{[Trayectoria A]}
		\UCpaso[\UCsist] Verifica que ningún caso de uso se encuentre asociado a la pantalla. \hyperlink{CU11-3:TAB}{[Trayectoria B]}
		\UCpaso[\UCsist] Verifica que ningún caso de uso se encuentre asociado a alguna acción de la pantalla. \hyperlink{CU11-3:TAC}{[Trayectoria C]}
		\UCpaso[\UCsist] Elimina la información referente a la pantalla.
		\UCpaso[\UCsist] Muestra el mensaje \cdtIdRef{MSG1}{Operación exitosa} en la pantalla \IUref{IU7}{Gestionar Pantallas} para indicar al actor que el registro se ha eliminado exitosamente.
	\end{UCtrayectoria}		
%--------------------------------------
\hypertarget{CU11-3:TAA}{\textbf{Trayectoria alternativa A}}\\
\noindent \textbf{Condición:} El actor desea cancelar la operación.
\begin{enumerate}
	\UCpaso[\UCactor] Oprime el botón \IUbutton{Cancelar} de la pantalla emergente.
	\UCpaso[\UCsist] Muestra la pantalla \IUref{IU7}{Gestionar Pantallas}.
	\item[- -] - - {\em {Fin del caso de uso}}.%
\end{enumerate}		
%--------------------------------------
\hypertarget{CU11-3:TAB}{\textbf{Trayectoria alternativa B}}\\
\noindent \textbf{Condición:} La pantalla está siendo referenciado en un caso de uso.
\begin{enumerate}
	\UCpaso[\UCsist] Muestra el mensaje \cdtIdRef{MSG13}{Eliminación no permitida} en la pantalla \IUref{IU7}{Gestionar Pantallas}.
	\item[- -] - - {\em {Fin del caso de uso}}.
\end{enumerate}
%--------------------------------------
\hypertarget{CU11-3:TAC}{\textbf{Trayectoria alternativa C}}\\
\noindent \textbf{Condición:} Algunas acciones de la pantalla están siendo referenciadas en algún caso de uso.
\begin{enumerate}
	\UCpaso[\UCsist] Muestra el mensaje \cdtIdRef{MSG13}{Eliminación no permitida} en la pantalla \IUref{IU7}{Gestionar Pantallas}.
	\item[- -] - - {\em {Fin del caso de uso}}.
\end{enumerate}
	

