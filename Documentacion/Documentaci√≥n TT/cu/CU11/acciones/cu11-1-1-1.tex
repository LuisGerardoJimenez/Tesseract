	\begin{UseCase}{CU11.1.1.1}{Registrar Acción}{
			
			Este caso de uso permite al colaborador (\hyperlink{jefe}{Líder de Análisis} o \hyperlink{analista}{Analista}) registrar alguna \hyperlink{EntidadAccion}{Acción} de la \hyperlink{pantalla}{Pantalla} sobre la cual se está operando.
			
			A través de un formulario el sistema solicita la información general de la acción que se desea registrar, así como su ícono, tipo de acción y pantalla destino.\\
			Los datos se almacenan con el propósito de conocer cuales son las operaciones que se pueden solicitar desde la pantalla, regularmente son botones, ligas u opciones de un menú. 
			
	}
	\UCitem{Actor}{\hyperlink{jefe}{Líder de Análisis}, \hyperlink{analista}{Analista}}
	\UCitem{Propósito}{Registrar la información de una acción perteneciente a una pantalla.}
	\UCitem{Entradas}{
		\begin{itemize}
			\item \cdtRef{EntidadAccion:nombreACC}{Nombre del acción}: Se escribe desde el teclado.
			\item \cdtRef{EntidadAccion:descripcionACC}{Descripción de la acción}: Se escribe desde el teclado.
			\item \cdtRef{EntidadAccion:tipoACC}{Tipo de acción}: Se selecciona de una lista.
			\item \cdtRef{EntidadAccion:destinoACC}{Pantalla destino}: Se selecciona de una lista.
			\item \cdtRef{EntidadAccion:imagenACC}{Imagen de la acción}: Se selecciona de los archivos locales.
		\end{itemize}	
	}
	\UCitem{Salidas}{
		\begin{itemize}
			\item \cdtRef{proyectoEntidad:claveProyecto}{Clave del proyecto}: Lo obtiene el sistema.
			\item \cdtRef{proyectoEntidad:nombreProyecto}{Nombre del proyecto}: Lo obtiene el sistema.
			\item \cdtRef{EntidadPantalla:claveIU}{Clave de la pantalla:} Lo obtiene el sistema.
			\item \cdtRef{EntidadPantalla:nombreIU}{Nombre de la pantalla:} Lo obtiene el sistema.
			\item \cdtIdRef{MSG1}{Operación exitosa}: Se muestra en la pantalla \IUref{IU7.1.1.1}{Gestionar Acciones} para indicar que el registro fue exitoso.
		\end{itemize}
		}
	\UCitem{Precondiciones}{\begin{itemize}
			\item Que exista al menos una pantalla registrada.
			\item Que el catálogo ''Tipo de acción'' contenga información. 
	\end{itemize}}
	\UCitem{Postcondiciones}{
		\begin{itemize}
			\item Se registrará una acción perteneciente a una pantalla de un proyecto.
			\item La acción podrá ser referenciada en casos de uso.
		\end{itemize}
	}
	\UCitem{Errores}{\begin{itemize}
			\item \cdtIdRef{MSG4}{Dato obligatorio}: Se muestra en la pantalla \IUref{IU7.1.1}{Registrar Acción} cuando no se ha ingresado un dato marcado como obligatorio.
			\item \cdtIdRef{MSG30}{Imagen obligatorio}: Se muestra en la pantalla \IUref{IU7.1.1}{Registrar Acción} cuando no se ha ingresado un campo de tipo archivo como obligatorio.
			\item \cdtIdRef{MSG5}{Formato incorrecto}: Se muestra en la pantalla \IUref{IU7.1.1}{Registrar Acción} cuando el tipo de dato ingresado no cumple con el tipo de dato solicitado en el campo.
			\item \cdtIdRef{MSG6}{Longitud inválida}: Se muestra en la pantalla \IUref{IU7.1.1}{Registrar Acción} cuando se ha excedido la longitud de alguno de los campos.
			\item \cdtIdRef{MSG7}{Registro repetido}: Se muestra en la pantalla \IUref{IU7.1.1}{Registrar Acción} cuando se registre un actor con un nombre que ya se encuentre registrado en el sistema.
			\item \cdtIdRef{MSG16}{Formato de archivo incorrecto}: Se muestra en la pantalla \IUref{IU7.1.1}{Registrar Acción} cuando la imagen de la acción no cumpla con el formato especificado.
			\item \cdtIdRef{MSG17}{Se ha excedido el tamaño del archivo}: Se muestra en la pantalla \IUref{IU7.1.1}{Registrar Acción} cuando la imagen de la acción exceda el tamaño especificado.
			\item \cdtIdRef{MSG29}{Los catálogos nos contienen información}: Se muestra en la pantalla \IUref{IU7.1.1.1}{Gestionar Acciones} cuando no exista información en el catálogo ''Tipo de acción''.
		\end{itemize}.
	}
	\UCitem{Tipo}{Secundario, extiende del caso de uso \UCref{CU11.1.1}{Gestionar Acciones}.}
\end{UseCase}
%--------------------------------------
\begin{UCtrayectoria}
	\UCpaso[\UCactor] Solicita registrar una acción oprimiendo el botón \IUbutton{Registrar} de la pantalla \IUref{IU7.1.1.1}{Gestionar Acciones}.
	\UCpaso[\UCsist] Verifica que el catálogo ''Tipo de acción'' cuente con información, con base en la regla de negocio \BRref{RN20}{Verificación de catálogos}. \hyperlink{CU11-1-1-1:TAA}{[Trayectoria A]}
	\UCpaso[\UCsist] Muestra la pantalla \IUref{IU7.1.1}{Registrar Acción}.
	\UCpaso[\UCactor] Ingresa la información solicitada. \label{CU11.1.1-P3}
	\UCpaso[\UCactor] Oprime el botón \IUbutton{Aceptar}. \hyperlink{CU11-1-1-1:TAB}{[Trayectoria B]} 
	\UCpaso[\UCsist] Verifica que el actor ingrese todos los campos obligatorios con base en la regla de negocio \BRref{RN8}{Datos obligatorios}. \hyperlink{CU11-1-1-1:TAC}{[Trayectoria C]} \hyperlink{CU11-1-1-1:TAD}{[Trayectoria D]}
	\UCpaso[\UCsist] Verificar que los datos ingresados cumpla con la longitud correcta, con base en la regla de negocio \BRref{RN37}{Longitud de datos}. \hyperlink{CU11-1-1-1:TAE}{[Trayectoria E]}
	\UCpaso[\UCsist] Verifica que los datos ingresados cumplan con el formato requerido, con base en la regla de negocio \BRref{RN7}{Información correcta}. \hyperlink{CU11-1-1-1:TAF}{[Trayectoria F]}\hyperlink{CU11-1-1-1:TAG}{[Trayectoria G]}
	\UCpaso[\UCsist] Verifica que el archivo ingresado cumpla con el peso correcto, con base en la regla de negocio \BRref{RN40}{Peso de archivos de imagen}. \hyperlink{CU11-1-1-1:TAH}{[Trayectoria H]}
	\UCpaso[\UCsist] Verifica que el nombre de la acción no se encuentre asociado a la pantalla con base en la regla de negocio \BRref{RN6}{Unicidad de nombres}. \hyperlink{CU11-1-1-1:TAI}{[Trayectoria I]}
	\UCpaso[\UCsist] Persiste la información de la acción en el sistema.
	\UCpaso[\UCsist] Muestra el mensaje \cdtIdRef{MSG1}{Operación exitosa} en la pantalla \IUref{IU7.1.1.1}{Gestionar Acciones} para indicar al actor que el registro se ha realizado exitosamente.
\end{UCtrayectoria}		

%--------------------------------------
\hypertarget{CU11-1-1-1:TAA}{\textbf{Trayectoria alternativa A}}\\
\noindent \textbf{Condición:} El catálogo de ''Tipo de acción'' no tiene información.
\begin{enumerate}
	\UCpaso[\UCsist] Muestra el mensaje \cdtIdRef{MSG29}{Los catálogos nos contienen información} en la pantalla \IUref{IU7.1.1.1}{Gestionar Acciones} para indicar que no es posible realizar la operación debido a la falta de información necesaria para el sistema.
	\item[- -] - - {\em {Fin del caso de uso}}.%
\end{enumerate}
%--------------------------------------
\hypertarget{CU11-1-1-1:TAB}{\textbf{Trayectoria alternativa B}}\\
\noindent \textbf{Condición:} El actor desea cancelar la operación.
\begin{enumerate}
	\UCpaso[\UCactor] Solicita cancelar la operación oprimiendo el botón \IUbutton{Cancelar} de la pantalla \IUref{IU7.1.1}{Registrar Acción}.
	\UCpaso[\UCsist] Muestra la pantalla \IUref{IU7.1.1.1}{Gestionar Acciones}.
	\item[- -] - - {\em {Fin del caso de uso}}.%
\end{enumerate}
%--------------------------------------
\hypertarget{CU11-1-1-1:TAC}{\textbf{Trayectoria alternativa C}}\\
\noindent \textbf{Condición:} El actor no ingresó algún dato marcado como obligatorio.
\begin{enumerate}
	\UCpaso[\UCsist] Muestra el mensaje \cdtIdRef{MSG4}{Dato obligatorio} señalando el campo que presenta el error en la pantalla \IUref{IU7.1.1}{Registrar Acción}.
	\UCpaso Regresa al paso \ref{CU11.1.1-P3} de la trayectoria principal.
	\item[- -] - - {\em {Fin de la trayectoria}}.%
\end{enumerate}
%--------------------------------------
\hypertarget{CU11-1-1-1:TAD}{\textbf{Trayectoria alternativa D}}\\
\noindent \textbf{Condición:} El actor no ingresó un dato de tipo imagen marcado como obligatorio.
\begin{enumerate}
	\UCpaso[\UCsist] Muestra el mensaje \cdtIdRef{MSG30}{Imagen obligatorio} señalando el campo que presenta el error en la pantalla \IUref{IU7.1.1}{Registrar Acción}.
	\UCpaso Regresa al paso \ref{CU11.1-P3} de la trayectoria principal.
	\item[- -] - - {\em {Fin de la trayectoria}}.%
\end{enumerate}
%--------------------------------------
\hypertarget{CU11-1-1-1:TAE}{\textbf{Trayectoria alternativa E}}\\
\noindent \textbf{Condición:} El actor ingresó un dato con un número de caracteres fuera del rango permitido.
\begin{enumerate}
	\UCpaso[\UCsist] Muestra el mensaje \cdtIdRef{MSG6}{Longitud inválida} señalando el campo que presenta el error en la pantalla \IUref{IU7.1.1}{Registrar Acción}.
	\UCpaso Regresa al paso \ref{CU11.1.1-P3} de la trayectoria principal.
	\item[- -] - - {\em {Fin de la trayectoria}}.%
\end{enumerate}
%--------------------------------------
\hypertarget{CU11-1-1-1:TAF}{\textbf{Trayectoria alternativa F}}\\
\noindent \textbf{Condición:} El actor ingresó un dato con un formato incorrecto.
\begin{enumerate}
	\UCpaso[\UCsist] Muestra el mensaje \cdtIdRef{MSG5}{Formato incorrecto} señalando el campo que presenta el error en la pantalla \IUref{IU7.1.1}{Registrar Acción}.
	\UCpaso Regresa al paso \ref{CU11.1.1-P3} de la trayectoria principal.
	\item[- -] - - {\em {Fin de la trayectoria}}.
\end{enumerate}
%--------------------------------------
\hypertarget{CU11-1-1-1:TAG}{\textbf{Trayectoria alternativa G}}\\
\noindent \textbf{Condición:} El actor proporciona una imagen de formato incorrecto.
\begin{enumerate}
	\UCpaso[\UCsist] Muestra el mensaje \cdtIdRef{MSG16}{Formato de archivo incorrecto} señalando el campo que presenta el error en la pantalla \IUref{IU7.1.1}{Registrar Acción}.
	\UCpaso Regresa al paso \ref{CU11.1.1-P3} de la trayectoria principal.
	\item[- -] - - {\em {Fin de la trayectoria}}.
\end{enumerate}
%--------------------------------------
\hypertarget{CU11-1-1-1:TAH}{\textbf{Trayectoria alternativa H}}\\
\noindent \textbf{Condición:} El actor proporciona una imagen que excede el tamaño máximo.
\begin{enumerate}
	\UCpaso[\UCsist] Muestra el mensaje \cdtIdRef{MSG17}{Se ha excedido el tamaño del archivo} señalando el campo que presenta el error en la pantalla \IUref{IU7.1.1}{Registrar Acción}.
	\UCpaso Regresa al paso \ref{CU11.1.1-P3} de la trayectoria principal.
	\item[- -] - - {\em {Fin de la trayectoria}}.
\end{enumerate}
%--------------------------------------
\hypertarget{CU11-1-1-1:TAI}{\textbf{Trayectoria alternativa I}}\\
\noindent \textbf{Condición:} El actor ingresó un nombre de acción que ya está asociado a la pantalla.
\begin{enumerate}
	\UCpaso[\UCsist] Muestra el mensaje \cdtIdRef{MSG7}{Registro repetido} señalando el campo que presenta la duplicidad en la pantalla \IUref{IU7.1.1}{Registrar Acción}.
	\UCpaso Regresa al paso \ref{CU11.1.1-P3} de la trayectoria principal.
	\item[- -] - - {\em {Fin de la trayectoria}}.
\end{enumerate}