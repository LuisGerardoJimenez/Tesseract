	\begin{UseCase}{CU11.1.2}{Modificar Acción}{
		Una acción es una operación que se puede solicitar desde una pantalla, regularmente son botones u opciones de un menú. Este caso de uso permite al analista modificar la información de una acción.
	}
	\UCitem{Versión}{\color{Gray}0.1}
	\UCitem{Actor}{\hyperlink{jefe}{Líder de Análisis}, \hyperlink{analista}{Analista}}
	\UCitem{Propósito}{Modificar la información de una pantalla.}
	\UCitem{Entradas}{
		\begin{itemize}
			\item \cdtRef{accion:nombreACC}{Nombre}: Se escribe desde el teclado.
			\item \cdtRef{accion:descripcionACC}{Descripción}: Se escribe desde el teclado.
			\item \cdtRef{accion:imagenACC}{Imagen}: Se selecciona de los archivos locales.
		\end{itemize}	
	}
	\UCitem{Salidas}{
		\begin{itemize}
			\item \cdtRef{accion:nombreACC}{Nombre}: Lo obtiene el sistema.
			\item \cdtRef{accion:descripcionACC}{Descripción}: Lo obtiene el sistema.
			\item \cdtRef{accion:imagenACC}{Imagen}: Lo obtiene el sistema.
		\end{itemize}
	}
	\UCitem{Destino}{Pantalla}
	\UCitem{Precondiciones}{Ninguna}
	\UCitem{Postcondiciones}{Ninguna}
	\UCitem{Errores}{\begin{itemize}
			\item \cdtIdRef{MSG4}{Dato obligatorio}: Se muestra en la pantalla \IUref{IU7.1.2}{Modificar Acción} cuando no se ha ingresado un dato marcado como obligatorio.
			\item \cdtIdRef{MSG29}{Formato incorrecto}: Se muestra en la pantalla \IUref{IU7.1.2}{Modificar Acción} cuando el tipo de dato ingresado no cumple con el tipo de dato solicitado en el campo.
			\item \cdtIdRef{MSG6}{Longitud inválida}: Se muestra en la pantalla \IUref{IU7.1.2}{Modificar Acción} cuando se ha excedido la longitud de alguno de los campos.
			\item \cdtIdRef{MSG7}{Registro repetido}: Se muestra en la pantalla \IUref{IU7.1.2}{Modificar Acción} cuando se registre un actor con un nombre que ya se encuentre registrado en el sistema.
			\item \cdtIdRef{MSG20}{Formato de archivo incorrecto}: Se muestra en la pantalla \IUref{IU7.1.2}{Modificar Acción} cuando la imagen de la acción no cumpla con el formato especificado.
			\item \cdtIdRef{MSG21}{Se ha excedido el tamaño del archivo}: Se muestra en la pantalla \IUref{IU7.1.2}{Modificar Acción} cuando la imagen de la acción exceda el tamaño especificado.
		\end{itemize}.
	}
	\UCitem{Tipo}{Secundario, extiende del los casos de uso \UCref{CU11.1}{Registrar Pantalla} y \UCref{CU11.2}{Modificar Pantalla}.}
\end{UseCase}
%--------------------------------------
\begin{UCtrayectoria}
	\UCpaso[\UCactor] Solicita modificar una acción oprimiendo el botón \editar de la acción que desea modificar en la pantalla \IUref{IU7.1}{Registrar Pantalla} o \IUref{IU7.2}{Modificar Pantalla}.
	\UCpaso[\UCsist] Obtiene la información de la acción seleccionada.
	\UCpaso[\UCsist] Muestra la información encontrada en pantalla \IUref{IU7.1.2}{Modificar Acción}.
	\UCpaso[\UCactor] Ingresa la información solicitada. \label{CU11.1.2-P4}
	\UCpaso[\UCactor] Solicita guardar la información del actor oprimiendo el botón \IUbutton{Aceptar} de la pantalla \IUref{IU7.1.2}{Modificar Acción}. \Trayref{MACC-A} 
	\UCpaso[\UCsist] Verifica que el actor ingrese todos los campos obligatorios con base en la regla de negocio \BRref{RN8}{Datos obligatorios}. \Trayref{MACC-B}
	\UCpaso[\UCsist] Verifica que los datos requeridos sean proporcionados correctamente con base en la regla de negocio \BRref{RN7}{Información correcta}. \Trayref{MACC-C} \Trayref{MACC-D} \Trayref{MACC-E} \Trayref{MACC-F}
	\UCpaso[\UCsist] Verifica que el nombre de la acción no se encuentre asociado a la pantalla con base en la regla de negocio \BRref{RN6}{Unicidad de nombres}. \Trayref{MACC-G} 
	\UCpaso[\UCsist] Agrega la acción a la tabla de la pantalla \IUref{IU7.1}{Registrar Pantalla} o \IUref{IU7.2}{Modificar Pantalla}.
\end{UCtrayectoria}		
%--------------------------------------

\begin{UCtrayectoriaA}{MACC-A}{El actor desea cancelar la operación.}
	\UCpaso[\UCactor] Solicita cancelar la operación oprimiendo el botón \IUbutton{Cancelar} de la pantalla \IUref{IU7.1.2}{Modificar Acción}.
	\UCpaso[\UCsist] Muestra la pantalla \IUref{IU7.1}{Registrar Pantalla} o \IUref{IU7.2}{Modificar Pantalla}.
\end{UCtrayectoriaA}

\begin{UCtrayectoriaA}{MACC-B}{El actor no ingresó algún dato marcado como obligatorio.}
	\UCpaso[\UCsist] Muestra el mensaje \cdtIdRef{MSG4}{Dato obligatorio} y señala el campo que presenta el error en la pantalla \IUref{IU7.1.2}{Modificar Acción}, indicando al actor que el dato es obligatorio.
	\UCpaso Regresa al paso \ref{CU11.1.2-P4} de la trayectoria principal.
\end{UCtrayectoriaA}

\begin{UCtrayectoriaA}{MACC-C}{El actor proporciona un dato que excede la longitud máxima.}
	\UCpaso[\UCsist] Muestra el mensaje \cdtIdRef{MSG6}{Longitud inválida} y señala el campo que excede la longitud en la pantalla \IUref{IU7.1.2}{Modificar Acción}, para indicar que el dato excede el tamaño máximo permitido.
	\UCpaso Regresa al paso \ref{CU11.1.2-P4} de la trayectoria principal.
\end{UCtrayectoriaA}

\begin{UCtrayectoriaA}{MACC-D}{El actor ingresó un tipo de dato incorrecto.}
	\UCpaso[\UCsist] Muestra el mensaje \cdtIdRef{MSG29}{Formato incorrecto} y señala el campo que presenta el dato inválido en la pantalla \IUref{IU7.1.2}{Modificar Acción}, para indicar que se ha ingresado un tipo de dato inválido.
	\UCpaso Regresa al paso \ref{CU11.1.2-P4} de la trayectoria principal.
\end{UCtrayectoriaA}

\begin{UCtrayectoriaA}{MACC-E}{El actor proporciona una imagen de formato incorrecto.}
	\UCpaso[\UCsist] Muestra el mensaje \cdtIdRef{MSG20}{Formato de archivo incorrecto} y señala el campo donde se solicita la imagen en la pantalla \IUref{IU7.1.2}{Modificar Acción}, para indicar que el formato es incorrecto.
	\UCpaso Regresa al paso \ref{CU11.1.2-P4} de la trayectoria principal.
\end{UCtrayectoriaA}

\begin{UCtrayectoriaA}{MACC-F}{El actor proporciona una imagen que excede el tamaño máximo.}
	\UCpaso[\UCsist] Muestra el mensaje \cdtIdRef{MSG21}{Se ha excedido el tamaño del archivo} y señala el campo donde se solicita la imagen en la pantalla \IUref{IU7.1.2}{Modificar Acción}, para indicar que el tamaño máximo ha sido rebasado.
	\UCpaso Regresa al paso \ref{CU11.1.2-P4} de la trayectoria principal.
\end{UCtrayectoriaA}

\begin{UCtrayectoriaA}{MACC-G}{El actor ingresó un nombre de acción que ya está asociado a la pantalla.}
	\UCpaso[\UCsist] Muestra el mensaje \cdtIdRef{MSG7}{Registro repetido} y señala el campo que presenta la duplicidad en la pantalla \IUref{IU7.1.2}{Modificar Acción}, indicando al actor que existe una acción con el mismo nombre.
	\UCpaso Regresa al paso \ref{CU11.1.2-P4} de la trayectoria principal.
\end{UCtrayectoriaA}