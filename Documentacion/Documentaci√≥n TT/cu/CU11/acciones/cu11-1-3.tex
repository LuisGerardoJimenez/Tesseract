	\begin{UseCase}{CU11.1.3}{Eliminar Acción}{
			
				Cuando el registro de alguna \hyperlink{EntidadAccion}{Acción} ya no tiene una razón de ser dentro de la \hyperlink{pantalla}{Pantalla}, Tesseract permitirá al colaborador (\hyperlink{jefe}{Líder} o \hyperlink{analista}{Analista}) eliminar en su totalidad el registro de la acción. \\

}
		\UCitem{Actor}{\hyperlink{jefe}{Líder de Análisis}, \hyperlink{analista}{Analista}}
		\UCitem{Propósito}{Eliminar la información de una acción de la tabla de acciones.}
		\UCitem{Entradas}{Ninguna}
		\UCitem{Salidas}{Ninguna}
		\UCitem{Precondiciones}{\begin{itemize}
				\item Que la acción no se encuentre asociada a un caso de uso.
				\item Que la acción no se encuentre asociada a un caso de uso liberado.
		\end{itemize}}
		\UCitem{Postcondiciones}{
		\begin{itemize}
			\item Se eliminará una acción de una pantalla perteneciente a un proyecto.
		\end{itemize}
		}
		\UCitem{Errores}{\begin{itemize}
		\item \cdtIdRef{MSG13}{Eliminación no permitida}: Se muestra en la pantalla \IUref{IU7.1}{Registrar Pantalla} o \IUref{IU7.2}{Modificar Pantalla} cuando no se pueda eleiminar la acción debido a que está siendo referenciada en algún caso de uso.
		\end{itemize}
		}
		\UCitem{Tipo}{Secundario, extiende del los casos de uso \UCref{CU11.1}{Registrar Pantalla} y \UCref{CU11.2}{Modificar Pantalla}.}
	\end{UseCase}
%--------------------------------------
	\begin{UCtrayectoria}
		\UCpaso[\UCactor] Da clic en el icono \eliminar del registro que desea eliminar de la pantalla \IUref{IU7.1}{Registrar Pantalla} o \IUref{IU7.2}{Modificar Pantalla}.
		\UCpaso[\UCsist] Verifica que la acción pueda eliminarse de acuerdo a la regla de negocio \BRref{RN18}{Eliminación de elementos}. \hyperlink{CU11-1-3:TAA}{[Trayectoria A]}
		\UCpaso[\UCsist] Verifica que ningún elemento esté referenciando a la acción. \hyperlink{CU11-1-3:TAB}{[Trayectoria B]}
		\UCpaso[\UCsist] Elimina la acción del sistema y actualiza la tabla \item \cdtRef{EntidadAccion}{Acciones} de la pantalla \IUref{IU7.1}{Registrar Pantalla} o \IUref{IU7.2}{Modificar Pantalla}.
	\end{UCtrayectoria}		
%--------------------------------------
\hypertarget{CU11-1-3:TAA}{\textbf{Trayectoria alternativa A}}\\
\noindent \textbf{Condición:} La acción se encuentra asociado a casos de uso liberados.
\begin{enumerate}
	\UCpaso[\UCsist] Oculta el botón \eliminar de la acción que esta asociado a casos de uso liberados.
	\item[- -] - - {\em {Fin del caso de uso}}.
\end{enumerate}
%--------------------------------------
\hypertarget{CU11-1-3:TAB}{\textbf{Trayectoria alternativa B}}\\
\noindent \textbf{Condición:} La acción está siendo referenciado en un elemento.
\begin{enumerate}
	\UCpaso[\UCsist] Muestra la pantalla \IUref{IU7.1}{Registrar Pantalla} o \IUref{IU7.2}{Modificar Pantalla} con el mensaje \cdtIdRef{MSG13}{Eliminación no permitida} mostrando una lista de elementos que están referenciando a la acción.
	\item[- -] - - {\em {Fin del caso de uso}}.%
\end{enumerate}
	

