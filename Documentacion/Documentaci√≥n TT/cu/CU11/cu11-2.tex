	\begin{UseCase}{CU11.2}{Modificar Pantalla}{
		Después de haber registrado una \hyperlink{pantalla}{Pantalla} en un \hyperlink{moduloEntidad}{Módulo} del \hyperlink{proyectoEntidad}{Proyecto} y el colaborador (\hyperlink{jefe}{Líder de Análisis} o \hyperlink{analista}{Analista}) requiera modificar su información, el sistema le permitirá editar cualquiera de los datos previamente registrados mediante un formulario, este formulario contendrá los datos precargados de la última actualización para poder modificarlos y posteriormente guardarlos.\\

		La única condición para modificar una pantalla exitosamente es que esta no se encuentre asociada a algún \hyperlink{casoUso}{Caso de Uso}.
	}
		
		\UCitem{Actor}{\hyperlink{jefe}{Líder de Análisis}, \hyperlink{analista}{Analista}}
		\UCitem{Propósito}{Registrar la información de una pantalla.}
		\UCitem{Entradas}{
		\begin{itemize}
			\item \cdtRef{pantalla:numeroIU}{Número de la pantalla}: Se escribe desde el teclado.
			\item \cdtRef{pantalla:nombreIU}{Nombre de la pantalla}: Se escribe desde el teclado.
			\item \cdtRef{pantalla:descripcionIU}{Descripción de la pantalla}: Se escribe desde el teclado.
			\item \cdtRef{pantalla:imagenIU}{Imagen de la pantalla}: Se selecciona de los archivos locales.
		\end{itemize}	
		}
		\UCitem{Salidas}{\begin{itemize}
				\item \cdtRef{proyectoEntidad:claveProyecto}{Clave del proyecto:} Lo obtiene el sistema.
				\item \cdtRef{proyectoEntidad:nombreProyecto}{Nombre del proyecto:} Lo obtiene el sistema.
				\item \cdtRef{moduloEntidad:claveModulo}{Clave del módulo:} Lo obtiene el sistema.
				\item \cdtRef{moduloEntidad:nombreModulo}{Nombre del módulo:} Lo obtiene el sistema.
				\item \cdtRef{pantalla:claveIU}{Clave}: Lo obtiene el sistema.
				\item \cdtRef{pantalla:numeroIU}{Número de la pantalla}: Lo obtiene el sistema.
				\item \cdtRef{pantalla:nombreIU}{Nombre de la pantalla}: Lo obtiene el sistema.
				\item \cdtRef{pantalla:descripcionIU}{Descripción de la pantalla}: Lo obtiene el sistema.
				\item \cdtRef{pantalla:imagenIU}{Imagen de la pantalla}: Lo obtiene el sistema.
				\item \cdtRef{pantalla:claveIU}{Clave:} Lo calcula el sistema mediante la regla de negocio \BRref{RN12}{Identifcador de elemento}.
				\item \cdtIdRef{MSG1}{Operación exitosa}: Se muestra en la pantalla \IUref{IU7}{Gestionar Pantallas} para indicar que la edición fue exitosa.
		\end{itemize}}
		
		\UCitem{Precondiciones}{
			\begin{itemize}
				\item Que exista al menos una pantalla registrada.
				\item Que la pantalla no se encuentre asociada a un caso de uso en estado ''Liberado''.
			\end{itemize}
		}
		\UCitem{Postcondiciones}{Se actualizará la información de una pantalla de un proyecto}
		\UCitem{Errores}{\begin{itemize}
		\item \cdtIdRef{MSG4}{Dato obligatorio}: Se muestra en la pantalla \IUref{IU7.2}{Modificar Pantalla} cuando no se ha ingresado un dato marcado como obligatorio.
		\item \cdtIdRef{MSG30}{Imagen obligatorio}: Se muestra en la pantalla \IUref{IU7.2}{Modificar Pantalla} cuando no se ha ingresado un campo de tipo archivo como obligatorio.
		\item \cdtIdRef{MSG5}{Formato incorrecto}: Se muestra en la pantalla \IUref{IU7.2}{Modificar Pantalla} cuando el tipo de dato ingresado no cumple con el tipo de dato solicitado en el campo.
		\item \cdtIdRef{MSG5}{Formato de campo incorrecto}: Se muestra en la pantalla \IUref{IU7.2}{Modificar Pantalla} cuando el número de la pantalla contiene un carácter no válido.
		\item \cdtIdRef{MSG6}{Longitud inválida}: Se muestra en la pantalla \IUref{IU7.2}{Modificar Pantalla} cuando se ha excedido la longitud de alguno de los campos.
		\item \cdtIdRef{MSG7}{Registro repetido}: Se muestra en la pantalla \IUref{IU7.2}{Modificar Pantalla} cuando se registre un actor con un nombre que ya se encuentre registrado en el sistema.
		\item \cdtIdRef{MSG16}{Formato de archivo incorrecto}: Se muestra en la pantalla \IUref{IU7.2}{Modificar Pantalla} cuando la imagen de la pantalla no cumpla con el formato especificado.
		\item \cdtIdRef{MSG17}{Se ha excedido el tamaño del archivo}: Se muestra en la pantalla \IUref{IU7.2}{Modificar Pantalla} cuando la imagen de la pantalla exceda el tamaño especificado.
		\end{itemize}.
		}
		\UCitem{Tipo}{Secundario, extiende del caso de uso \UCref{CU11}{Gestionar Pantallas}.}
	\end{UseCase}
%--------------------------------------
	\begin{UCtrayectoria}
		\UCpaso[\UCactor] Da clic en el icono \editar de la pantalla \IUref{IU7}{Gestionar Pantallas}.
		\UCpaso[\UCsist] Obtiene la información de la pantalla.
		\UCpaso[\UCsist] Muestra la pantalla \IUref{IU7.2}{Modificar Pantalla}. \label{CU11.2-P7}
		\UCpaso[\UCactor] Modifica la información de la pantalla. \label{CU11.2-P5}
		\UCpaso[\UCactor] Selecciona la nueva imagen de sus archivos locales. \label{CU11.2-P6}
		\UCpaso[\UCactor] Oprime el botón \IUbutton{Aceptar}.\hyperlink{CU11-2:TAA}{[Trayectoria A]} 
		\UCpaso[\UCsist] Verifica que el actor ingrese todos los campos obligatorios con base en la regla de negocio \BRref{RN8}{Datos obligatorios}. \hyperlink{CU11-2:TAB}{[Trayectoria B]} \hyperlink{CU11-2:TAC}{[Trayectoria C]}
		\UCpaso[\UCsist] Verificar que los datos ingresados cumpla con la longitud correcta, con base en la regla de negocio \BRref{RN37}{Longitud de datos}. \hyperlink{CU11-2:TAD}{[Trayectoria D]}
		\UCpaso[\UCsist] Verifica que los datos ingresados cumplan con el formato requerido, con base en la regla de negocio \BRref{RN7}{Información correcta}. \hyperlink{CU11-2:TAE}{[Trayectoria E]} \hyperlink{CU11-2:TAF}{[Trayectoria F]}
		\UCpaso[\UCsist] Verifica que el archivo ingresado cumpla con el peso correcto, con base en la regla de negocio \BRref{RN40}{Peso de archivos de imagen}. \hyperlink{CU11-2:TAG}{[Trayectoria G]}
		\UCpaso[\UCsist] Verifica que el número de la pantalla no se encuentre registrado en el sistema con base en la regla de negocio \BRref{RN1}{Unicidad de números}. \hyperlink{CU11-2:TAH}{[Trayectoria H]}
		\UCpaso[\UCsist] Verifica que el nombre de la pantalla no se encuentre registrado en el sistema con base en la regla de negocio \BRref{RN6}{Unicidad de nombres}. \hyperlink{CU11-2:TAI}{[Trayectoria I]} 
		\UCpaso[\UCsist] Actualiza la información de la pantalla en el sistema.
		\UCpaso[\UCsist] Muestra el mensaje \cdtIdRef{MSG1}{Operación exitosa} en la pantalla \IUref{IU7}{Gestionar Pantallas} para indicar al actor que la modificación se ha realizado exitosamente.
	\end{UCtrayectoria}		
%--------------------------------------
\hypertarget{CU11-2:TAA}{\textbf{Trayectoria alternativa A}}\\
\noindent \textbf{Condición:} El actor desea cancelar la operación.
\begin{enumerate}
	\UCpaso[\UCactor] Solicita cancelar la operación oprimiendo el botón \IUbutton{Cancelar} de la pantalla \IUref{IU7.2}{Modificar Pantalla}.
	\UCpaso[\UCsist] Muestra la pantalla \IUref{IU7}{Gestionar Pantallas}.
	\item[- -] - - {\em {Fin del caso de uso}}.%
\end{enumerate}
%--------------------------------------
\hypertarget{CU11-2:TAB}{\textbf{Trayectoria alternativa B}}\\
\noindent \textbf{Condición:} El actor no ingresó algún dato marcado como obligatorio.
\begin{enumerate}
	\UCpaso[\UCsist] Muestra el mensaje \cdtIdRef{MSG4}{Dato obligatorio} señalando el campo que presenta el error en la pantalla \IUref{IU7.2}{Modificar Pantalla}.
	\UCpaso Regresa al paso \ref{CU11.2-P5} de la trayectoria principal.
	\item[- -] - - {\em {Fin de la trayectoria}}.%
\end{enumerate}
%--------------------------------------
\hypertarget{CU11-2:TAC}{\textbf{Trayectoria alternativa C}}\\
\noindent \textbf{Condición:} El actor no ingresó un dato de tipo imagen marcado como obligatorio.
\begin{enumerate}
	\UCpaso[\UCsist] Muestra el mensaje \cdtIdRef{MSG30}{Imagen obligatorio} señalando el campo que presenta el error en la pantalla \IUref{IU7.2}{Modificar Pantalla}.
	\UCpaso Regresa al paso \ref{CU11.1-P3} de la trayectoria principal.
	\item[- -] - - {\em {Fin de la trayectoria}}.%
\end{enumerate}
%--------------------------------------
\hypertarget{CU11-2:TAD}{\textbf{Trayectoria alternativa D}}\\
\noindent \textbf{Condición:} El actor ingresó un dato con un número de caracteres fuera del rango permitido.
\begin{enumerate}
	\UCpaso[\UCsist] Muestra el mensaje \cdtIdRef{MSG6}{Longitud inválida} señalando el campo que presenta el error en la pantalla \IUref{IU7.2}{Modificar Pantalla}.
	\UCpaso Regresa al paso \ref{CU11.2-P5} de la trayectoria principal.
	\item[- -] - - {\em {Fin de la trayectoria}}.%
\end{enumerate}
%--------------------------------------	
\hypertarget{CU11-2:TAE}{\textbf{Trayectoria alternativa E}}\\
\noindent \textbf{Condición:} El actor ingresó un dato con un formato incorrecto.
\begin{enumerate}
	\UCpaso[\UCsist] Muestra el mensaje \cdtIdRef{MSG5}{Formato incorrecto} señalando el campo que presenta el error en la pantalla \IUref{IU7.2}{Modificar Pantalla}.
	\UCpaso Regresa al paso \ref{CU11.2-P5} de la trayectoria principal.
	\item[- -] - - {\em {Fin de la trayectoria}}.
\end{enumerate}
%--------------------------------------
\hypertarget{CU11-2:TAF}{\textbf{Trayectoria alternativa F}}\\
\noindent \textbf{Condición:} El actor proporciona una imagen de formato incorrecto.
\begin{enumerate}
	\UCpaso[\UCsist] Muestra el mensaje \cdtIdRef{MSG16}{Formato de archivo incorrecto} señalando el campo que presenta el error en la pantalla \IUref{IU7.2}{Modificar Pantalla}.
	\UCpaso Regresa al paso \ref{CU11.2-P6} de la trayectoria principal.
	\item[- -] - - {\em {Fin de la trayectoria}}.
\end{enumerate}
%--------------------------------------
\hypertarget{CU11-2:TAG}{\textbf{Trayectoria alternativa G}}\\
\noindent \textbf{Condición:} El actor proporciona una imagen que excede el tamaño máximo.
\begin{enumerate}
	\UCpaso[\UCsist] Muestra el mensaje \cdtIdRef{MSG17}{Se ha excedido el tamaño del archivo} señalando el campo que presenta el error en la pantalla \IUref{IU7.2}{Modificar Pantalla}.
	\UCpaso Regresa al paso \ref{CU11.2-P6} de la trayectoria principal.
	\item[- -] - - {\em {Fin de la trayectoria}}.
\end{enumerate}
%--------------------------------------	

\hypertarget{CU11-2:TAH}{\textbf{Trayectoria alternativa H}}\\
\noindent \textbf{Condición:} El actor ingresó un número de pantalla repetido.
\begin{enumerate}
	\UCpaso[\UCsist] Muestra el mensaje \cdtIdRef{MSG7}{Registro repetido} señalando el campo que presenta la duplicidad en la pantalla \IUref{IU7.2}{Modificar Pantalla}.
	\UCpaso Regresa al paso \ref{CU11.2-P5} de la trayectoria principal.
	\item[- -] - - {\em {Fin de la trayectoria}}.
\end{enumerate}
%--------------------------------------
\hypertarget{CU11-2:TAI}{\textbf{Trayectoria alternativa I}}\\
\noindent \textbf{Condición:} El actor ingresó un nombre de pantalla repetido.
\begin{enumerate}
	\UCpaso[\UCsist] Muestra el mensaje \cdtIdRef{MSG7}{Registro repetido} señalando el campo que presenta la duplicidad en la pantalla \IUref{IU7.2}{Modificar Pantalla}.
	\UCpaso Regresa al paso \ref{CU11.2-P5} de la trayectoria principal.
	\item[- -] - - {\em {Fin de la trayectoria}}.
\end{enumerate}