	\begin{UseCase}{CU11.1}{Registrar Pantalla}{
		Este caso de uso permite al colaborador (\hyperlink{jefe}{Líder de Análisis} o \hyperlink{analista}{Analista}) registrar la información del maquetado de una \hyperlink{pantalla}{Pantalla} correspondiente a algún \hyperlink{moduloEntidad}{Módulo} del \hyperlink{proyectoEntidad}{Proyecto}.\\
		
		A través de un formulario el sistema solicita la información general de la pantalla que se desea registrar, así como la imagen del maquetado. Al momento de registrar la pantalla, el sistema también le permitirá al colaborador gestionar las acciones que intervienen en la pantalla (de tal forma que podrá registrar, modificar y eliminar una acción). Las acciones son aquellos componentes como botones, hipervínculos y opciones del menú que cumplen una funcionalidad dentro de la pantalla. \\
		
		Los datos se almacenan con el propósito de conocer el maquetado de las pantallas y referenciarlas durante la elaboración de un \hyperlink{casoUso}{Caso de Uso}. Una vez registrada la pantalla, el colaborador podrá hacer uso de ella en el editor del \hyperlink{casoUso}{Caso de Uso} .
	}
		\UCitem{Actor}{\hyperlink{jefe}{Líder de Análisis}, \hyperlink{analista}{Analista}}
		\UCitem{Propósito}{Registrar la información de una pantalla.}
		\UCitem{Entradas}{
		\begin{itemize}
			\item \cdtRef{pantalla:numeroIU}{Número}: Se escribe desde el teclado.
			\item \cdtRef{pantalla:nombreIU}{Nombre}: Se escribe desde el teclado.
			\item \cdtRef{pantalla:descripcionIU}{Descripción}: Se escribe desde el teclado.
			\item \cdtRef{pantalla:imagenIU}{Imagen}: Se selecciona de los archivos locales.
		\end{itemize}	
		}
		\UCitem{Salidas}{\begin{itemize}
				\item \cdtRef{proyectoEntidad:claveProyecto}{Clave del proyecto:} Lo obtiene el sistema.
				\item \cdtRef{proyectoEntidad:nombreProyecto}{Nombre del proyecto:} Lo obtiene el sistema.
				\item \cdtRef{moduloEntidad:claveModulo}{Clave del módulo:} Lo obtiene el sistema.
				\item \cdtRef{moduloEntidad:nombreModulo}{Nombre del módulo:} Lo obtiene el sistema.
				\item \cdtRef{pantalla:claveIU}{Clave:} Lo calcula el sistema mediante la regla de negocio \BRref{RN12}{Identifcador de elemento}.
				\item \cdtRef{EntidadAccion}{Acciones:} Tabla que muestra \cdtRef{imagenACC}{Imagen}, \cdtRef{nombreACC}{nombre} de todos los los registros de las acciones.
				\item \cdtIdRef{MSG1}{Operación exitosa}: Se muestra en la pantalla \IUref{IU7}{Gestionar Pantallas} para indicar que el registro fue exitoso.
		\end{itemize}}
		
		\UCitem{Precondiciones}{Ninguna}
		\UCitem{Postcondiciones}{
		\begin{itemize}
			\item Se registrará una pantalla en el módulo de un proyecto en el sistema.
			\item La pantalla podrá ser referenciada en casos de uso.
		\end{itemize}
		}
		\UCitem{Errores}{\begin{itemize}
		\item \cdtIdRef{MSG4}{Dato obligatorio}: Se muestra en la pantalla \IUref{IU7.1}{Registrar Pantalla} cuando no se ha ingresado un dato marcado como obligatorio.
		\item \cdtIdRef{MSG5}{Formato incorrecto}: Se muestra en la pantalla \IUref{IU7.1}{Registrar Pantalla} cuando el tipo de dato ingresado no cumple con el tipo de dato solicitado en el campo.
		\item \cdtIdRef{MSG5}{Formato de campo incorrecto}: Se muestra en la pantalla \IUref{IU7.1}{Registrar Pantalla} cuando el número de la pantalla contiene un carácter no válido.
		\item \cdtIdRef{MSG6}{Longitud inválida}: Se muestra en la pantalla \IUref{IU7.1}{Registrar Pantalla} cuando se ha excedido la longitud de alguno de los campos.
		\item \cdtIdRef{MSG7}{Registro repetido}: Se muestra en la pantalla \IUref{IU7.1}{Registrar Pantalla} cuando se registre un actor con un nombre que ya se encuentre registrado en el sistema.
		\item \cdtIdRef{MSG16}{Formato de archivo incorrecto}: Se muestra en la pantalla \IUref{IU7.1}{Registrar Pantalla} cuando la imagen de la pantalla no cumpla con el formato especificado.
		\item \cdtIdRef{MSG17}{Se ha excedido el tamaño del archivo}: Se muestra en la pantalla \IUref{IU7.1}{Registrar Pantalla} cuando la imagen de la pantalla exceda el tamaño especificado.
		\end{itemize}.
		}
		\UCitem{Tipo}{Secundario, extiende del caso de uso \UCref{CU11}{Gestionar Pantallas}.}
	\end{UseCase}
%--------------------------------------
	\begin{UCtrayectoria}
		\UCpaso[\UCactor] Solicita registrar una pantalla oprimiendo el botón \IUbutton{Registrar} de la pantalla \IUref{IU7}{Gestionar Pantallas}.
		\UCpaso[\UCsist] Muestra la pantalla \IUref{IU7.1}{Registrar Pantalla}. \label{CU11.1-P5}
		\UCpaso[\UCactor] Ingresa la información solicitada. \label{CU11.1-P3}
		\UCpaso[\UCactor] Selecciona la imagen de sus archivos locales. \label{CU11.1-P4}
		\UCpaso[\UCactor] Gestiona las acciones de la pantalla a través de los botones: \IUbutton{Registrar}, \editar y \eliminar. 
		\UCpaso[\UCactor] Oprime el botón el botón \IUbutton{Aceptar}.\hyperlink{CU11-1:TAA}{[Trayectoria A]}
		\UCpaso[\UCsist] Verifica que el actor ingrese todos los campos obligatorios con base en la regla de negocio \BRref{RN8}{Datos obligatorios}. \hyperlink{CU11-1:TAB}{[Trayectoria B]}
		\UCpaso[\UCsist] Verificar que los datos ingresados cumpla con la longitud correcta, con base en la regla de negocio \BRref{RN37}{Longitud de datos}. \hyperlink{CU11-1:TAC}{[Trayectoria C]}
		\UCpaso[\UCsist] Verifica que los datos ingresados cumplan con el formato requerido, con base en la regla de negocio \BRref{RN7}{Información correcta}. \hyperlink{CU11-1:TAD}{[Trayectoria D]} \hyperlink{CU11-1:TAC}{[Trayectoria E]}
		\UCpaso[\UCsist] Verifica que el archivo ingresado cumpla con el peso correcto, con base en la regla de negocio \BRref{RN40}{Peso de archivos de imagen}. \hyperlink{CU11-1:TAF}{[Trayectoria F]}
		\UCpaso[\UCsist] Verifica que el número de la pantalla no se encuentre registrado en el sistema con base en la regla de negocio \BRref{RN1}{Unicidad de números}. \hyperlink{CU11-1:TAG}{[Trayectoria G]}
		\UCpaso[\UCsist] Verifica que el nombre de la pantalla no se encuentre registrado en el sistema con base en la regla de negocio \BRref{RN6}{Unicidad de nombres}. \hyperlink{CU11-1:TAH}{[Trayectoria H]}
		\UCpaso[\UCsist] Persiste la información de la pantalla en el sistema.
		\UCpaso[\UCsist] Muestra el mensaje \cdtIdRef{MSG1}{Operación exitosa} en la pantalla \IUref{IU7}{Gestionar Pantallas} para indicar al actor que el registro se ha realizado exitosamente.
	\end{UCtrayectoria}		
%--------------------------------------
\hypertarget{CU11-1:TAA}{\textbf{Trayectoria alternativa A}}\\
\noindent \textbf{Condición:} El actor desea cancelar la operación.
\begin{enumerate}
	\UCpaso[\UCactor] Solicita cancelar la operación oprimiendo el botón \IUbutton{Cancelar} de la pantalla \IUref{IU7.1}{Registrar Pantalla}.
	\UCpaso[\UCsist] Muestra la pantalla \IUref{IU7}{Gestionar Pantallas}.
	\item[- -] - - {\em {Fin del caso de uso}}.%
\end{enumerate}
%--------------------------------------
\hypertarget{CU11-1:TAB}{\textbf{Trayectoria alternativa B}}\\
\noindent \textbf{Condición:} El actor no ingresó algún dato marcado como obligatorio.
\begin{enumerate}
	\UCpaso[\UCsist] Muestra el mensaje \cdtIdRef{MSG4}{Dato obligatorio} señalando el campo que presenta el error en la pantalla \IUref{IU7.1}{Registrar Pantalla}.
	\UCpaso Regresa al paso \ref{CU11.1-P3} de la trayectoria principal.
	\item[- -] - - {\em {Fin de la trayectoria}}.%
\end{enumerate}
%--------------------------------------
\hypertarget{CU11-1:TAC}{\textbf{Trayectoria alternativa C}}\\
\noindent \textbf{Condición:} El actor ingresó un dato con un número de caracteres fuera del rango permitido.
\begin{enumerate}
	\UCpaso[\UCsist] Muestra el mensaje \cdtIdRef{MSG6}{Longitud inválida} señalando el campo que presenta el error en la pantalla \IUref{IU7.1}{Registrar Pantalla}.
	\UCpaso Regresa al paso \ref{CU11.1-P3} de la trayectoria principal.
	\item[- -] - - {\em {Fin de la trayectoria}}.%
\end{enumerate}
%--------------------------------------	
\hypertarget{CU11-1:TAD}{\textbf{Trayectoria alternativa D}}\\
\noindent \textbf{Condición:} El actor ingresó un dato con un formato incorrecto.
\begin{enumerate}
	\UCpaso[\UCsist] Muestra el mensaje \cdtIdRef{MSG5}{Formato incorrecto} señalando el campo que presenta el error en la pantalla \IUref{IU7.1}{Registrar Pantalla}.
	\UCpaso Regresa al paso \ref{CU11.1-P3} de la trayectoria principal.
	\item[- -] - - {\em {Fin de la trayectoria}}.
\end{enumerate}
%--------------------------------------
\hypertarget{CU11-1:TAE}{\textbf{Trayectoria alternativa E}}\\
\noindent \textbf{Condición:} El actor proporciona una imagen de formato incorrecto.
\begin{enumerate}
	\UCpaso[\UCsist] Muestra el mensaje \cdtIdRef{MSG16}{Formato de archivo incorrecto} señalando el campo que presenta el error en la pantalla \IUref{IU7.1}{Registrar Pantalla}.
	\UCpaso Regresa al paso \ref{CU11.1-P4} de la trayectoria principal.
	\item[- -] - - {\em {Fin de la trayectoria}}.
\end{enumerate}
%--------------------------------------
\hypertarget{CU11-1:TAF}{\textbf{Trayectoria alternativa F}}\\
\noindent \textbf{Condición:} El actor proporciona una imagen que excede el tamaño máximo.
\begin{enumerate}
	\UCpaso[\UCsist] Muestra el mensaje \cdtIdRef{MSG17}{Se ha excedido el tamaño del archivo} señalando el campo que presenta el error en la pantalla \IUref{IU7.1}{Registrar Pantalla}.
	\UCpaso Regresa al paso \ref{CU11.1-P4} de la trayectoria principal.
	\item[- -] - - {\em {Fin de la trayectoria}}.
\end{enumerate}
%--------------------------------------	

\hypertarget{CU11-1:TAG}{\textbf{Trayectoria alternativa G}}\\
\noindent \textbf{Condición:} El actor ingresó un número de pantalla repetido.
\begin{enumerate}
	\UCpaso[\UCsist] Muestra el mensaje \cdtIdRef{MSG7}{Registro repetido} señalando el campo que presenta la duplicidad en la pantalla \IUref{IU7.1}{Registrar Pantalla}.
	\UCpaso Regresa al paso \ref{CU11.1-P3} de la trayectoria principal.
	\item[- -] - - {\em {Fin de la trayectoria}}.
\end{enumerate}
%--------------------------------------
\hypertarget{CU11-1:TAH}{\textbf{Trayectoria alternativa H}}\\
\noindent \textbf{Condición:} El actor ingresó un nombre de pantalla repetido.
\begin{enumerate}
	\UCpaso[\UCsist] Muestra el mensaje \cdtIdRef{MSG7}{Registro repetido} señalando el campo que presenta la duplicidad en la pantalla \IUref{IU7.1}{Registrar Pantalla}.
	\UCpaso Regresa al paso \ref{CU11.1-P3} de la trayectoria principal.
	\item[- -] - - {\em {Fin de la trayectoria}}.
\end{enumerate}
%--------------------------------------

\subsubsection{Puntos de extensión}

\UCExtenssionPoint{El actor requiere registrar una acción}{Presionando el botón \IUbutton{Registrar} del paso \ref{CU11.1-P5} de la trayectoria principal.}{\UCref{CU11.1.1}{Registrar Acción}}
\UCExtenssionPoint{El actor requiere modificar una acción}{Presionando el icono \editar del paso \ref{CU11.1-P5} de la trayectoria principal.}{\UCref{CU11.1.2}{Modificar Acción}}
\UCExtenssionPoint{El actor requiere eliminar una acción}{Presionando el icono \eliminar del paso \ref{CU11.1-P5} de la trayectoria principal.}{\UCref{CU11.1.3}{Eliminar Acción}}