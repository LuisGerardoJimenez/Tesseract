	\begin{UseCase}{CU8.1}{Registrar Regla de negocio}{
			
			Este caso de uso permite al colaborador (\hyperlink{jefe}{Líder de Análisis} o \hyperlink{analista}{Analista}) registrar la información de una \hyperlink{BREntidad}{Regla de Negocio} en el \hyperlink{proyectoEntidad}{Proyecto}, los datos que se almacenan cumplen con el propósito de describir las políticas, normas, operaciones, definiciones y restricciones presentes en el \hyperlink{proyectoEntidad}{Proyecto} y que son de vital importancia para alcanzar los objetivos planteados por la organización.
			
			Una vez registrada la regla, el colaborador podrá hacer uso de la misma en el editor.
		
	}
		
		\UCitem{Actor}{\hyperlink{jefe}{Líder de Análisis}, \hyperlink{analista}{Analista}}
		\UCitem{Propósito}{Registrar la información de una regla de negocio.}
		\UCitem{Entradas}{
		\begin{itemize}
			\item \cdtRef{BRSEntidad:numeroBR}{Número de la Regla de Negocio:} Se escribe desde el teclado.
			\item \cdtRef{BREntidad:nombreBR}{Nombre de la Regla de Negocio:} Se escribe desde el teclado.
			\item \cdtRef{BREntidad:descripciónBR}{Descripción de la Regla de negocio:} Se escribe desde el teclado.
			\item \cdtRef{BREntidad:redaccionBR}{Redacción de la Regla de negocio:} Se escribe desde el teclado.
			\item \hyperlink{tTipoRN}{Tipo:} Se selecciona de una lista.
			\item Para el tipo ''Comparación de atributos'':
				\begin{itemize}
					\item \cdtRef{entidadEntidad}{Entidad 1}: Se selecciona de una lista.
					\item \cdtRef{entidadAtributo}{Atributo 1}: Se selecciona de una lista.
					\item Operador: Se selecciona de una lista.
					\item \cdtRef{entidadEntidad}{Entidad 2}: Se selecciona de una lista.
					\item \cdtRef{entidadAtributo}{Atributo 2}: Se selecciona de una lista.
				\end{itemize}
			\item Para el tipo ''Unicidad de párametros'':
				\begin{itemize}
					\item \cdtRef{entidadEntidad}{Entidad}: Se escribe desde el teclado.
					\item \cdtRef{entidadAtributo}{Atributo único}: Se escribe desde el teclado.
				\end{itemize}
			\item Para el tipo ''Formato correcto'':
				\begin{itemize}
					\item \cdtRef{entidadEntidad}{Entidad que contiene el atributo para verificar el formato}: Se selecciona de una lista.
					\item \cdtRef{entidadAtributo}{Atributo que se verificará con la expresión regular}: Se selecciona de una lista.
					\item Expresión regular: Se escribe desde el teclado.
				\end{itemize}
		\end{itemize}	
		}
		\UCitem{Salidas}{\begin{itemize}
				\item \cdtRef{proyectoEntidad:claveProyecto}{Clave del proyecto:} Lo obtiene el sistema.
				\item \cdtRef{proyectoEntidad:nombreProyecto}{Nombre del proyecto:} Lo obtiene el sistema.
				\item \cdtRef{BREntidad:claveBR}{Clave de la regla de negocio:} Lo calcula el sistma mediante la regla de negocio \BRref{RN12}{Idenficador de elemento}.
				\item \cdtIdRef{MSG1}{Operación exitosa}: Se muestra en la pantalla \IUref{IU9}{Gestionar Reglas de negocio} para indicar que el registro fue exitoso.
		\end{itemize}}
		
		\UCitem{Precondiciones}{Ninguna}
		\UCitem{Postcondiciones}{
		\begin{itemize}
			\item Se registrará una regla de negocio de un proyecto en el sistema.
			\item La regla de negocio podrá ser referenciada en casos de uso.
		\end{itemize}
		}
		\UCitem{Errores}{\begin{itemize}
		\item \cdtIdRef{MSG4}{Dato obligatorio}: Se muestra en la pantalla \IUref{IU9.1}{Registrar Regla de negocio} cuando no se ha ingresado un dato marcado como obligatorio.
		\item \cdtIdRef{MSG5}{Formato incorrecto}: Se muestra en la pantalla \IUref{IU9.1}{Registrar Regla de negocio} cuando el tipo de dato ingresado no cumple con el tipo de dato solicitado en el campo.
		\item \cdtIdRef{MSG5}{Formato de campo incorrecto}: Se muestra en la pantalla \IUref{IU9.1}{Registrar Regla de negocio} cuando el número de la regla de negocio contiene un carácter no válido.
		\item \cdtIdRef{MSG6}{Longitud inválida}: Se muestra en la pantalla \IUref{IU9.1}{Registrar Regla de negocio} cuando se ha excedido la longitud de alguno de los campos.
		\item \cdtIdRef{MSG7}{Registro repetido}: Se muestra en la pantalla \IUref{IU9.1}{Registrar Regla de negocio} cuando se registre un regla de negocio con un nombre o número que ya se encuentre registrado en el sistema.
		\item \cdtIdRef{MSG12}{Ha ocurrido un error}: Se muestra en la pantalla \IUref{IU9}{Gestionar Reglas de negocio} cuando no exista información en el catálogo ''tipo de regla de negocio''.
		\end{itemize}
		}
		\UCitem{Tipo}{Secundario, extiende del caso de uso \UCref{CU8}{Gestionar Reglas de negocio}.}
	\end{UseCase}
%--------------------------------------
	\begin{UCtrayectoria}
		\UCpaso[\UCactor] Solicita registrar una regla de negocio oprimiendo el botón \IUbutton{Registrar} de la pantalla \IUref{IU9}{Gestionar Reglas de negocio}.
		\UCpaso[\UCsist] Verifica que el catálogo ''tipo de regla de negocio'' cuente con información, con base en la regla de negocio \BRref{RN20}{Verificación de catálogos}. \hyperlink{CU8-1:TAA}{[Trayectoria A]}
		\UCpaso[\UCsist] Muestra la pantalla \IUref{IU9.1}{Registrar Regla de negocio}.
		\UCpaso[\UCactor] Ingresa la información solicitada en la pantalla. \label{CU8.1-P3}
		\UCpaso[\UCactor] Selecciona el tipo de regla de negocio.
		\UCpaso[\UCsist] Verifica que el tipo de regla de negocio requiera parámetros. \hyperlink{CU8-1:TAB}{[Trayectoria B]}
		\UCpaso[\UCsist] Muestra la pantalla \IUref{IU9.1A}{Registrar Regla de negocio: Comparación de atributos}, \IUref{IU9.1B}{Registrar Regla de negocio: Unicidad de parámetros} o \IUref{IU9.1C}{Registrar Regla de negocio: Formato correcto}, según corresponda.
		\UCpaso[\UCactor] Ingresa la información solicitada en la pantalla correspondiente.
		\UCpaso[\UCactor] Ingresa la redacción de la regla de negocio. \label{CU8.1-P9}
		\UCpaso[\UCactor] Oprime el botón \IUbutton{Aceptar}. \hyperlink{CU8-1:TAC}{[Trayectoria C]}
		\UCpaso[\UCsist] Verifica que el actor ingrese todos los campos obligatorios con base en la regla de negocio \BRref{RN8}{Datos obligatorios}. \hyperlink{CU8-1:TAD}{[Trayectoria D]}
		\UCpaso[\UCsist] Verifica que los datos ingresados cumpla con la longitud correcta, con base en la regla de negocio \BRref{RN37}{Longitud de datos}. \hyperlink{CU8-1:TAE}{[Trayectoria E]}
		\UCpaso[\UCsist] Verifica que los datos ingresados cumplan con el formato requerido, con base en la regla de negocio \BRref{RN7}{Información correcta}. \hyperlink{CU8-1:TAF}{[Trayectoria F]}
		\UCpaso[\UCsist] Verifica que el número de la regla de negocio no se encuentre registrado con base en la regla de negocio \BRref{RN1}{Unicidad de números}. \hyperlink{CU8-1:TAG}{[Trayectoria G]}
		\UCpaso[\UCsist] Verifica que el nombre de la regla de negocio no se encuentre registrado en el sistema con base en la regla de negocio \BRref{RN6}{Unicidad de nombres}. \hyperlink{CU8-1:TAH}{[Trayectoria H]}
		\UCpaso[\UCsist] Persiste la información de la regla de negocio en el sistema.
		\UCpaso[\UCsist] Muestra el mensaje \cdtIdRef{MSG1}{Operación exitosa} en la pantalla \IUref{IU9}{Gestionar Reglas de negocio} para indicar al actor que el registro se ha realizado exitosamente.
	\end{UCtrayectoria}		
%--------------------------------------
\hypertarget{CU8-1:TAA}{\textbf{Trayectoria alternativa A}}\\
\noindent \textbf{Condición:} El catálogo tipo de regla de negocio no tiene información.
\begin{enumerate}
	\UCpaso[\UCsist] Muestra el mensaje \cdtIdRef{MSG12}{Ha ocurrido un error} en la pantalla \IUref{IU9}{Gestionar Reglas de negocio} para indicar que no es posible realizar la operación debido a la falta de información necesaria para el sistema.
	\item[- -] - - {\em {Fin del caso de uso}}.%
\end{enumerate}
%--------------------------------------
\hypertarget{CU8-1:TAB}{\textbf{Trayectoria alternativa B}}\\
\noindent \textbf{Condición:} La regla de negocio no requiere parámetros.
\begin{enumerate}
	\UCpaso[\UCactor] Continúa con el paso \ref{CU8.1-P9} de la trayectoria principal.
	\item[- -] - - {\em {Fin de la trayectoria}}.%
\end{enumerate}
	
%--------------------------------------	
\hypertarget{CU8-1:TAC}{\textbf{Trayectoria alternativa C}}\\
\noindent \textbf{Condición:} El actor desea cancelar la operación.
\begin{enumerate}
	\UCpaso[\UCactor] Solicita cancelar la operación oprimiendo el botón \IUbutton{Cancelar} de la pantalla \IUref{IU9.1}{Registrar Regla de negocio}
	\UCpaso[\UCsist] Muestra la pantalla \IUref{IU9}{Gestionar Reglas de negocio}.
	\item[- -] - - {\em {Fin del caso de uso}}.%
\end{enumerate}
%--------------------------------------
\hypertarget{CU8-1:TAD}{\textbf{Trayectoria alternativa D}}\\
\noindent \textbf{Condición:} El actor no ingresó algún dato marcado como obligatorio.
\begin{enumerate}
	\UCpaso[\UCsist] Muestra el mensaje \cdtIdRef{MSG4}{Dato obligatorio} señalando el campo que presenta el error en la pantalla \IUref{IU9.1}{Registrar Regla de negocio}.
	\UCpaso Regresa al paso \ref{CU8.1-P3} de la trayectoria principal.
	\item[- -] - - {\em {Fin de la trayectoria}}.%
\end{enumerate}
%--------------------------------------
\hypertarget{CU8-1:TAE}{\textbf{Trayectoria alternativa E}}\\
\noindent \textbf{Condición:} El actor ingresó un dato con un número de caracteres fuera del rango permitido.
\begin{enumerate}
	\UCpaso[\UCsist] Muestra el mensaje \cdtIdRef{MSG6}{Longitud inválida} señalando el campo que presenta el error en la pantalla \IUref{IU9.1}{Registrar Regla de negocio}.
	\UCpaso Regresa al paso \ref{CU8.1-P3} de la trayectoria principal.
	\item[- -] - - {\em {Fin de la trayectoria}}.%
\end{enumerate}
%--------------------------------------	
\hypertarget{CU8-1:TAF}{\textbf{Trayectoria alternativa F}}\\
\noindent \textbf{Condición:} El actor ingresó un dato con un formato incorrecto.
\begin{enumerate}
	\UCpaso[\UCsist] Muestra el mensaje \cdtIdRef{MSG5}{Formato incorrecto} señalando el campo que presenta el error en la pantalla \IUref{IU9.1}{Registrar Regla de negocio}.
	\UCpaso Regresa al paso \ref{CU8.1-P3} de la trayectoria principal.
	\item[- -] - - {\em {Fin de la trayectoria}}.
\end{enumerate}
%--------------------------------------
\hypertarget{CU8-1:TAG}{\textbf{Trayectoria alternativa G}}\\
\noindent \textbf{Condición:} El actor ingresó un número de regla de negocio repetido.
\begin{enumerate}
	\UCpaso[\UCsist] Muestra el mensaje \cdtIdRef{MSG7}{Registro repetido} señalando el campo que presenta la duplicidad en la pantalla \IUref{IU9.1}{Registrar Regla de negocio}.
	\UCpaso Regresa al paso \ref{CU8.1-P3} de la trayectoria principal.
	\item[- -] - - {\em {Fin de la trayectoria}}.
\end{enumerate}

%--------------------------------------	
\hypertarget{CU8-1:TAH}{\textbf{Trayectoria alternativa H}}\\
\noindent \textbf{Condición:} El actor ingresó un nombre de regla de negocio repetido.
\begin{enumerate}
	\UCpaso[\UCsist] Muestra el mensaje \cdtIdRef{MSG7}{Registro repetido} señalando el campo que presenta la duplicidad en la pantalla \IUref{IU9.1}{Registrar Regla de negocio}.
	\UCpaso Regresa al paso \ref{CU8.1-P3} de la trayectoria principal.
	\item[- -] - - {\em {Fin de la trayectoria}}.
\end{enumerate}
%--------------------------------------