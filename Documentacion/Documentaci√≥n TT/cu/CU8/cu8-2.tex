	\begin{UseCase}{CU8.2}{Modificar Regla de negocio}{
		Este caso de uso permite al analista modificar la información de una regla de negocio.
	}
		\UCitem{Versión}{\color{Gray}0.1}
		\UCitem{Actor}{\hyperlink{jefe}{Líder de Análisis}, \hyperlink{analista}{Analista}}
		\UCitem{Propósito}{Modificar la información de una regla de negocio.}
		\UCitem{Entradas}{
		\begin{itemize}
			\item \cdtRef{BRSEntidad:numeroBR}{Número:} Se escribe desde el teclado.
			\item \cdtRef{BREntidad:nombreBR}{Nombre:} Se escribe desde el teclado.
			\item \cdtRef{BREntidad:descripciónBR}{Descripción:} Se escribe desde el teclado.
			\item \cdtRef{BREntidad:redaccionBR}{Redacción:} Se escribe desde el teclado.
			\item \hyperlink{tTipoRN}{Tipo:} Se selecciona de una lista.
			\item Para el tipo ''Comparación de atributos'':
				\begin{itemize}
					\item \cdtRef{entidadEntidad}{Entidad 1}: Se selecciona de una lista.
					\item \cdtRef{entidadAtributo}{Atributo 1}: Se selecciona de una lista.
					\item Operador: Se selecciona de una lista.
					\item \cdtRef{entidadEntidad}{Entidad 2}: Se selecciona de una lista.
					\item \cdtRef{entidadAtributo}{Atributo 2}: Se selecciona de una lista.
				\end{itemize}
			\item Para el tipo ''Unicidad de párametros'':
				\begin{itemize}
					\item \cdtRef{entidadEntidad}{Entidad}: Se escribe desde el teclado.
					\item \cdtRef{entidadAtributo}{Atributo único}: Se escribe desde el teclado.
				\end{itemize}
			\item Para el tipo ''Formato correcto'':
				\begin{itemize}
					\item \cdtRef{entidadEntidad}{Entidad que contiene el atributo para verificar el formato}: Se selecciona de una lista.
					\item \cdtRef{entidadAtributo}{Atributo que se verificará con la expresión regular}: Se selecciona de una lista.
					\item Expresión regular: Se escribe desde el teclado.
				\end{itemize}
		\end{itemize}	
		}
		\UCitem{Salidas}{\begin{itemize}
				\item \cdtRef{BREntidad:claveBR}{Clave:} Lo calcula el sistma mediante la regla de negocio \BRref{RN12}{Idenficador de elemento}.
				\item \cdtRef{BRSEntidad:numeroBR}{Número:} Lo obtiene el sistema.
				\item \cdtRef{BREntidad:nombreBR}{Nombre:} Lo obtiene el sistema.
				\item \cdtRef{BREntidad:descripciónBR}{Descripción:} Lo obtiene el sistema.
				\item \cdtRef{BREntidad:redaccionBR}{Redacción:} Lo obtiene el sistema.
				\item \hyperlink{tTipoRN}{Tipo:} Lo obtiene el sistema.
				\item Para el tipo ''Comparación de atributos'':
				\begin{itemize}
					\item \cdtRef{entidadEntidad}{Entidad 1}: Lo obtiene el sistema.
					\item \cdtRef{entidadAtributo}{Atributo 1}: Lo obtiene el sistema.
					\item Operador: Lo obtiene el sistema.
					\item \cdtRef{entidadEntidad}{Entidad 2}: Lo obtiene el sistema.
					\item \cdtRef{entidadAtributo}{Atributo 2}: Lo obtiene el sistema.
				\end{itemize}
				\item Para el tipo ''Unicidad de párametros'':
				\begin{itemize}
					\item \cdtRef{entidadEntidad}{Entidad}: Lo obtiene el sistema.
					\item \cdtRef{entidadAtributo}{Atributo único}: Lo obtiene el sistema.
				\end{itemize}
				\item Para el tipo ''Formato correcto'':
				\begin{itemize}
					\item \cdtRef{entidadEntidad}{Entidad que contiene el atributo para verificar el formato}: Lo obtiene el sistema.
					\item \cdtRef{entidadAtributo}{Atributo que se verificará con la expresión regular}: Lo obtiene el sistema.
					\item Expresión regular: Lo obtiene el sistema.
				\end{itemize}
				\item \cdtIdRef{MSG1}{Operación exitosa}: Se muestra en la pantalla \IUref{IU9}{Gestionar Reglas de negocio} para indicar que la modificación fue exitosa.
		\end{itemize}}
		\UCitem{Destino}{Pantalla}
		\UCitem{Precondiciones}{
			\begin{itemize}
				\item Que la regla de negocio no se encuentre asociada a un caso de uso en estado ''Liberado''.
			\end{itemize}
		}
		\UCitem{Postcondiciones}{Ninguna}
		\UCitem{Errores}{\begin{itemize}
		\item \cdtIdRef{MSG4}{Dato obligatorio}: Se muestra en la pantalla \IUref{IU9.2}{Modificar Regla de negocio} cuando no se ha ingresado un dato marcado como obligatorio.
		\item \cdtIdRef{MSG29}{Formato incorrecto}: Se muestra en la pantalla \IUref{IU9.1}{Registrar Regla de negocio} cuando el tipo de dato ingresado no cumple con el tipo de dato solicitado en el campo.
		\item \cdtIdRef{MSG5}{Formato de campo incorrecto}: Se muestra en la pantalla \IUref{IU9.1}{Registrar Regla de negocio} cuando el número de la regla de negocio contiene un carácter no válido.
		\item \cdtIdRef{MSG6}{Longitud inválida}: Se muestra en la pantalla \IUref{IU9.2}{Modificar Regla de negocio} cuando se ha excedido la longitud de alguno de los campos.
		\item \cdtIdRef{MSG7}{Registro repetido}: Se muestra en la pantalla \IUref{IU9.2}{Modificar Regla de negocio} cuando se registre un regla de negocio con un nombre o número que ya se encuentre registrado en el sistema.
		\item \cdtIdRef{MSG18}{Caracteres inválidos}: Se muestra en la pantalla \IUref{IU9.2}{Modificar Regla de negocio} cuando el nombre de la regla de negocio contiene un carácter no válido.
		\end{itemize}
		}
		\UCitem{Tipo}{Secundario, extiende del caso de uso \UCref{CU8}{Gestionar Reglas de negocio}.}
	\end{UseCase}
%--------------------------------------
	\begin{UCtrayectoria}
		\UCpaso[\UCactor] Solicita registrar un término oprimiendo el botón \IUbutton{Registrar} del registro que desea modificar en la pantalla \IUref{IU9}{Gestionar Reglas de negocio}.
		\UCpaso[\UCsist] Obtiene la información de la regla de negocio.
		\UCpaso[\UCsist] Verifica que la regla de negocio pueda modificarse con base en la regla de negocio \BRref{RN5}{Modificación de elementos asociados a casos de uso liberados}. \Trayref{MBR-I}
		\UCpaso[\UCsist] Muestra la pantalla \IUref{IU9.2}{Modificar Regla de negocio}.
		\UCpaso[\UCactor] Modifica la información solicitada en la pantalla. \label{CU8.2-P5}
		\UCpaso[\UCactor] Selecciona el tipo de regla de negocio.
		\UCpaso[\UCsist] Verifica que el tipo de regla de negocio requiera parámetros. \Trayref{RBR-A}
		\UCpaso[\UCsist] Muestra la pantalla \IUref{IU9.2A}{Modificar Regla de negocio: Comparación de atributos}, \IUref{IU9.2B}{Modificar Regla de negocio: Unicidad de parámetros} o \IUref{IU9.2C}{Modificar Regla de negocio: Formato correcto}, según corresponda.
		\UCpaso[\UCsist] Ingresa la información solicitada en la pantalla correspondiente.
		\UCpaso[\UCsist] Ingresa la redacción de la regla de negocio. \label{CU8.2-P11}
		\UCpaso[\UCactor] Solicita guardar los cambio de la regla de negocio oprimiendo el botón \IUbutton{Aceptar} de la pantalla \IUref{IU9.2A}{Modificar Regla de negocio: Comparación de atributos}, \IUref{IU9.2B}{Modificar Regla de negocio: Unicidad de parámetros} o \IUref{IU9.2C}{Modificar Regla de negocio: Formato correcto}, según corresponda. \Trayref{MBR-B}
		\UCpaso[\UCsist] Verifica que el actor ingrese todos los campos obligatorios con base en la regla de negocio \BRref{RN8}{Datos obligatorios}. \Trayref{MBR-C}
		\UCpaso[\UCsist] Verifica que los datos requeridos sean proporcionados correctamente con base en la regla de negocio \BRref{RN7}{Información correcta}. \Trayref{MBR-D} \Trayref{MBR-E}
		\UCpaso[\UCsist] Verifica que el número de la regla de negocio sea proporcionado correctamente con base en la regla de negocio \BRref{RN7}{Información correcta}. \Trayref{RBR-F}
		\UCpaso[\UCsist] Verifica que el número de la regla de negocio no se encuentre registrado con base en la regla de negocio \BRref{RN1}{Unicidad de números}. \Trayref{MBR-G}
		\UCpaso[\UCsist] Verifica que el nombre de la regla de negocio no se encuentre registrado en el sistema con base en la regla de negocio \BRref{RN6}{Unicidad de nombres}. \Trayref{RBR-H}
		\UCpaso[\UCsist] Registra los cambios de la regla de negocio en el sistema.
		\UCpaso[\UCsist] Muestra el mensaje \cdtIdRef{MSG1}{Operación exitosa} en la pantalla \IUref{IU9}{Gestionar Reglas de negocio} para indicar al actor que la modificación se ha realizado exitosamente.
	\end{UCtrayectoria}		
%--------------------------------------
	
	\begin{UCtrayectoriaA}{MBR-A}{La regla de negocio no requiere parámetros..}
		\UCpaso[\UCactor] Continúa con el paso \ref{CU8.2-P11} de la trayectoria principal.
	\end{UCtrayectoriaA}
	
	\begin{UCtrayectoriaA}{MBR-B}{El actor desea cancelar la operación.}
		\UCpaso[\UCactor] Solicita cancelar la operación oprimiendo el botón \IUbutton{Cancelar} de la pantalla \IUref{IU9.2}{Modificar Regla de negocio}
		\UCpaso[\UCsist] Muestra la pantalla \IUref{IU9}{Gestionar Reglas de negocio}.
	\end{UCtrayectoriaA}

	\begin{UCtrayectoriaA}{MBR-C}{El actor no ingresó algún dato marcado como obligatorio.}
		\UCpaso[\UCsist] Muestra el mensaje \cdtIdRef{MSG4}{Dato obligatorio} y señala el campo que presenta el error en la pantalla \IUref{IU9.2}{Modificar Regla de negocio}, indicando al actor que el dato es obligatorio.
		\UCpaso Regresa al paso \ref{CU8.2-P5} de la trayectoria principal.
	\end{UCtrayectoriaA}

	\begin{UCtrayectoriaA}{MBR-D}{El actor proporciona un dato que excede la longitud máxima.}
		\UCpaso[\UCsist] Muestra el mensaje \cdtIdRef{MSG6}{Longitud inválida} y señala el campo que excede la longitud en la pantalla \IUref{IU9.2}{Modificar Regla de negocio}, para indicar que el dato excede el tamaño máximo permitido.
		\UCpaso Regresa al paso \ref{CU8.2-P5} de la trayectoria principal.
	\end{UCtrayectoriaA}

	\begin{UCtrayectoriaA}{MBR-E}{El actor ingresó un tipo de dato incorrecto.}
		\UCpaso[\UCsist] Muestra el mensaje \cdtIdRef{MSG29}{Formato incorrecto} y señala el campo que presenta el dato inválido en la pantalla \IUref{IU9.1}{Registrar Regla de negocio}, para indicar que se ha ingresado un tipo de dato inválido.
		\UCpaso Regresa al paso \ref{CU8.2-P5} de la trayectoria principal.
	\end{UCtrayectoriaA}

	\begin{UCtrayectoriaA}{MBR-F}{El actor ingresó un número de regla negocio con un tipo de dato incorrecto.}
		\UCpaso[\UCsist] Muestra el mensaje \cdtIdRef{MSG5}{Formato de campo incorrecto} y señala el campo que presenta el dato inválido en la pantalla \IUref{IU9.1}{Registrar Regla de negocio}, para indicar que se ha ingresado un tipo de dato inválido.
		\UCpaso Regresa al paso \ref{CU8.1-P3} de la trayectoria principal.
	\end{UCtrayectoriaA}
	
	\begin{UCtrayectoriaA}{MBR-G}{El actor ingresó un número de regla de negocio repetido.}
		\UCpaso[\UCsist] Muestra el mensaje \cdtIdRef{MSG7}{Registro repetido} y señala el campo que presenta la duplicidad en
		la pantalla \IUref{IU9.2}{Modificar Regla de negocio}, indicando al actor que existe una regla de negocio con el mismo número.
		\UCpaso Regresa al paso \ref{CU8.2-P5} de la trayectoria principal.
	\end{UCtrayectoriaA}
	
	\begin{UCtrayectoriaA}{MBR-H}{El actor ingresó un nombre de regla de negocio repetido.}
		\UCpaso[\UCsist] Muestra el mensaje \cdtIdRef{MSG7}{Registro repetido} y señala el campo que presenta la duplicidad en la pantalla \IUref{IU9.2}{Modificar Regla de negocio}, indicando al actor que existe una regla de negocio con el mismo nombre.
		\UCpaso Regresa al paso \ref{CU8.2-P5} de la trayectoria principal.
	\end{UCtrayectoriaA}

	\begin{UCtrayectoriaA}{MBR-I}{La regla de negocio no puede modificarse debido a que se encuentra asociada a casos de uso liberados.}
		\UCpaso[\UCsist] Oculta el botón \editar de la regla de negocio que esta asociada a casos de uso liberados.
	\end{UCtrayectoriaA}
