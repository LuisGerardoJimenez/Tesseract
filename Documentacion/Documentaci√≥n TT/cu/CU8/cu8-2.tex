	\begin{UseCase}{CU8.2}{Modificar Regla de negocio}{
			
				Después de haber registrado una \hyperlink{BREntidad}{Regla de Negocio} y el colaborador (\hyperlink{jefe}{Líder de Análisis} o \hyperlink{analista}{Analista}) requiera modificar su información, el sistema le permitirá editar cualquiera de los datos previamente registrados (a excepción de la clave) mediante un formulario con los datos precargados de la última actualización para modificarlos y posteriormente guardarlos.\\
				
			    La única condición para modificar una Regla de negocio exitosamente es que esta no se encuentre asociada a algún caso de uso con estado ”Liberado”.
		
	}
		
		\UCitem{Actor}{\hyperlink{jefe}{Líder de Análisis}, \hyperlink{analista}{Analista}}
		\UCitem{Propósito}{Modificar la información de una regla de negocio.}
		\UCitem{Entradas}{
		\begin{itemize}
			\item \cdtRef{BRSEntidad:numeroBR}{Número de la Regla de Negocio:} Se escribe desde el teclado.
			\item \cdtRef{BREntidad:nombreBR}{Nombre de la Regla de Negocio:} Se escribe desde el teclado.
			\item \cdtRef{BREntidad:descripciónBR}{Descripción de la Regla de Negocio:} Se escribe desde el teclado.
			\item \cdtRef{BREntidad:redaccionBR}{Redacción de la Regla de negocio:} Se escribe desde el teclado.
			\item \hyperlink{tTipoRN}{Tipo:} Se selecciona de una lista.
			\item Para el tipo ''Comparación de atributos'':
				\begin{itemize}
					\item \cdtRef{entidadEntidad}{Entidad 1}: Se selecciona de una lista.
					\item \cdtRef{entidadAtributo}{Atributo 1}: Se selecciona de una lista.
					\item Operador: Se selecciona de una lista.
					\item \cdtRef{entidadEntidad}{Entidad 2}: Se selecciona de una lista.
					\item \cdtRef{entidadAtributo}{Atributo 2}: Se selecciona de una lista.
				\end{itemize}
			\item Para el tipo ''Unicidad de párametros'':
				\begin{itemize}
					\item \cdtRef{entidadEntidad}{Entidad}: Se escribe desde el teclado.
					\item \cdtRef{entidadAtributo}{Atributo único}: Se escribe desde el teclado.
				\end{itemize}
			\item Para el tipo ''Formato correcto'':
				\begin{itemize}
					\item \cdtRef{entidadEntidad}{Entidad que contiene el atributo para verificar el formato}: Se selecciona de una lista.
					\item \cdtRef{entidadAtributo}{Atributo que se verificará con la expresión regular}: Se selecciona de una lista.
					\item Expresión regular: Se escribe desde el teclado.
				\end{itemize}
		\end{itemize}	
		}
		\UCitem{Salidas}{\begin{itemize}
				\item \cdtRef{proyectoEntidad:claveProyecto}{Clave del proyecto:} Lo obtiene el sistema.
				\item \cdtRef{proyectoEntidad:nombreProyecto}{Nombre del proyecto:} Lo obtiene el sistema.
				\item \cdtRef{BREntidad:claveBR}{Clave de la Regla de negocio:} Lo calcula el sistma mediante la regla de negocio \BRref{RN12}{Idenficador de elemento}.
				\item \cdtRef{BRSEntidad:numeroBR}{Número de la Regla de negocio:} Lo obtiene el sistema.
				\item \cdtRef{BREntidad:nombreBR}{Nombre de la Regla de negocio:} Lo obtiene el sistema.
				\item \cdtRef{BREntidad:descripciónBR}{Descripción de la Regla de negocio:} Lo obtiene el sistema.
				\item \cdtRef{BREntidad:redaccionBR}{Redacción de la Regla de negocio:} Lo obtiene el sistema.
				\item \hyperlink{tTipoRN}{Tipo:} Lo obtiene el sistema.
				\item Para el tipo ''Comparación de atributos'':
				\begin{itemize}
					\item \cdtRef{entidadEntidad}{Entidad 1}: Lo obtiene el sistema.
					\item \cdtRef{entidadAtributo}{Atributo 1}: Lo obtiene el sistema.
					\item Operador: Lo obtiene el sistema.
					\item \cdtRef{entidadEntidad}{Entidad 2}: Lo obtiene el sistema.
					\item \cdtRef{entidadAtributo}{Atributo 2}: Lo obtiene el sistema.
				\end{itemize}
				\item Para el tipo ''Unicidad de párametros'':
				\begin{itemize}
					\item \cdtRef{entidadEntidad}{Entidad}: Lo obtiene el sistema.
					\item \cdtRef{entidadAtributo}{Atributo único}: Lo obtiene el sistema.
				\end{itemize}
				\item Para el tipo ''Formato correcto'':
				\begin{itemize}
					\item \cdtRef{entidadEntidad}{Entidad que contiene el atributo para verificar el formato}: Lo obtiene el sistema.
					\item \cdtRef{entidadAtributo}{Atributo que se verificará con la expresión regular}: Lo obtiene el sistema.
					\item Expresión regular: Lo obtiene el sistema.
				\end{itemize}
				\item \cdtIdRef{MSG1}{Operación exitosa}: Se muestra en la pantalla \IUref{IU9}{Gestionar Reglas de negocio} para indicar que la modificación fue exitosa.
		\end{itemize}}
		
		\UCitem{Precondiciones}{\begin{itemize}
				\item Que exista al menos un regla de negocio registrada en el proyecto
				\item Que la regla de negocio no se encuentre asociada a un caso de uso en estado ''Liberado''.
			\end{itemize}
		}
		\UCitem{Postcondiciones}{Se actualizará una regla de negocio de un proyecto.}
		\UCitem{Errores}{\begin{itemize}
		\item \cdtIdRef{MSG4}{Dato obligatorio}: Se muestra en la pantalla \IUref{IU9.2}{Modificar Regla de negocio} cuando no se ha ingresado un dato marcado como obligatorio.
		\item \cdtIdRef{MSG5}{Formato incorrecto}: Se muestra en la pantalla \IUref{IU9.2}{Modificar Regla de negocio} cuando el tipo de dato ingresado no cumple con el tipo de dato solicitado en el campo.
		\item \cdtIdRef{MSG5}{Formato de campo incorrecto}: Se muestra en la pantalla \IUref{IU9.2}{Modificar Regla de negocio} cuando el número de la regla de negocio contiene un carácter no válido.
		\item \cdtIdRef{MSG6}{Longitud inválida}: Se muestra en la pantalla \IUref{IU9.2}{Modificar Regla de negocio} cuando se ha excedido la longitud de alguno de los campos.
		\item \cdtIdRef{MSG7}{Registro repetido}: Se muestra en la pantalla \IUref{IU9.2}{Modificar Regla de negocio} cuando se registre un regla de negocio con un nombre o número que ya se encuentre registrado en el sistema.
		\item \cdtIdRef{MSG29}{No existe información necesaria en el sistema}: Se muestra en la pantalla \IUref{IU9}{Gestionar Reglas de negocio} cuando no exista información en el catálogo ''tipo de regla de negocio''.
		\end{itemize}
		}
		\UCitem{Tipo}{Secundario, extiende del caso de uso \UCref{CU8}{Gestionar Reglas de negocio}.}
	\end{UseCase}
%--------------------------------------
	\begin{UCtrayectoria}
		\UCpaso[\UCactor] Da clic en el icono \editar del registro que desea modificar en la pantalla \IUref{IU9}{Gestionar Reglas de negocio}.
		\UCpaso[\UCsist] Verifica que el catálogo ''tipo de regla de negocio'' cuente con información, con base en la regla de negocio \BRref{RN20}{Verificación de catálogos}. \hyperlink{CU8-2:TAA}{[Trayectoria A]}
		\UCpaso[\UCsist] Obtiene la información de la regla de negocio.
		\UCpaso[\UCsist] Muestra la pantalla \IUref{IU9.2}{Modificar Regla de negocio}.
		\UCpaso[\UCactor] Modifica la información de la regla de negocio. \label{CU8.2-P5}
		\UCpaso[\UCactor] Selecciona el tipo de regla de negocio.
		\UCpaso[\UCsist] Verifica que el tipo de regla de negocio requiera parámetros. \hyperlink{CU8-2:TAB}{[Trayectoria B]}
		\UCpaso[\UCsist] Muestra la pantalla \IUref{IU9.2A}{Modificar Regla de negocio: Comparación de atributos}, \IUref{IU9.2B}{Modificar Regla de negocio: Unicidad de parámetros} o \IUref{IU9.2C}{Modificar Regla de negocio: Formato correcto}, según corresponda.
		\UCpaso[\UCactor] Modifica la información solicitada en la pantalla correspondiente.
		\UCpaso[\UCactor] Modifica la redacción de la regla de negocio. \label{CU8.2-P11}
		\UCpaso[\UCactor] Oprime el botón \IUbutton{Aceptar}. \hyperlink{CU8-2:TAC}{[Trayectoria C]}
		\UCpaso[\UCsist] Verifica que el actor ingrese todos los campos obligatorios con base en la regla de negocio \BRref{RN8}{Datos obligatorios}. \hyperlink{CU8-2:TAD}{[Trayectoria D]}
		\UCpaso[\UCsist] Verifica que los datos ingresados cumpla con la longitud correcta, con base en la regla de negocio \BRref{RN37}{Longitud de datos}. \hyperlink{CU8-2:TAE}{[Trayectoria E]}
		\UCpaso[\UCsist] Verifica que los datos ingresados cumplan con el formato requerido, con base en la regla de negocio \BRref{RN7}{Información correcta}. \hyperlink{CU8-2:TAF}{[Trayectoria F]}
		\UCpaso[\UCsist] Verifica que el nombre de la regla de negocio no se encuentre registrado en el sistema con base en la regla de negocio \BRref{RN6}{Unicidad de nombres}. \hyperlink{CU8-2:TAG}{[Trayectoria G]}
		\UCpaso[\UCsist] Actualiza la información de la regla de negocio en el sistema.
		\UCpaso[\UCsist] Muestra el mensaje \cdtIdRef{MSG1}{Operación exitosa} en la pantalla \IUref{IU9}{Gestionar Reglas de negocio} para indicar al actor que la modificación se ha realizado exitosamente.
	\end{UCtrayectoria}		
%--------------------------------------
\hypertarget{CU8-2:TAA}{\textbf{Trayectoria alternativa A}}\\
\noindent \textbf{Condición:} El catálogo tipo de regla de negocio no tiene información.
\begin{enumerate}
	\UCpaso[\UCsist] Muestra el mensaje \cdtIdRef{MSG29}{No existe información necesaria en el sistema} en la pantalla \IUref{IU9}{Gestionar Reglas de negocio} para indicar que no es posible realizar la operación debido a la falta de información necesaria para el sistema.
	\item[- -] - - {\em {Fin del caso de uso}}.%
\end{enumerate}
%--------------------------------------
\hypertarget{CU8-2:TAB}{\textbf{Trayectoria alternativa B}}\\
\noindent \textbf{Condición:} La regla de negocio no requiere parámetros.
\begin{enumerate}
	\UCpaso[\UCactor] Continúa con el paso \ref{CU8.2-P11} de la trayectoria principal.
	\item[- -] - - {\em {Fin de la trayectoria}}.%
\end{enumerate}

%--------------------------------------	
\hypertarget{CU8-2:TAC}{\textbf{Trayectoria alternativa C}}\\
\noindent \textbf{Condición:} El actor desea cancelar la operación.
\begin{enumerate}
	\UCpaso[\UCactor] Solicita cancelar la operación oprimiendo el botón \IUbutton{Cancelar} de la pantalla \IUref{IU9.2}{Modificar Regla de negocio}
	\UCpaso[\UCsist] Muestra la pantalla \IUref{IU9}{Gestionar Reglas de negocio}.
	\item[- -] - - {\em {Fin del caso de uso}}.%
\end{enumerate}
%--------------------------------------
\hypertarget{CU8-2:TAD}{\textbf{Trayectoria alternativa D}}\\
\noindent \textbf{Condición:} El actor no ingresó algún dato marcado como obligatorio.
\begin{enumerate}
	\UCpaso[\UCsist] Muestra el mensaje \cdtIdRef{MSG4}{Dato obligatorio} señalando el campo que presenta el error en la pantalla \IUref{IU9.2}{Modificar Regla de negocio}.
	\UCpaso Regresa al paso \ref{CU8.2-P5} de la trayectoria principal.
	\item[- -] - - {\em {Fin de la trayectoria}}.%
\end{enumerate}
%--------------------------------------
\hypertarget{CU8-2:TAE}{\textbf{Trayectoria alternativa E}}\\
\noindent \textbf{Condición:} El actor ingresó un dato con un número de caracteres fuera del rango permitido.
\begin{enumerate}
	\UCpaso[\UCsist] Muestra el mensaje \cdtIdRef{MSG6}{Longitud inválida} señalando el campo que presenta el error en la pantalla \IUref{IU9.2}{Modificar Regla de negocio}.
	\UCpaso Regresa al paso \ref{CU8.2-P5} de la trayectoria principal.
	\item[- -] - - {\em {Fin de la trayectoria}}.%
\end{enumerate}
%--------------------------------------	
\hypertarget{CU8-2:TAF}{\textbf{Trayectoria alternativa F}}\\
\noindent \textbf{Condición:} El actor ingresó un dato con un formato o tipo de dato incorrecto.
\begin{enumerate}
	\UCpaso[\UCsist] Muestra el mensaje \cdtIdRef{MSG5}{Formato incorrecto} señalando el campo que presenta el error en la pantalla \IUref{IU9.2}{Modificar Regla de negocio}.
	\UCpaso Regresa al paso \ref{CU8.2-P5} de la trayectoria principal.
	\item[- -] - - {\em {Fin de la trayectoria}}.
\end{enumerate}
%--------------------------------------	
\hypertarget{CU8-2:TAG}{\textbf{Trayectoria alternativa G}}\\
\noindent \textbf{Condición:} El actor ingresó un nombre de regla de negocio repetido.
\begin{enumerate}
	\UCpaso[\UCsist] Muestra el mensaje \cdtIdRef{MSG7}{Registro repetido} señalando el campo que presenta la duplicidad en la pantalla \IUref{IU9.2}{Modificar Regla de negocio}.
	\UCpaso Regresa al paso \ref{CU8.2-P5} de la trayectoria principal.
	\item[- -] - - {\em {Fin de la trayectoria}}.
\end{enumerate}
