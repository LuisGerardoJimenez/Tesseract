	\begin{UseCase}{CU12.2}{Modificar Caso de uso}{
			
		Después de haber registrado un \hyperlink{casoUso}{Caso de Uso} en un \hyperlink{moduloEntidad}{Módulo} del \hyperlink{proyectoEntidad}{Proyecto} y el colaborador (\hyperlink{jefe}{Líder de Análisis} o \hyperlink{analista}{Analista}) requiera modificar su contenido, el sistema le permitirá editar cualquiera de los datos previamente registrados mediante un formulario, este formulario contendrá los datos precargados de la última actualización para poder corregirlos y posteriormente guardarlos.\\
		%Dentro de este caso de uso se le permitirá también al colaborador gestionar las precondiciones y postcondiciones propios del Caso de uso (de tal forma que podrá registrarlas y eliminarlas).\\
	    Al igual que en el registro, el colaborador podrá hacer uso de tokens que le mostrará una lista de \hyperlink{tElemento}{elementos} disponibles para su utilización.
		
		Para que esta acción se pueda llevar a cabo es necesario que el caso de uso se encuentre en estado de Edición o Pendiente de corrección con base en el \hyperlink{edoCU}{Modelo de estados del Caso de Uso}.

}

		\UCitem{Actor}{\hyperlink{jefe}{Líder de Análisis}, \hyperlink{analista}{Analista}}
		\UCitem{Propósito}{Modificar la información de un caso de uso.}
		\UCitem{Entradas}{
		\begin{itemize}
			\item De la sección \textbf{Información general del caso de uso}.
			\begin{itemize}
				\item \cdtRef{casoUso:numeroCU}{Número}: Se escribe desde el teclado. 
				\item \cdtRef{casoUso:nombreCU}{Nombre}: Se escribe desde el teclado.
				\item \cdtRef{casoUso:resumenCU}{Resumen}: Se escribe desde el teclado.
			\end{itemize}
			\item De la sección \textbf{Descripción del caso de uso}:
			\begin{itemize}
				\item \cdtRef{actorEntidad}{Actores}: Se escribe una palabra y se selecciona una sugerencia de la lista.
				\item Entradas: Se escribe una palabra y se selecciona una sugerencia de la lista.
				\item Salidas: Se escribe una palabra y se selecciona una sugerencia de la lista.
				\item \cdtRef{BREntidad}{Reglas de Negocio}: Se escribe una palabra y se selecciona una sugerencia de la lista.
			\end{itemize}
		\end{itemize}	
		}
		\UCitem{Salidas}{\begin{itemize}
				\item \cdtRef{proyectoEntidad:claveProyecto}{Clave del proyecto}: Lo obtiene el sistema.
				\item \cdtRef{proyectoEntidad:nombreProyecto}{Nombre del proyecto}: Lo obtiene el sistema.
				\item \cdtRef{moduloEntidad:claveModulo}{Clave del Módulo}: Lo obtiene el sistema.
				\item \cdtRef{moduloEntidad:nombreModulo}{Nombre del Módulo}: Lo obtiene el sistema.
				\item \cdtRef{casoUso:claveCU}{Clave}: Lo calcula el sistema mediante la regla de negocio \BRref{RN12}{Identificador de elemento}.
				\item De la sección \textbf{Información general del caso de uso}.
				\begin{itemize}
					\item \cdtRef{casoUso:numeroCU}{Número}: Lo obtiene el sistema.
					\item \cdtRef{casoUso:nombreCU}{Nombre}: Lo obtiene el sistema.
					\item \cdtRef{casoUso:resumenCU}{Resumen}: Lo obtiene el sistema.
				\end{itemize}
				\item De la sección \textbf{Descripción del caso de uso}:
				\begin{itemize}
					\item \cdtRef{actorEntidad}{Actores}: Lo obtiene el sistema.
					\item Entradas: Lo obtiene el sistema.
					\item Salidas: Lo obtiene el sistema.
					\item \cdtRef{BREntidad}{Reglas de Negocio}: Lo obtiene el sistema.
				\end{itemize}
				\item \cdtIdRef{MSG1}{Operación exitosa}: Se muestra en la pantalla \IUref{IU6}{Gestionar Casos de uso} para indicar que la modificación fue exitosa.
		\end{itemize}}
		\UCitem{Precondiciones}{
			\begin{itemize}
				\item Que exista al menos un caso de uso registrado en el módulo.
				\item Que el caso de uso se encuentre en estado ''Edición'' o ''Pendiente de corrección".
			\end{itemize}
		}
		\UCitem{Postcondiciones}{
		\begin{itemize}
			\item Se modificará la información del caso de uso en el sistema.
			\item El caso de uso pasará a estado ''Edición''.
		\end{itemize}
		}
		\UCitem{Errores}{\begin{itemize}
		\item \cdtIdRef{MSG4}{Dato obligatorio}: Se muestra en la pantalla \IUref{IU6.2}{Modificar Caso de uso} cuando no se ha ingresado un dato marcado como obligatorio.
		\item \cdtIdRef{MSG5}{Formato incorrecto}: Se muestra en la pantalla \IUref{IU6.2}{Modificar Caso de uso} cuando el tipo de dato ingresado no cumple con el tipo de dato solicitado en el campo.
		\item \cdtIdRef{MSG6}{Longitud inválida}: Se muestra en la pantalla \IUref{IU6.2}{Modificar Caso de uso} cuando se ha excedido la longitud de alguno de los campos.
		\item \cdtIdRef{MSG7}{Registro repetido}: Se muestra en la pantalla \IUref{IU6.2}{Modificar Caso de uso} cuando se registre un caso de uso con un nombre que ya se encuentre registrado en el sistema.
		\end{itemize}.
		}
		\UCitem{Tipo}{Secundario, extiende del caso de uso \UCref{CU12}{Gestionar Casos de uso}.}
	\end{UseCase}
%--------------------------------------
	\begin{UCtrayectoria}
		\UCpaso[\UCactor] Da clic en el icono \editar del registro que desee de la pantalla \IUref{IU6}{Gestionar Casos de uso}.
		\UCpaso[\UCsist] Obtiene la información del caso de uso seleccionado.
		\UCpaso[\UCsist] Muestra la pantalla \IUref{IU6.2}{Modificar Caso de uso}.\label{CU12.2-P4}
		\UCpaso[\UCsist] Marca como etiqueta el campo ''Clave'' de acuerdo a la regla de negocio \BRref{RN13}{Modificación del identificador}.
		\UCpaso[\UCactor] Modifica la información general del caso de uso. \label{CU12.2-P5}
		\UCpaso[\UCactor] Modifica la descripción del caso de uso. \label{CU12.2-P6} \hyperlink{CU12-2:TAA}{[Trayectoria A]}
		\UCpaso[\UCactor] Oprime el botón \IUbutton{Aceptar}. \label{CU12.2-P9} \hyperlink{CU12-2:TAB}{[Trayectoria B]}
		\UCpaso[\UCsist] Verifica que el actor ingrese todos los campos obligatorios con base en la regla de negocio \BRref{RN8}{Datos obligatorios}. \hyperlink{CU12-2:TAC}{[Trayectoria C]}
		\UCpaso[\UCsist] Verifica que los datos ingresados cumpla con la longitud correcta, con base en la regla de negocio \BRref{RN37}{Longitud de datos}. \hyperlink{CU12-2:TAD}{[Trayectoria D]}
		\UCpaso[\UCsist] Verifica que los datos ingresados cumplan con el formato requerido, con base en la regla de negocio \BRref{RN7}{Información correcta}. \hyperlink{CU12-2:TAE}{[Trayectoria E]}
		\UCpaso[\UCsist] Verifica que el número del caso de uso no se encuentre registrado en el sistema con base en la regla de negocio \BRref{RN29}{Unicidad de casos de uso}. \hyperlink{CU12-2:TAF}{[Trayectoria F]}
		\UCpaso[\UCsist] Verifica que el nombre del caso de uso no se encuentre registrado en el sistema con base en la regla de negocio \BRref{RN6}{Unicidad de nombres}. \hyperlink{CU12-2:TAG}{[Trayectoria G]}
		\UCpaso[\UCsist] Actualiza la información del caso de uso.
		\UCpaso[\UCsist] Cambia el estado del caso de uso a ''Edición''.
		\UCpaso[\UCsist] Muestra el mensaje \cdtIdRef{MSG1}{Operación exitosa} en la pantalla \IUref{IU6}{Gestionar Casos de uso} para indicar al actor que la modificación se ha realizado exitosamente.
	\end{UCtrayectoria}		
%--------------------------------------
\hypertarget{CU12-2:TAA}{\textbf{Trayectoria alternativa A}}\\
\noindent \textbf{Condición:} El actor desea hacer referencia a un elemento existente en el proyecto.
\begin{enumerate}
	\UCpaso[\UCactor] Ingresa el token correspondiente al elemento a referenciar.
	\UCpaso[\UCsist] Verifica que los tokens utilizados se encuentren correctamente estructurados, con base en la regla de negocio \BRref{RN31}{Estructura de Tokens}.
	\UCpaso[\UCsist] Obtiene los \hyperlink{tElemento}{elementos} registrados en el proyecto correspondientes al token ingresado. 
	\UCpaso[\UCsist] Muestra una lista con los \hyperlink{tElemento}{elementos} encontrados.
	\UCpaso[\UCactor] Selecciona un elemento de la lista.
	\UCpaso[\UCsist] Verifica que el nombre del elemento seleccionado no contenga espacios. \hyperlink{CU12-2:TAH}{[Trayectoria H]}
	\UCpaso[\UCsist] Agrega la referencia del elemento al texto. \label{CU12.2-TA1}
	\UCpaso Continúa en el paso \ref{CU12.2-P9} de la trayectoria principal.
	\item[- -] - - {\em {Fin de la trayectoria}}.%
\end{enumerate}
%--------------------------------------
\hypertarget{CU12-2:TAB}{\textbf{Trayectoria alternativa B}}\\
\noindent \textbf{Condición:} El actor desea cancelar la operación.
\begin{enumerate}
	\UCpaso[\UCactor] Solicita cancelar la operación oprimiendo el botón \IUbutton{Cancelar} de la pantalla \IUref{IU6.2}{Modificar Caso de uso}.
	\UCpaso[\UCsist] Muestra la pantalla \IUref{IU6}{Gestionar Casos de uso}.
	\item[- -] - - {\em {Fin del caso de uso}}.%
\end{enumerate}
%--------------------------------------
\hypertarget{CU12-2:TAC}{\textbf{Trayectoria alternativa C}}\\
\noindent \textbf{Condición:} El actor no ingresó algún dato marcado como obligatorio.
\begin{enumerate}
	\UCpaso[\UCsist] Muestra el mensaje \cdtIdRef{MSG4}{Dato obligatorio} señalando el campo que presenta el error en la pantalla \IUref{IU6.2}{Modificar Caso de uso}.
	\UCpaso Regresa al paso \ref{CU12.2-P6} de la trayectoria principal.
	\item[- -] - - {\em {Fin de la trayectoria}}.%
\end{enumerate}
%--------------------------------------
\hypertarget{CU12-2:TAD}{\textbf{Trayectoria alternativa D}}\\
\noindent \textbf{Condición:} El actor ingresó un dato con un número de caracteres fuera del rango permitido.
\begin{enumerate}
	\UCpaso[\UCsist] Muestra el mensaje \cdtIdRef{MSG6}{Longitud inválida} señalando el campo que presenta el error en la pantalla \IUref{IU6.2}{Modificar Caso de uso}.
	\UCpaso Regresa al paso \ref{CU12.2-P6} de la trayectoria principal.
	\item[- -] - - {\em {Fin de la trayectoria}}.%
\end{enumerate}
%--------------------------------------
\hypertarget{CU12-2:TAE}{\textbf{Trayectoria alternativa E}}\\
\noindent \textbf{Condición:} El actor ingresó un dato con un formato incorrecto.
\begin{enumerate}
	\UCpaso[\UCsist] Muestra el mensaje \cdtIdRef{MSG5}{Formato incorrecto} señalando el campo que presenta el error en la pantalla \IUref{IU6.2}{Modificar Caso de uso}.
	\UCpaso Regresa al paso \ref{CU12.2-P6} de la trayectoria principal.
	\item[- -] - - {\em {Fin de la trayectoria}}.
\end{enumerate}
%--------------------------------------
\hypertarget{CU12-2:TAF}{\textbf{Trayectoria alternativa F}}\\
\noindent \textbf{Condición:} El actor ingresó un número de un caso de uso repetido.
\begin{enumerate}
	\UCpaso[\UCsist] Muestra el mensaje \cdtIdRef{MSG7}{Registro repetido} señalando el campo que presenta la duplicidad en la pantalla \IUref{IU6.2}{Modificar Caso de uso}.
	\UCpaso Regresa al paso \ref{CU12.2-P6} de la trayectoria principal.
	\item[- -] - - {\em {Fin de la trayectoria}}.
\end{enumerate}
%--------------------------------------	
\hypertarget{CU12-2:TAG}{\textbf{Trayectoria alternativa G}}\\
\noindent \textbf{Condición:} El actor ingresó un nombre de un caso de uso repetido.
\begin{enumerate}
	\UCpaso[\UCsist] Muestra el mensaje \cdtIdRef{MSG7}{Registro repetido} señalando el campo que presenta la duplicidad en la pantalla \IUref{IU6.2}{Modificar Caso de uso}.
	\UCpaso Regresa al paso \ref{CU12.2-P6} de la trayectoria principal.
	\item[- -] - - {\em {Fin de la trayectoria}}.
\end{enumerate}
%--------------------------------------
\hypertarget{CU12-2:TAH}{\textbf{Trayectoria alternativa H}}\\
\noindent \textbf{Condición:} El texto contiene espacios.
\begin{enumerate}
	\UCpaso[\UCsist] Sustituye los espacios por guiones bajos.
	\UCpaso Continua en el \ref{CU12.2-TA1} de la trayectoria alternativa A.
	\item[- -] - - {\em {Fin de la trayectoria}}.
\end{enumerate}