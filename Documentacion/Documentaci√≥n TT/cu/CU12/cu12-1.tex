	\begin{UseCase}{CU12.1}{Registrar Caso de uso}{
			Este caso de uso permite al actor registrar un caso de uso. Al ingresar los datos, el actor podrá utilizar un token que le desplegará una lista de elementos disponibles para su utilización.
	}
		\UCitem{Actor}{\hyperlink{jefe}{Líder de Análisis}, \hyperlink{analista}{Analista}}
		\UCitem{Propósito}{Registrar la información de un caso de uso.}
		\UCitem{Entradas}{
		\begin{itemize}
			\item De la sección \textbf{Información general del caso de uso}.
			\begin{itemize}
				\item \cdtRef{casoUso:numeroCU}{Número}: Se escribe desde el teclado. 
				\item \cdtRef{casoUso:nombreCU}{Nombre}: Se escribe desde el teclado.
				\item \cdtRef{casoUso:resumenCU}{Resumen}: Se escribe desde el teclado.
			\end{itemize}
			\item De la sección \textbf{Descripción del caso de uso}:
			\begin{itemize}
				\item \cdtRef{actorEntidad}{Actores}: Se escribe una palabra y se selecciona una sugerencia de la lista.
				\item Entradas: Se escribe una palabra y se selecciona una sugerencia de la lista.
				\item Salidas: Se escribe una palabra y se selecciona una sugerencia de la lista.
				\item \cdtRef{BREntidad}{Reglas de Negocio}: Se escribe una palabra y se selecciona una sugerencia de la lista.
			\end{itemize}
		\end{itemize}	
		}
		\UCitem{Salidas}{\begin{itemize}
				\item \cdtRef{proyectoEntidad:claveProyecto}{Clave del proyecto}: Lo obtiene el sistema.
				\item \cdtRef{proyectoEntidad:nombreProyecto}{Nombre del proyecto}: Lo obtiene el sistema.
				\item \cdtRef{moduloEntidad:claveModulo}{Clave del Módulo}: Lo obtiene el sistema.
				\item \cdtRef{moduloEntidad:nombreModulo}{Nombre del Módulo}: Lo obtiene el sistema.
				\item \cdtRef{casoUso:claveCU}{Clave}: Lo calcula el sistema mediante la regla de negocio \BRref{RN12}{Identificador de elemento}.
				\item \cdtIdRef{MSG1}{Operación exitosa}: Se muestra en la pantalla \IUref{IU6}{Gestionar Casos de uso} para indicar que el registro fue exitoso.
		\end{itemize}}
		\UCitem{Precondiciones}{Ninguno.}
		\UCitem{Postcondiciones}{
		\begin{itemize}
			\item Se registrará un nuevo caso de uso en el sistema con estado ''Edición''.
		\end{itemize}
		}
		\UCitem{Errores}{\begin{itemize}
		\item \cdtIdRef{MSG4}{Dato obligatorio}: Se muestra en la pantalla \IUref{IU6.1}{Registrar Caso de uso} cuando no se ha ingresado un dato marcado como obligatorio.
		\item \cdtIdRef{MSG29}{Formato incorrecto}: Se muestra en la pantalla \IUref{IU6.1}{Registrar Caso de uso} cuando el tipo de dato ingresado no cumple con el tipo de dato solicitado en el campo.
		\item \cdtIdRef{MSG6}{Longitud inválida}: Se muestra en la pantalla \IUref{IU6.1}{Registrar Caso de uso} cuando se ha excedido la longitud de alguno de los campos.
		\item \cdtIdRef{MSG7}{Registro repetido}: Se muestra en la pantalla \IUref{IU6.1}{Registrar Caso de uso} cuando se registre un caso de uso con un nombre que ya se encuentre registrado en el sistema.
		\end{itemize}.
		}
		\UCitem{Tipo}{Secundario, extiende del caso de uso \UCref{CU12}{Gestionar Casos de uso}.}
	\end{UseCase}
%--------------------------------------
	\begin{UCtrayectoria}
		\UCpaso[\UCactor] Solicita registrar un caso de uso oprimiendo el botón \IUbutton{Registrar} de la pantalla \IUref{IU6}{Gestionar Casos de uso}.
		\UCpaso[\UCsist] Muestra la pantalla \IUref{IU6.1}{Registrar Caso de uso}.\label{CU12.1-P2}
		\UCpaso[\UCactor] Ingresa la información general del caso de uso. \label{CU12.1-P12}
		\UCpaso[\UCactor] Ingresa la descripción del caso de uso. \hyperlink{CU12-1:TAA}{[Trayectoria A]}
		\UCpaso[\UCactor] Gestiona las precondiciones a través de los botones: \IUbutton{Registrar}, \editar y \eliminar. \label{CU12.1-P17}
		\UCpaso[\UCactor] Gestiona las postcondiciones a través de los botones: \IUbutton{Registrar}, \editar y \eliminar. \label{CU12.1-P18}
		\UCpaso[\UCactor] Oprime el botón \IUbutton{Aceptar}. \label{CU10.1-P5} \hyperlink{CU12-1:TAB}{[Trayectoria B]}
		\UCpaso[\UCsist] Verifica que el actor ingrese todos los campos obligatorios con base en la regla de negocio \BRref{RN8}{Datos obligatorios}. \hyperlink{CU12-1:TAC}{[Trayectoria C]}
		\UCpaso[\UCsist] Verifica que los datos ingresados cumpla con la longitud correcta, con base en la regla de negocio \BRref{RN37}{Longitud de datos}. \hyperlink{CU12-1:TAD}{[Trayectoria D]}
		\UCpaso[\UCsist] Verifica que los datos ingresados cumplan con el formato requerido, con base en la regla de negocio \BRref{RN7}{Información correcta}. \hyperlink{CU12-1:TAE}{[Trayectoria E]}
		\UCpaso[\UCsist] Verifica que el número del caso de uso no se encuentre registrado en el sistema con base en la regla de negocio \BRref{RN29}{Unicidad de casos de uso}. \hyperlink{CU12-1:TAF}{[Trayectoria F]}
		\UCpaso[\UCsist] Verifica que el nombre del caso de uso no se encuentre registrado en el sistema con base en la regla de negocio \BRref{RN6}{Unicidad de nombres}. \hyperlink{CU12-1:TAG}{[Trayectoria G]}
		\UCpaso[\UCsist] Registra el caso de uso con el estado ''Edición''.
		\UCpaso[\UCsist] Muestra el mensaje \cdtIdRef{MSG1}{Operación exitosa} en la pantalla \IUref{IU6}{Gestionar Casos de uso} para indicar al actor que el registro se ha realizado exitosamente.
	\end{UCtrayectoria}		
%--------------------------------------
\hypertarget{CU12-1:TAA}{\textbf{Trayectoria alternativa A}}\\
\noindent \textbf{Condición:} El actor desea hacer referencia a un elemento existente en el proyecto.
\begin{enumerate}
	\UCpaso[\UCactor] Ingresa el token correspondiente al elemento a referenciar.
	\UCpaso[\UCsist] Verifica que los tokens utilizados se encuentren correctamente estructurados, con base en la regla de negocio \BRref{RN31}{Estructura de Tokens}. 
	\UCpaso[\UCsist] Obtiene los elementos registrados en el proyecto correspondientes al token ingresado. 
	\UCpaso[\UCsist] Muestra una lista con los elementos encontrados.
	\UCpaso[\UCactor] Selecciona un elemento de la lista.
	\UCpaso[\UCsist] Verifica que el nombre del elemento seleccionado no contenga espacios. \hyperlink{CU12-1:TAH}{[Trayectoria H]}
	\UCpaso[\UCsist] Agrega la referencia del elemento al texto. \label{CU12.1-TA1}
	\UCpaso Continúa en el paso \ref{CU12.1-P17} de la trayectoria principal.
	\item[- -] - - {\em {Fin de la trayectoria}}.%
\end{enumerate}
%--------------------------------------
\hypertarget{CU12-1:TAB}{\textbf{Trayectoria alternativa B}}\\
\noindent \textbf{Condición:} El actor desea cancelar la operación.
\begin{enumerate}
	\UCpaso[\UCactor] Solicita cancelar la operación oprimiendo el botón \IUbutton{Cancelar} de la pantalla \IUref{IU6.1}{Registrar Caso de uso}.
	\UCpaso[\UCsist] Muestra la pantalla \IUref{IU6}{Gestionar Casos de uso}.
	\item[- -] - - {\em {Fin del caso de uso}}.%
\end{enumerate}
%--------------------------------------
\hypertarget{CU12-1:TAC}{\textbf{Trayectoria alternativa C}}\\
\noindent \textbf{Condición:} El actor no ingresó algún dato marcado como obligatorio.
\begin{enumerate}
	\UCpaso[\UCsist] Muestra el mensaje \cdtIdRef{MSG4}{Dato obligatorio} señalando el campo que presenta el error en la pantalla \IUref{IU6.1}{Registrar Caso de uso}.
	\UCpaso Regresa al paso \ref{CU12.1-P12} de la trayectoria principal.
	\item[- -] - - {\em {Fin de la trayectoria}}.%
\end{enumerate}
%--------------------------------------
\hypertarget{CU12-1:TAD}{\textbf{Trayectoria alternativa D}}\\
\noindent \textbf{Condición:} El actor ingresó un dato con un número de caracteres fuera del rango permitido.
\begin{enumerate}
	\UCpaso[\UCsist] Muestra el mensaje \cdtIdRef{MSG6}{Longitud inválida} señalando el campo que presenta el error en la pantalla \IUref{IU6.1}{Registrar Caso de uso}.
	\UCpaso Regresa al paso \ref{CU12.1-P12} de la trayectoria principal.
	\item[- -] - - {\em {Fin de la trayectoria}}.%
\end{enumerate}
%--------------------------------------
\hypertarget{CU12-1:TAE}{\textbf{Trayectoria alternativa E}}\\
\noindent \textbf{Condición:} El actor ingresó un dato con un formato incorrecto.
\begin{enumerate}
	\UCpaso[\UCsist] Muestra el mensaje \cdtIdRef{MSG29}{Formato incorrecto} señalando el campo que presenta el error en la pantalla \IUref{IU6.1}{Registrar Caso de uso}.
	\UCpaso Regresa al paso \ref{CU12.1-P12} de la trayectoria principal.
	\item[- -] - - {\em {Fin de la trayectoria}}.
\end{enumerate}
%--------------------------------------
\hypertarget{CU12-1:TAF}{\textbf{Trayectoria alternativa F}}\\
\noindent \textbf{Condición:} El actor ingresó un número de un caso de uso repetido.
\begin{enumerate}
	\UCpaso[\UCsist] Muestra el mensaje \cdtIdRef{MSG7}{Registro repetido} señalando el campo que presenta la duplicidad en la pantalla \IUref{IU6.1}{Registrar Caso de uso}.
	\UCpaso Regresa al paso \ref{CU12.1-P12} de la trayectoria principal.
	\item[- -] - - {\em {Fin de la trayectoria}}.
\end{enumerate}
%--------------------------------------	
\hypertarget{CU12-1:TAG}{\textbf{Trayectoria alternativa G}}\\
\noindent \textbf{Condición:} El actor ingresó un nombre de un caso de uso repetido.
\begin{enumerate}
	\UCpaso[\UCsist] Muestra el mensaje \cdtIdRef{MSG7}{Registro repetido} señalando el campo que presenta la duplicidad en la pantalla \IUref{IU6.1}{Registrar Caso de uso}.
	\UCpaso Regresa al paso \ref{CU12.1-P12} de la trayectoria principal.
	\item[- -] - - {\em {Fin de la trayectoria}}.
\end{enumerate}
%--------------------------------------
\hypertarget{CU12-1:TAH}{\textbf{Trayectoria alternativa H}}\\
\noindent \textbf{Condición:} El texto contiene espacios.
\begin{enumerate}
	\UCpaso[\UCsist] Sustituye los espacios por guiones bajos.
	\UCpaso Continua en el \ref{CU12.1-TA1} de la trayectoria alternativa A.
	\item[- -] - - {\em {Fin de la trayectoria}}.
\end{enumerate}

\UCExtenssionPoint{El actor requiere registrar una precondición}{Paso \ref{CU12.1-P2} de la trayectoria principal.}{\UCref{CU12.1.2}{Registrar Precondición}}
\UCExtenssionPoint{El actor requiere eliminar una precondición}{Paso \ref{CU12.1-P2} de la trayectoria principal.}{\UCref{CU12.1.3}{Modificar Precondición}}
\UCExtenssionPoint{El actor requiere registrar una postcondición}{Paso \ref{CU12.1-P2} de la trayectoria principal.}{\UCref{CU12.1.4}{Registrar postcondición}}
\UCExtenssionPoint{El actor requiere eliminar una postcondición}{Paso \ref{CU12.1-P2} de la trayectoria principal.}{\UCref{CU12.1.5}{Eliminar postcondición}}