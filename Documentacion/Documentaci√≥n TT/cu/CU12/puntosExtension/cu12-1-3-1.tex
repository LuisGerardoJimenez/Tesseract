	\begin{UseCase}{CU12.1.3.1}{Registrar Punto de extensión}{
			
			Este caso de uso permite al colaborador (\hyperlink{jefe}{Líder de Análisis} o \hyperlink{analista}{Analista}) registrar la información de un \hyperlink{entidadExtension}{Punto de Extensión} correspondiente a algún \hyperlink{casoUso}{Caso de Uso} del \hyperlink{moduloEntidad}{Módulo} sobre el cual se está operando.\\
			
			A través de un formulario el sistema solicita la información del Punto de Extensión que se desea registrar, a que Caso de Uso extiende, cual es la causa y por último la región de la Trayectoria.\\
			
			Los puntos de extensión describen una región de la trayectoria en la que se puede extender el funcionamiento a través de otro caso de uso. Esta relación significa que un caso de uso puede estar basado en otro caso de uso más básico. El término basado-en es usado para indicar que un caso de uso es una especialización de un caso de uso más general. De esta manera solamente la diferencia tendrá que ser especificada en el caso de uso especializado. Esta relación es una forma extendida de la relación de herencia.\\
			
			Una vez registrado el Punto de Extensión, el colaborador podrá acceder a las acciones correspondientes.
	}
		\UCitem{Actor}{\hyperlink{jefe}{Líder de Análisis}, \hyperlink{analista}{Analista}}
		\UCitem{Propósito}{Registrar los puntos de extensión de un caso de uso.}
		\UCitem{Entradas}{
		\begin{itemize}
			\item \cdtRef{entidadExtension:causaPE}{Causa}: Se escribe desde el teclado.
			\item \cdtRef{entidadExtension:regionPE}{Región de la trayectoria}: Se escribe desde el teclado.
			\item \cdtRef{entidadExtension:ExtiendePE}{Caso de uso que extiende}: Se selecciona de una lista.
		\end{itemize}	
		}
		\UCitem{Salidas}{\begin{itemize}
				\item \cdtRef{proyectoEntidad:claveProyecto}{Clave del proyecto}: Lo obtiene el sistema.
				\item \cdtRef{proyectoEntidad:nombreProyecto}{Nombre del proyecto}: Lo obtiene el sistema.
				\item \cdtRef{moduloEntidad:claveModulo}{Clave del Módulo}: Lo obtiene el sistema.
				\item \cdtRef{moduloEntidad:nombreModulo}{Nombre del Módulo}: Lo obtiene el sistema.
				\item \cdtRef{casoUso:claveCU}{Clave} del caso de uso: Lo obtiene el sistema. 
				\item \cdtRef{casoUso:numeroCU}{Número} del caso de uso: Lo obtiene el sistema. 
				\item \cdtRef{casoUso:nombreCU}{Nombre} del caso de uso: Lo obtiene el sistema.
				\item \cdtIdRef{MSG1}{Operación exitosa}: Se muestra en la pantalla \IUref{IU6.1.3}{Gestionar Puntos de extensión} para indicar que el registro fue exitoso.
		\end{itemize}}
		\UCitem{Precondiciones}{
			\begin{itemize}
				\item Que existan al menos 2 casos de uso registrados.
				\item Que el caso de uso al que pertenece el punto de extensión se encuentre en estado ''Edición'' o ''Pendiente de corrección''.
				\item Que el catálogo ''Extiende a'' tenga información.
			\end{itemize}
		}
		\UCitem{Postcondiciones}{
			Se registrará un nueva punto de extensión para un caso de uso en el sistema.
		}
		\UCitem{Errores}{\begin{itemize}
		\item \cdtIdRef{MSG4}{Dato obligatorio}: Se muestra en la pantalla \IUref{IU6.1.3.1}{Registrar Punto de extensión} cuando no se ha ingresado un dato marcado como obligatorio.
		\item \cdtIdRef{MSG5}{Formato incorrecto}: Se muestra en la pantalla \IUref{IU6.1.3.1}{Registrar Punto de extensión} cuando el tipo de dato ingresado no cumple con el tipo de dato solicitado en el campo.
		\item \cdtIdRef{MSG6}{Longitud inválida}: Se muestra en la pantalla \IUref{IU6.1.3.1}{Registrar Punto de extensión} cuando se ha excedido la longitud de alguno de los campos.
		\item \cdtIdRef{MSG15}{Falta información}: Se muestra en la pantalla \IUref{IU6.1.3}{Gestionar Puntos de extensión} cuando no existan casos de uso registrados.
		\item \cdtIdRef{MSG7}{Registro repetido}: Se muestra en la pantalla \IUref{IU6.1.3.1}{Registrar Punto de extensión} cuando se registre un punto de extensión con un caso de uso que ya se encuentre registrado en el sistema.
		\end{itemize}.
		}
		\UCitem{Tipo}{Secundario, extiende del caso de uso \UCref{CU12.1.6}{Gestionar Puntos de extensión}.}
	\end{UseCase}
%--------------------------------------
	\begin{UCtrayectoria}
		\UCpaso[\UCactor] Solicita registrar un punto de extensión oprimiendo el botón \IUbutton{Registrar} de la pantalla \IUref{IU6.1.3}{Gestionar Puntos de extensión}.
		\UCpaso[\UCsist] Verifica que exista al menos un caso de uso, con base en la regla de negocio \BRref{RN20}{Verificación de catálogos}. \hyperlink{CU12-1-6-1:TAA}{[Trayectoria A]}
		\UCpaso[\UCsist] Muestra la pantalla \IUref{IU6.1.3.1}{Registrar Punto de extensión}.
		\UCpaso[\UCactor] Selecciona el caso de uso que extiende. 
		\UCpaso[\UCactor] Ingresa la información solicitada. \hyperlink{CU12-1-6-1:TAB}{[Trayectoria B]} \label{CU12.1.6.1-P4}
		\UCpaso[\UCactor] Oprime el botón \IUbutton{Aceptar}. \hyperlink{CU12-1-6-1:TAC}{[Trayectoria C]}\label{CU12.1.6.1-P6}
		\UCpaso[\UCsist] Verifica que el actor ingrese todos los campos obligatorios con base en la regla de negocio \BRref{RN8}{Datos obligatorios}. \hyperlink{CU12-1-6-1:TAD}{[Trayectoria D]}
		\UCpaso[\UCsist] Verifica que los datos ingresados cumpla con la longitud correcta, con base en la regla de negocio \BRref{RN37}{Longitud de datos}. \hyperlink{CU12-1-6-1:TAE}{[Trayectoria E]}
		\UCpaso[\UCsist] Verifica que los datos ingresados cumplan con el formato requerido, con base en la regla de negocio \BRref{RN7}{Información correcta}. \hyperlink{CU12-1-6-1:TAF}{[Trayectoria F]}
		\UCpaso[\UCsist] Verifica que el punto de extensión no se encuentre registrado en el sistema con base en la regla de negocio \BRref{RN17}{Unicidad de puntos de extensión}. \hyperlink{CU12-1-6-1:TAG}{[Trayectoria G]}
		\UCpaso[\UCsist] Persiste la información del punto de extensión.
		\UCpaso[\UCsist] Muestra el mensaje \cdtIdRef{MSG1}{Operación exitosa} en la pantalla \IUref{IU6.1.3}{Gestionar Puntos de extensión} para indicar al actor que el registro se ha realizado exitosamente.
	\end{UCtrayectoria}		
%--------------------------------------
\hypertarget{CU12-1-6-1:TAA}{\textbf{Trayectoria alternativa A}}\\
\noindent \textbf{Condición:} El catálogo de ''Extiende a'' no tiene información.
\begin{enumerate}
	\UCpaso[\UCsist] Muestra el mensaje \cdtIdRef{MSG15}{Falta información} en la pantalla \IUref{IU6.1.3}{Gestionar Puntos de extensión} para indicar que no existen casos de uso registrados.
	\item[- -] - - {\em {Fin del caso de uso}}.%
\end{enumerate}
%--------------------------------------
\hypertarget{CU12-1-6-1:TAB}{\textbf{Trayectoria alternativa B}}\\
\noindent \textbf{Condición:} El actor desea seleccionar un paso.
\begin{enumerate}
	\UCpaso[\UCactor] Ingresa el token {\em P·}. 
	\UCpaso[\UCsist] Obtiene los pasos del caso de uso.
	\UCpaso[\UCsist] Muestra una lista con los pasos encontrados.
	\UCpaso[\UCactor] Selecciona un paso de la lista.
	\UCpaso[\UCsist] Verifica que el nombre del caso de uso al que pertenece el paso no contenga espacios. \hyperlink{CU12-1-6-1:TAH}{[Trayectoria H]}
	\UCpaso[\UCsist] Agrega la clave del caso de uso al que pertenece el paso al texto, seguido del signo ''·''. \label{CU12.1.6.1-TA6}
	\UCpaso[\UCsist] Agrega el número del caso de uso al texto, seguido del signo '':''.
	\UCpaso[\UCsist] Agrega el nombre del caso de uso al texto, seguido del signo '':''.
	\UCpaso[\UCsist] Agrega la clave de la trayectoria a la que pertenece el paso al texto, seguido del signo ''·''.
	\UCpaso[\UCsist] Agrega el número del paso seleccionado al texto.
	\UCpaso Continúa en el paso \ref{CU12.1.6.1-P6} de la trayectoria principal.
	\item[- -] - - {\em {Fin de la trayectoria}}.%
\end{enumerate}
%--------------------------------------
\hypertarget{CU12-1-6-1:TAC}{\textbf{Trayectoria alternativa C}}\\
\noindent \textbf{Condición:} El actor desea cancelar la operación.
\begin{enumerate}
	\UCpaso[\UCactor] Solicita cancelar la operación oprimiendo el botón \IUbutton{Cancelar} de la pantalla \IUref{IU6.1.3.1}{Registrar Punto de extensión}.
	\UCpaso[\UCsist] Muestra la pantalla \IUref{IU6.1.3}{Gestionar Puntos de extensión}.
	\item[- -] - - {\em {Fin del caso de uso}}.%
\end{enumerate}
%--------------------------------------
\hypertarget{CU12-1-6-1:TAD}{\textbf{Trayectoria alternativa D}}\\
\noindent \textbf{Condición:} El actor no ingresó algún dato marcado como obligatorio.
\begin{enumerate}
	\UCpaso[\UCsist] Muestra el mensaje \cdtIdRef{MSG4}{Dato obligatorio} señalando el campo que presenta el error en la pantalla \IUref{IU6.1.3.1}{Registrar Punto de extensión}.
	\UCpaso Regresa al paso \ref{CU12.1.6.1-P4} de la trayectoria principal.
	\item[- -] - - {\em {Fin de la trayectoria}}.%
\end{enumerate}
%--------------------------------------
\hypertarget{CU12-1-6-1:TAE}{\textbf{Trayectoria alternativa E}}\\
\noindent \textbf{Condición:} El actor ingresó un dato con un número de caracteres fuera del rango permitido.
\begin{enumerate}
	\UCpaso[\UCsist] Muestra el mensaje \cdtIdRef{MSG6}{Longitud inválida} señalando el campo que presenta el error en la pantalla \IUref{IU6.1.3.1}{Registrar Punto de extensión}.
	\UCpaso Regresa al paso \ref{CU12.1.6.1-P4} de la trayectoria principal.
	\item[- -] - - {\em {Fin de la trayectoria}}.%
\end{enumerate}
%--------------------------------------
\hypertarget{CU12-1-6-1:TAF}{\textbf{Trayectoria alternativa F}}\\
\noindent \textbf{Condición:} El actor ingresó un dato con un formato incorrecto.
\begin{enumerate}
	\UCpaso[\UCsist] Muestra el mensaje \cdtIdRef{MSG5}{Formato incorrecto} señalando el campo que presenta el error en la pantalla \IUref{IU6.1.3.1}{Registrar Punto de extensión}.
	\UCpaso Regresa al paso \ref{CU12.1.6.1-P4} de la trayectoria principal.
	\item[- -] - - {\em {Fin de la trayectoria}}.
\end{enumerate}
%-------------------------------------
\hypertarget{CU12-1-6-1:TAG}{\textbf{Trayectoria alternativa G}}\\
\noindent \textbf{Condición:} El actor ingresó un punto de extensión que ya existe.
\begin{enumerate}
	\UCpaso[\UCsist] Muestra el mensaje \cdtIdRef{MSG7}{Registro repetido} señalando el campo que presenta la duplicidad en la pantalla \IUref{IU6.1.3.1}{Registrar Punto de extensión}.
	\UCpaso Regresa al paso \ref{CU12.1.6.1-P4} de la trayectoria principal.
	\item[- -] - - {\em {Fin de la trayectoria}}.
\end{enumerate}
%--------------------------------------
\hypertarget{CU12-1-6-1:TAH}{\textbf{Trayectoria alternativa H}}\\
\noindent \textbf{Condición:} El texto contiene espacios.
\begin{enumerate}
	\UCpaso[\UCsist] Sustituye los espacios por guiones bajos.
	\UCpaso Continua en el paso \ref{CU12.1.6.1-TA6} de la trayectoria alternativa B.
	\item[- -] - - {\em {Fin de la trayectoria}}.
\end{enumerate}
