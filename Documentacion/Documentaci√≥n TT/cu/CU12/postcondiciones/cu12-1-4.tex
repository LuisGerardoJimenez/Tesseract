	\begin{UseCase}{CU12.1.4}{Registrar Postcondición}{
			
			Cuando el colaborador (\hyperlink{jefe}{Líder de Análisis} o \hyperlink{analista}{Analista}) está registrando la información de un \hyperlink{casoUso}{Caso de Uso} correspondiente a algún \hyperlink{moduloEntidad}{Módulo} del proyecto, el sistema le permitirá registrar las postcondiciones correspondientes al caso de uso para que pueda ser utilizado en sus Trayectorias. Una \hyperlink{entidadPostcondicion}{Postcondición}describen un conjunto de condiciones que deberán cumplirse después de ejecutar el caso de uso. A través de un campo el sistema solicita la redacción de la postcondición que se desea registrar.\\
		
	}
		\UCitem{Actor}{\hyperlink{jefe}{Líder de Análisis}, \hyperlink{analista}{Analista}}
		\UCitem{Propósito}{Registrar las precondiciones de un caso de uso.}
		\UCitem{Entradas}{ \cdtRef{entidadPostcondicion:redaccionPostcondicion}{Redacción de la postcondición}: Se escribe desde el teclado.	
		}
		\UCitem{Salidas}{Ninguna}
		\UCitem{Precondiciones}{Que exista al menos un caso de uso registrado.}
		\UCitem{Postcondiciones}{Ninguna}
		\UCitem{Errores}{\begin{itemize}
		\item \cdtIdRef{MSG4}{Dato obligatorio}: Se muestra en la pantalla \IUref{IU6.1.3}{Registrar Postcondición} cuando no se ha ingresado un dato marcado como obligatorio.
		\item \cdtIdRef{MSG29}{Formato incorrecto}: Se muestra en la pantalla \IUref{IU6.1.3}{Registrar Postcondición} cuando el tipo de dato ingresado no cumple con el tipo de dato solicitado en el campo.
		\item \cdtIdRef{MSG6}{Longitud inválida}: Se muestra en la pantalla \IUref{IU6.1.3}{Registrar Postcondición} cuando se ha excedido la longitud de alguno de los campos.
		\end{itemize}.}
		\UCitem{Tipo}{Secundario, extiende del caso de uso \UCref{CU12.1}{Registrar Caso de uso} y \UCref{CU12.2}{Modificar Caso de uso}.}
	\end{UseCase}
%--------------------------------------
	\begin{UCtrayectoria}
		\UCpaso[\UCactor] Solicita registrar una postcondición oprimiendo el botón \IUbutton{Registrar} de la pantalla \IUref{IU6.1}{Registrar Caso de uso} o \IUref{IU6.2}{Modificar Caso de uso} en la sección ''Postcondiciones''.
		\UCpaso[\UCsist] Muestra la pantalla emergente \IUref{IU6.1.3}{Registrar Postcondición}.
		\UCpaso[\UCactor] Ingresa la redacción de la postcondición. \hyperlink{CU12-1-4:TAA}{[Trayectoria A]} \label{CU12.1.4-P3}
		\UCpaso[\UCactor] Oprime el botón \IUbutton{Aceptar}. \hyperlink{CU12-1-4:TAB}{[Trayectoria B]} 
		\UCpaso[\UCsist] Verifica que el actor ingrese todos los campos obligatorios con base en la regla de negocio \BRref{RN8}{Datos obligatorios}. \hyperlink{CU12-1-4:TAC}{[Trayectoria C]}
		\UCpaso[\UCsist] Verifica que los datos requeridos sean proporcionados correctamente con base en la regla de negocio \BRref{RN7}{Información correcta}. \hyperlink{CU12-1-4:TAD}{[Trayectoria D]}
		\UCpaso[\UCsist] Verifica que los datos ingresados cumplan con el formato requerido, con base en la regla de negocio \BRref{RN7}{Información correcta}. \hyperlink{CU12-1-4:TAE}{[Trayectoria E]}
		\UCpaso[\UCsist] Agrega la postcondición en la tabla de la pantalla \IUref{IU6.1}{Registrar Caso de uso} o \IUref{IU6.2}{Modificar Caso de uso}, en la sección''Postcondiciones''.
	\end{UCtrayectoria}		
%-------------------------------------
\hypertarget{CU12-1-4:TAA}{\textbf{Trayectoria alternativa A}}\\
\noindent \textbf{Condición:} El actor desea hacer referencia a un elemento existente en el proyecto.
\begin{enumerate}
	\UCpaso[\UCactor] Ingresa el token correspondiente al elemento a referenciar.
	\UCpaso[\UCsist] Verifica que los tokens utilizados se encuentren correctamente estructurados, con base en la regla de negocio \BRref{RN31}{Estructura de Tokens}. 
	\UCpaso[\UCsist] Obtiene los elementos registrados en el proyecto correspondientes al token ingresado. 
	\UCpaso[\UCsist] Muestra una lista con los elementos encontrados.
	\UCpaso[\UCactor] Selecciona un elemento de la lista.
	\UCpaso[\UCsist] Verifica que el nombre del elemento seleccionado no contenga espacios. \hyperlink{CU12-1-4:TAF}{[Trayectoria F]}
	\UCpaso[\UCsist] Agrega la referencia del elemento al texto. \label{CU12.1.4-TA1}
	\UCpaso Continúa en el paso \ref{CU12.1.4-P3} de la trayectoria principal.
	\item[- -] - - {\em {Fin de la trayectoria}}.%
\end{enumerate}
%--------------------------------------
\hypertarget{CU12-1-4:TAB}{\textbf{Trayectoria alternativa B}}\\
\noindent \textbf{Condición:} El actor desea cancelar la operación.
\begin{enumerate}
	\UCpaso[\UCactor] Solicita cancelar la operación oprimiendo el botón \IUbutton{Cancelar} de la pantalla .
	\UCpaso[\UCsist] Muestra la pantalla \IUref{IU6}{Gestionar Casos de uso}.
	\item[- -] - - {\em {Fin del caso de uso}}.%
\end{enumerate}
%--------------------------------------
\hypertarget{CU12-1-4:TAC}{\textbf{Trayectoria alternativa C}}\\
\noindent \textbf{Condición:} El actor no ingresó algún dato marcado como obligatorio.
\begin{enumerate}
	\UCpaso[\UCsist] Muestra el mensaje \cdtIdRef{MSG4}{Dato obligatorio} señalando el campo que presenta el error en la pantalla \IUref{IU6.1.3}{Registrar postcondición}.
	\UCpaso Regresa al paso \ref{CU12.1.4-P3} de la trayectoria principal.
	\item[- -] - - {\em {Fin de la trayectoria}}.%
\end{enumerate}
%--------------------------------------
\hypertarget{CU12-1-4:TAD}{\textbf{Trayectoria alternativa D}}\\
\noindent \textbf{Condición:} El actor ingresó un dato con un número de caracteres fuera del rango permitido.
\begin{enumerate}
	\UCpaso[\UCsist] Muestra el mensaje \cdtIdRef{MSG6}{Longitud inválida} señalando el campo que presenta el error en la pantalla \IUref{IU6.1.3}{Registrar Postcondición}.
	\UCpaso Regresa al paso \ref{CU12.1.4-P3} de la trayectoria principal.
	\item[- -] - - {\em {Fin de la trayectoria}}.%
\end{enumerate}
%--------------------------------------
\hypertarget{CU12-1-4:TAE}{\textbf{Trayectoria alternativa E}}\\
\noindent \textbf{Condición:} El actor ingresó un dato con un formato incorrecto.
\begin{enumerate}
	\UCpaso[\UCsist] Muestra el mensaje \cdtIdRef{MSG29}{Formato incorrecto} señalando el campo que presenta el error en la pantalla \IUref{IU6.1.3}{Registrar Postcondición}.
	\UCpaso Regresa al paso \ref{CU12.1.4-P3} de la trayectoria principal.
	\item[- -] - - {\em {Fin de la trayectoria}}.
\end{enumerate}
%--------------------------------------
\hypertarget{CU12-1-4:TAH}{\textbf{Trayectoria alternativa F}}\\
\noindent \textbf{Condición:} El texto contiene espacios.
\begin{enumerate}
	\UCpaso[\UCsist] Sustituye los espacios por guiones bajos.
	\UCpaso Continua en el \ref{CU12.1.4-TA1} de la trayectoria alternativa A.
	\item[- -] - - {\em {Fin de la trayectoria}}.
\end{enumerate}