	\begin{UseCase}{CU12.1.3.2}{Eliminar Postcondición}{
		Cuando el registro de una  alguna \hyperlink{entidadPostcondicion}{Postcondición} en el \hyperlink{casoUso}{Caso de Uso} sobre el cual se está operando ya no tiene una razón de ser, Tesseract permitirá al colaborador (\hyperlink{jefe}{Líder} o \hyperlink{analista}{Analista}) eliminar en su totalidad su registro de la tabla y no podrá ser utilizado en las trayectorias del \hyperlink{casoUso}{Caso de Uso}.
	}
	\UCitem{Actor}{\hyperlink{jefe}{Líder de Análisis}, \hyperlink{analista}{Analista}}
	\UCitem{Propósito}{Eliminar una postcondición de un caso de uso.}
	\UCitem{Entradas}{Ninguna.}
	\UCitem{Salidas}{
			\begin{itemize}
				\item \cdtIdRef{MSG10}{Confirmar eliminación}: Se muestra en la pantalla \IUref{IU6.1}{Registrar Caso de uso} o \IUref{IU6.2}{Modificar Caso de uso} preguntando al actor si desea continuar con la eliminación de la Postcondición.
				\item \cdtIdRef{MSG1}{Operación exitosa}: Se muestra en la pantalla \IUref{IU12}{Pendiente} para indicar que la postcondición fue eliminada correctamente.
			\end{itemize}
	}
	\UCitem{Precondiciones}{Ninguna}
	\UCitem{Postcondiciones}{
			Se eliminará una postcondición de un caso de uso perteneciente a un proyecto.
	}
	\UCitem{Errores}{Ninguno.}
	\UCitem{Tipo}{Secundario, extiende del caso de uso \UCref{CU12.1.3}{Gestionar Postcondiciones}}
\end{UseCase}
%--------------------------------------
\begin{UCtrayectoria}
	\UCpaso[\UCactor] Da clic en el icono \eliminar del registro que desea eliminar de la pantalla \IUref{IU7}{Pendiente}.
	\UCpaso[\UCsist] Muestra el mensaje emergente \cdtIdRef{MSG10}{Confirmar eliminación} con los botones \IUbutton{Aceptar} y \IUbutton{Cancelar} en la pantalla \IUref{IU7}{Pendiente}.
	\UCpaso[\UCactor] Oprime el botón \IUbutton{Aceptar}. \hyperlink{CU12-1-3-2:TAA}{[Trayectoria A]}
	\UCpaso[\UCsist] Elimina la información referente a la postcondición.
	\UCpaso[\UCsist] Muestra el mensaje \cdtIdRef{MSG1}{Operación exitosa} en la pantalla \IUref{IU7}{Pendiente} para indicar al actor que el registro se ha eliminado exitosamente.
\end{UCtrayectoria}		
%-------------------------------------
\hypertarget{CU12-1-3-2:TAA}{\textbf{Trayectoria alternativa A}}\\
\noindent \textbf{Condición:} El actor desea cancelar la operación.
\begin{enumerate}
	\UCpaso[\UCactor] Solicita cancelar la operación oprimiendo el botón \IUbutton{Cancelar} de la pantalla .
	\UCpaso[\UCsist] Muestra la pantalla \IUref{IU7}{Pendiente}.
	\item[- -] - - {\em {Fin del caso de uso}}.%
\end{enumerate}
