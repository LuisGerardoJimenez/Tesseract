	\begin{UseCase}{CU12.1.1.1}{Registrar Trayectoria}{
			
		Este caso de uso permite al colaborador (\hyperlink{jefe}{Líder de Análisis} o \hyperlink{analista}{Analista}) registrar la información de una \hyperlink{entidadTray}{Trayectoria} (Principal o Alternativa) correspondiente al \hyperlink{casoUso}{Caso de Uso} del \hyperlink{moduloEntidad}{Módulo} sobre el cual se está operando.\\
		
		A través de un formulario el sistema solicita la información de la Trayectoria que se desea registrar (como su clave, tipo y condición). Por caso de uso solo se puede registrar una trayectoria principal y una o varias alternativas.
		
		Las trayectorias están compuestas por una serie de pasos, mismos que podrán ser gestionados al momento de registrar las trayectorias. Los pasos son cada una de las actividades que realizan ya sea el sistema o el actor en su interacción y se podrán registrar, modificar, cambiar su posición o en su defecto ser eliminados. \\
		
		Para que esta acción se pueda llevar a cabo es necesario que el caso de uso al que pertenece la trayectoria se encuentre en estado de Edición o Pendiente de corrección con base en el \hyperlink{edoCU}{Modelo de estados del Caso de Uso} y que exista al menos un Caso de Uso previamente registrado.
	
}
		\UCitem{Actor}{\hyperlink{jefe}{Líder de Análisis}, \hyperlink{analista}{Analista}}
		\UCitem{Propósito}{Registrar la información de una trayectoria principal o alternativa.}
		\UCitem{Entradas}{
		\begin{itemize}
			\item \cdtRef{entidadTray:nombreTray}{Clave de la trayectoria}: Se escribe desde el teclado.
			\item \cdtRef{entidadTray:alternativaTray}{Tipo de la trayectoria:} Se escribe desde el teclado.
			\item \cdtRef{entidadTray:condicionTray}{Condición de la trayectoria:} Se escribe desde el teclado.
			\item \cdtRef{entidadTray:finTray}{Fin del caso de uso:} Se selecciona de una lista.
		\end{itemize}	
		}
		\UCitem{Salidas}{\begin{itemize}
				\item \cdtRef{proyectoEntidad:claveProyecto}{Clave del proyecto}: Lo obtiene el sistema.
				\item \cdtRef{proyectoEntidad:nombreProyecto}{Nombre del proyecto}: Lo obtiene el sistema.
				\item \cdtRef{moduloEntidad:claveModulo}{Clave del Módulo}: Lo obtiene el sistema.
				\item \cdtRef{moduloEntidad:nombreModulo}{Nombre del Módulo}: Lo obtiene el sistema.
				\item \cdtRef{casoUso:numeroCU}{Número} del caso de uso: Lo obtiene el sistema. 
				\item \cdtRef{casoUso:nombreCU}{Nombre} del caso de uso: Lo obtiene el sistema.
				\item \cdtIdRef{MSG1}{Operación exitosa}: Se muestra en la pantalla \IUref{IU6.1.1}{Gestionar Trayectorias} para indicar que el registro fue exitoso.
		\end{itemize}}
		\UCitem{Precondiciones}{
			\begin{itemize}
				\item Que el caso de uso al que pertenece la trayectoria se encuentre en estado ''Edición'' o ''Pendiente de corrección''.
				\item Que exista al menos un caso de uso registrado.
			\end{itemize}
		}
		\UCitem{Postcondiciones}{
			Se registrará una nueva trayectoria para un caso de uso en el sistema.
		}
		\UCitem{Errores}{\begin{itemize}
		\item \cdtIdRef{MSG4}{Dato obligatorio}: Se muestra en la pantalla \IUref{IU6.1.1.1}{Registrar Trayectoria} cuando no se ha ingresado un dato marcado como obligatorio.
		\item \cdtIdRef{MSG5}{Formato incorrecto}: Se muestra en la pantalla \IUref{IU6.1.1.1}{Registrar Trayectoria} cuando el tipo de dato ingresado no cumple con el tipo de dato solicitado en el campo.
		\item \cdtIdRef{MSG6}{Longitud inválida}: Se muestra en la pantalla \IUref{IU6.1.1.1}{Registrar Trayectoria} cuando se ha excedido la longitud de alguno de los campos.
		\item \cdtIdRef{MSG7}{Registro repetido}: Se muestra en la pantalla \IUref{IU6.1.1.1}{Registrar Trayectoria} cuando se registre un caso de uso con un nombre que ya se encuentre registrado en el sistema.
		\item \cdtIdRef{MSG29}{Los catálogos nos contienen información}: Se muestra en la pantalla \IUref{IU6.1.1}{Gestionar Trayectorias} cuando no exista información en el catálogo ''tipo''.
		\end{itemize}.
		}
		\UCitem{Tipo}{Secundario, extiende del caso de uso \UCref{CU12.1.1}{Gestionar Trayectorias}.}
	\end{UseCase}
%--------------------------------------
	\begin{UCtrayectoria}
		\UCpaso[\UCactor] Solicita registrar una trayectoria oprimiendo el botón \IUbutton{Registrar} de la pantalla \IUref{IU6.1.1}{Gestionar Trayectorias}.
		\UCpaso[\UCsist] Verifica que el catálogo ''tipo'' cuente con información, con base en la regla de negocio \BRref{RN20}{Verificación de catálogos}. \hyperlink{CU12-1-1-1:TAA}{[Trayectoria A]}
		\UCpaso[\UCsist] Verifica que no exista una trayectoria del tipo ''principal'' registrada con base en la regla de negocio \BRref{RN11}{Registro de trayectorias}. \hyperlink{CU12-1-1-1:TAB}{[Trayectoria B]}
		\UCpaso[\UCsist] Muestra la pantalla \IUref{IU6.1.1.1}{Registrar Trayectoria}. \label{CU12.1.1.1-P17}
		\UCpaso[\UCactor] Ingresa la clave de la trayectoria. \label{CU12.1.1.1-P16}
		\UCpaso[\UCactor] Selecciona la opción ''Principal'' del campo ''Tipo''. \hyperlink{CU12-1-1-1:TAC}{[Trayectoria C]} 
		\UCpaso[\UCsist] Marca la casilla del campo ''Fin del caso de uso''.
		\UCpaso[\UCactor] Oprime el botón \IUbutton{Aceptar} . \hyperlink{CU12-1-1-1:TAD}{[Trayectoria D]} \label{CU12.1.1.1-P18}
		\UCpaso[\UCsist] Verifica que el actor ingrese todos los campos obligatorios con base en la regla de negocio \BRref{RN8}{Datos obligatorios}. \hyperlink{CU12-1-1-1:TAE}{[Trayectoria E]}
		\UCpaso[\UCsist] Verifica que los datos ingresados cumpla con la longitud correcta, con base en la regla de negocio \BRref{RN37}{Longitud de datos}. \hyperlink{CU12-1-1-1:TAF}{[Trayectoria F]}
		\UCpaso[\UCsist] Verifica que los datos ingresados cumplan con el formato requerido, con base en la regla de negocio \BRref{RN7}{Información correcta}. \hyperlink{CU12-1-1-1:TAG}{[Trayectoria G]}
		\UCpaso[\UCsist] Verifica que la clave de la trayectoria no se encuentre registrada en el sistema con base en la regla de negocio \BRref{RN24}{Unicidad de la clave de la trayectoria}. \hyperlink{CU12-1-1-1:TAH}{[Trayectoria H]}
		\UCpaso[\UCsist] Persiste la información de la trayectoria.
		\UCpaso[\UCsist] Muestra el mensaje \cdtIdRef{MSG1}{Operación exitosa} en la pantalla \IUref{IU6.1.1}{Gestionar Trayectorias} para indicar al actor que el registro se ha realizado exitosamente.
	\end{UCtrayectoria}		
%--------------------------------------
\hypertarget{CU12-1-1-1:TAA}{\textbf{Trayectoria alternativa A}}\\
\noindent \textbf{Condición:} El catálogo ''tipo'' no tiene información.
\begin{enumerate}
	\UCpaso[\UCsist] Muestra el mensaje \cdtIdRef{MSG29}{Los catálogos nos contienen información} en la pantalla \IUref{IU6.1.1}{Gestionar Trayectorias} para indicar que no es posible realizar la operación debido a la falta de información necesaria para el sistema.
	\item[- -] - - {\em {Fin del caso de uso}}.%
\end{enumerate}
%--------------------------------------
\hypertarget{CU12-1-1-1:TAB}{\textbf{Trayectoria alternativa B}}\\
\noindent \textbf{Condición:} Ya existe una trayectoria principal registrada.
\begin{enumerate}
	\UCpaso[\UCsist] Selecciona automáticamente la opción ''Alternativa'' del campo ''Tipo''.
	\UCpaso[\UCsist] Inhabilita el campo ''tipo''.
	\UCpaso[\UCsist] Muestra el campo de condición.
	\UCpaso Continúa con el paso \ref{CU12.1.1.1-TB3} de la trayectoria alternativa C.
	\item[- -] - - {\em {Fin de la trayectoria}}.%
\end{enumerate}
%--------------------------------------
\hypertarget{CU12-1-1-1:TAC}{\textbf{Trayectoria alternativa C}}\\
\noindent \textbf{Condición:} El actor desea registrar una trayectoria alternativa.
\begin{enumerate}
	\UCpaso[\UCactor] Selecciona la opción ''Alternativa'' del campo ''Tipo''.
	\UCpaso[\UCsist] Muestra el campo de condición.
	\UCpaso[\UCactor] Ingresa la condición de la trayectoria. \label{CU12.1.1.1-TB3}
	\UCpaso[\UCactor] Selecciona si en la trayectoria se termina el caso de uso.
	\UCpaso Continúa con el paso \ref{CU12.1.1.1-P18} de la trayectoria principal.
	\item[- -] - - {\em {Fin de la trayectoria}}.%
\end{enumerate}
%--------------------------------------
\hypertarget{CU12-1-1-1:TAD}{\textbf{Trayectoria alternativa D}}\\
\noindent \textbf{Condición:} El actor desea cancelar la operación.
\begin{enumerate}
	\UCpaso[\UCactor] Solicita cancelar la operación oprimiendo el botón \IUbutton{Cancelar} de la pantalla \IUref{IU6.1.1.1}{Registrar Trayectoria}.
	\UCpaso[\UCsist] Muestra la pantalla \IUref{IU6.1.1}{Gestionar Trayectorias}.
	\item[- -] - - {\em {Fin del caso de uso}}.%
\end{enumerate}
%--------------------------------------
\hypertarget{CU12-1-1-1:TAE}{\textbf{Trayectoria alternativa E}}\\
\noindent \textbf{Condición:} El actor no ingresó algún dato marcado como obligatorio.
\begin{enumerate}
	\UCpaso[\UCsist] Muestra el mensaje \cdtIdRef{MSG4}{Dato obligatorio} señalando el campo que presenta el error en la pantalla \IUref{IU6.1.1.1}{Registrar Trayectoria}.
	\UCpaso Regresa al paso \ref{CU12.1.1.1-P16} de la trayectoria principal.
	\item[- -] - - {\em {Fin de la trayectoria}}.%
\end{enumerate}
%--------------------------------------
\hypertarget{CU12-1-1-1:TAF}{\textbf{Trayectoria alternativa F}}\\
\noindent \textbf{Condición:} El actor ingresó un dato con un número de caracteres fuera del rango permitido.
\begin{enumerate}
	\UCpaso[\UCsist] Muestra el mensaje \cdtIdRef{MSG6}{Longitud inválida} señalando el campo que presenta el error en la pantalla \IUref{IU6.1.1.1}{Registrar Trayectoria}.
	\UCpaso Regresa al paso \ref{CU12.1.1.1-P16} de la trayectoria principal.
	\item[- -] - - {\em {Fin de la trayectoria}}.%
\end{enumerate}
%--------------------------------------
\hypertarget{CU12-1-1-1:TAG}{\textbf{Trayectoria alternativa G}}\\
\noindent \textbf{Condición:} El actor ingresó un dato con un formato incorrecto.
\begin{enumerate}
	\UCpaso[\UCsist] Muestra el mensaje \cdtIdRef{MSG5}{Formato incorrecto} señalando el campo que presenta el error en la pantalla \IUref{IU6.1.1.1}{Registrar Trayectoria}.
	\UCpaso Regresa al paso \ref{CU12.1.1.1-P16} de la trayectoria principal.
	\item[- -] - - {\em {Fin de la trayectoria}}.
\end{enumerate}
%--------------------------------------	
\hypertarget{CU2-1-1-1:TAH}{\textbf{Trayectoria alternativa H}}\\
\noindent \textbf{Condición:} El actor ingresó una clave de trayectoria que ya existe dentro del sistema.
\begin{enumerate}
	\UCpaso[\UCsist] Muestra el mensaje \cdtIdRef{MSG7}{Registro repetido} señalando el campo que presenta la duplicidad en la pantalla \IUref{IU6.1.1.1}{Registrar Trayectoria}.
	\UCpaso Regresa al paso \ref{CU12.1.1.1-P16} de la trayectoria principal.
	\item[- -] - - {\em {Fin de la trayectoria}}.
\end{enumerate}