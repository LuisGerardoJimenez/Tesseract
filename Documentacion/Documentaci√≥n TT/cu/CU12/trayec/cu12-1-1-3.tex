	\begin{UseCase}{CU12.1.1.3}{Eliminar Trayectoria}{
				
		Cuando el registro de alguna \hyperlink{entidadTray}{Trayectoria} ya no tiene una razón de ser dentro del \hyperlink{casoUso}{Caso de Uso}, Tesseract permitirá al colaborador (\hyperlink{jefe}{Líder} o \hyperlink{analista}{Analista}) eliminar en su totalidad su registro (así como el de sus pasos). \\
		
		Una Trayectoria podrá ser eliminada del Caso de Uso siempre y cuando la trayectoria no se encuentre referenciada en otro caso de uso.
	}
		\UCitem{Actor}{\hyperlink{jefe}{Líder de Análisis}, \hyperlink{analista}{Analista}}
		\UCitem{Propósito}{Eliminar la información de una trayectoria.}
		\UCitem{Entradas}{Ninguna}
		\UCitem{Salidas}{\begin{itemize}
				\item \cdtIdRef{MSG1}{Operación exitosa}: Se muestra en la pantalla \IUref{IU6.1.1}{Gestionar Trayectorias} para indicar que la trayectoria fue eliminada correctamente.
				\item \cdtIdRef{MSG10}{Confirmar eliminación}: Se muestra en la pantalla \IUref{IU6.1.1}{Gestionar Trayectorias} preguntando al actor si desea continuar con la eliminación de la trayectoria.
		\end{itemize}}
		\UCitem{Precondiciones}{Que la trayectoria no se encuentre referenciada en otro caso de uso.}
		\UCitem{Postcondiciones}{
			Se eliminará la trayectoria seleccionada del sistema.
		}
		\UCitem{Errores}{
		\cdtIdRef{MSG13}{Eliminación no permitida}: Se muestra en la pantalla \IUref{IU6.1.1}{Gestionar Traycetorias} cuando no se pueda eliminar la trayectoria debido a que está siendo referenciada en algún caso de uso.
		}
		\UCitem{Tipo}{Secundario, extiende del caso de uso \UCref{CU12.1.1}{Gestionar Trayectorias}.}
	\end{UseCase}
%--------------------------------------
	\begin{UCtrayectoria}
		\UCpaso[\UCactor] Da clic en el icono \eliminar del registro que desea eliminar de la pantalla \IUref{IU6.1.1}{Gestionar Trayectorias}.
		\UCpaso[\UCsist] Muestra el mensaje emergente \cdtIdRef{MSG10}{Confirmar eliminación} con los botones \IUbutton{Aceptar} y \IUbutton{Cancelar} en la pantalla \IUref{IU6.1.1}{Gestionar Trayectorias}.
		\UCpaso[\UCactor] Confirma la eliminación de la pantalla oprimiendo el botón \IUbutton{Aceptar}. \hyperlink{CU12-1-1-3:TAA}{[Trayectoria A]}
		\UCpaso[\UCsist] Verifica que ningún elemento se encuentre referenciando a la trayectoria. \hyperlink{CU12-1-1-3:TAB}{[Trayectoria B]}	
		\UCpaso[\UCsist] Elimina la información referente a la trayectoria.
		\UCpaso[\UCsist] Muestra el mensaje \cdtIdRef{MSG1}{Operación exitosa} en la pantalla \IUref{IU6.1.1}{Gestionar Trayectorias} para indicar al actor que el registro se ha eliminado exitosamente.
	\end{UCtrayectoria}		
%--------------------------------------
\hypertarget{CU12-1-1-3:TAA}{\textbf{Trayectoria alternativa A}}\\
\noindent \textbf{Condición:} El actor desea cancelar la operación.
\begin{enumerate}
	\UCpaso[\UCactor] Oprime el botón \IUbutton{Cancelar} de la pantalla emergente.
	\UCpaso[\UCsist] Muestra la pantalla \IUref{IU6.1.1}{Gestionar Trayectorias}.
	\item[- -] - - {\em {Fin del caso de uso}}.%
\end{enumerate}
%--------------------------------------
	\hypertarget{CU12-1-1-3:TAB}{\textbf{Trayectoria alternativa B}}\\
	\noindent \textbf{Condición:} La trayectoria está siendo referenciada en algún elemento.
	\begin{enumerate}
		\UCpaso[\UCsist] Muestra el mensaje \cdtIdRef{MSG13}{Eliminación no permitida} en la pantalla \IUref{IU6.1.1}{Gestionar Trayectorias} en una pantalla emergente con la lista de elementos que están referenciando a la trayectoria.
		\UCpaso[\UCactor] Oprime el botón \IUbutton{Aceptar} de la pantalla emergente.
		\UCpaso[\UCsist] Muestra la pantalla \IUref{IU6.1.1}{Gestionar Trayectorias}.
		\item[- -] - - {\em {Fin del caso de uso}}.
	\end{enumerate}

