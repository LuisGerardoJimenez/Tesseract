	\begin{UseCase}{CU12.1.1.1.1.3}{Eliminar Paso}{
			
		Cuando el registro de algún \hyperlink{entidadPaso}{Paso} ya no tiene una razón de ser dentro del \hyperlink{casoUso}{Caso de Uso}, Tesseract permitirá al colaborador (\hyperlink{jefe}{Líder} o \hyperlink{analista}{Analista}) eliminar en su totalidad su registro. \\
			
		Un Paso podrá ser eliminado del sistema siempre y cuando el paso no se encuentre referenciado en otro caso de uso o el paso no se encuentre referenciado en otro paso del mismo caso de uso.

	}
		\UCitem{Actor}{\hyperlink{jefe}{Líder de Análisis}, \hyperlink{analista}{Analista}}
		\UCitem{Propósito}{Eliminar un paso perteneciente a una trayectoria.}
		\UCitem{Entradas}{Ninguna.}
		\UCitem{Salidas}{
				\begin{itemize}
					\item \cdtIdRef{MSG10}{Confirmar eliminación}: Se muestra en la pantalla \IUref{IU6.1.1.1}{Pendiente} preguntando al actor si desea continuar con la eliminación del paso.
					\item \cdtIdRef{MSG1}{Operación exitosa}: Se muestra en la pantalla \IUref{IU6.1.1}{Pendiente} para indicar que el paso fue eliminado correctamente.
				\end{itemize}
		}
		\UCitem{Precondiciones}{
			\begin{itemize}
				\item Que el paso no se encuentre referenciado en otro caso de uso.
				\item Que el paso no se encuentre referenciado en otro paso del mismo caso de uso.
			\end{itemize}}
		\UCitem{Postcondiciones}{
			Se eliminará un paso perteneciente a un trayectoria del sistema.
		}
		\UCitem{Errores}{
		\cdtIdRef{MSG13}{Eliminación no permitida}: Se muestra en la pantalla \IUref{IU6.1.1.1}{Pendiente} cuando no se pueda eliminar el paso debido a que está siendo utilizado por otro elemento.
		}
		\UCitem{Tipo}{Secundario, extiende del caso de uso \UCref{CU12.1.1.1.1}{Gestionar Pasos}.}
	\end{UseCase}
%--------------------------------------
	\begin{UCtrayectoria}
		\UCpaso[\UCactor] Da clic en el icono \eliminar del registro que desea eliminar de la pantalla \IUref{IU6.1.1}{Pendiente}.
		\UCpaso[\UCsist] Muestra el mensaje emergente \cdtIdRef{MSG10}{Confirmar eliminación} con los botones \IUbutton{Aceptar} y \IUbutton{Cancelar}.
		\UCpaso[\UCactor] Confirma la eliminación de la pantalla oprimiendo el botón \IUbutton{Aceptar}. \hyperlink{CU12-1-1-1-1-3:TAA}{[Trayectoria A]}
		\UCpaso[\UCsist] Verifica que ningún elemento se encuentre referenciando al paso. \hyperlink{CU12-1-1-1-1-3:TAB}{[Trayectoria B]}
		\UCpaso[\UCsist] Elimina la información referente al paso.
		\UCpaso[\UCsist] Muestra el mensaje \cdtIdRef{MSG1}{Operación exitosa} en la pantalla \IUref{IU6.1.1}{Pendiente} para indicar al actor que el registro se ha eliminado exitosamente.
	\end{UCtrayectoria}		
%--------------------------------------
\hypertarget{CU12-1-1-1-1-3:TAA}{\textbf{Trayectoria alternativa A}}\\
\noindent \textbf{Condición:} El actor desea cancelar la operación.
\begin{enumerate}
	\UCpaso[\UCactor] Oprime el botón \IUbutton{Cancelar} de la pantalla emergente.
	\UCpaso[\UCsist] Muestra la pantalla \IUref{IU6.1.1.1}{Pendiente}.
	\item[- -] - - {\em {Fin del caso de uso}}.%
\end{enumerate}
%--------------------------------------
\hypertarget{CU12-1-1-1-1-3:TAB}{\textbf{Trayectoria alternativa B}}\\
\noindent \textbf{Condición:} Existen elementos referenciando al paso que se desea eliminar.
\begin{enumerate}
	\UCpaso[\UCsist] Muestra el mensaje \cdtIdRef{MSG13}{Eliminación no permitida} en la pantalla \IUref{IU6.1.1.1}{Pendiente}.
	\item[- -] - - {\em {Fin del caso de uso}}.
\end{enumerate}
	

