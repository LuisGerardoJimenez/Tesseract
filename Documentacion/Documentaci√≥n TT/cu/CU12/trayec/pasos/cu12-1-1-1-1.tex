	\begin{UseCase}{CU12.1.1.1.1}{Gestionar Pasos}{
     	Resumen de la gestión de pasos PENDIENTE.
	}
	\UCitem{Actor}{\hyperlink{jefe}{Líder de análisis}, \hyperlink{analista}{Analista}}
	\UCitem{Propósito}{Proporcionar al actor un mecanismo para llevar el control de pasos que componen a una trayectoria de un caso de uso.}
	\UCitem{Entradas}{Ninguna}
	\UCitem{Salidas}{\begin{itemize}
			\item \cdtRef{proyectoEntidad:claveProyecto}{Clave del proyecto}: Lo obtiene el sistema.
			\item \cdtRef{proyectoEntidad:nombreProyecto}{Nombre del proyecto}: Lo obtiene el sistema.
			\item \cdtRef{moduloEntidad:nombreModulo}{Nombre del Módulo}: Lo obtiene el sistema.
			\item \cdtRef{casoUso:numeroCU}{Número} del caso de uso: Lo obtiene el sistema. 
			\item \cdtRef{casoUso:nombreCU}{Nombre} del caso de uso: Lo obtiene el sistema.
			\item \cdtRef{entidadTray:nombreTray}{Nombre} de la trayectoria: Lo obtiene el sistema.
			\item \cdtRef{entidadPaso}{Paso}: Tabla que muestra el \cdtRef{entidadPaso:numeroPaso}{Número} y \cdtRef{entidadPaso:numeroPaso}{redaccionPaso} de todos los pasos registrados en una trayectoria.
			\item \cdtIdRef{MSG2}{No existe información}: Se muestra en la pantalla \IUref{IU12A}{Pendiente} cuando no existen pasos registrados.
	\end{itemize}}
	
	\UCitem{Precondiciones}{Que exista al menos una trayectoria registrada.}
	\UCitem{Postcondiciones}{Ninguna}
	\UCitem{Errores}{Ninguno}
	\UCitem{Tipo}{Secundario, extiende del caso de uso \UCref{CU12.1.1}{Gestionar Trayectorias}.}
\end{UseCase}
%--------------------------------------
\begin{UCtrayectoria}
	\UCpaso[\UCactor] Solicita gestionar los pasos de una trayectoria seleccionando el icono (Pendiente) de la pantalla \IUref{IU6.1.1}{Gestionar Trayectorias}.
	\UCpaso[\UCsist] Obtiene la información de los pasos registrados de la trayectoria seleccionada. \hyperlink{CU12-1-1-1-1:TAA}{[Trayectoria A]}
	\UCpaso[\UCsist] Ordena los pasos en base a los números de los mismos.
	\UCpaso[\UCsist] Muestra la información de las pasos en la pantalla \IUref{IU12}{Pendiente} y las operaciones disponibles de acuerdo a la regla de negocio \BRref{RN15}{Operaciones disponibles}. \label{CU12-1-1-1-1-P4}
	\UCpaso[\UCactor] Gestiona las pasos a través de los botones: \IUbutton{Registrar}, \editar y \eliminar. 
\end{UCtrayectoria}		
%--------------------------------------
\hypertarget{CU12-1-1-1-1:TAA}{\textbf{Trayectoria alternativa A}}\\
\noindent \textbf{Condición:} No existen registros de pasos.
\begin{enumerate}
	\UCpaso[\UCsist] Muestra el mensaje \cdtIdRef{MSG2}{No existe información} en la pantalla \IUref{IU7A}{Pendiente} para indicar que no hay registros de pasos para mostrar.  \label{CU12-1-1-1-1-TA1}
	\UCpaso[\UCactor] Gestiona las pasos a través del botón: \IUbutton{Registrar}. 
	\item[- -] - - {\em {Fin de la trayectoria}}.%
\end{enumerate}
%--------------------------------------
\subsubsection{Puntos de extensión}

\UCExtenssionPoint{El actor requiere registrar un paso de la trayectoria.}{Presionando el botón \IUbutton{Registrar} del paso \ref{CU12-1-1-1-1-P4} de la trayectoria principal o del paso \ref{CU12-1-1-1-1-TA1} de la Trayectoria alternativa A.}{\UCref{CU12.1.1.1.1.1}{Registrar Paso}}
\UCExtenssionPoint{El actor requiere modificar un paso de la trayectoria.}{Presionando el icono \editar del paso \ref{CU12-1-1-1-1-P4} de la trayectoria principal.}{\UCref{CU12.1.1.1.1.2}{Modificar Paso}}
\UCExtenssionPoint{El actor requiere eliminar un paso de la trayectoria.}{Presionando el icono \eliminar del paso \ref{CU12-1-1-1-1-P4} de la trayectoria principal.}{\UCref{CU12.1.1.1.1.3}{Eliminar Paso}}
