	\begin{UseCase}{CU12.1.2.2}{Modificar Pre/Postcondición}{
		
		Pendiente
		
	}
	\UCitem{Actor}{\hyperlink{jefe}{Líder de Análisis}, \hyperlink{analista}{Analista}}
	\UCitem{Propósito}{Modificar la información de una precondición/postcondición de un caso de uso.}
	\UCitem{Entradas}{
		\begin{itemize}
			\item Tipo de condición: Se selecciona de una lista.
			\item \cdtRef{entidadPrecondicion:redaccionPrecondicion}{Redacción de la precondición}: Se escribe desde el teclado o \cdtRef{entidadPostcondicion:redaccionPostcondicion}{Redacción de la postcondición}: Se escribe desde el teclado.
		\end{itemize}
	}
	\UCitem{Salidas}{\begin{itemize}
			\item \cdtRef{proyectoEntidad:claveProyecto}{Clave del proyecto}: Lo obtiene el sistema.
			\item \cdtRef{proyectoEntidad:nombreProyecto}{Nombre del proyecto}: Lo obtiene el sistema.
			\item \cdtRef{moduloEntidad:nombreModulo}{Nombre del Módulo}: Lo obtiene el sistema.
			\item \cdtRef{casoUso:numeroCU}{Número} del caso de uso: Lo obtiene el sistema. 
			\item \cdtRef{casoUso:nombreCU}{Nombre} del caso de uso: Lo obtiene el sistema.
			\item Tipo de condición: Lo obtiene el sistema..
			\item \cdtRef{entidadPrecondicion:redaccionPrecondicion}{Redacción de la precondición}: Lo obtiene el sistema o \cdtRef{entidadPostcondicion:redaccionPostcondicion}{Redacción de la postcondición}: Lo obtiene el sistema.
			\item \cdtIdRef{MSG1}{Operación exitosa}: Se muestra en la pantalla \IUref{IU6.1.2}{Gestionar Pre/Postcondiciones} para indicar que la modificación fue exitosa.
		\end{itemize}
	}
	\UCitem{Precondiciones}{
		\begin{itemize}
			\item Que exista al menos un caso de uso registrado.
			\item Que exista al menos una condición registrada.
			\item Que el caso de uso se encuentre en estado ''Edición'' o ''Pendiente de corrección".
		\end{itemize}
	}
	\UCitem{Postcondiciones}{Se actualizará la información de una condición perteneciente a un caso de uso en el sistema.}
	\UCitem{Errores}{\begin{itemize}
			\item \cdtIdRef{MSG4}{Dato obligatorio}: Se muestra en la pantalla \IUref{IU6.1.2.2}{Modificar Pre/Postcondición} cuando no se ha ingresado un dato marcado como obligatorio.
			\item \cdtIdRef{MSG5}{Formato incorrecto}: Se muestra en la pantalla \IUref{IU6.1.2.2}{Modificar Pre/Postcondición} cuando el tipo de dato ingresado no cumple con el tipo de dato solicitado en el campo.
			\item \cdtIdRef{MSG6}{Longitud inválida}: Se muestra en la pantalla \IUref{IU6.1.2.2}{Modificar Pre/Postcondición} cuando se ha excedido la longitud de alguno de los campos.
		\end{itemize}.}
	\UCitem{Tipo}{Secundario, extiende del caso de uso \UCref{CU12.1.2}{Gestionar Pre/Postcondiciones}.}
\end{UseCase}
%--------------------------------------
\begin{UCtrayectoria}
	\UCpaso[\UCactor] Da clic en el icono \editar del registro que desee de la pantalla \IUref{IU6.1.2}{Gestionar Pre/Postcondiciones}.
	\UCpaso[\UCsist] Obtiene la información de la condición.
	\UCpaso[\UCsist] Muestra la pantalla \IUref{IU6.1.2.2}{Modificar Pre/Postcondición}.
	\UCpaso[\UCactor] Modifica la información de la condición. \hyperlink{CU12-1-2-2:TAA}{[Trayectoria A]} \label{CU12.1.2.2-P3}
	\UCpaso[\UCactor] Oprime el botón \IUbutton{Aceptar}. \hyperlink{CU12-1-2-2:TAB}{[Trayectoria B]} \label{CU12.1.2.2-P4}
	\UCpaso[\UCsist] Verifica que el actor ingrese todos los campos obligatorios con base en la regla de negocio \BRref{RN8}{Datos obligatorios}. \hyperlink{CU12-1-2-2:TAC}{[Trayectoria C]}
	\UCpaso[\UCsist] Verificar que los datos ingresados cumpla con la longitud correcta, con base en la regla de negocio \BRref{RN37}{Longitud de datos} \hyperlink{CU12-1-2-2:TAD}{[Trayectoria D]}
	\UCpaso[\UCsist] Verifica que los datos ingresados cumplan con el formato requerido, con base en la regla de negocio \BRref{RN7}{Información correcta}. \hyperlink{CU12-1-2-2:TAE}{[Trayectoria E]}
	\UCpaso[\UCsist] Actualiza la información de la condición en el sistema.
	\UCpaso[\UCsist] Muestra el mensaje \cdtIdRef{MSG1}{Operación exitosa} en la pantalla \IUref{IU6.1.2}{Gestionar Pre/Postcondiciones} para indicar al actor que la modificación se ha realizado exitosamente.
\end{UCtrayectoria}		
%--------------------------------------
\hypertarget{CU12-1-2-2:TAA}{\textbf{Trayectoria alternativa A}}\\
\noindent \textbf{Condición:} El actor desea hacer referencia a un elemento existente en el proyecto.
\begin{enumerate}
	\UCpaso[\UCactor] Ingresa el token correspondiente al elemento a referenciar.
	\UCpaso[\UCsist] Verifica que los tokens utilizados se encuentren correctamente estructurados, con base en la regla de negocio \BRref{RN31}{Estructura de Tokens}. 
	\UCpaso[\UCsist] Obtiene los \hyperlink{tElemento}{elementos} registrados en el proyecto correspondientes al token ingresado. 
	\UCpaso[\UCsist] Muestra una lista con los \hyperlink{tElemento}{elementos} encontrados.
	\UCpaso[\UCactor] Selecciona un elemento de la lista.
	\UCpaso[\UCsist] Verifica que el nombre del elemento seleccionado no contenga espacios. \hyperlink{CU12-1-2-2:TAF}{[Trayectoria F]}
	\UCpaso[\UCsist] Agrega la referencia del elemento al texto. \label{CU12.1.2.2-TA1}
	\UCpaso Continúa en el paso \ref{CU12.1.2.2-P4} de la trayectoria principal.
	\item[- -] - - {\em {Fin de la trayectoria}}.%
\end{enumerate}
%--------------------------------------
\hypertarget{CU12-1-2-2:TAB}{\textbf{Trayectoria alternativa B}}\\
\noindent \textbf{Condición:} El actor desea cancelar la operación.
\begin{enumerate}
	\UCpaso[\UCactor] Solicita cancelar la operación oprimiendo el botón \IUbutton{Cancelar} de la pantalla \IUref{IU6.1.2.2}{Modificar Pre/Postcondición}.
	\UCpaso[\UCsist] Muestra la pantalla \IUref{IU6.1.2}{Gestionar Pre/Postcondiciones}.
	\item[- -] - - {\em {Fin del caso de uso}}.%
\end{enumerate}
%--------------------------------------
\hypertarget{CU12-1-2-2:TAC}{\textbf{Trayectoria alternativa C}}\\
\noindent \textbf{Condición:} El actor no ingresó algún dato marcado como obligatorio.
\begin{enumerate}
	\UCpaso[\UCsist] Muestra el mensaje \cdtIdRef{MSG4}{Dato obligatorio} señalando el campo que presenta el error en la pantalla \IUref{IU6.1.2.2}{Modificar Pre/Postcondición}.
	\UCpaso Regresa al paso \ref{CU12.1.2.2-P3} de la trayectoria principal.
	\item[- -] - - {\em {Fin de la trayectoria}}.%
\end{enumerate}
%--------------------------------------
\hypertarget{CU12-1-2-2:TAD}{\textbf{Trayectoria alternativa D}}\\
\noindent \textbf{Condición:} El actor ingresó un dato con un número de caracteres fuera del rango permitido.
\begin{enumerate}
	\UCpaso[\UCsist] Muestra el mensaje \cdtIdRef{MSG6}{Longitud inválida} señalando el campo que presenta el error en la pantalla \IUref{IU6.1.2.2}{Modificar Pre/Postcondición}.
	\UCpaso Regresa al paso \ref{CU12.1.2.2-P3} de la trayectoria principal.
	\item[- -] - - {\em {Fin de la trayectoria}}.%
\end{enumerate}
%--------------------------------------
\hypertarget{CU12-1-2-2:TAE}{\textbf{Trayectoria alternativa E}}\\
\noindent \textbf{Condición:} El actor ingresó un dato con un formato incorrecto.
\begin{enumerate}
	\UCpaso[\UCsist] Muestra el mensaje \cdtIdRef{MSG5}{Formato incorrecto} señalando el campo que presenta el error en la pantalla \IUref{IU6.1.2.2}{Modificar Pre/Postcondición}.
	\UCpaso Regresa al paso \ref{CU12.1.2.2-P3} de la trayectoria principal.
	\item[- -] - - {\em {Fin de la trayectoria}}.
\end{enumerate}
%--------------------------------------
\hypertarget{CU12-1-2-2:TAF}{\textbf{Trayectoria alternativa F}}\\
\noindent \textbf{Condición:} El texto contiene espacios.
\begin{enumerate}
	\UCpaso[\UCsist] Sustituye los espacios por guiones bajos.
	\UCpaso Continua en el paso \ref{CU12.1.2.2-TA1} de la trayectoria alternativa A.
	\item[- -] - - {\em {Fin de la trayectoria}}.
\end{enumerate}