	\begin{UseCase}{CU12.1.2}{Gestionar Precondiciones}{
     	Al momento de registrar un caso de uso, el sistema le permitirá al colaborador gestionar las precondiciones del caso de uso (de tal forma que podrá registrarlas o eliminarlas). Las precondiciones son condiciones que ha de satisfacerse justo antes del comienzo de la ejecución de las instrucciones del caso de uso.
	}
	\UCitem{Actor}{\hyperlink{jefe}{Líder de análisis}, \hyperlink{analista}{Analista}}
	\UCitem{Propósito}{Proporcionar al actor un mecanismo para llevar el control de las precondiciones pertenecientes a caso de uso.}
	\UCitem{Entradas}{Ninguna}
	\UCitem{Salidas}{\begin{itemize}
			\item \cdtRef{proyectoEntidad:claveProyecto}{Clave del proyecto}: Lo obtiene el sistema.
			\item \cdtRef{proyectoEntidad:nombreProyecto}{Nombre del proyecto}: Lo obtiene el sistema.
			\item \cdtRef{moduloEntidad:nombreModulo}{Nombre del Módulo}: Lo obtiene el sistema.
			\item \cdtRef{casoUso:numeroCU}{Número} del caso de uso: Lo obtiene el sistema. 
			\item \cdtRef{casoUso:nombreCU}{Nombre} del caso de uso: Lo obtiene el sistema.
			\item \cdtRef{entidadPrecondicion}{Precondición}: Tabla que muestra \cdtRef{entidadPrecondicion:redaccionPrecondicion}{Redacción} de todas las precondiciones registradas en un caso de uso.
			\item \cdtIdRef{MSG2}{No existe información}: Se muestra en la pantalla \IUref{IU12A}{Pendiente} cuando no existen precondiciones registradas.
	\end{itemize}}
	
	\UCitem{Precondiciones}{Que exista al menos un caso de uso registrado.}
	\UCitem{Postcondiciones}{Ninguna}
	\UCitem{Errores}{Ninguno}
	\UCitem{Tipo}{Secundario, extiende del caso de uso \UCref{CU12}{Gestionar Casos de uso}.}
\end{UseCase}
%--------------------------------------
\begin{UCtrayectoria}
	\UCpaso[\UCactor] Solicita gestionar las precondiciones de un caso de uso seleccionando el icono (Pendiente) de la pantalla \IUref{IU6}{Gestionar Casos de uso}.
	\UCpaso[\UCsist] Obtiene la información de las precondiciones registradas de caso de uso seleccionado. \hyperlink{CU12-1-2:TAA}{[Trayectoria A]}
	\UCpaso[\UCsist] Ordena las precondiciones alfabéticamente basándose en la redacción de las mismas.
	\UCpaso[\UCsist] Muestra la información de las precondiciones en la pantalla \IUref{IU12}{Pendiente} y las operaciones disponibles de acuerdo a la regla de negocio \BRref{RN15}{Operaciones disponibles}. \label{CU12-1-2-P4}
	\UCpaso[\UCactor] Gestiona las precondiciones a través de los botones: \IUbutton{Registrar} y \eliminar. 
\end{UCtrayectoria}		
%--------------------------------------
\hypertarget{CU12-1-2:TAA}{\textbf{Trayectoria alternativa A}}\\
\noindent \textbf{Condición:} No existen registros de precondiciones.
\begin{enumerate}
	\UCpaso[\UCsist] Muestra el mensaje \cdtIdRef{MSG2}{No existe información} en la pantalla \IUref{IU7A}{Pendiente} para indicar que no hay registros de precondiciones para mostrar.  \label{CU12-1-2-TA1}
	\UCpaso[\UCactor] Gestiona las precondiciones a través del botón: \IUbutton{Registrar}. 
	\item[- -] - - {\em {Fin de la trayectoria}}.%
\end{enumerate}
%--------------------------------------
\subsubsection{Puntos de extensión}

\UCExtenssionPoint{El actor requiere registrar una precondición}{Presionando el botón \IUbutton{Registrar} del paso \ref{CU12-1-2-P4} de la trayectoria principal o del paso \ref{CU12-1-2-TA1} de la Trayectoria Alternativa A.}{\UCref{CU12.1.2.1}{Registrar Precondición}}
\UCExtenssionPoint{El actor requiere eliminar una precondición}{Presionando el icono \eliminar del paso \ref{CU12-1-2-P4} de la trayectoria principal.}{\UCref{CU12.1.2.2}{Eliminar Precondición}}
