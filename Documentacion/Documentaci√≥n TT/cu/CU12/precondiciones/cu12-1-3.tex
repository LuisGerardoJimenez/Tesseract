	\begin{UseCase}{CU12.1.3}{Eliminar Precondición}{
		Este caso de uso permite al actor eliminar un registro de la tabla de precondiciones perteneciente a un trayectoria de un caso de uso.
	}
	\UCitem{Versión}{\color{Gray}0.1}
	\UCitem{Actor}{\hyperlink{jefe}{Líder de Análisis}, \hyperlink{analista}{Analista}}
	\UCitem{Propósito}{Eliminar una precondición de un caso de uso.}
	\UCitem{Entradas}{Ninguna.}
	\UCitem{Salidas}{
		\begin{itemize}
			\item \cdtIdRef{MSG10}{Confirmar eliminación}: Se muestra en la pantalla \IUref{IU6.1}{Registrar Caso de uso} o \IUref{IU6.2}{Modificar Caso de uso} para que el actor confirme la eliminación.
		\end{itemize}
	}
	\UCitem{Destino}{Pantalla}
	\UCitem{Precondiciones}{Ninguna}
	\UCitem{Postcondiciones}{
		\begin{itemize}
			\item Se eliminará la precondición de la tabla.
		\end{itemize}
	}
	\UCitem{Errores}{Ninguno.}
	\UCitem{Tipo}{Secundario, extiende del caso de uso \UCref{CU12.1}{Registrar Caso de uso} y \UCref{CU12.2}{Modificar Caso de uso}.}
\end{UseCase}
%--------------------------------------
\begin{UCtrayectoria}
	\UCpaso[\UCactor] Solicita eliminar una precondición oprimiendo el botón \eliminar del registro que desea eliminar de la tabla de precondiciones de la pantalla \IUref{IU6.1}{Registrar Caso de uso} o \IUref{IU6.2}{Modificar Caso de uso}.
	\UCpaso[\UCsist] Muestra el mensaje emergente \cdtIdRef{MSG10}{Confirmar eliminación} con los botones \IUbutton{Aceptar} y \IUbutton{Cancelar} en la pantalla \IUref{IU6.1}{Registrar Caso de uso} o \IUref{IU6.2}{Modificar Caso de uso}.
	\UCpaso[\UCactor] Confirma la eliminación del paso oprimiendo el botón \IUbutton{Aceptar}. \Trayref{EPRE-A}
	\UCpaso[\UCsist] Elimina la precondición de la tabla correspondiente.
\end{UCtrayectoria}		
%-------------------------------------

\begin{UCtrayectoriaA}{EPRE-A}{El actor desea cancelar la operación.}
	\UCpaso[\UCactor] Oprime el botón \IUbutton{Cancelar} de la pantalla emergente.
	\UCpaso[\UCsist] Muestra la pantalla \IUref{IU6.1}{Registrar Caso de uso} o \IUref{IU6.2}{Modificar Caso de uso}.
\end{UCtrayectoriaA}


