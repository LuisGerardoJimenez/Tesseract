	\begin{UseCase}{CU7.3}{Eliminar Entidad}{
			
			Cuando ya no se hará uso de alguna \hyperlink{entidadEntidad}{Entidad} que estaba considerada o simplemente no tiene una razón de ser dentro del \hyperlink{proyectoEntidad}{Proyecto}, Tesseract permitirá al colaborador (\hyperlink{jefe}{Líder} o \hyperlink{analista}{Analista}) eliminar en su totalidad el registro de dicha entidad. \\
			Una entidad podrá ser eliminada siempre y cuando no se encuentre asociada a algún caso de uso con estado ”Liberado”.

	}
		\UCitem{Actor}{\hyperlink{jefe}{Líder de Análisis}, \hyperlink{analista}{Analista}}
		\UCitem{Propósito}{Eliminar la información de una entidad.}
		\UCitem{Entradas}{Ninguna}
		\UCitem{Salidas}{\begin{itemize}
				\item \cdtIdRef{MSG1}{Operación exitosa}: Se muestra en la pantalla \IUref{IU12}{Gestionar Entidade} para indicar que la entidad fue eliminada correctamente.
				\item \cdtIdRef{MSG10}{Confirmar eliminación}: Se muestra en la pantalla \IUref{IU12}{Gestionar Entidades} preguntando al actor si desea continuar con la eliminación de la entidad.
		\end{itemize}}
		
		\UCitem{Precondiciones}{\begin{itemize}
				\item Que la entidad no se encuentre asociada a un caso de uso.
				\item Que la entidad no se encuentre asociada a un caso de uso liberado
		\end{itemize}}
		\UCitem{Postcondiciones}{
			Se eliminará una entidad de un proyecto del sistema.
		}
		\UCitem{Errores}{
		\cdtIdRef{MSG13}{Eliminación no permitida}: Se muestra en la pantalla \IUref{IU12}{Gestionar Entidades} cuando la entidad está siendo referenciado en algún caso de uso.
		}
		\UCitem{Tipo}{Secundario, extiende del caso de uso \UCref{CU7}{Gestionar Entidades}.}
	\end{UseCase}
%--------------------------------------
	\begin{UCtrayectoria}
		\UCpaso[\UCactor] Da clic en el icono \eliminar del registro que desea eliminar de la pantalla \IUref{IU12}{Gestionar Entidades}.
		\UCpaso[\UCsist] Muestra el mensaje emergente \cdtIdRef{MSG10}{Confirmar eliminación} con los botones \IUbutton{Aceptar} y \IUbutton{Cancelar} en la pantalla \IUref{IU12}{Gestionar entidades}.
		\UCpaso[\UCactor] Confirma la eliminación de la entidad oprimiendo el botón \IUbutton{Aceptar}. \hyperlink{CU7-3:TAA}{[Trayectoria A]}
		\UCpaso[\UCsist] Verifica que ningún caso de uso se encuentre asociado a la entidad. \hyperlink{CU7-3:TAB}{[Trayectoria B]}
		\UCpaso[\UCsist] Verifica que ningún caso de uso se encuentre asociado a alguno de los atributos de la entidad. \hyperlink{CU7-3:TAC}{[Trayectoria C}
		\UCpaso[\UCsist] Elimina la información referente a la entidad.
		\UCpaso[\UCsist] Muestra el mensaje \cdtIdRef{MSG1}{Operación exitosa} en la pantalla \IUref{IU12}{Gestionar Entidades} para indicar al actor que el registro se ha eliminado exitosamente.
	\end{UCtrayectoria}

%--------------------------------------
\hypertarget{CU7-3:TAA}{\textbf{Trayectoria alternativa A}}\\
\noindent \textbf{Condición:} El actor desea cancelar la operación.
\begin{enumerate}
	\UCpaso[\UCactor] Oprime el botón \IUbutton{Cancelar} de la pantalla emergente.
	\UCpaso[\UCsist] Muestra la pantalla \IUref{IU12}{Gestionar Entidades}.
	\item[- -] - - {\em {Fin del caso de uso}}.%
\end{enumerate}
%--------------------------------------
\hypertarget{CU7-3:TAB}{\textbf{Trayectoria alternativa B}}\\
\noindent \textbf{Condición:} La entidad está siendo referenciado en un caso de uso.
\begin{enumerate}
	\UCpaso[\UCsist] Muestra el mensaje \cdtIdRef{MSG13}{Eliminación no permitida} en la pantalla \IUref{IU12}{Gestionar Entidades}.
	\item[- -] - - {\em {Fin del caso de uso}}.
\end{enumerate}
%--------------------------------------
\hypertarget{CU7-3:TAC}{\textbf{Trayectoria alternativa C}}\\
\noindent \textbf{Condición:} Algunos atributos de la entidad están siendo referenciados en algún caso de uso.
\begin{enumerate}
	\UCpaso[\UCsist] Muestra el mensaje \cdtIdRef{MSG13}{Eliminación no permitida} en la pantalla \IUref{IU12}{Gestionar Entidades}.
	\item[- -] - - {\em {Fin del caso de uso}}.
\end{enumerate}
	

