	\begin{UseCase}{CU7.3}{Eliminar Entidad}{
		Este caso de uso permite al actor eliminar del sistema una entidad.
	}
		\UCitem{Versión}{\color{Gray}0.1}
		\UCitem{Actor}{\hyperlink{jefe}{Líder de Análisis}, \hyperlink{analista}{Analista}}
		\UCitem{Propósito}{Eliminar la información de una entidad.}
		\UCitem{Entradas}{Ninguna}
		\UCitem{Salidas}{\begin{itemize}
				\item \cdtIdRef{MSG1}{Operación exitosa}: Se muestra en la pantalla \IUref{IU12}{Gestionar Entidade} para indicar que el término fue eliminado correctamente.
				\item \cdtIdRef{MSG10}{Confirmar eliminación}: Se muestra en la pantalla \IUref{IU12}{Gestionar Entidades} para que el actor confirme la eliminación.
		\end{itemize}}
		\UCitem{Destino}{Pantalla}
		\UCitem{Precondiciones}{Ninguna}
		\UCitem{Postcondiciones}{
		\begin{itemize}
			\item Se eliminará una entidad de un proyecto del sistema.
		\end{itemize}
		}
		\UCitem{Errores}{\begin{itemize}
		\item \cdtIdRef{MSG12}{Ha ocurrido un error}: Se muestra en una pantalla emergente cuando no se pueda eliminar una entidad debido a que está siendo referenciada en algún caso de uso.
		\item \cdtIdRef{MSG13}{Eliminación no permitida}: Se muestra en la pantalla \IUref{IU12}{Gestionar Entidades} cuando la entidad del glosario no se encuentre en un estado que permita la eliminación.
		\end{itemize}
		}
		\UCitem{Tipo}{Secundario, extiende del caso de uso \UCref{CU6}{Gestionar Términos del glosario}.}
	\end{UseCase}
%--------------------------------------
	\begin{UCtrayectoria}
		\UCpaso[\UCactor] Solicita eliminar una entidad oprimiendo el botón \eliminar del registro que desea eliminar de la pantalla \IUref{IU12}{Gestionar Entidades}.
		\UCpaso[\UCsist] Verifica que la entidad pueda eliminarse de acuerdo a la regla de negocio \BRref{RN18}{Eliminación de elementos}. \hyperlink{CU7-3:TAA}{[Trayectoria A]}
		\UCpaso[\UCsist] Verifica que ningún caso de uso se encuentre asociado a la entidad. \hyperlink{CU7-3:TAB}{[Trayectoria B]}
		\UCpaso[\UCsist] Verifica que ningún caso de uso se encuentre asociado a alguno de los atributos de la entidad. \hyperlink{CU7-3:TAC}{[Trayectoria C}
		\UCpaso[\UCsist] Muestra el mensaje emergente \cdtIdRef{MSG10}{Confirmar eliminación} con los botones \IUbutton{Aceptar} y \IUbutton{Cancelar} en la pantalla \IUref{IU12}{Gestionar entidades}.
		\UCpaso[\UCactor] Confirma la eliminación de la entidad oprimiendo el botón \IUbutton{Aceptar}. \hyperlink{CU7-3:TAD}{[Trayectoria D]}
		\UCpaso[\UCsist] Elimina la información referente a la entidad.
		\UCpaso[\UCsist] Muestra el mensaje \cdtIdRef{MSG1}{Operación exitosa} en la pantalla \IUref{IU12}{Gestionar Entidades} para indicar al actor que el registro se ha eliminado exitosamente.
	\end{UCtrayectoria}		
%--------------------------------------
\hypertarget{CU7-3:TAA}{\textbf{Trayectoria alternativa A}}\\
\noindent \textbf{Condición:} La entidad se encuentra asociada a casos de uso liberados.
\begin{enumerate}
	\UCpaso[\UCsist] Oculta el botón \eliminar de la entidad que esta asociada a casos de uso liberados.
	\item[- -] - - {\em {Fin del caso de uso}}.
\end{enumerate}
%--------------------------------------
\hypertarget{CU7-3:TAB}{\textbf{Trayectoria alternativa B}}\\
\noindent \textbf{Condición:} La entidad está siendo referenciado en un caso de uso.
\begin{enumerate}
	\UCpaso[\UCsist] Muestra el mensaje \cdtIdRef{MSG13}{Eliminación no permitida} en la pantalla \IUref{IU12}{Gestionar Entidades} en una pantalla emergente con la lista de casos de uso que están referenciando a la entidad.
	\UCpaso[\UCactor] Oprime el botón \IUbutton{Aceptar} de la pantalla emergente.
	\UCpaso[\UCsist] Muestra la pantalla \IUref{IU12}{Gestionar Entidades}.
	\item[- -] - - {\em {Fin del caso de uso}}.
\end{enumerate}
%--------------------------------------
\hypertarget{CU7-3:TAC}{\textbf{Trayectoria alternativa C}}\\
\noindent \textbf{Condición:} Algunos atributos de la entidad están siendo referenciados en algún caso de uso.
\begin{enumerate}
	\UCpaso[\UCsist] Muestra el mensaje \cdtIdRef{MSG13}{Eliminación no permitida} en la pantalla \IUref{IU12}{Gestionar Entidades} en una pantalla emergente con la lista de elementos que están referenciando al atributo de la entidad.
	\UCpaso[\UCactor] Oprime el botón \IUbutton{Aceptar} de la pantalla emergente.
	\UCpaso[\UCsist] Muestra la pantalla \IUref{IU12}{Gestionar Entidades}.
	\item[- -] - - {\em {Fin del caso de uso}}.
\end{enumerate}
%--------------------------------------
\hypertarget{CU7-3:TAD}{\textbf{Trayectoria alternativa D}}\\
\noindent \textbf{Condición:} El actor desea cancelar la operación.
\begin{enumerate}
	\UCpaso[\UCactor] Oprime el botón \IUbutton{Cancelar} de la pantalla emergente.
	\UCpaso[\UCsist] Muestra la pantalla \IUref{IU12}{Gestionar Entidades}.
	\item[- -] - - {\em {Fin del caso de uso}}.%
\end{enumerate}
	

