	\begin{UseCase}{CU7.2}{Modificar Entidad}{
			
		Este caso de uso permite al colaborador (\hyperlink{jefe}{Líder de Análisis} o \hyperlink{analista}{Analista}) modificar la información de una \hyperlink{entidadEntidad}{Entidad} previamente registrada en el sistema. La actualización de dicha información se lleva a cabo por medio de un formulario en donde se encuentran precargados los datos de la última actualización realizada, con la posibilidad de editarlos y almacenarlos. Dentro de este caso de uso se le permitirá también al colaborador gestionar los atributos que componen la entidad (de tal forma que podrá registrar, modificar y eliminar un atributo).
		Una entidad podrá ser modificada siempre y cuando no se encuentre asociada a algún caso de uso con estado ”Liberado”.

	}
		
		\UCitem{Actor}{\hyperlink{jefe}{Líder de Análisis}, \hyperlink{analista}{Analista}}
		\UCitem{Propósito}{Modificar la información de una entidad y gestionar sus atributos.}
		\UCitem{Entradas}{
		\begin{itemize}
			\item \cdtRef{entidadEntidad:nombreEntidad}{Nombre de la entidad:} Se escribe desde el teclado.
			\item \cdtRef{entidadEntidad:descripcionEntidad}{Descripción de la entidad:} Se escribe desde el teclado.
		\end{itemize}	
		}
		\UCitem{Salidas}{\begin{itemize}
				\item \cdtRef{proyectoEntidad:claveProyecto}{Clave del proyecto:} Lo obtiene el sistema.
				\item \cdtRef{proyectoEntidad:nombreProyecto}{Nombre del proyecto:} Lo obtiene el sistema.
				\item \cdtRef{entidadEntidad:nombreEntidad}{Nombre de la entidad:} Lo obtiene el sistema.
				\item \cdtRef{entidadEntidad:descripcionEntidad}{Descripción de la entidad:} Lo obtiene el sistema.
				\item \cdtIdRef{MSG1}{Operación exitosa}: Se muestra en la pantalla \IUref{IU12}{Gestionar Entidades} para indicar que la modificación fue exitosa.
		\end{itemize}}
		
		\UCitem{Precondiciones}{
			\begin{itemize}
				\item Que exista al menos una entidad registrada
				\item Que la entidad no se encuentre asociada a un caso de uso en estado ''Liberado''.
			\end{itemize}
		}
		\UCitem{Postcondiciones}{Se actualizará la información de una entidad de un proyecto}
		\UCitem{Errores}{\begin{itemize}
		\item \cdtIdRef{MSG4}{Dato obligatorio}: Se muestra en la pantalla \IUref{IU12.2}{Modificar Entidad} cuando no se ha ingresado un dato marcado como obligatorio.
		\item \cdtIdRef{MSG5}{Formato incorrecto}: Se muestra en la pantalla \IUref{IU12.2}{Modificar Entidad} cuando el tipo de dato ingresado no cumple con el tipo de dato solicitado en el campo.
		\item \cdtIdRef{MSG6}{Longitud inválida}: Se muestra en la pantalla \IUref{IU12.2}{Modificar Entidad} cuando se ha excedido la longitud de alguno de los campos.
		\item \cdtIdRef{MSG7}{Registro repetido}: Se muestra en la pantalla \IUref{IU12.2}{Modificar Entidad} cuando se registre una entidad con un nombre que ya se encuentra registrado en el sistema.
		%\item \cdtIdRef{MSG14}{Registro necesario}: Se muestra en la pantalla \IUref{IU12.2}{Modificar Entidad} cuando el actor no ingrese ningún atributo.
		\end{itemize}
		}
		\UCitem{Tipo}{Secundario, extiende del caso de uso \UCref{CU7}{Gestionar Entidades}.}
	\end{UseCase}
%--------------------------------------
	\begin{UCtrayectoria}
		\UCpaso[\UCactor] Da clic en el icono \editar de algún registro existente en la pantalla \IUref{IU12}{Gestionar Entidades}.
		\UCpaso[\UCsist] Obtiene la información de la entidad.
		\UCpaso[\UCsist] Muestra la pantalla \IUref{IU12.2}{Modificar Entidad}. \label{CU7.2-P4}
		\UCpaso[\UCactor] Modifica la información de la entidad. \label{CU7.2-P5}
		\UCpaso[\UCactor] Oprime el botón \IUbutton{Aceptar}. \hyperlink{CU7-2:TAA}{[Trayectoria A]}
		\UCpaso[\UCsist] Verifica que el actor ingrese todos los campos obligatorios con base en la regla de negocio \BRref{RN8}{Datos obligatorios}. \hyperlink{CU7-2:TAB}{[Trayectoria B]}
		\UCpaso[\UCsist] Verificar que los datos ingresados cumpla con la longitud correcta, con base en la regla de negocio \BRref{RN37}{Longitud de datos}. \hyperlink{CU7-2:TAC}{[Trayectoria C]}
		\UCpaso[\UCsist] Verifica que los datos ingresados cumplan con el formato requerido, con base en la regla de negocio \BRref{RN7}{Información correcta}. \hyperlink{CU7-2:TAD}{[Trayectoria D]}
		\UCpaso[\UCsist] Verifica que el nombre de la entidad no se encuentre registrado en el sistema con base en la regla de negocio \BRref{RN6}{Unicidad de nombres}. \hyperlink{CU7-2:TAE}{[Trayectoria E]} 
		\UCpaso[\UCsist] Actualiza la información de la entidad en el sistema.
		\UCpaso[\UCsist] Muestra el mensaje \cdtIdRef{MSG1}{Operación exitosa} en la pantalla \IUref{IU12}{Gestionar Entidades} para indicar al actor que la modificación se ha realizado exitosamente.
	\end{UCtrayectoria}		
%--------------------------------------
\hypertarget{CU7-2:TAA}{\textbf{Trayectoria alternativa A}}\\
\noindent \textbf{Condición:} El actor desea cancelar la operación.
\begin{enumerate}
	\UCpaso[\UCactor] Solicita cancelar la operación oprimiendo el botón \IUbutton{Cancelar} de la pantalla \IUref{IU12.2}{Modificar Entidad}
	\UCpaso[\UCsist] Muestra la pantalla \IUref{IU12}{Gestionar Entidades}.
	\item[- -] - - {\em {Fin del caso de uso}}.%
\end{enumerate}
%--------------------------------------
\hypertarget{CU7-2:TAB}{\textbf{Trayectoria alternativa B}}\\
\noindent \textbf{Condición:} El actor no ingresó algún dato marcado como obligatorio.
\begin{enumerate}
	\UCpaso[\UCsist] Muestra el mensaje \cdtIdRef{MSG4}{Dato obligatorio} señalando el campo que presenta el error en la pantalla \IUref{IU12.2}{Modificar Entidad}.
	\UCpaso Regresa al paso \ref{CU7.2-P5} de la trayectoria principal.
	\item[- -] - - {\em {Fin de la trayectoria}}.%
\end{enumerate}
%--------------------------------------
\hypertarget{CU7-2:TAC}{\textbf{Trayectoria alternativa C}}\\
\noindent \textbf{Condición:} El actor ingresó un dato con un número de caracteres fuera del rango permitido.
\begin{enumerate}
	\UCpaso[\UCsist] Muestra el mensaje \cdtIdRef{MSG6}{Longitud inválida} señalando el campo que presenta el error en la pantalla \IUref{IU12.2}{Modificar Entidad}.
	\UCpaso Regresa al paso \ref{CU7.2-P5} de la trayectoria principal.
	\item[- -] - - {\em {Fin de la trayectoria}}.%
\end{enumerate}
%--------------------------------------
\hypertarget{CU7-2:TAD}{\textbf{Trayectoria alternativa D}}\\
\noindent \textbf{Condición:} El actor ingresó un dato con un formato de dato incorrecto.
\begin{enumerate}
	\UCpaso[\UCsist] Muestra el mensaje \cdtIdRef{MSG5}{Formato incorrecto} señalando el campo que presenta el error en la pantalla \IUref{IU12.2}{Modificar Entidad}.
	\UCpaso Regresa al paso \ref{CU7.2-P5} de la trayectoria principal.
	\item[- -] - - {\em {Fin de la trayectoria}}.
\end{enumerate}
%--------------------------------------	
\hypertarget{CU7-2:TAE}{\textbf{Trayectoria alternativa E}}\\
\noindent \textbf{Condición:} El actor ingresó una entidad ya existe dentro del sistema.
\begin{enumerate}
	\UCpaso[\UCsist] Muestra el mensaje \cdtIdRef{MSG7}{Registro repetido} señalando el campo que presenta la duplicidad en la pantalla \IUref{IU12.2}{Modificar Entidad}.
	\UCpaso Regresa al paso \ref{CU7.2-P5} de la trayectoria principal.
	\item[- -] - - {\em {Fin de la trayectoria}}.
\end{enumerate}