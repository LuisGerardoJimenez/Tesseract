	\begin{UseCase}{CU7.1.1}{Gestionar Atributos}{
     	Las acciones disponibles para cada \hyperlink{atributoEntidad}{Atributo} dependerán del estado en el que se encuentre el caso de uso donde son referenciadas. 
	}
	\UCitem{Actor}{\hyperlink{jefe}{Líder de análisis}, \hyperlink{analista}{Analista}}
	\UCitem{Propósito}{Proporcionar al actor un mecanismo para llevar el control de los atributos pertenecientes a una entidad.}
	\UCitem{Entradas}{Ninguna}
	\UCitem{Salidas}{\begin{itemize}
			\item \cdtRef{proyectoEntidad:claveProyecto}{Clave del proyecto}: Lo obtiene el sistema.
			\item \cdtRef{proyectoEntidad:nombreProyecto}{Nombre del proyecto}: Lo obtiene el sistema.
			\item \cdtRef{entidadEntidad:nombreEntidad}{Nombre de la entidad:} Lo obtiene el sistema.
			\item \cdtRef{atributoEntidad}{Atributos:} Tabla que muestra \cdtRef{atributoEntidad:nombreATR}{Nombre}, \cdtRef{atributoEntidad:obligatorioATR}{Obligatorio (si o no)} y \hyperlink{tTipoDatoP}{Tipo de Dato} de todos los los registros de los atributos
			\item \cdtIdRef{MSG2}{No existe información}: Se muestra en la pantalla \IUref{IU12.1.1.1}{Gestionar Atributos} cuando no existen atributos registrados.
	\end{itemize}}
	
	\UCitem{Precondiciones}{Que exista al menos una entidad registrada.}
	\UCitem{Postcondiciones}{Ninguna}
	\UCitem{Errores}{Ninguno}
	\UCitem{Tipo}{Secundario, extiende del caso de uso \UCref{CU7}{Gestionar Entidades}.}
\end{UseCase}
%--------------------------------------
\begin{UCtrayectoria}
	\UCpaso[\UCactor] Solicita gestionar los atributos de una entidad seleccionando el icono \raisebox{-1mm}{\includegraphics[height=11pt]{images/Iconos/Atributo}} de la pantalla \IUref{IU12}{Gestionar Entidades}.
	\UCpaso[\UCsist] Obtiene la información de los atributos registrados de la entidad seleccionada. \hyperlink{CU7-1-1:TAA}{[Trayectoria A]}
	\UCpaso[\UCsist] Ordena los atributos alfabéticamente basándose en el nombres de los mismos.
	\UCpaso[\UCsist] Muestra la información de las atributos en la pantalla \IUref{IU12.1.1.1}{Gestionar Atributos} y las operaciones disponibles de acuerdo a la regla de negocio \BRref{RN15}{Operaciones disponibles}. \label{CU7-1-1-P4}
	\UCpaso[\UCactor] Gestiona los atributos a través de los botones: \IUbutton{Registrar}, \editar y \eliminar. 
\end{UCtrayectoria}		
%--------------------------------------
\hypertarget{CU7-1-1:TAA}{\textbf{Trayectoria alternativa A}}\\
\noindent \textbf{Condición:} No existen registros de atributos.
\begin{enumerate}
	\UCpaso[\UCsist] Muestra el mensaje \cdtIdRef{MSG2}{No existe información} en la pantalla \IUref{IU12A.1.1.1}{Gestionar Atributos: Sin registros} para indicar que no hay registros de atributos para mostrar.  \label{CU7-1-1-TA1}
	\UCpaso[\UCactor] Gestiona los atributos a través del botón: \IUbutton{Registrar}. 
	\item[- -] - - {\em {Fin de la trayectoria}}.%
\end{enumerate}
%--------------------------------------
\subsubsection{Puntos de extensión}

\UCExtenssionPoint{El actor requiere registrar un atributo.}{Presionando el botón \IUbutton{Registrar} del paso \ref{CU7-1-1-P4} de la trayectoria principal o del paso \ref{CU7-1-1-TA1} de la trayectoria alternativa A.}{\UCref{CU7.1.1.1}{Registrar Atributo}}
\UCExtenssionPoint{El actor requiere modificar un atributo.}{Presionando el icono \editar del paso \ref{CU7-1-1-P4} de la trayectoria principal.}{\UCref{CU7.1.1.2}{Modificar Atributo}}
\UCExtenssionPoint{El actor requiere eliminar un atributo.}{Presionando el icono \eliminar del paso \ref{CU7-1-1-P4} de la trayectoria principal.}{\UCref{CU7.1.1.3}{Eliminar Atributo}}
