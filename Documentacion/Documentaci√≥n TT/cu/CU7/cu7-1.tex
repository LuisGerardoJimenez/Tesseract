	\begin{UseCase}{CU7.1}{Registrar Entidad}{
			
		Este caso de uso permite al colaborador (\hyperlink{jefe}{Líder de Análisis} o \hyperlink{analista}{Analista}) registrar la información de una \hyperlink{entidadEntidad}{Entidad} del proyecto, al momento de registrar la entidad, el sistema también le permitirá al colaborador gestionar los atributos que la componen (de tal forma que podrá registrar, modificar y eliminar un atributo) .
		Para poder registrar una entidad es necesario que al menos se registre uno de sus atributos y una vez registrada la entidad, el colaborador podrá referenciarla en el editor.
	}
		\UCitem{Actor}{\hyperlink{jefe}{Líder de Análisis}, \hyperlink{analista}{Analista}}
		\UCitem{Propósito}{Registrar la información de una entidad y gestionar sus atributos.}
		\UCitem{Entradas}{
		\begin{itemize}
			\item \cdtRef{entidadEntidad:nombreEntidad}{Nombre de la entidad:} Se escribe desde el teclado.
			\item \cdtRef{entidadEntidad:descripcionEntidad}{Descripción de la entidad:} Se escribe desde el teclado.
		\end{itemize}	
		}
		\UCitem{Salidas}{\begin{itemize}
				\item \cdtRef{proyectoEntidad:claveProyecto}{Clave del proyecto:} Lo obtiene el sistema.
				\item \cdtRef{proyectoEntidad:nombreProyecto}{Nombre del proyecto:} Lo obtiene el sistema.
				\item \cdtRef{atributoEntidad}{Atributos:} Tabla que muestra \cdtRef{atributoEntidad:nombreATR}{Nombre}, \cdtRef{atributoEntidad:obligatorioATR}{Obligatorio (si o no)} y \hyperlink{tTipoDatoP}{Tipo de Dato} de todos los los registros de los atributos
				\item \cdtIdRef{MSG1}{Operación exitosa}: Se muestra en la pantalla \IUref{IU12}{Gestionar Entidades} para indicar que el registro fue exitoso.
		\end{itemize}}
		
		\UCitem{Precondiciones}{Ninguna}
		\UCitem{Postcondiciones}{
		\begin{itemize}
			\item Se registrará un entidad de un proyecto en el sistema.
			\item Se podrán gestionar los atributos de una entidad.
			\item La entidad podrá ser referenciada en casos de uso.
		\end{itemize}
		}
		\UCitem{Errores}{\begin{itemize}
		\item \cdtIdRef{MSG4}{Dato obligatorio}: Se muestra en la pantalla \IUref{IU12.1}{Registrar Entidad} cuando no se ha ingresado un dato marcado como obligatorio.
		\item \cdtIdRef{MSG5}{Formato incorrecto}: Se muestra en la pantalla \IUref{IU12.1}{Registrar Entidad} cuando el tipo de dato ingresado no cumple con el tipo de dato solicitado en el campo.
		\item \cdtIdRef{MSG6}{Longitud inválida}: Se muestra en la pantalla \IUref{IU12.1}{Registrar Entidad} cuando se ha excedido la longitud de alguno de los campos.
		\item \cdtIdRef{MSG7}{Registro repetido}: Se muestra en la pantalla \IUref{IU12.1}{Registrar Entidad} cuando se registre una entidad con un nombre que ya se encuentra registrado en el sistema.
		\item \cdtIdRef{MSG14}{Registro necesario}: Se muestra en la pantalla \IUref{IU12.1}{Registrar Entidad} cuando el actor no ingrese ningún atributo.
		\end{itemize}
		}
		\UCitem{Tipo}{Secundario, extiende del caso de uso \UCref{CU7}{Gestionar Entidades}.}
	\end{UseCase}
%--------------------------------------
	\begin{UCtrayectoria}
		\UCpaso[\UCactor] Solicita registrar una entidad oprimiendo el botón \IUbutton{Registrar} de la pantalla \IUref{IU12}{Gestionar Entidades}.
		\UCpaso[\UCsist] Muestra la pantalla \IUref{IU12.1}{Registrar Entidad}. \label{CU7.1-P2}
		\UCpaso[\UCactor] Ingresa la información solicitada. \label{CU7.1-P3}
		\UCpaso[\UCactor] Gestiona los atributos a través de los botones: \IUbutton{Registrar}, \editar y \eliminar. \label{CU7.1-P4}
		\UCpaso[\UCactor] Oprime el botón \IUbutton{Aceptar}. \hyperlink{CU7-1:TAA}{[Trayectoria A]}
		\UCpaso[\UCsist] Verifica que el actor ingrese todos los campos obligatorios con base en la regla de negocio \BRref{RN8}{Datos obligatorios}. \hyperlink{CU7-1:TAB}{[Trayectoria B]}
		\UCpaso[\UCsist] Verificar que los datos ingresados cumpla con la longitud correcta, con base en la regla de negocio \BRref{RN37}{Longitud de datos}. \hyperlink{CU7-1:TAC}{[Trayectoria C]}
		\UCpaso[\UCsist] Verifica que los datos ingresados cumplan con el formato requerido, con base en la regla de negocio \BRref{RN7}{Información correcta}. \hyperlink{CU7-1:TAD}{[Trayectoria D]}
		\UCpaso[\UCsist] Verifica que el nombre de la entidad no se encuentre registrado en el sistema con base en la regla de negocio \BRref{RN6}{Unicidad de nombres}. \hyperlink{CU7-1:TAE}{[Trayectoria E]}
		\UCpaso[\UCsist] Verifica que el actor haya ingresado al menos un atributo. \hyperlink{CU7-1:TAF}{[Trayectoria F]}
		\UCpaso[\UCsist] Registra la información de la entidad.
		\UCpaso[\UCsist] Muestra el mensaje \cdtIdRef{MSG1}{Operación exitosa} en la pantalla \IUref{IU12}{Gestionar Entidades} para indicar al actor que el registro se ha realizado exitosamente.
	\end{UCtrayectoria}		
%--------------------------------------
\hypertarget{CU7-1:TAA}{\textbf{Trayectoria alternativa A}}\\
\noindent \textbf{Condición:} El actor desea cancelar la operación.
\begin{enumerate}
	\UCpaso[\UCactor] Solicita cancelar la operación oprimiendo el botón \IUbutton{Cancelar} de la pantalla \IUref{IU12.1}{Registrar Entidad}
	\UCpaso[\UCsist] Muestra la pantalla \IUref{IU12}{Gestionar Entidades}.
	\item[- -] - - {\em {Fin del caso de uso}}.%
\end{enumerate}
%--------------------------------------
\hypertarget{CU7-1:TAB}{\textbf{Trayectoria alternativa B}}\\
\noindent \textbf{Condición:} El actor no ingresó algún dato marcado como obligatorio.
\begin{enumerate}
	\UCpaso[\UCsist] Muestra el mensaje \cdtIdRef{MSG4}{Dato obligatorio} señalando el campo que presenta el error en la pantalla \IUref{IU12.1}{Registrar Entidad}.
	\UCpaso Regresa al paso \ref{CU7.1-P3} de la trayectoria principal.
	\item[- -] - - {\em {Fin de la trayectoria}}.%
\end{enumerate}
%--------------------------------------
\hypertarget{CU7-1:TAC}{\textbf{Trayectoria alternativa C}}\\
\noindent \textbf{Condición:} El actor ingresó un dato con un número de caracteres fuera del rango permitido.
\begin{enumerate}
	\UCpaso[\UCsist] Muestra el mensaje \cdtIdRef{MSG6}{Longitud inválida} señalando el campo que presenta el error en la pantalla \IUref{IU12.1}{Registrar Entidad}.
	\UCpaso Regresa al paso \ref{CU7.1-P3} de la trayectoria principal.
	\item[- -] - - {\em {Fin de la trayectoria}}.%
\end{enumerate}
%--------------------------------------
\hypertarget{CU7-1:TAD}{\textbf{Trayectoria alternativa D}}\\
\noindent \textbf{Condición:} El actor ingresó un dato con un formato de dato incorrecto.
\begin{enumerate}
	\UCpaso[\UCsist] Muestra el mensaje \cdtIdRef{MSG5}{Formato incorrecto} señalando el campo que presenta el error en la pantalla \IUref{IU12.1}{Registrar Entidad}.
	\UCpaso Regresa al paso \ref{CU7.1-P3} de la trayectoria principal.
	\item[- -] - - {\em {Fin de la trayectoria}}.
\end{enumerate}
%--------------------------------------	
\hypertarget{CU7-1:TAE}{\textbf{Trayectoria alternativa E}}\\
\noindent \textbf{Condición:} El actor ingresó una entidad ya existe dentro del sistema.
\begin{enumerate}
	\UCpaso[\UCsist] Muestra el mensaje \cdtIdRef{MSG7}{Registro repetido} señalando el campo que presenta la duplicidad en la pantalla \IUref{IU12.1}{Registrar Entidad}.
	\UCpaso Regresa al paso \ref{CU7.1-P3} de la trayectoria principal.
	\item[- -] - - {\em {Fin de la trayectoria}}.
\end{enumerate}
%--------------------------------------
\hypertarget{CU7-1:TAF}{\textbf{Trayectoria alternativa F}}\\
\noindent \textbf{Condición:} El actor no registró ningún atributo.
\begin{enumerate}
	\UCpaso[\UCsist] Muestra el mensaje \cdtIdRef{MSG14}{Registro necesario} en la sección de atributos de la pantalla \IUref{IU12.1}{Registrar Entidad}.
	\UCpaso Regresa al paso \ref{CU7.1-P3} de la trayectoria principal.
	\item[- -] - - {\em {Fin de la trayectoria}}.
\end{enumerate}
%--------------------------------------
\subsubsection{Puntos de extensión}

\UCExtenssionPoint{El actor requiere registrar un atributo.}{Paso \ref{CU7.1-P2} de la trayectoria principal.}{\UCref{CU7.1.1}{Registrar Atributo}}
\UCExtenssionPoint{El actor requiere modificar un atributo.}{Paso \ref{CU7.1-P2} de la trayectoria principal.}{\UCref{CU7.1.2}{Modificar Atributo}}
\UCExtenssionPoint{El actor requiere eliminar un atributo.}{Paso \ref{CU7.1-P2} de la trayectoria principal.}{\UCref{CU7.1.3}{Eliminar Atributo}}