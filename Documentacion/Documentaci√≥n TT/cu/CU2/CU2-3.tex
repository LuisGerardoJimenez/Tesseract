	\begin{UseCase}{CU2.3}{Eliminar proyecto de Administrador}{
			Cuando no se logró concretar la negociación de un \hyperlink{proyectoEntidad}{Proyecto} o este fué cancelado por algún motivo y ya no se continuará con su desarrollo, Tesseract permitirá al {\hyperlink{admin}{Administrador}} eliminar en su totalidad su registro del sistema. \\
			Un \hyperlink{proyectoEntidad}{Proyecto} podrá ser eliminado siempre y cuando no se haya asociado algún caso de uso a algúno de sus módulos por parte de algún colaborador ya sea el {\hyperlink{jefe}{líder de proyecto}} o algún {\hyperlink{analista}{analista}}.
	}
		\UCitem{Actor}{\hyperlink{admin}{Administrador}}
		\UCitem{Propósito}{Eliminar un proyecto del sistema.}
		\UCitem{Entradas}{Niguna}	
		\UCitem{Salidas}{
		\begin{itemize}
			\item \cdtIdRef{MSG1}{Operación Exitosa}: Se muestra en la pantalla \IUref{IU2}{Gestionar proyectos de Administrador} para indicar el proyecto fue eliminado correctamente.
			\item \cdtIdRef{MSG10}{Confirmar eliminación}: Se muestra en la pantalla \IUref{IU2}{Gestionar proyectos de Administrador} preguntando al actor si desea continuar con la eliminación del proyecto.
		\end{itemize}
		}
		\UCitem{Precondiciones}{
				 Que el proyecto no tenga casos de uso asociados.
		}
		\UCitem{Postcondiciones}{
			Se eliminará la información del proyecto en el sistema.
		}
		\UCitem{Errores}{
		\cdtIdRef{MSG26}{No es posible eliminar proyecto}: Se muestra en la pantalla \IUref{IU2}{Gestionar proyectos de Adminstrador} cuando el proyecto tiene \hyperlink{tElemento}{elementos} asociados.
		}
		\UCitem{Tipo}{Secundario, extiende del caso de uso \UCref{CU2}{Gestionar proyectos de Administrador}}
	\end{UseCase}
%--------------------------------------
	\begin{UCtrayectoria}
		\UCpaso[\UCactor] Da clic en el icono \eliminar del registro que desea eliminar de la pantalla \IUref{IU2}{Gestionar proyectos de Administrador}.
		\UCpaso[\UCsist] Muestra el mensaje \cdtIdRef{MSG10}{Confirmar eliminación} en la pantalla \IUref{IU2}{Gestionar proyectos de Administrador} con los botones \IUbutton{Aceptar} y \IUbutton{Cancelar}
		\UCpaso[\UCactor] Confirma la eliminación del proyecto oprimiendo el botón \IUbutton{Aceptar}. \hyperlink{CU2-3:TAA}{[Trayectoria A]}
		\UCpaso[\UCsist] Verifica que el proyecto pueda eliminarse, con base en la regla de negocio \BRref{RN34}{Eliminación de proyectos}. \hyperlink{CU2-3:TAB}{[Trayectoria B]}
		\UCpaso[\UCsist] Elimina el proyecto del sistema.
		\UCpaso[\UCsist] Muestra el mensaje \cdtIdRef{MSG1}{Operación exitosa} en la pantalla \IUref{IU2}{Gestionar proyectos de Administrador} para indicar al actor que se ha eliminado el registro exitosamente.
	\end{UCtrayectoria}		
%--------------------------------------	
	\hypertarget{CU2-3:TAA}{\textbf{Trayectoria alternativa A}}\\
	\noindent \textbf{Condición:} El actor desea cancelar la operación.
	\begin{enumerate}
		\UCpaso[\UCactor] Solicita cancelar la operación oprimiendo el botón \IUbutton{Cancelar} de la ventana emergente.
		\UCpaso[\UCsist] Muestra la pantalla \IUref{IU2}{Gestionar proyectos de Administrador}.
		\item[- -] - - {\em {Fin del caso de uso}}.%
	\end{enumerate}

%--------------------------------------
	\hypertarget{CU2-3:TAB}{\textbf{Trayectoria alternativa B}}\\
	\noindent \textbf{Condición:} El proyecto tiene \hyperlink{tElemento}{elementos} asociados.
	\begin{enumerate}
		\UCpaso[\UCsist] Muestra la pantalla \IUref{IU2}{Gestionar proyectos de Administrador} con el mensaje \cdtIdRef{MSG26}{No es posible eliminar proyecto}.
		\item[- -] - - {\em {Fin del caso de uso}}.%
	\end{enumerate}