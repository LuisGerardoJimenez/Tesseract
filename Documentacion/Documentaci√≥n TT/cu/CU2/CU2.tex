	\begin{UseCase}{CU2}{Gestionar proyectos de Administrador}{
			
			Permite al \hyperlink{admin}{Administrador} realizar las acciones necesarias para controlar los proyectos que han sido previamente registrados en el sistema, como lo son visualizar el listado de los proyectos, registrar un nuevo proyecto, modificar uno existente (para actualizar la información de su presupuesto, fechas, estado, nombre, clave, etc) o eliminarlo (en caso de que el proyecto ya no tenga una razón de ser dentro del sistema).\\
			%La gestión está disponible en cualquier estado en el que se encuentre el proyecto con base en el \hyperlink{edoProy}{Modelo de estados del proyecto}%
			La gestión muestra todos los proyectos no importando en que estado se encuentren con base en el Modelo \hyperlink{edoProy}{Modelo de estados del proyecto}.
				
	}
		\UCitem{Actor}{\hyperlink{admin}{Administrador}}
		\UCitem{Propósito}{Proporcionar al actor un mecanismo para llevar el control de los proyectos.}
		\UCitem{Entradas}{Ninguna}
		\UCitem{Salidas}{\begin{itemize}
				\item \hyperlink{proyectoEntidad}{Proyecto}: Tabla que muestra la \cdtRef{proyectoEntidad:claveProyecto}{clave}, \cdtRef{proyectoEntidad:nombreProyecto}{nombre} y el \cdtRef{proyectoEntidad:liderProyecto}{Líder de Proyecto} de todos los proyectos existentes.
				\item \cdtIdRef{MSG2}{No existe información}: Se muestra en la pantalla \IUref{IU2A}{Gestionar proyectos de Administrador} cuando no existen proyectos registrados.
			\end{itemize}	
			}
		
		\UCitem{Precondiciones}{Ninguna}
		\UCitem{Postcondiciones}{Ninguna}
		\UCitem{Errores}{Ninguno}
		\UCitem{Tipo}{Caso de uso primario}
		
	\end{UseCase}
%--------------------------------------
	\begin{UCtrayectoria}
		\UCpaso[\UCactor] Solicita gestionar los proyectos presionando la opción ''Proyectos'' del menú \IUref{MN1}{Menú de Administrador}.
		\UCpaso[\UCsist] Obtiene la información de los proyectos registrados en cualquier estado. \hyperlink{CU2:TAA}{[Trayectoria A]}
		\UCpaso[\UCsist] Ordena los proyectos alfabéticamente basándose en la clave del mismo.
		\UCpaso[\UCsist] Muestra la información de los proyectos en la pantalla \IUref{IU2}{Gestionar proyectos de Administrador}. \label{GP-P3}
		\UCpaso[\UCactor] Gestiona los proyectos a través de los botones: \IUbutton{Registrar}, \editar  y \eliminar. 
	\end{UCtrayectoria}		
%--------------------------------------
	\hypertarget{CU2:TAA}{\textbf{Trayectoria alternativa A}}\\
\noindent \textbf{Condición:} No existen registros de proyectos.
\begin{enumerate}
	\UCpaso[\UCsist] Muestra el mensaje \cdtIdRef{MSG2}{No existe información} dentro de la tabla de proyectos de la pantalla \IUref{IU2A}{Gestionar proyectos de Administrador} para indicar que no hay registros de proyectos para mostrar. \label{TA-P1}
	\UCpaso[\UCactor] Gestiona los proyectos a través del botón: \IUbutton{Registrar}.
	\item[- -] - - {\em {Fin del caso de uso}}.%
\end{enumerate}
%--------------------------------------

\subsubsection{Puntos de extensión}

\UCExtenssionPoint{El actor requiere registrar un proyecto.}{Presionando el botón \IUbutton{Registrar} del paso \ref{GP-P3} de la trayectoria principal o del paso \ref{TA-P1} de la trayectoria alternativa A.}{\UCref{CU2.1}{Registrar Proyecto}}
\UCExtenssionPoint{El actor requiere modificar un proyecto.}{Presionando el icono \editar del paso \ref{GP-P3} de la trayectoria principal.}{\UCref{CU2.2}{Modificar Proyecto}}
\UCExtenssionPoint{El actor requiere eliminar un proyecto.}{Presionando el icono \eliminar del paso \ref{GP-P3} de la trayectoria principal.}{\UCref{CU2.3}{Eliminar Proyecto}}
