	\begin{UseCase}{CU3.2}{Modificar Colaborador}{
		Este caso de uso permite modificar la información de una Colaborador que podrá ser elegida como colaborador de un proyecto.
	}
		\UCitem{Versión}{\color{Gray}0.1}
		\UCitem{Actor}{\hyperlink{admin}{Administrador}}
		\UCitem{Propósito}{Modificar la información de una Colaborador.}
		\UCitem{Entradas}{
		\begin{itemize}
			\item \cdtRef{colaboradorEntidad:nombreColaborador}{Nombre}: Se escribe desde el teclado.
			\item \cdtRef{colaboradorEntidad:pApellidoColaborador}{Primer Apellido}: Se escribe desde el teclado.
			\item \cdtRef{colaboradorEntidad:sApellidoColaborador}{Segundo Apellido}: Se escribe desde el teclado.
			\item \cdtRef{colaboradorEntidad:correoColaborador}{Correo electrónico}: Se escribe desde el teclado.
			\item \cdtRef{colaboradorEntidad:passColaborador}{Contraseña}: Se escribe desde el teclado.
		\end{itemize}	
		}
		\UCitem{Salidas}{\begin{itemize}
			\item \cdtRef{colaboradorEntidad:curpColaborador}{CURP}: Lo obtiene el sistema.
			\item \cdtRef{colaboradorEntidad:nombreColaborador}{Nombre}: Lo obtiene el sistema.
			\item \cdtRef{colaboradorEntidad:pApellidoColaborador}{Primer Apellido}: Lo obtiene el sistema.
			\item \cdtRef{colaboradorEntidad:sApellidoColaborador}{Segundo Apellido}: Lo obtiene el sistema.
			\item \cdtRef{colaboradorEntidad:correoColaborador}{Correo electrónico}: Lo obtiene el sistema.
			\item \cdtRef{colaboradorEntidad:passColaborador}{Contraseña}: Lo obtiene el sistema.
			\item \cdtIdRef{MSG1}{Operación exitosa}: Se muestra en la pantalla \IUref{IU3}{Gestionar Colaboradores} para indicar que la modificación fue exitosa.
		\end{itemize}}
		\UCitem{Precondiciones}{Ninguna}
		\UCitem{Postcondiciones}{
		\begin{itemize}
			\item Se actualizará la información de un Colaborador en el sistema.
		\end{itemize}
		}
		\UCitem{Errores}{\begin{itemize}
		\item \cdtIdRef{MSG4}{Dato obligatorio}: Se muestra en la pantalla \IUref{IU3.2}{Modificar Colaborador} cuando no se ha ingresado un dato marcado como obligatorio.
		\item \cdtIdRef{MSG29}{Formato incorrecto}: Se muestra en la pantalla \IUref{IU3.2}{Modificar Colaborador} cuando el tipo de dato ingresado no cumple con el tipo de dato solicitado en el campo.
		\item \cdtIdRef{MSG6}{Longitud inválida}: Se muestra en la pantalla \IUref{IU3.2}{Modificar Colaborador} cuando se ha excedido la longitud de alguno de los campos.
		\item \cdtIdRef{MSG7}{Registro repetido}: Se muestra en la pantalla \IUref{IU3.2}{Modificar Colaborador} cuando se registre un Colaborador con un correo que ya se encuentra registrado en el sistema.
		\end{itemize}
		}
		\UCitem{Tipo}{Secundario, extiende del caso de uso \UCref{CU3}{Gestionar Colaboradores}}
	\end{UseCase}
%--------------------------------------
	\begin{UCtrayectoria}
		\UCpaso[\UCactor] Solicita registrar un proyecto oprimiendo el botón \editar de la pantalla \IUref{IU3}{Gestionar Colaboradores}.
		\UCpaso[\UCsist] Obtiene la información del Colaborador seleccionado.
		\UCpaso[\UCsist] Deshabilita el campo de la \cdtRef{colaboradorEntidad:curpColaborador}{CURP}.
		\UCpaso[\UCsist] Muestra la pantalla \IUref{IU3.2}{Modificar Colaborador}.
		\UCpaso[\UCactor] Modifica la información solicitada en la pantalla. \label{CU3.2-P5}
		\UCpaso[\UCactor] Solicita guardar el proyecto oprimiendo el botón \IUbutton{Aceptar} de la pantalla \IUref{IU3.2}{Modificar Colaborador}. \hyperlink{CU3-2:TAA}{[Trayectoria A]}
		\UCpaso[\UCsist] Verifica que el actor ingrese todos los campos obligatorios con base en la regla de negocio \BRref{RN8}{Datos obligatorios}. \hyperlink{CU3-2:TAB}{[Trayectoria B]}
		\UCpaso[\UCsist] Verificar que los datos ingresados cumpla con la longitud correcta, con base en la regla de negocio \BRref{RN37}{Longitud de datos}. \hyperlink{CU3-2:TAC}{[Trayectoria C]} 
		\UCpaso[\UCsist] Verifica que los datos ingresados cumplan con el formato requerido, con base en la regla de negocio \BRref{RN7}{Información correcta}. \hyperlink{CU3-2:TAD}{[Trayectoria D]} 
		\UCpaso[\UCsist] Verifica que el correo del Colaborador no se encuentre registrado en el sistema con base en la regla de negocio \BRref{RN36}{Unicidad de correos}. \hyperlink{CU3-2:TAE}{[Trayectoria E]}\label{CU3.2-P10}
		\UCpaso[\UCsist] Actualiza la información del proyecto en el sistema.
		\UCpaso[\UCsist] Verifica que el correo electrónico y la contraseña del Colaborador no hayan cambiado. \hyperlink{CU3-2:TAF}{[Trayectoria F]}
		\UCpaso[\UCsist] Muestra el mensaje \cdtIdRef{MSG1}{Operación exitosa} en la pantalla \IUref{IU3}{Gestionar Colaboradores} para indicar al actor que la modificación se ha realizado exitosamente. 
	\end{UCtrayectoria}		
%--------------------------------------	
	\hypertarget{CU3-2:TAA}{\textbf{Trayectoria alternativa A}}\\
	\noindent \textbf{Condición:} El actor desea cancelar la operación.
	\begin{enumerate}
		\UCpaso[\UCactor] Solicita cancelar la operación oprimiendo el botón \IUbutton{Cancelar} de la pantalla \IUref{IU3.2}{Modificar Colaborador}.
		\UCpaso[\UCsist] Muestra la pantalla \IUref{IU3}{Gestionar Colaboradores}.
		\item[- -] - - {\em {Fin del caso de uso}}.%
	\end{enumerate}
%--------------------------------------	
\hypertarget{CU3-2:TAB}{\textbf{Trayectoria alternativa B}}\\
\noindent \textbf{Condición:} El actor no ingresó algún dato marcado como obligatorio.
\begin{enumerate}
	\UCpaso[\UCsist] Muestra el mensaje \cdtIdRef{MSG4}{Dato obligatorio} señalando el campo que presenta el error en la pantalla \IUref{IU3.2}{Modificar Colaborador}.
	\UCpaso Regresa al paso \ref{CU3.2-P5} de la trayectoria principal.
	\item[- -] - - {\em {Fin de la trayectoria}}.%
\end{enumerate}
%--------------------------------------	
\hypertarget{CU3-2:TAC}{\textbf{Trayectoria alternativa C}}\\
\noindent \textbf{Condición:} El actor ingresó un dato con un número de caracteres fuera del rango permitido.
\begin{enumerate}
	\UCpaso[\UCsist] Muestra el mensaje \cdtIdRef{MSG6}{Longitud inválida} señalando el campo que presenta el error en la pantalla \IUref{IU3.2}{Modificar Colaborador}.
	\UCpaso Regresa al paso \ref{CU3.2-P5} de la trayectoria principal.
	\item[- -] - - {\em {Fin de la trayectoria}}.%
\end{enumerate}
%--------------------------------------	
\hypertarget{CU3-2:TAD}{\textbf{Trayectoria alternativa D}}\\
\noindent \textbf{Condición:} El actor ingresó un dato con un formato incorrecto.
\begin{enumerate}
	\UCpaso[\UCsist] Muestra el mensaje \cdtIdRef{MSG29}{Formato incorrecto} señalando el campo que presenta el error en la pantalla \IUref{IU3.2}{Modificar Colaborador}.
	\UCpaso Regresa al paso \ref{CU3.2-P5} de la trayectoria principal.
	\item[- -] - - {\em {Fin de la trayectoria}}.
\end{enumerate}
%--------------------------------------	
\hypertarget{CU3-2:TAE}{\textbf{Trayectoria alternativa E}}\\
\noindent \textbf{Condición:} El actor ingresó una correo electrónico repetido.
\begin{enumerate}
	\UCpaso[\UCsist] Muestra el mensaje \cdtIdRef{MSG7}{Registro repetido} señalando el campo que presenta la duplicidad en la pantalla\IUref{IU3.2}{Modificar Colaborador}.
	\UCpaso Regresa al paso \ref{CU3.2-P5} de la trayectoria principal.
	\item[- -] - - {\em {Fin de la trayectoria}}.
\end{enumerate}

%--------------------------------------	
\hypertarget{CU3-2:TAF}{\textbf{Trayectoria alternativa F}}\\
\noindent \textbf{Condición:} El actor ingresó un nuevo correo electrónico.
\begin{enumerate}
	\UCpaso[\UCsist] Envía un correo con el mensaje \cdtIdRef{MSG25}{Datos de sesión} a la nueva cuenta de correo electrónico proporcionada por el actor.
	\UCpaso Regresa al paso \ref{CU3.2-P10} de la trayectoria principal.
	\item[- -] - - {\em {Fin de la trayectoria}}.
\end{enumerate}