	\begin{UseCase}{CU3}{Gestionar Colaboradores}{
		Permite al actor visualizar todas las personas de la organización, así como solicitar el registro, consulta, modificación y eliminación de una persona.Permite al {\hyperlink{admin}{Administrador} visualizar en una tabla, el registro de todas las personas que forman parte de la organización, así como solicitar el registro de un nuevo miembro, modificar alguno existente (con el fin de actualizar su información personal y de acceso al sistema) o eliminarlo (en caso de que la persona por algún motivo ya no forme parte de la compañía).\\}
	}
	\UCitem{Versión}{\color{Gray}0.1}
	\UCitem{Actor}{\hyperlink{admin}{Administrador}}
	\UCitem{Propósito}{Proporcionar al actor un mecanismo para llevar el control de los Colaboradores.}
	\UCitem{Entradas}{Ninguna}
	\UCitem{Salidas}{\begin{itemize}
			\item \cdtRef{colaboradorEntidad}{Colaborador}: Tabla que muestra \cdtRef{colaboradorEntidad:curpColaborador}{CURP}, \cdtRef{colaboradorEntidad:nombreColaborador}{Nombre}, \cdtRef{colaboradorEntidad:pApellidoColaborador}{Primer Apellido}, \cdtRef{colaboradorEntidad:sApellidoColaborador}{Segundo Apellido} de todos los registros de las personas.
			\item \cdtIdRef{MSG2}{No existe información}: Se muestra en la pantalla \IUref{IU3A}{Gestionar Colaboradores} cuando no existen Colaboradores registrado.
	\end{itemize}}
	\UCitem{Precondiciones}{Ninguna}
	\UCitem{Postcondiciones}{Ninguna}
	\UCitem{Errores}{Ninguno}
	\UCitem{Tipo}{Caso de uso primario}
\end{UseCase}
%--------------------------------------
\begin{UCtrayectoria}
	\UCpaso[\UCactor] Solicita gestionar a los Colaboradores presionando la opción ''Colaboradores'' del menú \IUref{MN1}{Menú de Administrador}.
	\UCpaso[\UCsist] Obtiene la información de los Colaboradores registrado. \hyperlink{CU3:TAA}{[Trayectoria A]}
	\UCpaso[\UCsist] Muestra la información de los Colaboradores en la pantalla \IUref{IU3}{Gestionar Colaboradores}. \label{CU3-P3}
	\UCpaso[\UCactor] Gestiona a los colaboradores a través de los botones: \IUbutton{Registrar}, \editar  y \eliminar.
\end{UCtrayectoria}		
%--------------------------------------
\hypertarget{CU3:TAA}{\textbf{Trayectoria alternativa A}}\\
\noindent \textbf{Condición:} No existen registros de colaboradores.
\begin{enumerate}
	\UCpaso[\UCsist] Muestra el mensaje \cdtIdRef{MSG2}{No existe información} en la pantalla \IUref{IU3A}{Gestionar Colaboradores} para indicar que no hay registros de colaboradores para mostrar. \label{CU3-A2}
	\UCpaso[\UCactor] Gestiona a los colaboradores a través del botón: \IUbutton{Registrar}. 
	\item[- -] - - {\em {Fin de la trayectoria}}.%
\end{enumerate}
%--------------------------------------

\subsubsection{Puntos de extensión}

\UCExtenssionPoint{El actor requiere registrar una persona.}{Paso \ref{CU3-P3} de la trayectoria principal o del paso \ref{CU3-A2} de la trayectoria alternativa A.}{\UCref{CU3.1}{Registrar Colaborador}}
\UCExtenssionPoint{El actor requiere modificar una persona.}{Paso \ref{CU3-P3} de la trayectoria principal.}{\UCref{CU3.2}{Modificar Colaborador}}
\UCExtenssionPoint{El actor requiere eliminar una persona.}{Paso \ref{CU3-P3} de la trayectoria principal.}{\UCref{CU3.3}{Eliminar Colaborador}}
