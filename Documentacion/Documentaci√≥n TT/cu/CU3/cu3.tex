	\begin{UseCase}{CU3}{Gestionar Personal}{
		Permite al actor visualizar todas las personas de la organización, así como solicitar el registro, consulta, modificación y eliminación de una persona.
	}
	\UCitem{Versión}{\color{Gray}0.1}
	\UCitem{Actor}{\hyperlink{admin}{Administrador}}
	\UCitem{Propósito}{Proporcionar al actor un mecanismo para llevar el control del personal.}
	\UCitem{Entradas}{Ninguna}
	\UCitem{Salidas}{\begin{itemize}
			\item : \cdtRef{colaboradorEntidad}{Colaborador}: Tabla que muestra \cdtRef{colaboradorEntidad:curpColaborador}{CURP}, \cdtRef{colaboradorEntidad:nombreColaborador}{Nombre}, \cdtRef{colaboradorEntidad:pApellidoColaborador}{Primer Apellido}, \cdtRef{colaboradorEntidad:sApellidoColaborador}{Segundo Apellido} de todos los registros de las personas.
			\item \cdtIdRef{MSG2}{No existe información}: Se muestra en la pantalla \IUref{IU2}{Gestionar proyectos de Administrador} cuando no existe personal registrado.
	\end{itemize}}
	\UCitem{Destino}{Pantalla}
	\UCitem{Precondiciones}{Ninguna}
	\UCitem{Postcondiciones}{Ninguna}
	\UCitem{Errores}{Ninguno}
	\UCitem{Tipo}{Caso de uso primario}
\end{UseCase}
%--------------------------------------
\begin{UCtrayectoria}
	\UCpaso[\UCactor] Solicita gestionar al personal presionando la opción ''Personal'' del menú \IUref{MN1}{Menú de Administrador}.
	\UCpaso[\UCsist] Obtiene la información del personal registrado. \hyperlink{CU3:TAA}{[Trayectoria A]}
	\UCpaso[\UCsist] Muestra la información del personal en la pantalla \IUref{IU3}{Gestionar Personal}.
	\UCpaso[\UCactor] Gestiona los proyectos a través de los botones: \IUbutton{Registrar}, \editar  y \eliminar. \label{CU3-P4}
\end{UCtrayectoria}		
%--------------------------------------
\hypertarget{CU3:TAA}{\textbf{Trayectoria alternativa A}}\\
\noindent \textbf{Condición:} No existen registros de colaboradores.
\begin{enumerate}
	\UCpaso[\UCsist] Muestra el mensaje \cdtIdRef{MSG2}{No existe información} en la pantalla \IUref{IU3}{Gestionar Personal} para indicar que no hay registros de colaboradores para mostrar.
	\item[- -] - - {\em {Fin de la trayectoria}}.%
\end{enumerate}
%--------------------------------------

\subsubsection{Puntos de extensión}

\UCExtenssionPoint{El actor requiere registrar una persona.}{Paso \ref{CU3-P4} de la trayectoria principal.}{\UCref{CU3.1}{Registrar Persona}}
\UCExtenssionPoint{El actor requiere modificar una persona.}{Paso \ref{CU3-P4} de la trayectoria principal.}{\UCref{CU3.2}{Modificar Persona}}
\UCExtenssionPoint{El actor requiere eliminar una persona.}{Paso \ref{CU3-P4} de la trayectoria principal.}{\UCref{CU3.3}{Eliminar Persona}}
