	\begin{UseCase}{CU10.1}{Registrar Actor}{
		Este caso de uso permite al colaborador (\hyperlink{jefe}{Líder de Análisis} o \hyperlink{analista}{Analista}) registrar la información de un \hyperlink{actorEntidad}{Actor} en el \hyperlink{proyectoEntidad}{Proyecto}.\\
		
		Los datos de un actor están determinados por su nombre, descripción y su cardinalidad. La cardinalidad define el número de entidades con la cual otra entidad se puede asociar mediante una relación binaria.\\
		
		Los datos que se almacenan cumplen con el propósito de conocer cuales son las entidades que pueden interactuar con el sistema. Esta puede ser una persona o un grupo de personas homogéneas, otro sistema, o una máquina. 
		Una vez registrado el actor, el colaborador podrá hacer uso del mismo en el editor del \hyperlink{casoUso}{Caso de Uso} .
	}
		\UCitem{Actor}{\hyperlink{jefe}{Líder de Análisis}, \hyperlink{analista}{Analista}}
		\UCitem{Propósito}{Registrar la información de actor.}
		\UCitem{Entradas}{
		\begin{itemize}
			\item \cdtRef{actorEntidad:nombreACT}{Nombre del actor:} Se escribe desde el teclado.
			\item \cdtRef{actorEntidad:descripcionACT}{Descripción del actor:} Se escribe desde el teclado.
			\item \cdtRef{actorEntidad:oCardinalidadACT}{Cardinalidad del actor:} Se selecciona de una lista.
		\end{itemize}	
		}
		\UCitem{Salidas}{\begin{itemize}
				\item \cdtRef{proyectoEntidad:claveProyecto}{Clave del proyecto:} Lo obtiene el sistema.
				\item \cdtRef{proyectoEntidad:nombreProyecto}{Nombre del proyecto:} Lo obtiene el sistema.
				\item \cdtIdRef{MSG1}{Operación exitosa}: Se muestra en la pantalla \IUref{IU8}{Gestionar Actores} para indicar que el registro fue exitoso.
		\end{itemize}}
		\UCitem{Destino}{Pantalla}
		\UCitem{Precondiciones}{Ninguna}
		\UCitem{Postcondiciones}{
		\begin{itemize}
			\item Se registrará un actor en el sistema.
			\item El actor podrá ser referenciado en casos de uso.
		\end{itemize}
		}
		\UCitem{Errores}{\begin{itemize}
		\item \cdtIdRef{MSG4}{Dato obligatorio}: Se muestra en la pantalla \IUref{IU8.1}{Registrar Actor} cuando no se ha ingresado un dato marcado como obligatorio.
		\item \cdtIdRef{MSG29}{Formato incorrecto}: Se muestra en la pantalla \IUref{IU8.1}{Registrar Actor} cuando el tipo de dato ingresado no cumple con el tipo de dato solicitado en el campo.
		\item \cdtIdRef{MSG6}{Longitud inválida}: Se muestra en la pantalla \IUref{IU8.1}{Registrar Actor} cuando se ha excedido la longitud de alguno de los campos.
		\item \cdtIdRef{MSG7}{Registro repetido}: Se muestra en la pantalla \IUref{IU8.1}{Registrar Actor} cuando se registre un actor con un nombre que ya se encuentre registrado en el sistema.
		\item \cdtIdRef{MSG12}{Ha ocurrido un error}: Se muestra en la pantalla \IUref{IU8}{Gestionar Actores} cuando no existe información en el catálogo de cardilanidad.
		\end{itemize}.
		}
		\UCitem{Tipo}{Secundario, extiende del caso de uso \UCref{CU10}{Gestionar Actores}.}
	\end{UseCase}
%--------------------------------------
	\begin{UCtrayectoria}
		\UCpaso[\UCactor] Solicita registrar una actor oprimiendo el botón \IUbutton{Registrar} de la pantalla \IUref{IU8}{Gestionar Actores}.
		\UCpaso[\UCactor] Verifica que exista información referente a la Cardinalidad con base en la regla de negocio \BRref{RN20}{Verificación de catálogos}. \hyperlink{CU10-1:TAA}{[Trayectoria A]}
		\UCpaso[\UCsist] Muestra la pantalla \IUref{IU8.1}{Registrar Actor}.
		\UCpaso[\UCactor] Ingresa la información solicitada. \label{CU10.1-P4}
		\UCpaso[\UCactor] Oprime el botón \IUbutton{Aceptar}. \hyperlink{CU10-1:TAB}{[Trayectoria B]}
		\UCpaso[\UCsist] Verifica que el actor ingrese todos los campos obligatorios con base en la regla de negocio \BRref{RN8}{Datos obligatorios}. \hyperlink{CU10-1:TAC}{[Trayectoria C]}
		\UCpaso[\UCsist] Verificar que los datos ingresados cumpla con la longitud correcta, con base en la regla de negocio \BRref{RN37}{Longitud de datos}. \hyperlink{CU10-1:TAD}{[Trayectoria D]}
		\UCpaso[\UCsist] Verifica que los datos ingresados cumplan con el formato requerido, con base en la regla de negocio \BRref{RN7}{Información correcta}. \hyperlink{CU10-1:TAE}{[Trayectoria E]}
		\UCpaso[\UCsist] Verifica que el nombre del actor no se encuentre registrado en el sistema con base en la regla de negocio \BRref{RN6}{Unicidad de nombres}. \hyperlink{CU10-1:TAF}{[Trayectoria F]} 
		\UCpaso[\UCsist] Registra la información del actor en el sistema.
		\UCpaso[\UCsist] Muestra el mensaje \cdtIdRef{MSG1}{Operación exitosa} en la pantalla \IUref{IU8}{Gestionar Actores} para indicar al actor que el registro se ha realizado exitosamente.
	\end{UCtrayectoria}		
%--------------------------------------
\hypertarget{CU10-1:TAA}{\textbf{Trayectoria alternativa A}}\\
\noindent \textbf{Condición:} No existe información en el catálogo de ''cardinalidad''.
\begin{enumerate}
	\UCpaso[\UCactor] Muestra el mensaje \cdtIdRef{MSG12}{Ha ocurrido un error} en la pantalla \IUref{IU8}{Gestionar Actores}.
	\item[- -] - - {\em {Fin del caso de uso}}.%
\end{enumerate}

%--------------------------------------
	\hypertarget{CU10-1:TAB}{\textbf{Trayectoria alternativa B}}\\
	\noindent \textbf{Condición:} El actor desea cancelar la operación.
	\begin{enumerate}
		\UCpaso[\UCactor] Solicita cancelar la operación oprimiendo el botón \IUbutton{Cancelar} de la pantalla \IUref{IU8.1}{Registrar Actor}.
		\UCpaso[\UCsist] Muestra la pantalla \IUref{IU8}{Gestionar Actores}.
		\item[- -] - - {\em {Fin del caso de uso}}.%
	\end{enumerate}
%--------------------------------------
\hypertarget{CU10-1:TAC}{\textbf{Trayectoria alternativa C}}\\
\noindent \textbf{Condición:} El actor no ingresó algún dato marcado como obligatorio.
\begin{enumerate}
	\UCpaso[\UCsist] Muestra el mensaje \cdtIdRef{MSG4}{Dato obligatorio} señalando el campo que presenta el error en la pantalla \IUref{IU8.1}{Registrar Actor}.
	\UCpaso Regresa al paso \ref{CU10.1-P4} de la trayectoria principal.
	\item[- -] - - {\em {Fin de la trayectoria}}.%
\end{enumerate}
%--------------------------------------
\hypertarget{CU10-1:TAD}{\textbf{Trayectoria alternativa D}}\\
\noindent \textbf{Condición:} El actor ingresó un dato con un número de caracteres fuera del rango permitido.
\begin{enumerate}
	\UCpaso[\UCsist] Muestra el mensaje \cdtIdRef{MSG6}{Longitud inválida} señalando el campo que presenta el error en la pantalla \IUref{IU8.1}{Registrar Actor}.
	\UCpaso Regresa al paso \ref{CU10.1-P4} de la trayectoria principal.
	\item[- -] - - {\em {Fin de la trayectoria}}.%
\end{enumerate}
%--------------------------------------
\hypertarget{CU10-1:TAE}{\textbf{Trayectoria alternativa E}}\\
\noindent \textbf{Condición:} El actor ingresó un dato con un formato incorrecto.
\begin{enumerate}
	\UCpaso[\UCsist] Muestra el mensaje \cdtIdRef{MSG29}{Formato incorrecto} señalando el campo que presenta el error en la pantalla \IUref{IU8.1}{Registrar Actor}.
	\UCpaso Regresa al paso \ref{CU10.1-P4} de la trayectoria principal.
	\item[- -] - - {\em {Fin de la trayectoria}}.
\end{enumerate}
%--------------------------------------	
\hypertarget{CU10-1:TAF}{\textbf{Trayectoria alternativa F}}\\
\noindent \textbf{Condición:} El actor ingresó el nombre de un actor repetido.
\begin{enumerate}
	\UCpaso[\UCsist] Muestra el mensaje \cdtIdRef{MSG7}{Registro repetido} señalando el campo que presenta la duplicidad en la pantalla \IUref{IU8.1}{Registrar Actor}.
	\UCpaso Regresa al paso \ref{CU10.1-P4} de la trayectoria principal.
	\item[- -] - - {\em {Fin de la trayectoria}}.
\end{enumerate}
%--------------------------------------