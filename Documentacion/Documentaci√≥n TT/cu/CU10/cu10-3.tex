	\begin{UseCase}{CU10.3}{Eliminar Actor}{
			
		Cuando el registro de algún \hyperlink{actorEntidad}{Actor} ya no es requerido o simplemente no tiene una razón de ser dentro del \hyperlink{proyectoEntidad}{Proyecto}, Tesseract permitirá al colaborador (\hyperlink{jefe}{Líder} o \hyperlink{analista}{Analista}) eliminar en su totalidad el registro del actor. \\
		
		Una Actor podrá ser eliminado siempre y cuando no se encuentre asociado con algún \hyperlink{casoUso}{Caso de Uso} en estado ”Liberado”.	
	}

		\UCitem{Actor}{\hyperlink{jefe}{Líder de Análisis}, \hyperlink{analista}{Analista}}
		\UCitem{Propósito}{Eliminar la información de un actor.}
		\UCitem{Entradas}{Ninguna}
		\UCitem{Salidas}{\begin{itemize}
				\item \cdtIdRef{MSG1}{Operación exitosa}: Se muestra en la pantalla \IUref{IU8}{Gestionar Actores} para indicar que el actor fue eliminado correctamente.
				\item \cdtIdRef{MSG10}{Confirmar eliminación}: Se muestra en la pantalla \IUref{IU8}{Gestionar Actores} preguntando al actor si desea continuar con la eliminación del actor.
		\end{itemize}}
		\UCitem{Precondiciones}{\begin{itemize}
				\item Que el actor no se encuentre asociado a un caso de uso.
				\item Quel actor no se encuentre asociado a un caso de uso liberado.
		\end{itemize}}
		\UCitem{Postcondiciones}{
		Se eliminará un actor de un proyecto del sistema.
		}
		\UCitem{Errores}{
		\cdtIdRef{MSG13}{Eliminación no permitida}: Se muestra en la pantalla \IUref{IU8}{Gestionar Actores} cuando el actor está siendo referenciado en algún caso de uso.		
		}
		\UCitem{Tipo}{Secundario, extiende del caso de uso \UCref{CU10}{Gestionar Actores}.}
	\end{UseCase}
%--------------------------------------
	\begin{UCtrayectoria}
		\UCpaso[\UCactor] Da clic en el icono \eliminar del registro que desea eliminar de la pantalla \IUref{IU8}{Gestionar Actores}.
		\UCpaso[\UCsist] Muestra el mensaje emergente \cdtIdRef{MSG10}{Confirmar eliminación} con los botones \IUbutton{Aceptar} y \IUbutton{Cancelar}.
		\UCpaso[\UCactor] Confirma la eliminación del actor oprimiendo el botón \IUbutton{Aceptar}. \hyperlink{CU10-3:TAA}{[Trayectoria A]}
		\UCpaso[\UCsist] Verifica que ningún caso de uso se encuentre asociado al actor. \hyperlink{CU10-3:TAB}{[Trayectoria B]}
		\UCpaso[\UCsist] Elimina la información referente al actor.
		\UCpaso[\UCsist] Muestra el mensaje \cdtIdRef{MSG1}{Operación exitosa} en la pantalla \IUref{IU8}{Gestionar Actores} para indicar al actor que el registro se ha eliminado exitosamente.
	\end{UCtrayectoria}		
%--------------------------------------
\hypertarget{CU10-3:TAA}{\textbf{Trayectoria alternativa A}}\\
\noindent \textbf{Condición:} El actor desea cancelar la operación.
\begin{enumerate}
	\UCpaso[\UCactor] Oprime el botón \IUbutton{Cancelar} de la pantalla emergente.
	\UCpaso[\UCsist] Muestra la pantalla \IUref{IU8}{Gestionar Actores}.
	\item[- -] - - {\em {Fin del caso de uso}}.%
\end{enumerate}	
%--------------------------------------
\hypertarget{CU10-3:TAB}{\textbf{Trayectoria alternativa B}}\\
\noindent \textbf{Condición:} El actor está siendo referenciado en un caso de uso..
\begin{enumerate}
	\UCpaso[\UCsist] Muestra el mensaje \cdtIdRef{MSG13}{Eliminación no permitida} en la pantalla \IUref{IU8}{Gestionar Actores}.
	\item[- -] - - {\em {Fin del caso de uso}}.
\end{enumerate}
