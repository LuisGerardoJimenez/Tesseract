	\begin{UseCase}{CU5.3}{Eliminar Módulo}{
		Cuando algún \hyperlink{moduloEntidad}{Módulo} está de más o simplemente no tiene una razón de ser dentro del \hyperlink{proyectoEntidad}{Proyecto}, Tesseract permitirá al colaborador (ya sea \hyperlink{jefe}{Líder} o \hyperlink{analista}{Analista}) eliminar en su totalidad su registro. \\
		
		Un módulo podrá ser eliminado siempre y cuando no existan referencias al contenido del módulo a eliminar desde \hyperlink{tElemento}{elementos} de otro módulo. 
	}
		\UCitem{Actor}{\hyperlink{jefe}{Líder de Análisis}, \hyperlink{analista}{Analista}}
		\UCitem{Propósito}{Eliminar la información de un módulo.}
		\UCitem{Entradas}{Ninguna}
		\UCitem{Salidas}{\begin{itemize}
				\item \cdtIdRef{MSG1}{Operación exitosa}: Se muestra en la pantalla \IUref{IU4}{Gestionar Módulos} para indicar que el módulo fue eliminado correctamente.
				\item \cdtIdRef{MSG10}{Confirmar eliminación}: Se muestra en la pantalla \IUref{IU4}{Gestionar Módulos} preguntando al actor si desea continuar con la eliminación del módulo.
		\end{itemize}}
		
		\UCitem{Precondiciones}{\begin{itemize}
				\item Que no tenga casos de uso asociados
				\item Que no se existan referencias al contenido del módulo a eliminar desde \hyperlink{tElemento}{elementos} de otro módulo.
			\end{itemize}
		}
		\UCitem{Postcondiciones}{
			Se eliminará un módulo del sistema.
		}
		\UCitem{Errores}{
		\cdtIdRef{MSG13}{Eliminación no permitida}: Se muestra en la pantalla \IUref{IU4}{Gestionar Módulos} cuando no sea posible eliminar el módulo, con base en la regla de negocio \BRref{RN28}{Eliminación de módulos}.
		}
		\UCitem{Tipo}{Secundario, extiende del caso de uso \UCref{CU5}{Gestionar Módulos}.}
	\end{UseCase}
%--------------------------------------
	\begin{UCtrayectoria}
		\UCpaso[\UCactor] Da clic \eliminar del módulo que desea eliminar de la pantalla \IUref{IU4}{Gestionar Módulos}.
		\UCpaso[\UCsist] Muestra el mensaje emergente \cdtIdRef{MSG10}{Confirmar eliminación} con los botones \IUbutton{Aceptar} y \IUbutton{Cancelar} en la pantalla \IUref{IU4}{Gestionar Módulos}.
		\UCpaso[\UCactor] Confirma la eliminación del módulo oprimiendo el botón \IUbutton{Aceptar}. \hyperlink{CU5-3:TAA}{[Trayectoria A]}
		\UCpaso[\UCsist] Verifica que el módulo pueda eliminarse, con base en la regla de negocio \BRref{RN28}{Eliminación de módulos}. \hyperlink{CU5-3:TAB}{[Trayectoria B]}
		\UCpaso[\UCsist] Elimina la información referente al módulo.
		\UCpaso[\UCsist] Muestra el mensaje \cdtIdRef{MSG1}{Operación exitosa} en la pantalla \IUref{IU4}{Gestionar Módulos} para indicar al actor que el registro se ha eliminado exitosamente.
	\end{UCtrayectoria}		
%--------------------------------------
\hypertarget{CU5-3:TAA}{\textbf{Trayectoria alternativa A}}\\
\noindent \textbf{Condición:} El actor desea cancelar la operación.
\begin{enumerate}
	\UCpaso[\UCactor] Solicita cancelar la operación oprimiendo el botón \IUbutton{Cancelar} del mensaje emergente \cdtIdRef{MSG10}{Confirmar eliminación}.
	\UCpaso[\UCsist] Muestra la pantalla \IUref{IU4}{Gestionar Módulos}.
	\item[- -] - - {\em {Fin del caso de uso}}.%
\end{enumerate}

%--------------------------------------
\hypertarget{CU5-3:TAB}{\textbf{Trayectoria alternativa B}}\\
\noindent \textbf{Condición:} El módulo tiene \hyperlink{tElemento}{elementos} asociados.
\begin{enumerate}
	\UCpaso[\UCsist] Muestra el mensaje \cdtIdRef{MSG13}{Eliminación no permitida} en la pantalla \IUref{IU4}{Gestionar Módulos}.
	\item[- -] - - {\em {Fin del caso de uso}}.%
\end{enumerate}
	

