	\begin{UseCase}{CU6.3}{Eliminar Término de Glosario}{
			Cuando algún \hyperlink{terminoGLSEntidad}{Término de glosario} está de más o simplemente no tiene una razón de ser dentro del \hyperlink{proyectoEntidad}{Proyecto}, Tesseract permitirá al colaborador (\hyperlink{jefe}{Líder} o \hyperlink{analista}{Analista}) eliminar en su totalidad su registro. \\
		
			Un Término de glosario podrá ser eliminado siempre y cuando no se encuentre asociado a algún caso de uso con estado ”Liberado”.
	}
		
		\UCitem{Actor}{\hyperlink{jefe}{Líder de Análisis}, \hyperlink{analista}{Analista}}
		\UCitem{Propósito}{Eliminar la información de un término del glosario.}
		\UCitem{Entradas}{Ninguna}
		\UCitem{Salidas}{\begin{itemize}
				\item \cdtIdRef{MSG1}{Operación exitosa}: Se muestra en la pantalla \IUref{IU11}{Gestionar Términos de glosario} para indicar que el término fue eliminado correctamente.
				\item \cdtIdRef{MSG10}{Confirmar eliminación}: Se muestra en la pantalla \IUref{IU11}{Gestionar Términos de glosario} preguntando al actor si desea continuar con la eliminación del término.
		\end{itemize}}
		
		\UCitem{Precondiciones}{\begin{itemize}
				\item Que el término no se encuentre asociado a un caso de uso.
				\item Quel término no se encuentre asociado a un caso de uso liberado.
		\end{itemize}}
		\UCitem{Postcondiciones}{
			Se eliminará el término del glosario del sistema.
		}
		\UCitem{Errores}{
		\cdtIdRef{MSG13}{Eliminación no permitida}: Se muestra en la pantalla \IUref{IU11}{Gestionar Términos del glosario} cuando el término del glosario está siendo referenciado en algún caso de uso.
		}
		\UCitem{Tipo}{Secundario, extiende del caso de uso \UCref{CU6}{Gestionar Términos del glosario}.}
	\end{UseCase}
%--------------------------------------
	\begin{UCtrayectoria}
		\UCpaso[\UCactor] Da clic en el icono \eliminar del registro que desea eliminar de la pantalla \IUref{IU11}{Gestionar Términos de glosario}.
		\UCpaso[\UCsist] Muestra el mensaje emergente \cdtIdRef{MSG10}{Confirmar eliminación} con los botones \IUbutton{Aceptar} y \IUbutton{Cancelar} en la pantalla \IUref{IU11}{Gestionar Términos del glosario}.
		\UCpaso[\UCactor] Confirma la eliminación del término oprimiendo el botón \IUbutton{Aceptar}. \hyperlink{CU6-3:TAA}{[Trayectoria A]}
		\UCpaso[\UCsist] Verifica que el término del glosario pueda eliminarse de acuerdo a la regla de negocio \BRref{RN18}{Eliminación de elementos}. \hyperlink{CU6-3:TAB}{[Trayectoria B]}
		\UCpaso[\UCsist] Verifica que ningún caso de uso se encuentre asociado al término del glosario. \hyperlink{CU6-3:TAC}{[Trayectoria C]}
		\UCpaso[\UCsist] Elimina la información referente al término del glosario.
		\UCpaso[\UCsist] Muestra el mensaje \cdtIdRef{MSG1}{Operación exitosa} en la pantalla \IUref{IU11}{Gestionar Términos del glosario} para indicar al actor que el registro se ha eliminado exitosamente.
	\end{UCtrayectoria}		


%--------------------------------------
\hypertarget{CU6-3:TAA}{\textbf{Trayectoria alternativa A}}\\
\noindent \textbf{Condición:} El actor desea cancelar la operación.
\begin{enumerate}
	\UCpaso[\UCactor] Oprime el botón \IUbutton{Cancelar} de la pantalla emergente.
	\UCpaso[\UCsist] Muestra la pantalla \IUref{IU11}{Gestionar Términos del glosarios}.
	\item[- -] - - {\em {Fin del caso de uso}}.%
\end{enumerate}	
%--------------------------------------
\hypertarget{CU6-3:TAB}{\textbf{Trayectoria alternativa B}}\\
\noindent \textbf{Condición:} El término se encuentra asociado a casos de uso liberados.
\begin{enumerate}
	\UCpaso[\UCsist] Oculta el botón \eliminar del término que esta asociado a casos de uso liberados.
	\item[- -] - - {\em {Fin del caso de uso}}.
\end{enumerate}
%--------------------------------------
\hypertarget{CU6-3:TAC}{\textbf{Trayectoria alternativa C}}\\
\noindent \textbf{Condición:} El término del glosario está siendo referenciado en un caso de uso.
\begin{enumerate}
	\UCpaso[\UCsist] Muestra el mensaje \cdtIdRef{MSG13}{Eliminación no permitida} en la pantalla \IUref{IU11}{Gestionar Términos del glosario}.
	\item[- -] - - {\em {Fin del caso de uso}}.
\end{enumerate}

