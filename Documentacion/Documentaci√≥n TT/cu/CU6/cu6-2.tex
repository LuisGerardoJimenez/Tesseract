	\begin{UseCase}{CU6.2}{Modificar Término de Glosario}{
			
			Después de haber registrado un nuevo \hyperlink{terminoGLSEntidad}{Término de glosario} y el colaborador requiera modificar la información del término, el sistema le mostrará un formulario con los datos precargados de la última actualización, mediante este podrá modificar cualquiera de los datos previamente registrados.\\
		    La única condición para modificar un Término de glosario de forma exitosa es que este no se encuentre asociado a algún caso de uso con estado ”Liberado”.
		
	}
		
		\UCitem{Actor}{\hyperlink{jefe}{Líder de Análisis}, \hyperlink{analista}{Analista}}
		\UCitem{Propósito}{Modificar la información de un término.}
		\UCitem{Entradas}{
		\begin{itemize}
			\item \cdtRef{terminoGLSEntidad:nombreTGLS}{Nombre del término:} Se escribe desde el teclado.
			\item \cdtRef{terminoGLSEntidad:descripcionTGLS}{Descripción del término:} Se escribe desde el teclado.
		\end{itemize}	
		}
		\UCitem{Salidas}{\begin{itemize}
				\item \cdtRef{proyectoEntidad:claveProyecto}{Clave del proyecto:} Lo obtiene el sistema.
				\item \cdtRef{proyectoEntidad:nombreProyecto}{Nombre del proyecto:} Lo obtiene el sistema.
				\item \cdtRef{terminoGLSEntidad:nombreTGLS}{Nombre del término:} Lo obtiene el sistema.
				\item \cdtRef{terminoGLSEntidad:descripcionTGLS}{Descripción del término:} Lo obtiene el sistema
				\item \cdtIdRef{MSG1}{Operación exitosa}: Se muestra en la pantalla \IUref{IU11}{Gestionar Términos de glosario} para indicar que la modificación fue exitosa.
		\end{itemize}}
		
		\UCitem{Precondiciones}{\begin{itemize}
				\item Que exista al menos un término registrado.
				\item Que el término no se encuentre asociado a un caso de uso en estado ''Liberado''.
			\end{itemize}
		}
		\UCitem{Postcondiciones}{Se actualizará un término de un proyecto.}
		\UCitem{Errores}{\begin{itemize}
		\item \cdtIdRef{MSG4}{Dato obligatorio}: Se muestra en la pantalla \IUref{IU11.2}{Modificar Término} cuando no se ha ingresado un dato marcado como obligatorio.
		\item \cdtIdRef{MSG29}{Formato incorrecto}: Se muestra en la pantalla \IUref{IU11.2}{Modificar Término} cuando el tipo de dato ingresado no cumple con el tipo de dato solicitado en
		el campo.
		\item \cdtIdRef{MSG6}{Longitud inválida}: Se muestra en la pantalla \IUref{IU11.2}{Modificar Término} cuando se ha excedido la longitud de alguno de los campos.
		\item \cdtIdRef{MSG7}{Registro repetido}: Se muestra en la pantalla \IUref{IU11.2}{Modificar Término} cuando se registre un término con un nombre que ya se encuentra registrado en el sistema.
		\end{itemize}
		}
		\UCitem{Tipo}{Secundario, extiende del caso de uso \UCref{CU6}{Gestionar Términos}.}
	\end{UseCase}
%--------------------------------------
	\begin{UCtrayectoria}
		\UCpaso[\UCactor] Da clic en el icono \editar de la pantalla \IUref{IU11}{Gestionar Términos del glosario}.
		\UCpaso[\UCsist] Obtiene la información del término.
		\UCpaso[\UCsist] Verifica que el término pueda modificarse con base en la regla de negocio \BRref{RN5}{Modificación de elementos asociados a casos de uso liberados}. \hyperlink{CU6-2:TAF}{[Trayectoria F]}
		\UCpaso[\UCsist] Muestra la pantalla \IUref{IU11.2}{Modificar Término}.
		\UCpaso[\UCactor] Modifica la información del término. \label{CU6.2-P5}
		\UCpaso[\UCactor] Oprime el botón \IUbutton{Aceptar}. \hyperlink{CU6-2:TAB}{[Trayectoria A]}
		\UCpaso[\UCsist] Verifica que el actor ingrese todos los campos obligatorios con base en la regla de negocio \BRref{RN8}{Datos obligatorios}. \hyperlink{CU6-2:TAB}{[Trayectoria B]}
		\UCpaso[\UCsist] Verificar que los datos ingresados cumpla con la longitud correcta, con base en la regla de negocio \BRref{RN37}{Longitud de datos}. \hyperlink{CU6-2:TAC}{[Trayectoria C]}
		\UCpaso[\UCsist] Verifica que los datos ingresados cumplan con el formato requerido, con base en la regla de negocio \BRref{RN7}{Información correcta}. \hyperlink{CU6-2:TAD}{[Trayectoria D]}
		\UCpaso[\UCsist] Verifica que el nombre del término no se encuentre registrado en el sistema con base en la regla de negocio \BRref{RN6}{Unicidad de nombres}. \hyperlink{CU6-2:TAE}{[Trayectoria E]}
		\UCpaso[\UCsist] Actualiza la información del término.
		\UCpaso[\UCsist] Muestra el mensaje \cdtIdRef{MSG1}{Operación exitosa} en la pantalla \IUref{IU11}{Gestionar Términos del glosario} para indicar al actor que la modificación se ha realizado exitosamente.
	\end{UCtrayectoria}		
%--------------------------------------
	\hypertarget{CU6-2:TAA}{\textbf{Trayectoria alternativa A}}\\
	\noindent \textbf{Condición:} El actor desea cancelar la operación.
	\begin{enumerate}
		\UCpaso[\UCactor] Solicita cancelar la operación oprimiendo el botón \IUbutton{Cancelar} de la pantalla \IUref{IU11.2}{Modificar Término}
		\UCpaso[\UCsist] Muestra la pantalla \IUref{IU11}{Gestionar Términos}.
		\item[- -] - - {\em {Fin del caso de uso}}.%
	\end{enumerate}
%--------------------------------------
	\hypertarget{CU6-2:TAB}{\textbf{Trayectoria alternativa B}}\\
	\noindent \textbf{Condición:} El actor no ingresó algún dato marcado como obligatorio.
	\begin{enumerate}
		\UCpaso[\UCsist] Muestra el mensaje \cdtIdRef{MSG4}{Dato obligatorio} señalando el campo que presenta el error en la pantalla \IUref{IU11.2}{Modificar Término}.
		\UCpaso Regresa al paso \ref{CU6.2-P5} de la trayectoria principal.
		\item[- -] - - {\em {Fin de la trayectoria}}.%
	\end{enumerate}
%--------------------------------------
	\hypertarget{CU6-2:TAC}{\textbf{Trayectoria alternativa C}}\\
	\noindent \textbf{Condición:} El actor ingresó un dato con un número de caracteres fuera del rango permitido.
	\begin{enumerate}
		\UCpaso[\UCsist] Muestra el mensaje \cdtIdRef{MSG6}{Longitud inválida} señalando el campo que presenta el error en la pantalla \IUref{IU11.2}{Modificar Término}.
		\UCpaso Regresa al paso \ref{CU6.2-P5} de la trayectoria principal.
		\item[- -] - - {\em {Fin de la trayectoria}}.%
	\end{enumerate}
%--------------------------------------
	\hypertarget{CU6-2:TAD}{\textbf{Trayectoria alternativa D}}\\
	\noindent \textbf{Condición:} El actor ingresó un dato con un formato de dato incorrecto.
	\begin{enumerate}
		\UCpaso[\UCsist] Muestra el mensaje \cdtIdRef{MSG29}{Formato incorrecto} señalando el campo que presenta el error en la pantalla \IUref{IU11.2}{Modificar Término}.
		\UCpaso Regresa al paso \ref{CU6.2-P5} de la trayectoria principal.
		\item[- -] - - {\em {Fin de la trayectoria}}.
	\end{enumerate}
%--------------------------------------	
	\hypertarget{CU6-2:TAE}{\textbf{Trayectoria alternativa E}}\\
	\noindent \textbf{Condición:} El actor ingresó un nombre de término repetido.
	\begin{enumerate}
		\UCpaso[\UCsist] Muestra el mensaje \cdtIdRef{MSG7}{Registro repetido} señalando el campo que presenta la duplicidad en la pantalla \IUref{IU11.2}{Modificar Término}.
		\UCpaso Regresa al paso \ref{CU6.2-P5} de la trayectoria principal.
		\item[- -] - - {\em {Fin de la trayectoria}}.
	\end{enumerate}
%--------------------------------------	
\hypertarget{CU6-2:TAF}{\textbf{Trayectoria alternativa F}}\\
\noindent \textbf{Condición:} El término se encuentra asociado a casos de uso liberados.
\begin{enumerate}
	\UCpaso[\UCsist] Oculta el botón \editar del término que esta asociado a casos de uso liberados.
	\item[- -] - - {\em {Fin del caso de uso}}.
\end{enumerate}