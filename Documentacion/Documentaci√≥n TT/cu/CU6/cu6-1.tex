	\begin{UseCase}{CU6.1}{Registrar Término de Glosario}{
		Permite al colaborador (ya sea \hyperlink{jefe}{Líder} o \hyperlink{analista}{Analista}) registrar la información general de un  \hyperlink{terminoGLSEntidad}{Término de glosario}, dichos datos se guardan con el objetivo de conocer el significado de las palabras que puedan resultar difíciles de comprender al momento de dar lectura al caso de uso.

		Una vez registrado el término del glosario, el colaborador podrá hacer uso haciendo referencia del término en el editor.
	}
		
		\UCitem{Actor}{\hyperlink{jefe}{Líder de Análisis}, \hyperlink{analista}{Analista}}
		\UCitem{Propósito}{Registrar la información de un término.}
		\UCitem{Entradas}{
		\begin{itemize}
			\item \cdtRef{terminoGLSEntidad:nombreTGLS}{Nombre del término:} Se escribe desde el teclado.
			\item \cdtRef{terminoGLSEntidad:descripcionTGLS}{Descripción del término:} Se escribe desde el teclado.
		\end{itemize}	
		}
		\UCitem{Salidas}{\begin{itemize}
				\item \cdtRef{proyectoEntidad:claveProyecto}{Clave del proyecto:} Lo obtiene el sistema.
				\item \cdtRef{proyectoEntidad:nombreProyecto}{Nombre del proyecto:} Lo obtiene el sistema.
				\item \cdtIdRef{MSG1}{Operación exitosa}: Se muestra en la pantalla \IUref{IU11}{Gestionar Términos de glosario} para indicar que el registro fue exitoso.
		\end{itemize}}
		
		\UCitem{Precondiciones}{Ninguna}
		\UCitem{Postcondiciones}{
		\begin{itemize}
			\item Se registrará término de un proyecto en el sistema.
			\item El término podrá ser referenciado en casos de uso.
		\end{itemize}
		}
		\UCitem{Errores}{\begin{itemize}
		\item \cdtIdRef{MSG4}{Dato obligatorio}: Se muestra en la pantalla \IUref{IU11.1}{Registrar Término} cuando no se ha ingresado un dato marcado como obligatorio.
		\item \cdtIdRef{MSG29}{Formato incorrecto}: Se muestra en la pantalla \IUref{IU11.1}{Registrar Término} cuando el tipo de dato ingresado no cumple con el tipo de dato solicitado en el campo.
		\item \cdtIdRef{MSG6}{Longitud inválida}: Se muestra en la pantalla \IUref{IU11.1}{Registrar Término} cuando se ha excedido la longitud de alguno de los campos.
		\item \cdtIdRef{MSG7}{Registro repetido}: Se muestra en la pantalla \IUref{IU11.1}{Registrar Término} cuando se registre un término con un nombre que ya se encuentra registrado en el sistema.
		\end{itemize}
		}
		\UCitem{Tipo}{Secundario, extiende del caso de uso \UCref{CU6}{Gestionar Términos}.}
	\end{UseCase}
%--------------------------------------
	\begin{UCtrayectoria}
		\UCpaso[\UCactor] Solicita registrar un término oprimiendo el botón \IUbutton{Registrar} de la pantalla \IUref{IU11}{Gestionar Términos}.
		\UCpaso[\UCsist] Muestra la pantalla \IUref{IU11.1}{Registrar Término}.
		\UCpaso[\UCactor] Ingresa la información solicitada. \label{CU6.1-P3}
		\UCpaso[\UCactor] Oprime el botón \IUbutton{Aceptar}. \hyperlink{CU6-1:TAA}{[Trayectoria A]}
		\UCpaso[\UCsist] Verifica que el actor ingrese todos los campos obligatorios con base en la regla de negocio \BRref{RN8}{Datos obligatorios}. \hyperlink{CU6-1:TAB}{[Trayectoria B]}
		\UCpaso[\UCsist] Verificar que los datos ingresados cumpla con la longitud correcta, con base en la regla de negocio \BRref{RN37}{Longitud de datos}. \hyperlink{CU6-1:TAC}{[Trayectoria C]}
		\UCpaso[\UCsist] Verifica que los datos ingresados cumplan con el formato requerido, con base en la regla de negocio \BRref{RN7}{Información correcta}. \hyperlink{CU6-1:TAD}{[Trayectoria D]}
		\UCpaso[\UCsist] Verifica que el nombre del término no se encuentre registrado en el sistema con base en la regla de negocio \BRref{RN6}{Unicidad de nombres}. \hyperlink{CU6-1:TAE}{[Trayectoria E]}
		\UCpaso[\UCsist] Registra la información del término en el sistema.
		\UCpaso[\UCsist] Muestra el mensaje \cdtIdRef{MSG1}{Operación exitosa} en la pantalla \IUref{IU11}{Gestionar Términos del glosario} para indicar al actor que el registro se ha realizado exitosamente.
	\end{UCtrayectoria}		
%--------------------------------------	
	\hypertarget{CU6-1:TAA}{\textbf{Trayectoria alternativa A}}\\
	\noindent \textbf{Condición:} El actor desea cancelar la operación.
	\begin{enumerate}
		\UCpaso[\UCactor] Solicita cancelar la operación oprimiendo el botón \IUbutton{Cancelar} de la pantalla \IUref{IU11.1}{Registrar Término}
		\UCpaso[\UCsist] Muestra la pantalla \IUref{IU11}{Gestionar Términos}.
		\item[- -] - - {\em {Fin del caso de uso}}.%
	\end{enumerate}
%--------------------------------------
\hypertarget{CU6-1:TAB}{\textbf{Trayectoria alternativa B}}\\
\noindent \textbf{Condición:} El actor no ingresó algún dato marcado como obligatorio.
\begin{enumerate}
	\UCpaso[\UCsist] Muestra el mensaje \cdtIdRef{MSG4}{Dato obligatorio} señalando el campo que presenta el error en la pantalla \IUref{IU11.1}{Registrar Término}.
	\UCpaso Regresa al paso \ref{CU6.1-P3} de la trayectoria principal.
	\item[- -] - - {\em {Fin de la trayectoria}}.%
\end{enumerate}
%--------------------------------------
\hypertarget{CU6-1:TAC}{\textbf{Trayectoria alternativa C}}\\
\noindent \textbf{Condición:} El actor ingresó un dato con un número de caracteres fuera del rango permitido.
\begin{enumerate}
	\UCpaso[\UCsist] Muestra el mensaje \cdtIdRef{MSG6}{Longitud inválida} señalando el campo que presenta el error en la pantalla \IUref{IU11.1}{Registrar Término}.
	\UCpaso Regresa al paso \ref{CU6.1-P3} de la trayectoria principal.
	\item[- -] - - {\em {Fin de la trayectoria}}.%
\end{enumerate}
%--------------------------------------
\hypertarget{CU6-1:TAD}{\textbf{Trayectoria alternativa D}}\\
\noindent \textbf{Condición:} El actor ingresó un dato con un formato incorrecto.
\begin{enumerate}
	\UCpaso[\UCsist] Muestra el mensaje \cdtIdRef{MSG29}{Formato incorrecto} señalando el campo que presenta el error en la pantalla \IUref{IU11.1}{Registrar Término}.
	\UCpaso Regresa al paso \ref{CU6.1-P3} de la trayectoria principal.
	\item[- -] - - {\em {Fin de la trayectoria}}.
\end{enumerate}
%--------------------------------------	
\hypertarget{CU6-1:TAE}{\textbf{Trayectoria alternativa E}}\\
\noindent \textbf{Condición:} El actor ingresó un nombre de término repetido.
\begin{enumerate}
	\UCpaso[\UCsist] Muestra el mensaje \cdtIdRef{MSG7}{Registro repetido} señalando el campo que presenta la duplicidad en la pantalla \IUref{IU11.1}{Registrar Término}.
	\UCpaso Regresa al paso \ref{CU6.1-P3} de la trayectoria principal.
	\item[- -] - - {\em {Fin de la trayectoria}}.
\end{enumerate}


