	\begin{UseCase}{CU5.2}{Modificar Módulo}{
		Cuando se ha registrado un nuevo \hyperlink{moduloEntidad}{Módulo} y se requiere modificar su información, el sistema mostrará al colaborador (ya sea \hyperlink{jefe}{Líder} o \hyperlink{analista}{Analista}) un formulario con la información precargada de los datos que con anterioridad se guardaron, teniendo la posibilidad de editarlos y almacenarlos nuevamente sobreescribiendo lo previamente registrado.\\
		
		Debe considerarse que la \cdtRef{claveModulo}{Clave} del módulo no puede modificarse.
	}
		\UCitem{Actor}{\hyperlink{jefe}{Líder de Análisis}, \hyperlink{analista}{Analista}}
		\UCitem{Propósito}{Editar la información de un módulo.}
		\UCitem{Entradas}{
		\begin{itemize}
			\item \cdtRef{moduloEntidad:nombreModulo}{Nombre del módulo:} Se escribe desde el teclado.
			\item \cdtRef{moduloEntidad:descripcionModulo}{Descripción del módulo:} Se escribe desde el teclado.
		\end{itemize}	
		}
		\UCitem{Salidas}{\begin{itemize}
				\item \cdtRef{proyectoEntidad:claveProyecto}{Clave del proyecto:} Lo obtiene el sistema.
				\item \cdtRef{proyectoEntidad:nombreProyecto}{Nombre del proyecto:} Lo obtiene el sistema.
				\item \cdtRef{moduloEntidad:claveModulo}{Clave del módulo:} Lo obtiene el sistema.
				\item \cdtRef{moduloEntidad:nombreModulo}{Nombre del módulo:} Lo obtiene el sistema.
				\item \cdtRef{moduloEntidad:descripcionModulo}{Descripción del módulo:} Lo obtiene el sistema.
				\item \cdtIdRef{MSG1}{Operación exitosa}: Se muestra en la pantalla \IUref{IU4}{Gestionar Módulos} para indicar que la modificación fue exitosa.
		\end{itemize}}
		\UCitem{Precondiciones}{\begin{itemize}
				\item Que exista al menos un proyecto registrado
				\item Que exista al menos un módulo registrado
			\end{itemize}
		}
		\UCitem{Postcondiciones}{
			Se actualizará un módulo de un proyecto.
		}
		\UCitem{Errores}{\begin{itemize}
		\item \cdtIdRef{MSG4}{Dato obligatorio}: Se muestra en la pantalla \IUref{IU4.2}{Modificar Módulo} cuando no se ha ingresado un dato marcado como obligatorio.
		\item \cdtIdRef{MSG5}{Formato incorrecto}: Se muestra en la pantalla \IUref{IU4.2}{Modificar Módulo} cuando el tipo de dato ingresado no cumple con el tipo de dato solicitado en
		el campo.
		\item \cdtIdRef{MSG6}{Longitud inválida}: Se muestra en la pantalla \IUref{IU4.2}{Modificar Módulo} cuando se ha excedido la longitud de alguno de los campos.
		\item \cdtIdRef{MSG7}{Registro repetido}: Se muestra en la pantalla \IUref{IU4.2}{Modificar Módulo} cuando se registre un módulo con un nombre que ya se encuentra registrado en el sistema.
		\end{itemize}
		}
		\UCitem{Tipo}{Secundario, extiende del caso de uso \UCref{CU5}{Gestionar Módulos}.}
	\end{UseCase}
%--------------------------------------
	\begin{UCtrayectoria}
		\UCpaso[\UCactor] Da clic en icono \editar de la pantalla \IUref{IU4}{Gestionar Módulos}.
		\UCpaso[\UCsist] Obtiene la información del módulo seleccionado.
		\UCpaso[\UCsist] Muestra la pantalla \IUref{IU4.2}{Modificar Módulo}.
		\UCpaso[\UCsist] Marca como etiqueta el campo ''Clave'' de acuerdo a la regla de negocio \BRref{RN13}{Modificación del identificador}.
		\UCpaso[\UCactor] Modifica la información del módulo. \label{CU5.2-P4}
		\UCpaso[\UCactor] Oprime el botón \IUbutton{Aceptar}. \hyperlink{CU5-2:TAA}{[Trayectoria A]}
		\UCpaso[\UCsist] Verifica que el actor ingrese todos los campos obligatorios con base en la regla de negocio \BRref{RN8}{Datos obligatorios}. \hyperlink{CU5-2:TAB}{[Trayectoria B]}
		\UCpaso[\UCsist] Verifica que los datos ingresados cumpla con la longitud correcta, con base en la regla de negocio \BRref{RN37}{Longitud de datos}. \hyperlink{CU5-2:TAC}{[Trayectoria C]}
		\UCpaso[\UCsist] Verifica que los datos ingresados cumplan con el formato requerido, con base en la regla de negocio \BRref{RN7}{Información correcta}. \hyperlink{CU5-2:TAD}{[Trayectoria D]}
		\UCpaso[\UCsist] Verifica que el nombre del módulo no se encuentre registrado en el sistema con base en la regla de negocio \BRref{RN6}{Unicidad de nombres}. \hyperlink{CU5-2:TAE}{[Trayectoria E]}
		\UCpaso[\UCsist] Actualiza la información del módulo.
		\UCpaso[\UCsist] Muestra el mensaje \cdtIdRef{MSG1}{Operación exitosa} en la pantalla \IUref{IU4}{Gestionar Módulos} para indicar al actor que la modificación se ha realizado exitosamente.
	\end{UCtrayectoria}		
%--------------------------------------
	
%--------------------------------------
\hypertarget{CU5-2:TAA}{\textbf{Trayectoria alternativa A}}\\
\noindent \textbf{Condición:} El actor desea cancelar la operación.
\begin{enumerate}
	\UCpaso[\UCactor] Solicita cancelar la operación oprimiendo el botón \IUbutton{Cancelar} de la pantalla \IUref{IU4.2}{Modificar Módulo}
	\UCpaso[\UCsist] Muestra la pantalla \IUref{IU4}{Gestionar Módulos}.
	\item[- -] - - {\em {Fin del caso de uso}}.%
\end{enumerate}
%--------------------------------------
\hypertarget{CU5-2:TAB}{\textbf{Trayectoria alternativa B}}\\
\noindent \textbf{Condición:} El actor no ingresó algún dato marcado como obligatorio.
\begin{enumerate}
	\UCpaso[\UCsist] Muestra el mensaje \cdtIdRef{MSG4}{Dato obligatorio} señalando el campo que presenta el error en la pantalla \IUref{IU4.2}{Modificar Módulo}.
	\UCpaso Regresa al paso \ref{CU5.2-P4} de la trayectoria principal.
	\item[- -] - - {\em {Fin de la trayectoria}}.%
\end{enumerate}
%--------------------------------------
\hypertarget{CU5-2:TAC}{\textbf{Trayectoria alternativa C}}\\
\noindent \textbf{Condición:} El actor ingresó un dato con un número de caracteres fuera del rango permitido.
\begin{enumerate}
	\UCpaso[\UCsist] Muestra el mensaje \cdtIdRef{MSG6}{Longitud inválida} señalando el campo que presenta el error en la pantalla \IUref{IU4.2}{Modificar Módulo}.
	\UCpaso Regresa al paso \ref{CU5.2-P4} de la trayectoria principal.
	\item[- -] - - {\em {Fin de la trayectoria}}.%
\end{enumerate}
%--------------------------------------	
\hypertarget{CU5-2:TAD}{\textbf{Trayectoria alternativa D}}\\
\noindent \textbf{Condición:} El actor ingresó un dato con un formato o tipo de dato incorrecto.
\begin{enumerate}
	\UCpaso[\UCsist] Muestra el mensaje \cdtIdRef{MSG5}{Formato incorrecto} señalando el campo que presenta el error en la pantalla \IUref{IU4.2}{Modificar Módulo}.
	\UCpaso Regresa al paso \ref{CU5.2-P4} de la trayectoria principal.
	\item[- -] - - {\em {Fin de la trayectoria}}.
\end{enumerate}
%--------------------------------------	
\hypertarget{CU5-2:TAE}{\textbf{Trayectoria alternativa E}}\\
\noindent \textbf{Condición:} El actor ingresó un nombre de módulo repetido.
\begin{enumerate}
	\UCpaso[\UCsist] Muestra el mensaje \cdtIdRef{MSG7}{Registro repetido} señalando el campo que presenta la duplicidad en la pantalla \IUref{IU4.2}{Modificar Módulo}
	\UCpaso Regresa al paso \ref{CU5.2-P4} de la trayectoria principal.
	\item[- -] - - {\em {Fin de la trayectoria}}.
\end{enumerate}