	\begin{UseCase}{CU5}{Gestionar Módulos}{
	Permite al Analista visualizar todos los módulos en que se divide el proyecto, así como solicitar el registro, modificación y eliminación de un módulo.
	}
	\UCitem{Versión}{\color{Gray}0.1}
	\UCitem{Actor}{\hyperlink{jefe}{Líder de análisis}, \hyperlink{analista}{Analista}}
	\UCitem{Propósito}{Proporcionar al actor un mecanismo para llevar el control de los módulos de un proyecto.}
	\UCitem{Entradas}{Ninguna}
	\UCitem{Salidas}{\begin{itemize}
			\item \cdtRef{proyectoEntidad:claveProyecto}{Clave del proyecto}: Lo obtiene el sistema.
			\item \cdtRef{proyectoEntidad:nombreProyecto}{Nombre del proyecto}: Lo obtiene el sistema.
			\item \cdtRef{moduloEntidad}{Módulo}: Tabla que muestra \cdtRef{moduloEntidad:claveModulo}{Clave}, y el \cdtRef{moduloEntidad:nombreModulo}{Nombre} de los módulos de un proyecto.
			\item \cdtIdRef{MSG2}{No existe información}: Se muestra en la pantalla \IUref{IU4}{Gestionar Módulos} cuando no existen módulos registrados.
	\end{itemize}}
	\UCitem{Precondiciones}{Ninguna}
	\UCitem{Postcondiciones}{Ninguna}
	\UCitem{Errores}{Ninguno}
	\UCitem{Tipo}{Secundario, extiende de \UCref{CU4}{Gestionar Proyectos de Colaborador}}
\end{UseCase}
%--------------------------------------
\begin{UCtrayectoria}
	\UCpaso[\UCactor] Solicita gestionar los términos presionando el botón \raisebox{-1mm}{\includegraphics[height=11pt]{images/Iconos/entrar}} de algún proyecto existente de la pantalla \IUref{IU5}{Gestionar Proyectos de Colaborador}.
	\UCpaso[\UCsist] Obtiene la información de los módulos del proyecto seleccionado. \hyperlink{CU5:TAA}{[Trayectoria A]}
	\UCpaso[\UCsist] Muestra la información de los módulos en la pantalla \IUref{IU4}{Gestionar Módulos}.\label{CU5-P3}
	\UCpaso[\UCactor] Gestiona los módulos a través de los botones: \IUbutton{Registrar}, \editar, \eliminar, \UCsist  y \raisebox{-1mm}{\includegraphics[height=11pt]{images/Iconos/pantalla}}.
\end{UCtrayectoria}		
%--------------------------------------
\hypertarget{CU5:TAA}{\textbf{Trayectoria alternativa A}}\\
\noindent \textbf{Condición:} No existen registros de módulos
\begin{enumerate}
	\UCpaso[\UCsist] Muestra el mensaje \cdtIdRef{MSG2}{No existe información} en la pantalla \IUref{IU4}{Gestionar Módulos} para indicar que no hay registros de módulos para mostrar. \label{CU5-TA1}
	\UCpaso[\UCactor] Gestiona los módulos a través del botón: \IUbutton{Registrar}. 
	\item[- -] - - {\em {Fin del caso de uso}}.%
\end{enumerate}
%--------------------------------------

\subsubsection{Puntos de extensión}

\UCExtenssionPoint{El actor requiere registrar un módulo.}{Paso \ref{CU5-P3} de la trayectoria principal o del paso \ref{CU5-TA1} de la trayectoria alternativa A.}{\UCref{CU5.1}{Registrar Módulo}}
\UCExtenssionPoint{El actor requiere modificar un módulo.}{Paso \ref{CU5-P3} de la trayectoria principal.}{\UCref{CU5.2}{Modificar Módulo}}
\UCExtenssionPoint{El actor requiere eliminar un módulo.}{Paso \ref{CU5-P3} de la trayectoria principal.}{\UCref{CU5.3}{Eliminar Módulo}}
\UCExtenssionPoint{El actor requiere gestionar los casos de uso de un módulo.}{Paso \ref{CU5-P3} de la trayectoria principal.}{\UCref{CU11}{Gestionar Casos de uso}}
\UCExtenssionPoint{El actor requiere gestionar las pantallas de un módulo.}{Paso \ref{CU5-P3} de la trayectoria principal.}{\UCref{CU3.3}{Eliminar Persona}}