	\begin{UseCase}{CU4.1}{Elegir Colaboradores}{
		Este caso de uso permite al {\hyperlink{jefe}{Líder de proyecto}} asignar a los Colaboradores con los que trabajará en conjunto durante el desarrollo del mismo, esta selección podrá realizarse a través de una lista o una búsqueda en donde se mostrará el registro del personal de la organización.
	
		Al momento de seleccionar a los analistas, estos tendrán acceso al proyecto y a sus acciones correspondientes. El administrador no podrá participar como colaborador en ningún proyecto.
	}
		\UCitem{Actor}{\hyperlink{jefe}{Líder de análisis}}
		\UCitem{Propósito}{Elegir a los colaboradores que participarán en un proyecto.}
		\UCitem{Entradas}{
		\begin{itemize}
			\item \cdtRef{colaboradorEntidad}{Colaborador}: Se selecciona de un combobox.
		\end{itemize}	
		}
		\UCitem{Salidas}{\begin{itemize}
				\item \cdtIdRef{MSG1}{Operación exitosa}: Se muestra en la pantalla \IUref{IU5}{Gestionar Proyectos de colaborador} para indicar que la asociación se realizó exitosamente.
				\item  \cdtIdRef{MSG2}{No existe información}: Se muestra en la pantalla \IUref{IU5.1A}{Elegir Colaboradores} cuando no hay personal registrado.
		\end{itemize}}
		\UCitem{Precondiciones}{\begin{itemize}
				\item Que el actor haya iniciado sesión como \hyperlink{jefe}{Líder de proyecto}.
				\item Que existan al menos 2 colaboradores en el sistema.
			\end{itemize}}
		\UCitem{Postcondiciones}{
			Los colaboradores seleccionados podrán entrar al proyecto en cuestión.
		}
		\UCitem{Errores}{
				\cdtIdRef{MSG12}{Ha ocurrido un error}: Se muestra en la pantalla \IUref{IU5}{Gestionar Proyectos de colaborador} cuando la asignación del colaborador a un proyecto no se realizó correctamente.
		}
		\UCitem{Tipo}{Secundario, extiende del caso de uso \UCref{CU4}{Gestionar Proyectos de Colaborador}.}
	\end{UseCase}
%--------------------------------------
	\begin{UCtrayectoria}
		\UCpaso[\UCactor] Da clic en el icono \raisebox{-1mm}{\includegraphics[height=11pt]{images/Iconos/colaboradores}} del proyecto que desee de la pantalla \IUref{IU5}{Gestionar Proyectos de Colaborador}.
		\UCpaso[\UCsist] Obtiene la información del personal de la organización. \hyperlink{CU4-1:TAA}{[Trayectoria A]}
		\UCpaso[\UCsist] Discrimina al \hyperlink{admin}{Administrador} de la lista de colaboradores con base en la regla de negocio \BRref{RN38}{Funciones de administrador}.
		\UCpaso[\UCsist] Muestra los colaboradores encontrados en la pantalla \IUref{IU5.1}{Elegir Colaboradores}
		\UCpaso[\UCactor] Selecciona los colaboradores que participarán en el proyecto, marcando o desmarcando las casillas de la columna ''Elegir''.
		\UCpaso[\UCactor] Oprime el botón \IUbutton{Aceptar}.
		\UCpaso[\UCsist] Actualiza el personal asociado al proyecto en cuestión. \hyperlink{CU4-1:TAB}{[Trayectoria B]}
		\UCpaso[\UCsist] Muestra el mensaje \cdtIdRef{MSG1}{Operación exitosa} en la pantalla \IUref{IU5}{Gestionar Proyectos de Colaborador} para indicar al actor que los colaboradores del proyecto han sido asociados exitosamente.
	\end{UCtrayectoria}		
%--------------------------------------
	\hypertarget{CU4-1:TAA}{\textbf{Trayectoria alternativa A}}\\
	\noindent \textbf{Condición:} No existen colaboradores.
	\begin{enumerate}
		\UCpaso[\UCsist] Muestra el mensaje \cdtIdRef{MSG2}{No existe información} en la pantalla \IUref{IU5.1A}{Elegir Colaboradores} para indicar que no hay registros para mostrar.
		\item[- -] - - {\em {Fin del caso de uso}}.%
	\end{enumerate}

\hypertarget{CU4-1:TAB}{\textbf{Trayectoria alternativa B}}\\
\noindent \textbf{Condición:} La asociación del colaborador falló.
\begin{enumerate}
	\UCpaso[\UCsist] Muestra el mensaje \cdtIdRef{MSG12}{Ha ocurido un error} en la pantalla \IUref{IU5}{Gestionar Proyectos de Colaborador} para indicar que ocurrió un error al asociar a los colaboradores.
	\item[- -] - - {\em {Fin del caso de uso}}.%
\end{enumerate}
	
%--------------------------------------