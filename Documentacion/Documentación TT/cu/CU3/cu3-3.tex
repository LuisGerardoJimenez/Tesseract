	\begin{UseCase}{CU3.3}{Eliminar Colaborador}{
			Cuando algún miembro deja de pertenecer a la compañia y ya no se contará con su participación dentro de algún \hyperlink{proyectoEntidad}{Proyecto}, Tesseract permitirá al {\hyperlink{admin}{Administrador}} eliminar en su totalidad el registro de un colaborador. \\
			Un colaborador podrá ser eliminado siempre y cuando no sea el {\hyperlink{jefe}{líder}} de algún \hyperlink{proyectoEntidad}{Proyecto}.
	}
		\UCitem{Versión}{\color{Gray}0.1}
		\UCitem{Actor}{\hyperlink{admin}{Administrador}}
		\UCitem{Propósito}{Eliminar un Colaborador del sistema.}
		\UCitem{Entradas}{Niguna}	
		\UCitem{Salidas}{
		\begin{itemize}
			\item \cdtIdRef{MSG1}{Operación Exitosa}: Se muestra en la pantalla \IUref{IU3}{Gestionar Colaboradores} para indicar el proyecto fue eliminado correctamente.
			\item \cdtIdRef{MSG10}{Confirmar eliminación}: Se muestra para que el actor confirme la eliminación.
		\end{itemize}
		}
		\UCitem{Precondiciones}{
		\begin{itemize}
			\item Que el Colaborador no lidere ningún proyecto.
		\end{itemize}
		}
		\UCitem{Postcondiciones}{
		\begin{itemize}
			\item Se eliminará un Colaborador del sistema.
		\end{itemize}
		}
		\UCitem{Errores}{\begin{itemize}
		\item \cdtIdRef{MSG13}{Eliminación no permitida}: Se muestra en la pantalla \IUref{IU3}{Gestionar Colaboradores} cuando no se pueda eliminar un Colaborador.
		\end{itemize}
		}
		\UCitem{Tipo}{Secundario, extiende del caso de uso \UCref{CU3}{Gestionar Colaboradores}}
	\end{UseCase}
%--------------------------------------
	\begin{UCtrayectoria}
		\UCpaso[\UCactor] Solicita eliminar un Colaborador oprimiendo el botón \eliminar del registro que desea eliminar de la pantalla \IUref{IU3}{Gestionar Colaboradores}.
		\UCpaso[\UCsist] Obtiene la información del proyecto seleccionado.
		\UCpaso[\UCsist] Verifica que el Colaborador pueda eliminarse, con base en la regla de negocio \BRref{RN27}{Eliminación de Colaboradores}. \hyperlink{CU3-3:TAA}{[Trayectoria A]}
		\UCpaso[\UCsist] Muestra el mensaje \cdtIdRef{MSG10}{Confirmar eliminación} en la pantalla \IUref{IU3}{Gestionar Colaboradores} con los botones \IUbutton{Aceptar} y \IUbutton{Cancelar}
		\UCpaso[\UCsist] Confirma la eliminación del proyecto oprimiendo el botón \IUbutton{Aceptar}. \hyperlink{CU3-3:TAB}{[Trayectoria B]}
		\UCpaso[\UCsist] Elimina el proyecto del sistema.
		\UCpaso[\UCsist] Muestra el mensaje \cdtIdRef{MSG1}{Operación exitosa} en la pantalla \IUref{IU3}{Gestionar Colaboradores} para indicar al actor que se ha eliminado el registro exitosamente.
	\end{UCtrayectoria}		
%--------------------------------------	
	\hypertarget{CU3-3:TAA}{\textbf{Trayectoria alternativa A}}\\
	\noindent \textbf{Condición:} El Colaborador tiene elementos asociados.
	\begin{enumerate}
		\UCpaso[\UCsist] Muestra el mensaje \cdtIdRef{MSG13}{Eliminación no permitida} en la pantalla \IUref{IU3}{Gestionar Colaboradores}.
		\item[- -] - - {\em {Fin del caso de uso}}.%
	\end{enumerate}
%--------------------------------------
	\hypertarget{CU3-3:TAB}{\textbf{Trayectoria alternativa B}}\\
	\noindent \textbf{Condición:} El actor desea cancelar la operación.
	\begin{enumerate}
		\UCpaso[\UCactor] Solicita cancelar la operación oprimiendo el botón \IUbutton{Cancelar} de la ventana emergente.
		\UCpaso[\UCsist] Muestra la pantalla \IUref{IU3}{Gestionar Colaboradores}.
		\item[- -] - - {\em {Fin del caso de uso}}.%
	\end{enumerate}
%--------------------------------------
