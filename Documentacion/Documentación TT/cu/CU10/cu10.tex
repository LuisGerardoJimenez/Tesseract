	\begin{UseCase}{CU10}{Gestionar Actores}{
			Cuando un colaborador (\hyperlink{jefe}{Líder de análisis} o \hyperlink{analista}{Analista}) está operando en un \hyperlink{proyectoEntidad}{Proyecto}, tiene la posibilidad de gestionar los \hyperlink{tElemento}{elementos} que serán referenciados al \hyperlink{casoUso}{Caso de Uso}. Uno de estos \hyperlink{tElemento}{elementos} es el \hyperlink{actorEntidad}{Actor}.\\
			
			Un Actor es una entidad que interactua con el sistema y que le demanda una funcionalidad. Esto incluye a los humanos pero también  a todos los sistemas externos y entidades abstractas como el tiempo. En el caso de los seres humanos se pueden ver a los actores como definiciones de rol por lo que un mismo individuo puede corresponder a uno o más Actores.\\
			
			Este caso de uso permite al colaborador visualizar en una tabla, el registro de los actores que componen el proyecto (con su respectiva información), así como solicitar el registro de un actor nuevo, modificar alguno existente (teniendo la posibilidad de actualizar algún dato de registro), eliminarlo (en caso de que el actor por algún motivo ya no forme parte del proyecto) o simplemente consultar su información (nombre, descripción y cardinalidad) \\
			
			Las acciones disponibles para cada actor dependerán del estado en el que se encuentre el \hyperlink{casoUso}{Caso de Uso} donde es referenciado.
	}
	\UCitem{Actor}{\hyperlink{jefe}{Líder de análisis}, \hyperlink{analista}{Analista}}
	\UCitem{Propósito}{Proporcionar al actor un mecanismo para llevar el control de los actores de un proyecto.}
	\UCitem{Entradas}{Ninguna}
	\UCitem{Salidas}{\begin{itemize}
			\item \cdtRef{proyectoEntidad:claveProyecto}{Clave del proyecto}: Lo obtiene el sistema.
			\item \cdtRef{proyectoEntidad:nombreProyecto}{Nombre del proyecto}: Lo obtiene el sistema.
			\item \cdtRef{actorEntidad}{Actor}: Tabla que muestra \cdtRef{actorEntidad:nombreACT}{nombre} de todos los actores registrados de un proyecto.
			\item \cdtIdRef{MSG2}{No existe información}: Se muestra en la pantalla \IUref{IU8A}{Gestionar Actores} cuando no existen actores registrados.
	\end{itemize}}
	
	\UCitem{Precondiciones}{Que exista al menos un proyecto registrado.}
	\UCitem{Postcondiciones}{Ninguna}
	\UCitem{Errores}{Ninguno}
	\UCitem{Tipo}{Primario}
\end{UseCase}
%--------------------------------------
\begin{UCtrayectoria}
	\UCpaso[\UCactor] Solicita gestionar los actores seleccionando la opción ''Actores'' del menú \IUref{MN3}{Menú de Proyecto}.
	\UCpaso[\UCsist] Obtiene la información de los actores registrados del proyecto seleccionado. \hyperlink{CU10:TAA}{[Trayectoria A]}
	\UCpaso[\UCsist] Ordena los actores alfabéticamente basándose en el nombre de los mismos.
	\UCpaso[\UCsist] Muestra la información de los actores en la pantalla \IUref{IU8}{Gestionar Actores} y las operaciones disponibles de acuerdo a la regla de negocio \BRref{RN15}{Operaciones disponibles}. \label{CU10-P4}
	\UCpaso[\UCactor] Gestiona los actores a través de los botones: \IUbutton{Registrar}, \editar, \eliminar. y \raisebox{-1mm}{\includegraphics[height=11pt]{images/Iconos/consultar}}. 
\end{UCtrayectoria}		
%--------------------------------------
\hypertarget{CU10:TAA}{\textbf{Trayectoria alternativa A}}\\
\noindent \textbf{Condición:} No existen registros de actores.
\begin{enumerate}
	\UCpaso[\UCsist] Muestra el mensaje \cdtIdRef{MSG2}{No existe información} en la pantalla \IUref{IU8A}{Gestionar Actores} para indicar que no hay registros de actores para mostrar. \label{CU10-TA1}
	\UCpaso[\UCactor] Gestiona los actores a través del botón: \IUbutton{Registrar}. 
	\item[- -] - - {\em {Fin del caso de uso}}.%
\end{enumerate}

%--------------------------------------

\subsubsection{Puntos de extensión}

\UCExtenssionPoint{El actor requiere registrar un actor.}{Presionando el botón \IUbutton{Registrar} del paso \ref{CU10-P4} de la trayectoria principal o  del paso \ref{CU10-TA1} de la trayectoria alternativa A.}{\UCref{CU10.1}{Registrar Actor}}
\UCExtenssionPoint{El actor requiere modificar un actor.}{Presionando el icono \editar del paso \ref{CU10-P4} de la trayectoria principal.}{\UCref{CU10.2}{Modificar Actor}}
\UCExtenssionPoint{El actor requiere eliminar un actor.}{Presionando el icono \eliminar del paso \ref{CU10-P4} de la trayectoria principal.}{\UCref{CU10.3}{Eliminar Actor}}
\UCExtenssionPoint{El actor requiere consultar un actor.}{Presionando el icono \raisebox{-1mm}{\includegraphics[height=11pt]{images/Iconos/consultar}} del paso \ref{CU10-P4} de la trayectoria principal.}{\UCref{CU10.4}{Consultar Actor}}