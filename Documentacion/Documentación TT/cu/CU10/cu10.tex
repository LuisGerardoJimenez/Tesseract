	\begin{UseCase}{CU10}{Gestionar Actores}{
	Este caso de uso permite al analista visualizar los registros de los actores registrados en el sistema. También permite al actor acceder a las operaciones de registro, consulta, modificación, y eliminación de un actor.
	}
	\UCitem{Actor}{\hyperlink{jefe}{Líder de análisis}, \hyperlink{analista}{Analista}}
	\UCitem{Propósito}{Proporcionar al actor un mecanismo para llevar el control de los actores de un proyecto.}
	\UCitem{Entradas}{Ninguna}
	\UCitem{Salidas}{\begin{itemize}
			\item \cdtRef{proyectoEntidad:claveProyecto}{Clave del proyecto}: Lo obtiene el sistema.
			\item \cdtRef{proyectoEntidad:nombreProyecto}{Nombre del proyecto}: Lo obtiene el sistema.
			\item \cdtRef{actorEntidad}{Actor}: Tabla que muestra \cdtRef{actorEntidad:nombreACT}{nombre} de todos los actores registrados de un proyecto.
			\item \cdtIdRef{MSG2}{No existe información}: Se muestra en la pantalla \IUref{IU8A}{Gestionar Actores} cuando no existen actores registrados.
	\end{itemize}}
	
	\UCitem{Precondiciones}{\begin{itemize}
			\item Que exista al menos un proyecto registrado.
			\item Que exista al menos un actor registrado.
	\end{itemize}}
	\UCitem{Postcondiciones}{Ninguna}
	\UCitem{Errores}{Ninguno}
	\UCitem{Tipo}{Primario}
\end{UseCase}
%--------------------------------------
\begin{UCtrayectoria}
	\UCpaso[\UCactor] Solicita gestionar los actores seleccionando la opción ''Actores'' del menú \IUref{MN3}{Menú de Proyecto}.
	\UCpaso[\UCsist] Obtiene la información de los actores registrados del proyecto seleccionado. \hyperlink{CU10:TAA}{[Trayectoria A]}
	\UCpaso[\UCsist] Ordena los actores alfabéticamente basándose en nombre de los mismos.
	\UCpaso[\UCsist] Muestra la información de los actores en la pantalla \IUref{IU8}{Gestionar Actores} y las operaciones disponibles de acuerdo a la regla de negocio \BRref{RN15}{Operaciones disponibles}. \label{CU10-P4}
	\UCpaso[\UCactor] Gestiona los actores a través de los botones: \IUbutton{Registrar}, \editar, \eliminar. y \raisebox{-1mm}{\includegraphics[height=11pt]{images/Iconos/consultar}}. 
\end{UCtrayectoria}		
%--------------------------------------
\hypertarget{CU10:TAA}{\textbf{Trayectoria alternativa A}}\\
\noindent \textbf{Condición:} No existen registros de actores.
\begin{enumerate}
	\UCpaso[\UCsist] Muestra el mensaje \cdtIdRef{MSG2}{No existe información} en la pantalla \IUref{IU8A}{Gestionar Actores} para indicar que no hay registros de actores para mostrar. \label{CU10-TA1}
	\UCpaso[\UCactor] Gestiona los actores a través del botón: \IUbutton{Registrar}. 
	\item[- -] - - {\em {Fin del caso de uso}}.%
\end{enumerate}

%--------------------------------------

\subsubsection{Puntos de extensión}

\UCExtenssionPoint{El actor requiere registrar un actor.}{Paso \ref{CU10-P4} de la trayectoria principal o  del paso \ref{CU10-TA1} de la trayectoria alternativa A.}{\UCref{CU10.1}{Registrar Actor}}
\UCExtenssionPoint{El actor requiere modificar un actor.}{Paso \ref{CU10-P4} de la trayectoria principal.}{\UCref{CU10.2}{Modificar Actor}}
\UCExtenssionPoint{El actor requiere eliminar un actor.}{Paso \ref{CU10-P4} de la trayectoria principal.}{\UCref{CU10.3}{Eliminar Actor}}
\UCExtenssionPoint{El actor requiere consultar un actor.}{Paso \ref{CU10-P4} de la trayectoria principal.}{\UCref{CU10.4}{Consultar Actor}}