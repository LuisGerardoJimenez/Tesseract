	\begin{UseCase}{CU10.4}{Consultar Actor}{
			
	Este caso de uso permite al colaborador (\hyperlink{jefe}{Líder} o \hyperlink{analista}{Analista}) consultar la información de un \hyperlink{actorEntidad}{Actor} previamente registrado en el \hyperlink{proyectoEntidad}{Proyecto} sobre el cual se está operando.
	
	Además del nombre del actor, el colaborador podrá consultar su descripción y cardinalidad.
	
	}
		\UCitem{Versión}{\color{Gray}0.1}
		\UCitem{Actor}{\hyperlink{jefe}{Líder de Análisis}, \hyperlink{analista}{Analista}}
		\UCitem{Propósito}{Consultar la información de un actor de un proyecto.}
		\UCitem{Entradas}{Ninguna}
		\UCitem{Salidas}{
			\begin{itemize}
				\item \cdtRef{proyectoEntidad:claveProyecto}{Clave del proyecto:} Lo obtiene el sistema.
				\item \cdtRef{proyectoEntidad:nombreProyecto}{Nombre del proyecto:} Lo obtiene el sistema.
				\item \cdtRef{actorEntidad:nombreACT}{Nombre del actor:} Se escribe desde el teclado.
				\item \cdtRef{actorEntidad:descripcionACT}{Descripción del actor:} Se escribe desde el teclado.
		\end{itemize}}
		\UCitem{Destino}{Pantalla}
		\UCitem{Precondiciones}{Ninguna}
		\UCitem{Postcondiciones}{Ninguna}
		\UCitem{Errores}{\begin{itemize}
		\item \cdtIdRef{MSG12}{Ha ocurrido un error}: Se muestra en la pantalla \IUref{IU8}{Gestionar Actores} cuando el actor que se desea consultar no existe.
		\end{itemize}
		}
		\UCitem{Tipo}{Secundario, extiende del caso de uso \UCref{CU10}{Gestionar Actores}.}
	\end{UseCase}
%--------------------------------------
	\begin{UCtrayectoria}
		\UCpaso[\UCactor] Da clic en el icono \raisebox{-1mm}{\includegraphics[height=11pt]{images/Iconos/consultar}} del registro que desea consultar de la pantalla \IUref{IU8}{Gestionar Actores} o la liga correspondiente a un actor en la pantalla \IUref{IU6.3}{Consultar caso de uso}.
		\UCpaso[\UCsist] Obtiene la información del actor seleccionado. \hyperlink{CU10-4:TAA}{[Trayectoria A]}
		\UCpaso[\UCsist] Muestra la pantalla \IUref{IU8.3}{Consultar Actor}.
		\UCpaso[\UCactor] Consulta la información del actor.
		\UCpaso[\UCactor] Finaliza la consulta oprimiendo el botón \IUbutton{Regresar}.
		\UCpaso[\UCsist] Muestra la pantalla \IUref{IU8}{Gestionar Actores}.
	\end{UCtrayectoria}		
%--------------------------------------
\hypertarget{CU10-4:TAA}{\textbf{Trayectoria alternativa A}}\\
\noindent \textbf{Condición:} El actor que se desea consultar no existe.
\begin{enumerate}
	\UCpaso[\UCsist] Muestra el mensaje \cdtIdRef{MSG12}{Ha ocurrido un error} en la pantalla \IUref{IU8}{Gestionar Actores}.
	\item[- -] - - {\em {Fin del caso de uso}}.
\end{enumerate}

	
