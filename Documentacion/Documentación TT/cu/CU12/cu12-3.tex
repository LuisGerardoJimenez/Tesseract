	\begin{UseCase}{CU12.3}{Eliminar Caso de uso}{
	Este caso de uso permite al actor eliminar del sistema un caso de uso.
	}
		\UCitem{Actor}{\hyperlink{jefe}{Líder de Análisis}, \hyperlink{analista}{Analista}}
		\UCitem{Propósito}{Eliminar la información de un caso de uso.}
		\UCitem{Entradas}{Ninguna}
		\UCitem{Salidas}{\begin{itemize}
				\item \cdtIdRef{MSG1}{Operación exitosa}: Se muestra en la pantalla \IUref{IU6}{Gestionar Casos de uso} para indicar que la pantalla fue eliminada correctamente.
				\item \cdtIdRef{MSG10}{Confirmar eliminación}: Se muestra en la pantalla \IUref{IU6}{Gestionar Casos de uso} para que el actor confirme la eliminación.
		\end{itemize}}
		\UCitem{Precondiciones}{
				Que el caso de uso se encuentre en estado ''Edición'' o ''Pendiente de corrección''.
		}
		\UCitem{Postcondiciones}{
			Se eliminará una pantalla un caso de uso del sistema.
		}
		\UCitem{Errores}{\begin{itemize}
		\item \cdtIdRef{MSG13}{Eliminación no permitida}: Se muestra en la pantalla \IUref{IU6}{Gestionar Casos de uso} cuando no se pueda eliminar el caso de uso debido a que está siendo referenciado en algún caso de uso.
		\end{itemize}
		}
		\UCitem{Tipo}{Secundario, extiende del caso de uso \UCref{CU12}{Gestionar Casos de uso}.}
	\end{UseCase}
%--------------------------------------
	\begin{UCtrayectoria}
		\UCpaso[\UCactor] Da clic en el icono \eliminar del registro que desea eliminar de la pantalla \IUref{IU6}{Gestionar Casos de uso}.
		\UCpaso[\UCsist] Verifica que el caso de uso se encuentre en estado ''Edición'' o ''Pendiente de corrección''. \hyperlink{CU12-3:TAC}{[Trayectoria C]}
		\UCpaso[\UCsist] Muestra el mensaje emergente \cdtIdRef{MSG10}{Confirmar eliminación} con los botones \IUbutton{Aceptar} y \IUbutton{Cancelar} en la pantalla \IUref{IU6}{Gestionar Casos de uso}.
		\UCpaso[\UCactor] Confirma la eliminación del caso de uso oprimiendo el botón \IUbutton{Aceptar}. \hyperlink{CU12-3:TAA}{[Trayectoria A]}
		\UCpaso[\UCsist] Verifica que ningún caso de uso se encuentre referenciando a algún elemento perteneciente caso de uso que desea eliminar. \hyperlink{CU12-3:TAB}{[Trayectoria B]}
		\UCpaso[\UCsist] Elimina la información referente al caso de uso.
		\UCpaso[\UCsist] Muestra el mensaje \cdtIdRef{MSG1}{Operación exitosa} en la pantalla \IUref{IU6}{Gestionar Casos de uso} para indicar al actor que el registro se ha eliminado exitosamente.
	\end{UCtrayectoria}		
%--------------------------------------
\hypertarget{CU12-3:TAA}{\textbf{Trayectoria alternativa A}}\\
\noindent \textbf{Condición:} El actor desea cancelar la operación.
\begin{enumerate}
	\UCpaso[\UCactor] Oprime el botón \IUbutton{Cancelar} de la pantalla emergente.
	\UCpaso[\UCsist] Muestra la pantalla \IUref{IU6}{Gestionar Casos de Uso}.
	\item[- -] - - {\em {Fin del caso de uso}}.%
\end{enumerate}
%--------------------------------------
\hypertarget{CU12-3:TAB}{\textbf{Trayectoria alternativa B}}\\
\noindent \textbf{Condición:} El caso de uso o uno de sus elementos se encuentra referenciado en otro caso de uso.
\begin{enumerate}
	\UCpaso[\UCsist] Muestra el mensaje \cdtIdRef{MSG13}{Eliminación no permitida} en la pantalla \IUref{IU6}{Gestionar Casos de uso} en una pantalla emergente con la lista de casos de uso que están referenciando a los elementos del caso de uso que se desea eliminar.
	\item[- -] - - {\em {Fin del caso de uso}}.
\end{enumerate}
%-------------------------------------	
\hypertarget{CU12-3:TAC}{\textbf{Trayectoria alternativa C}}\\
\noindent \textbf{Condición:} El caso de uso no se encuentra en estado ''Edición'' o ''Pendiente de corrección''.
\begin{enumerate}
	\UCpaso[\UCsist] Oculta el botón \eliminar del caso que no se encuentra en estado de ''Edición'' o ''Pendiente de corrección''.
	\item[- -] - - {\em {Fin del caso de uso}}.
\end{enumerate}
