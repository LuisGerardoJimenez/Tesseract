	\begin{UseCase}{CU12}{Gestionar Casos de uso}{
			Una vez que el colaborador (\hyperlink{jefe}{Líder de Análisis} o \hyperlink{analista}{Analista}) ha dividido el \hyperlink{proyectoEntidad}{Proyecto}, podrá acceder a las acciones que corresponden a cada \hyperlink{moduloEntidad}{Módulo} como la gestión de casos de uso.  Cada módulo contiene la gestión de sus respectivos casos de uso.
			
			En el documento de caso de uso los analistas plasmarán a través de trayectorias, precondiciones, postcondiciones, puntos de extensión, así como del uso de otros \hyperlink{tElemento}{elementos} (pantallas, entidades, reglas de negocio, mensajes, actores y términos del negocio), la interacción del actor con el sistema. 
			
			Este caso de uso permite al colaborador visualizar en una tabla, el registro de los casos de uso que conforman un módulo en específico, así como solicitar el registro de uno nuevo (en donde también es posible gestionar otros \hyperlink{tElemento}{elementos} propios del Caso de uso), modificar la información de algún Caso de uso existente (teniendo la posibilidad de actualizar algún dato de registro), gestionar sus trayectorias, puntos de extensión, entradas, así como eliminarlo (en caso de que el caso de uso por algún motivo ya no forme parte del proyecto), terminarlo o simplemente consultarlo (información general, trayectorias y puntos de extensión). \\
	}

	\UCitem{Actor}{\hyperlink{jefe}{Líder de análisis}, \hyperlink{analista}{Analista}}
	\UCitem{Propósito}{Proporcionar al actor un mecanismo para llevar el control de los casos de uso de un proyecto.}
	\UCitem{Entradas}{Ninguna}
	\UCitem{Salidas}{\begin{itemize}
			\item \cdtRef{proyectoEntidad:claveProyecto}{Clave del proyecto}: Lo obtiene el sistema.
			\item \cdtRef{proyectoEntidad:nombreProyecto}{Nombre del proyecto}: Lo obtiene el sistema.
			\item \cdtRef{moduloEntidad:claveModulo}{Clave del Módulo}: Lo obtiene el sistema.
			\item \cdtRef{moduloEntidad:nombreModulo}{Nombre del Módulo}: Lo obtiene el sistema.
			\item \cdtRef{casoUso}{Casos de uso}: Tabla que muestra \cdtRef{casoUso:claveCU}{clave} y \cdtRef{casoUso:nombreCU}{nombre} de todos los casos de uso registrados de un proyecto.
			\item \cdtIdRef{MSG2}{No existe información}: Se muestra en la pantalla \IUref{IU6}{Gestionar Casos de uso} cuando no existen casos de uso registradas.
	\end{itemize}}
	\UCitem{Precondiciones}{Que existe al menos un proyecto registrado.}
	\UCitem{Postcondiciones}{Ninguna}
	\UCitem{Errores}{Ninguno}
	\UCitem{Tipo}{Primario}
\end{UseCase}
%--------------------------------------
\begin{UCtrayectoria}
	\UCpaso[\UCactor] Solicita gestionar los casos de uso presionando el botón \UCsist de algún módulo de la pantalla \IUref{IU4}{Gestionar Módulos}
	\UCpaso[\UCsist] Obtiene la información de todos los casos de uso registrados en cualquier estado del módulo seleccionado. \hyperlink{CU12:TAA}{[Trayectoria A]}
	\UCpaso[\UCsist] Ordena los casos de uso alfabéticamente con base en la clave de los mismos.
	\UCpaso[\UCsist] Muestra la información de los casos de uso en la pantalla \IUref{IU6}{Gestionar Casos de uso} y las operaciones disponibles de acuerdo a la regla de negocio \BRref{RN15}{Operaciones disponibles}.\label{CU12-P4}
	\UCpaso[\UCactor] Gestiona los casos de uso a través de los botones: \IUbutton{Registrar}, \raisebox{-1mm}{\includegraphics[height=11pt]{images/Iconos/consultar}}, \editar, \raisebox{-1mm}{\includegraphics[height=11pt]{images/Iconos/tray}}, \raisebox{-1mm}{\includegraphics[height=11pt]{images/Iconos/talt}}, \raisebox{-1mm}{\includegraphics[height=11pt]{images/Iconos/revisar}} \raisebox{-1mm}{\includegraphics[height=11pt]{images/Iconos/terminar}}, \eliminar . 
\end{UCtrayectoria}		
%--------------------------------------
\hypertarget{CU12:TAA}{\textbf{Trayectoria alternativa A}}\\
\noindent \textbf{Condición:} No existen registros de casos de uso
\begin{enumerate}
	\UCpaso[\UCsist] Muestra el mensaje \cdtIdRef{MSG2}{No existe información} en la pantalla \IUref{IU6A}{Gestionar Casos de uso} para indicar que no hay registros de casos de uso para mostrar. \label{CU12-TA1}
	\UCpaso[\UCactor] Gestiona los casos de uso a través del botón: \IUbutton{Registrar}. 
	\item[- -] - - {\em {Fin del caso de uso}}.%
\end{enumerate}
%--------------------------------------

\subsubsection{Puntos de extensión}

\UCExtenssionPoint{El actor requiere registrar un caso de uso.}{Presionando el botón \IUbutton{Registrar} del paso \ref{CU12-P4} de la trayectoria principal o del paso \ref{CU12-TA1} de la trayectoria alternativa A.}{\UCref{CU12.1}{Registrar Caso de uso}}
\UCExtenssionPoint{El actor requiere modificar un casos de uso.}{Presionando el icono \editar del paso \ref{CU12-P4} de la trayectoria principal.}{\UCref{CU12.2}{Modificar Caso de uso}}
\UCExtenssionPoint{El actor requiere gestionar las trayectorias de un caso de uso.}{Presionando el icono \raisebox{-1mm}{\includegraphics[height=11pt]{images/Iconos/tray}} del paso \ref{CU12-P4} de la trayectoria principal.}{\UCref{CU12.1.1}{Gestionar Trayectorias}}
\UCExtenssionPoint{El actor requiere gestionar los puntos de extensión de un caso de uso.}{Presionando el icono \raisebox{-1mm}{\includegraphics[height=11pt]{images/Iconos/talt}} del paso \ref{CU12-P4} de la trayectoria principal.}{\UCref{CU12.1.4}{Gestionar Puntos de extensión}}
\UCExtenssionPoint{El actor requiere eliminar un caso de uso.}{Presionando el icono \eliminar del paso \ref{CU12-P4} de la trayectoria principal.}{\UCref{CU12.3}{Eliminar Caso de uso}}
\UCExtenssionPoint{El actor requiere consultar un actor.}{Presionando el icono \raisebox{-1mm}{\includegraphics[height=11pt]{images/Iconos/consultar}} del paso \ref{CU12-P4} de la trayectoria principal.}{\UCref{CU12.4}{Consultar Caso de uso}}
\UCExtenssionPoint{El actor requiere revisar un caso de uso.}{Presionando el icono \raisebox{-1mm}{\includegraphics[height=11pt]{images/Iconos/revisar}} del paso \ref{CU12-P4} de la trayectoria principal.}{\UCref{CU12.5}{Revisar Caso de uso}}
\UCExtenssionPoint{El actor requiere terminar un caso de uso.}{Presionando el icono \raisebox{-1mm}{\includegraphics[height=11pt]{images/Iconos/terminar}} del paso \ref{CU12-P4} de la trayectoria principal.}{\UCref{CU12.6}{Teminar Caso de uso}}
\UCExtenssionPoint{El actor requiere gestionar las precondiciones.}{Presionando el icono (Pendiente) del paso \ref{CU12-P4} de la trayectoria principal.}{\UCref{CU12.1.2}{Gestionar Precondiciones}}
\UCExtenssionPoint{El actor requiere terminar las postcondiciones.}{Presionando el icono (Pendiente) del paso \ref{CU12-P4} de la trayectoria principal.}{\UCref{CU12.1.3}{Gestionar Postcondiciones}}