	\begin{UseCase}{CU12.1.5}{Eliminar Postcondición}{
		Este caso de uso permite al actor eliminar un registro de la tabla de postcondiciones perteneciente a un trayectoria de un caso de uso.
	}
	\UCitem{Actor}{\hyperlink{jefe}{Líder de Análisis}, \hyperlink{analista}{Analista}}
	\UCitem{Propósito}{Eliminar una postcondición de un caso de uso.}
	\UCitem{Entradas}{Ninguna.}
	\UCitem{Salidas}{
			\cdtIdRef{MSG10}{Confirmar eliminación}: Se muestra en la pantalla \IUref{IU6.1}{Registrar Caso de uso} o \IUref{IU6.2}{Modificar Caso de uso} preguntando al actor si desea continuar con la eliminación de la Postcondición.
	}
	\UCitem{Precondiciones}{Ninguna}
	\UCitem{Postcondiciones}{
			Se eliminará la postcondición de la tabla.
	}
	\UCitem{Errores}{Ninguno.}
	\UCitem{Tipo}{Secundario, extiende del caso de uso \UCref{CU12.1}{Registrar Caso de uso} y \UCref{CU12.2}{Modificar Caso de uso}.}
\end{UseCase}
%--------------------------------------
\begin{UCtrayectoria}
	\UCpaso[\UCactor] Da clic en el icono \eliminar del registro que desea eliminar de la tabla de postcondiciones de la pantalla \IUref{IU6.1}{Registrar Caso de uso} o \IUref{IU6.2}{Modificar Caso de uso}.
	\UCpaso[\UCsist] Muestra el mensaje emergente \cdtIdRef{MSG10}{Confirmar eliminación} con los botones \IUbutton{Aceptar} y \IUbutton{Cancelar} en la pantalla \IUref{IU6.1}{Registrar Caso de uso} o \IUref{IU6.2}{Modificar Caso de uso}.
	\UCpaso[\UCactor] Oprime el botón \IUbutton{Aceptar}. \hyperlink{CU12-1-5:TAA}{[Trayectoria A]}
	\UCpaso[\UCsist] Elimina la postcondición de la tabla correspondiente.
\end{UCtrayectoria}		
%-------------------------------------
\hypertarget{CU12-1-5:TAA}{\textbf{Trayectoria alternativa A}}\\
\noindent \textbf{Condición:} El actor desea cancelar la operación.
\begin{enumerate}
	\UCpaso[\UCactor] Solicita cancelar la operación oprimiendo el botón \IUbutton{Cancelar} de la pantalla .
	\UCpaso[\UCsist] Muestra la pantalla \IUref{IU6.1}{Registrar Caso de uso} o \IUref{IU6.2}{Modificar Caso de uso}.
	\item[- -] - - {\em {Fin del caso de uso}}.%
\end{enumerate}
