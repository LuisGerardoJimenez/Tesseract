	\begin{UseCase}{CU12.6}{Terminar Caso de uso}{
			
			Cuando el colaborador (\hyperlink{jefe}{Líder} o \hyperlink{analista}{Analista}) considere que ha terminado por completo el registro de un \hyperlink{casoUso}{Caso de Uso}, podrá concluir la edición del mismo en el sistema para que pueda ser revisado y finalmente liberado.
			Una vez que el colaborador termine el caso de uso ya no se encontrarán disponibles las acciones que involucren su edición ni la de sus elementos, tampoco podrá eliminarlo, únicamente estarán disponibles las acciones para consultarlo y revisarlo \\
		
			Un Caso de uso podrá ser terminado siempre y cuando su estado sea Edición o Pendiente de corrección con base en el \hyperlink{edoCU}{Modelo de estados del Caso de Uso}. 
			
	}
		\UCitem{Actor}{\hyperlink{jefe}{Líder de Análisis}, \hyperlink{analista}{Analista}}
		\UCitem{Propósito}{Terminar la edición de un caso de uso para solicitar su revisión.}
		\UCitem{Entradas}{Ninguna}
		\UCitem{Salidas}{
			\begin{itemize}
				\item \cdtIdRef{MSG1}{Operación exitosa}: Se muestra en la pantalla \IUref{IU6}{Gestionar Casos de uso} para indicar que la operación fue exitosa.
				\item \cdtIdRef{MSG27}{Confirmación de termino}: Se muestra en la pantalla \IUref{IU6}{Gestionar Casos de uso} para que el actor confirme el término del caso de uso.
		\end{itemize}}
		\UCitem{Precondiciones}{
				Que el caso de uso se encuentre en estado ''Edición'' o ''Pendiente de corrección''.
		}
		\UCitem{Postcondiciones}{
				El caso de uso pasará a estado ''Revisión''.
		}
		\UCitem{Errores}{Ninguno}
		\UCitem{Tipo}{Secundario, extiende del caso de uso \UCref{CU12}{Gestionar Casos de uso}.}
	\end{UseCase}
%--------------------------------------
	\begin{UCtrayectoria}
		\UCpaso[\UCactor] Da clic en el icono \raisebox{-1mm}{\includegraphics[height=11pt]{images/Iconos/terminar}} del registro que desea realizar la operación de la pantalla \IUref{IU6}{Gestionar Casos de uso}.
		\UCpaso[\UCsist] Verifica que el estado del caso de uso sea ''Edición'' o ''Pendiente de corrección''. \hyperlink{CU12-5:TAA}{[Trayectoria A]}
		\UCpaso[\UCsist] Muestra el mensaje emergente \cdtIdRef{MSG27}{Confirmación de termino} con los botones \IUbutton{Aceptar} y \IUbutton{Cancelar} en la pantalla \IUref{IU6}{Gestionar Casos de uso}.
		\UCpaso[\UCactor] Oprime el botón \IUbutton{Aceptar}. \hyperlink{CU12-5:TAB}{[Trayectoria B]}
		\UCpaso[\UCsist] Cambia el estado del caso de uso a ''Revisión''.
		\UCpaso[\UCactor] Se muestra el mensaje \cdtIdRef{MSG1}{Operación exitosa} en la pantalla \IUref{IU6}{Gestionar Casos de uso}.
	\end{UCtrayectoria}		
%--------------------------------------
\hypertarget{CU12-5:TAA}{\textbf{Trayectoria alternativa A}}\\
\noindent \textbf{Condición:} El caso de uso que se desea terminar se encuentra en un estado diferente a ''Edición'' o ''Pendiente de corrección''.
\begin{enumerate}
	\UCpaso[\UCsist] Oculta el botón \raisebox{-1mm}{\includegraphics[height=11pt]{images/Iconos/terminar}} del caso que no se encuentra en estado de ''Edición'' o ''Pendiente de corrección''.
	\item[- -] - - {\em {Fin del caso de uso}}.
\end{enumerate}
%--------------------------------------
\hypertarget{CU12-5:TAB}{\textbf{Trayectoria alternativa B}}\\
\noindent \textbf{Condición:} El actor desea cancelar la operación.
\begin{enumerate}
	\UCpaso[\UCactor] Oprime el botón \IUbutton{Cancelar} de la pantalla emergente.
	\UCpaso[\UCsist] Muestra la pantalla \IUref{IU6}{Gestionar Casos de Uso}.
	\item[- -] - - {\em {Fin del caso de uso}}.%
\end{enumerate}
