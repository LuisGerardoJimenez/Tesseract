	\begin{UseCase}{CU12.5}{Revisar Caso de uso}{
		Este caso de uso permite al analista revisar la información de un caso de uso para su posterior liberación o corrección.
	}
		\UCitem{Versión}{\color{Gray}0.1}
		\UCitem{Actor}{\hyperlink{jefe}{Líder de Análisis}, \hyperlink{analista}{Analista}}
		\UCitem{Propósito}{Revisar que la información del caso de uso sea correcta.}
		\UCitem{Entradas}{
			\begin{itemize}
				\item ¿El resumen del caso de uso es correcto?
				\item \cdtRef{entidadRevision:seccionRelRevision}{Observaciones del resumen}: Se escribe desde el teclado.
				\item ¿Las trayectorias del caso de uso son correctas?
				\item \cdtRef{entidadRevision:seccionRelRevision}{Observaciones de las trayectorias}: Se escribe desde el teclado.
				\item ¿Los puntos de extensión son correctos?
				\item \cdtRef{entidadRevision:seccionRelRevision}{Observaciones de los puntos de extensión}: Se escribe desde el teclado.
			\end{itemize}
		}
		\UCitem{Salidas}{
			\begin{itemize}
				\item \cdtRef{proyectoEntidad:claveProyecto}{Clave del proyecto}: Lo obtiene el sistema.
				\item \cdtRef{proyectoEntidad:nombreProyecto}{Nombre del proyecto}: Lo obtiene el sistema.
				\item \cdtRef{moduloEntidad:claveModulo}{Clave del Módulo}: Lo obtiene el sistema.
				\item \cdtRef{moduloEntidad:nombreModulo}{Nombre del Módulo}: Lo obtiene el sistema.
				\item \cdtRef{casoUso:numeroCU}{Número del caso de uso}: Lo obtiene el sistema.
				\item \cdtRef{casoUso:nombreCU}{Nombre del caso de uso}: Lo obtiene el sistema.
				\item \cdtRef{casoUso:resumenCU}{Resumen del caso de uso}: Lo obtiene el sistema.
				\item Estado: Lo obtiene el sistema.
				\item De la sección \textbf{Información general del Caso de uso}: 
					\begin{itemize}
						\item \cdtRef{actorEntidad}{Actores}: Lo obtiene el sistema.
						\item Entradas: Lo obtiene el sistema.
						\item Salidas: Lo obtiene el sistema.
						\item \cdtRef{BREntidad}{Reglas de Negocio}: Lo obtiene el sistema.
						\item \cdtRef{entidadPrecondicion:redaccionPrecondicion}{Redacción de cada Precondición}: Lo obtiene el sistema.
						\item \cdtRef{entidadPostcondicion:redaccionPostcondicion}{Redacción de cada Precondición}: Lo obtiene el sistema.
					\end{itemize}
				\item De la sección \textbf{Trayectorias}:
					\begin{itemize}
						\item Se muestra la siguiente información para cada \cdtRef{entidadTray}{Trayectoria}:
							\begin{itemize}
								\item \cdtRef{entidadTray:nombreTray}{Clave}: Lo obtiene el sistema.
								\item \cdtRef{entidadTray:alternativaTray}{Tipo:} Lo obtiene el sistema.
								\item \cdtRef{entidadTray:condicionTray}{Condición:} Lo obtiene el sistema.
								\item Se muestra la siguiente información para cada \cdtRef{entidadPaso}{Paso}:
									\begin{itemize}
										\item \cdtRef{entidadPaso:numeroPaso}{Número del paso}: Lo obtiene el sistema.
										\item \cdtRef{entidadPaso:realizaPaso}{Quien realiza el paso}: Lo obtiene el sistema.
										\item \cdtRef{entidadPaso:redaccionPaso}{Redacción del paso:} Lo obtiene el sistema.
									\end{itemize}
								\item \cdtRef{entidadTray:finTray}{Fin del caso de uso:} Lo obtiene el sistema.
							\end{itemize}
						\item De la sección \textbf{Puntos de extensión}:
							\begin{itemize}
								\item Se muestra la siguiente información para cada \cdtRef{entidadExtension}{Punto de extensión}:
									\begin{itemize}
										\item \cdtRef{entidadExtension:observacionRevision}{Causa}: Lo obtiene el sistema.
										\item Región de la trayectoria: Lo obtiene el sistema.
										\item Caso de uso al que extiende: Lo obtiene el sistema.
									\end{itemize}
							\end{itemize}
					\end{itemize}
				\item \cdtIdRef{MSG1}{Operación exitosa}: Se muestra en la pantalla \IUref{IU6}{Gestionar Casos de uso} para indicar que la revisión se ha realizado exitosamente.
		\end{itemize}}
		\UCitem{Precondiciones}{
			\begin{itemize}
				\item Que el caso de uso se encuentre en estado ''Revisión".
				\item Que el actor haya iniciado sesión como analista.
			\end{itemize}
		}
		\UCitem{Postcondiciones}{
				El caso de uso pasará a estado ''Pendiente de corrección'', ''Liberado'' o ''Por liberar''.
		}
		\UCitem{Errores}{\begin{itemize}
				\item \cdtIdRef{MSG4}{Dato obligatorio}: Se muestra en la pantalla \IUref{IU6.4}{Revisar Caso de uso} cuando no se ha ingresado un dato marcado como obligatorio.
				\item \cdtIdRef{MSG6}{Longitud inválida}: Se muestra en la pantalla \IUref{IU6.4}{Revisar Caso de uso} cuando se ha excedido la longitud de alguno de los campos.
				\item \cdtIdRef{MSG29}{Formato incorrecto}: Se muestra en la pantalla \IUref{IU6.4}{Revisar Caso de uso} cuando el tipo de dato ingresado no cumple con el tipo de dato solicitado en el campo.
				\item \cdtIdRef{MSG12}{Ha ocurrido un error}: Se muestra en la pantalla \IUref{IU6}{Gestionar Casos de uso} cuando el actor que se desea consultar no existe.
		\end{itemize}
		}
		\UCitem{Tipo}{Secundario, extiende del caso de uso \UCref{CU12}{Gestionar Casos de uso}.}
	\end{UseCase}
%--------------------------------------
	\begin{UCtrayectoria}
		\UCpaso[\UCactor] Da clic en el icono \raisebox{-1mm}{\includegraphics[height=11pt]{images/Iconos/revisar}} del registro que desea revisar de la pantalla \IUref{IU6}{Gestionar Casos de uso}.
		\UCpaso[\UCsist] Obtiene la información del caso de uso seleccionado. 
		\UCpaso[\UCsist] Verifica que el estado del caso de uso sea ''Revisión''. \hyperlink{CU12-5:TAA}{[Trayectoria A]}
		\UCpaso[\UCsist] Muestra la pantalla \IUref{IU6.4}{Revisar Caso de uso}.
		\UCpaso[\UCactor] Selecciona la opción ''Sí'' para cada una de las secciones. \hyperlink{CU12-5:TAB}{[Trayectoria B]} \label{CU12.5-P5}
		\UCpaso[\UCactor] Oprime el botón \IUbutton{Aceptar}. \label{CU12.5-P6}
		\UCpaso[\UCsist] Verifica que el actor ingrese todos los campos obligatorios con base en la regla de negocio \BRref{RN8}{Datos obligatorios}. \hyperlink{CU12-5:TAC}{[Trayectoria C]}
		\UCpaso[\UCsist] Verifica que el actor en sesión sea ''Analista''. \hyperlink{CU12-5:TAF}{[Trayectoria F]}
		\UCpaso[\UCsist] Cambia el estado del caso de uso a ''Por liberar''.
		\UCpaso[\UCsist] Se muestra el mensaje \cdtIdRef{MSG1}{Operación exitosa} en la pantalla \IUref{IU6}{Gestionar Casos de uso}. \label{CU12.5-P12}
	\end{UCtrayectoria}		
%--------------------------------------
\hypertarget{CU12-5:TAA}{\textbf{Trayectoria alternativa A}}\\
\noindent \textbf{Condición:} El caso de uso que se desea revisar se encuentra en un estado diferente a ''Revisión''.
\begin{enumerate}
	\UCpaso[\UCsist] Oculta el botón \raisebox{-1mm}{\includegraphics[height=11pt]{images/Iconos/revisar}} del caso que no se encuentra en estado de ''Revisión''.
	\item[- -] - - {\em {Fin del caso de uso}}.
\end{enumerate}
%--------------------------------------
\hypertarget{CU12-5:TAB}{\textbf{Trayectoria alternativa B}}\\
\noindent \textbf{Condición:} El actor marcó como incorrecta alguna de las secciones del caso de uso.
\begin{enumerate}
	\UCpaso[\UCsist] Muestra el campo correspondiente a las observaciones de aquellas secciones en las que se hayan marcado como incorrecta.
	\UCpaso[\UCactor] Ingresa las observaciones en los campos. \label{CU12.5-TAP2}
	\UCpaso[\UCsist] Verifica que el actor ingrese todos los campos obligatorios con base en la regla de negocio \BRref{RN8}{Datos obligatorios}. \hyperlink{CU12-5:TAC}{[Trayectoria C]}
	\UCpaso[\UCsist] Verifica que los datos ingresados cumpla con la longitud correcta, con base en la regla de negocio \BRref{RN37}{Longitud de datos}. \hyperlink{CU12-5:TAD}{[Trayectoria D]}
	\UCpaso[\UCsist] Verifica que los datos ingresados cumplan con el formato requerido, con base en la regla de negocio \BRref{RN7}{Información correcta}. \hyperlink{CU12-5:TAE}{[Trayectoria E]}
	\UCpaso[\UCsist] Almacena las observaciones realizadas a cada sección marcada como incorrecta.
	\UCpaso[\UCsist] Cambia el estado del caso de uso a ''Pendiente de corrección''.
	\UCpaso  Continúa en el paso \ref{CU12.5-P12} de la Trayectoria principal.
	\item[- -] - - {\em {Fin de la trayectoria}}.%
\end{enumerate}
%--------------------------------------
\hypertarget{CU12-5:TAC}{\textbf{Trayectoria alternativa C}}\\
\noindent \textbf{Condición:} El actor no ingresó algún dato marcado como obligatorio.
\begin{enumerate}
	\UCpaso[\UCsist] Muestra el mensaje \cdtIdRef{MSG4}{Dato obligatorio} señalando el campo que presenta el error en la pantalla \IUref{IU6.4}{Revisar Caso de uso}.
	\UCpaso Regresa al paso \ref{CU12.5-P5} de la Trayectoria principal o al paso \ref{CU12.5-TAP2} de la Trayectoria alternativa B.
	\item[- -] - - {\em {Fin de la trayectoria}}.%
\end{enumerate}
%--------------------------------------
\hypertarget{CU12-5:TAD}{\textbf{Trayectoria alternativa D}}\\
\noindent \textbf{Condición:} El actor ingresó un dato con un número de caracteres fuera del rango permitido.
\begin{enumerate}
	\UCpaso[\UCsist] Muestra el mensaje \cdtIdRef{MSG6}{Longitud inválida} señalando el campo que presenta el error en la pantalla \IUref{IU6.4}{Revisar Caso de uso}.
	\UCpaso Regresa al paso \ref{CU12.5-P5} de la Trayectoria principal o al paso \ref{CU12.5-TAP2} de la Trayectoria alternativa B.
	\item[- -] - - {\em {Fin de la trayectoria}}.%
\end{enumerate}
%--------------------------------------
\hypertarget{CU12-5:TAE}{\textbf{Trayectoria alternativa E}}\\
\noindent \textbf{Condición:} El actor ingresó un dato con un formato incorrecto.
\begin{enumerate}
	\UCpaso[\UCsist] Muestra el mensaje \cdtIdRef{MSG29}{Formato incorrecto} señalando el campo que presenta el error en la pantalla \IUref{IU6.4}{Revisar Caso de uso}.
	\UCpaso Regresa al paso \ref{CU12.5-P5} de la Trayectoria principal o al paso \ref{CU12.5-TAP2} de la Trayectoria alternativa B.
	\item[- -] - - {\em {Fin de la trayectoria}}.
\end{enumerate}
%--------------------------------------
\hypertarget{CU12-5:TAF}{\textbf{Trayectoria alternativa F}}\\
\noindent \textbf{Condición:} El actor en sesión es un ''Líder de Proyecto''.
\begin{enumerate}
	\UCpaso[\UCsist] Cambia el estado del caso de uso a ''Liberado''.
	\UCpaso Continúa en el paso \ref{CU12.5-P12} de la Trayectoria principal.
	\item[- -] - - {\em {Fin de la trayectoria}}.
\end{enumerate}