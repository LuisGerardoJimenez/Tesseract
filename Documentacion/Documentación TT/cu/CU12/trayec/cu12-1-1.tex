	\begin{UseCase}{CU12.1.1}{Gestionar Trayectorias}{
			
				Una vez que el colaborador (\hyperlink{jefe}{Líder de Análisis} o \hyperlink{analista}{Analista}) ha registrado un caso de uso en algún \hyperlink{moduloEntidad}{Módulo} del \hyperlink{proyectoEntidad}{Proyecto}, podrá acceder a las acciones que corresponden a la Gestión de cada \hyperlink{casoUso}{Caso de Uso} como la gestión de Trayectorias.  Cada Caso de Uso contiene la gestión de sus respectivas Trayectorias.\\
				
				Las trayectorias describen los escenarios ideales y alternos de un sistema mediante una serie de pasos. Este caso de uso permite al colaborador visualizar en una tabla, el registro de la trayectoria principal y las trayectorias alternativas del caso de uso sobre el cual se está operando, del mismo modo le da la posibilidad de solicitar el registro de nuevas trayectorias (en donde también es posible gestionar los pasos involucrados), modificar la información de alguna de las trayectorias existentes (teniendo la posibilidad de actualizar algún dato de registro)o simplemente eliminarlas (en caso de que la trayectoria por algún motivo ya no tenga razón de ser dentro del Caso de uso). \\
				
	}
	\UCitem{Actor}{\hyperlink{jefe}{Líder de análisis}, \hyperlink{analista}{Analista}}
	\UCitem{Propósito}{Proporcionar al actor un mecanismo para llevar el control de las trayectorias de un caso de uso.}
	\UCitem{Entradas}{Ninguna}
	\UCitem{Salidas}{\begin{itemize}
			\item \cdtRef{proyectoEntidad:claveProyecto}{Clave del proyecto}: Lo obtiene el sistema.
			\item \cdtRef{proyectoEntidad:nombreProyecto}{Nombre del proyecto}: Lo obtiene el sistema.
			\item \cdtRef{moduloEntidad:claveModulo}{Clave del Módulo}: Lo obtiene el sistema.
			\item \cdtRef{moduloEntidad:nombreModulo}{Nombre del Módulo}: Lo obtiene el sistema.
			\item \cdtRef{casoUso:numeroCU}{Número} del caso de uso: Lo obtiene el sistema. 
			\item \cdtRef{casoUso:nombreCU}{Nombre} del caso de uso: Lo obtiene el sistema.
			\item \cdtRef{entidadTray}{Trayectoria}: Tabla que muestra \cdtRef{entidadTray:nombreTray}{nombre} y \cdtRef{entidadTray:condicionTray}{condición} de todos los registros de las trayectorias del caso de uso.
			\item \cdtIdRef{MSG2}{No existe información}: Se muestra en la pantalla \IUref{IU6.1.1}{Gestionar Trayectorias} cuando no existen trayectorias registradas.
	\end{itemize}}
	\UCitem{Precondiciones}{
		\begin{itemize}
			\item Que el caso de uso se encuentre en estado ''Edición'' o ''Pendiente de corrección''.
			\item Que exista al menos un trayectoria registrada.
		\end{itemize}
	}
	\UCitem{Postcondiciones}{Ninguna}
	\UCitem{Errores}{Ninguno}
	\UCitem{Tipo}{Secundario, extiende del caso de uso \UCref{CU12}{Gestionar Casos de uso}.}
\end{UseCase}
%--------------------------------------
\begin{UCtrayectoria}
	\UCpaso[\UCactor] Solicita gestionar las trayectorias de un caso de uso presionando el botón \raisebox{-1mm}{\includegraphics[height=11pt]{images/Iconos/tray}} del caso de uso que desee de la pantalla \IUref{IU6}{Gestionar Casos de uso}
	\UCpaso[\UCsist] Obtiene la información de las trayectorias del caso de uso. \hyperlink{CU12-1-1:TAA}{[Trayectoria A]}
	\UCpaso[\UCsist] Ordena las trayectorias alfabéticamente con base en la clave de las mismas.
	\UCpaso[\UCsist] Verifica que el caso de uso se encuentre en estado ''Edición'' o en estado ''Pendiente de corrección''.\hyperlink{CU12-1-1:TAB}{[Trayectoria B]}
	\UCpaso[\UCsist] Muestra la información de las trayectorias en la pantalla \IUref{IU6.1.1}{Gestionar Trayectorias}. 
	\UCpaso[\UCactor] Gestiona las trayectorias a través de los botones: \IUbutton{Registrar}, \editar, y \eliminar. \label{CU12.1.1-P5}
\end{UCtrayectoria}		
%--------------------------------------
\hypertarget{CU12-1-1:TAA}{\textbf{Trayectoria alternativa A}}\\
\noindent \textbf{Condición:} No existen registros de trayectorias.
\begin{enumerate}
	\UCpaso[\UCsist] Muestra el mensaje \cdtIdRef{MSG2}{No existe información} en la pantalla \IUref{IU6.1.1}{Gestionar Trayectorias} para indicar que no hay registros de trayectorias para mostrar. \label{CU12-1-1-TA1}
	\UCpaso[\UCactor] Gestiona las trayectorias a través del botón: \IUbutton{Registrar}. 
	\item[- -] - - {\em {Fin del caso de uso}}.%
\end{enumerate}
%--------------------------------------
\hypertarget{CU12-1-1:TAB}{\textbf{Trayectoria alternativa B}}\\
\noindent \textbf{Condición:} El caso de uso no se encuentra en estado ''Edición'' o ''Pendiente de corrección''.
\begin{enumerate}
	\UCpaso[\UCsist]  Oculta el botón \raisebox{-1mm}{\includegraphics[height=11pt]{images/Iconos/tray}} del caso que no se encuentra en estado de ''Edición'' o ''Pendiente de corrección''.
	\item[- -] - - {\em {Fin del caso de uso}}.
\end{enumerate}
%--------------------------------------

\subsubsection{Puntos de extensión}

\UCExtenssionPoint{El actor requiere registrar una trayectoria.}{Paso \ref{CU12.1.1-P5} de la trayectoria principal o del paso \ref{CU12-1-1-TA1} de la Trayectoria alternativa A.}{\UCref{CU12.1.1.1}{Registrar Trayectoria}}
\UCExtenssionPoint{El actor requiere modificar una trayectoria.}{Paso \ref{CU12.1.1-P5} de la trayectoria principal.}{\UCref{CU12.1.1.2}{Modificar Trayectoria}}
\UCExtenssionPoint{El actor requiere eliminar una trayectoria.}{Paso \ref{CU12.1.1-P5} de la trayectoria principal.}{\UCref{CU12.1.1.3}{Eliminar Trayectoria}}
