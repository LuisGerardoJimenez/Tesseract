	\begin{UseCase}{CU12.1.4.2}{Modificar Punto de extensión}{
		Los puntos de extensión describen una región de la trayectoria en la que se puede extender el funcionamiento a través de otro caso de uso. Este caso de uso permite al analista modificar un punto de extensión.
		
		Después de haber registrado un \hyperlink{entidadExtension}{Punto de Extensión} en un \hyperlink{casoUso}{Caso de Uso} del \hyperlink{moduloEntidad}{Módulo} y el colaborador (\hyperlink{jefe}{Líder de Análisis} o \hyperlink{analista}{Analista}) requiera modificar su contenido, el sistema le permitirá editar cualquiera de los datos previamente registrados mediante un formulario, este formulario contendrá los datos precargados de la última actualización para poder corregirlos y posteriormente guardarlos.\\
		
	}
		\UCitem{Actor}{\hyperlink{jefe}{Líder de Análisis}, \hyperlink{analista}{Analista}}
		\UCitem{Propósito}{Modificar los puntos de extensión de un caso de uso.}
		\UCitem{Entradas}{
		\begin{itemize}
			\item \cdtRef{entidadExtension:observacionRevision}{Causa}: Se escribe desde el teclado.
			\item Regiónde la trayectoria: Se escribe desde el teclado.
			\item Caso de uso al que extiende: Se escribe desde el teclado.
		\end{itemize}	
		}
		\UCitem{Salidas}{\begin{itemize}
				\item \cdtRef{proyectoEntidad:claveProyecto}{Clave del proyecto}: Lo obtiene el sistema.
				\item \cdtRef{proyectoEntidad:nombreProyecto}{Nombre del proyecto}: Lo obtiene el sistema.
				\item \cdtRef{moduloEntidad:claveModulo}{Clave del Módulo}: Lo obtiene el sistema.
				\item \cdtRef{moduloEntidad:nombreModulo}{Nombre del Módulo}: Lo obtiene el sistema.
				\item \cdtRef{moduloEntidad:nombreModulo}{Nombre del Módulo}: Lo obtiene el sistema.
				\item \cdtRef{casoUso:numeroCU}{Número} del caso de uso: Lo obtiene el sistema. 
				\item \cdtRef{casoUso:nombreCU}{Nombre} del caso de uso: Lo obtiene el sistema.
				\item \cdtRef{entidadExtension:observacionRevision}{Causa}: Lo obtiene el sistema.
				\item Región de la trayectoria: Lo obtiene el sistema.
				\item Caso de uso al que extiende: Lo obtiene el sistema.
				\item \cdtIdRef{MSG1}{Operación exitosa}: Se muestra en la pantalla \IUref{IU6.1.4}{Gestionar Puntos de extensión} para indicar que el registro fue exitoso.
		\end{itemize}}
		\UCitem{Precondiciones}{
			\begin{itemize}
				\item Que existan al menos 2 casos de uso registrados.
				\item Que el caso de uso al que pertenece el punto de extensión se encuentre en estado ''Edición'' o ''Pendiente de corrección''.
				\item Que el catálogo ''Extiende a'' tenga información.
			\end{itemize}
		}
		\UCitem{Postcondiciones}{
			Se actualizará un punto de extensión para un caso de uso.
		}
		\UCitem{Errores}{\begin{itemize}
		\item \cdtIdRef{MSG4}{Dato obligatorio}: Se muestra en la pantalla \IUref{IU6.1.4.2}{Modificar Punto de extensión} cuando no se ha ingresado un dato marcado como obligatorio.
		\item \cdtIdRef{MSG5}{Formato incorrecto}: Se muestra en la pantalla \IUref{IU6.1.4.2}{Modificar Punto de extensión} cuando el tipo de dato ingresado no cumple con el tipo de dato solicitado en el campo.
		\item \cdtIdRef{MSG6}{Longitud inválida}: Se muestra en la pantalla \IUref{IU6.1.4.2}{Modificar Punto de extensión} cuando se ha excedido la longitud de alguno de los campos.
		\item \cdtIdRef{MSG15}{Falta información}: Se muestra en la pantalla \IUref{IU6.1.4}{Gestionar Puntos de extensión} cuando no existan casos de uso registrados.
		\item \cdtIdRef{MSG7}{Registro repetido}: Se muestra en la pantalla \IUref{IU6.1.4.2}{Modificar Punto de extensión} cuando se registre un punto de extensión con un caso de uso que ya se encuentre registrado en el sistema.
		\end{itemize}.
		}
		\UCitem{Tipo}{Secundario, extiende del caso de uso \UCref{CU12.1.6}{Gestionar Puntos de extensión}.}
	\end{UseCase}
%--------------------------------------
	\begin{UCtrayectoria}
		\UCpaso[\UCactor] Da clic en el icono \editar de la pantalla \IUref{IU6.1.4}{Gestionar Puntos de extensión}.
		\UCpaso[\UCsist] Obtiene la información del punto de extensión.
		\UCpaso[\UCsist] Verifica que el caso de uso se encuentre en estado ''Edición'' o ''Pendiente de corrección''. \hyperlink{CU12-1-6-2:TAH}{[Trayectoria H]}
		\UCpaso[\UCsist] Verifica que exista al menos un caso de uso, con base en la regla de negocio \BRref{RN20}{Verificación de catálogos}. \hyperlink{CU12-1-6-2:TAA}{[Trayectoria A]}
		\UCpaso[\UCsist] Muestra la pantalla \IUref{IU6.1.4.2}{Modificar Punto de extensión}.
		\UCpaso[\UCactor] Modifica el caso de uso que extiende. 
		\UCpaso[\UCsist] Modifica la información del punto de extensión. \hyperlink{CU12-1-6-2:TAB}{[Trayectoria B]} \label{CU12.1.6.2-P4}
		\UCpaso[\UCactor] Oprime el botón \IUbutton{Aceptar}. \hyperlink{CU12-1-6-2:TAC}{[Trayectoria C]} \label{CU12.1.6.2-P6}
		\UCpaso[\UCsist] Verifica que el actor ingrese todos los campos obligatorios con base en la regla de negocio \BRref{RN8}{Datos obligatorios}. \hyperlink{CU12-1-6-2:TAD}{[Trayectoria D]}
		\UCpaso[\UCsist] Verifica que los datos ingresados cumpla con la longitud correcta, con base en la regla de negocio \BRref{RN37}{Longitud de datos}. \hyperlink{CU12-1-6-2:TAE}{[Trayectoria E]}
		\UCpaso[\UCsist] Verifica que los datos ingresados cumplan con el formato requerido, con base en la regla de negocio \BRref{RN7}{Información correcta}. \hyperlink{CU12-1-6-2:TAF}{[Trayectoria F]}
		\UCpaso[\UCsist] Verifica que el punto de extensión no se encuentre registrado en el sistema con base en la regla de negocio \BRref{RN17}{Unicidad de puntos de extensión}. \hyperlink{CU12-1-6-2:TAG}{[Trayectoria G]} 
		\UCpaso[\UCsist] Actualiza la información del punto de extensión.
		\UCpaso[\UCsist] Muestra el mensaje \cdtIdRef{MSG1}{Operación exitosa} en la pantalla \IUref{IU6.1.4}{Gestionar Puntos de extensión} para indicar al actor que la modificación se ha realizado exitosamente.
	\end{UCtrayectoria}		
%--------------------------------------
\hypertarget{CU12-1-6-2:TAA}{\textbf{Trayectoria alternativa A}}\\
\noindent \textbf{Condición:} El catálogo de ''Extiende a'' no tiene información.
\begin{enumerate}
	\UCpaso[\UCsist] Muestra el mensaje \cdtIdRef{MSG15}{Falta información} en la pantalla \IUref{IU6.1.4}{Gestionar Puntos de extensión} para indicar que no existen casos de uso registrados.
	\item[- -] - - {\em {Fin del caso de uso}}.%
\end{enumerate}
%--------------------------------------	
\hypertarget{CU12-1-6-2:TAB}{\textbf{Trayectoria alternativa B}}\\
\noindent \textbf{Condición:} El actor desea seleccionar un paso.
\begin{enumerate}
	\UCpaso[\UCactor] Ingresa el token {\em P·}. 
	\UCpaso[\UCsist] Obtiene los pasos del caso de uso.
	\UCpaso[\UCsist] Muestra una lista con los pasos encontradas.
	\UCpaso[\UCactor] Selecciona un paso de la lista.
	\UCpaso[\UCsist] Verifica que el nombre del caso de uso al que pertenece el paso no contenga espacios. \hyperlink{CU12-1-6-2:TAI}{[Trayectoria I]}
	\UCpaso[\UCsist] Agrega la clave del caso de uso al que pertenece el paso al texto, seguido del signo ''·''. \label{CU12.1.6.2-TA6}
	\UCpaso[\UCsist] Agrega el número del caso de uso al texto, seguido del signo '':''.
	\UCpaso[\UCsist] Agrega el nombre del caso de uso al texto, seguido del signo '':''.
	\UCpaso[\UCsist] Agrega la clave de la trayectoria a la que pertenece el paso al texto, seguido del signo ''·''.
	\UCpaso[\UCsist] Agrega el número del paso seleccionado al texto.
	\UCpaso Continúa en el paso \ref{CU12.1.6.2-P6} de la trayectoria principal.
	\item[- -] - - {\em {Fin de la trayectoria}}.%
\end{enumerate}
%--------------------------------------
\hypertarget{CU12-1-6-2:TAC}{\textbf{Trayectoria alternativa C}}\\
\noindent \textbf{Condición:} El actor desea cancelar la operación.
\begin{enumerate}
	\UCpaso[\UCactor] Solicita cancelar la operación oprimiendo el botón \IUbutton{Cancelar} de la pantalla \IUref{IU6.1.4.2}{Modificar Punto de extensión}.
	\UCpaso[\UCsist] Muestra la pantalla \IUref{IU6.1.4}{Gestionar Puntos de extensión}.
	\item[- -] - - {\em {Fin del caso de uso}}.%
\end{enumerate}
%--------------------------------------
\hypertarget{CU12-1-6-2:TAD}{\textbf{Trayectoria alternativa D}}\\
\noindent \textbf{Condición:} El actor no ingresó algún dato marcado como obligatorio.
\begin{enumerate}
	\UCpaso[\UCsist] Muestra el mensaje \cdtIdRef{MSG4}{Dato obligatorio} señalando el campo que presenta el error en la pantalla \IUref{IU6.1.4.2}{Modificar Punto de extensión}.
	\UCpaso Regresa al paso \ref{CU12.1.6.2-P4} de la trayectoria principal.
	\item[- -] - - {\em {Fin de la trayectoria}}.%
\end{enumerate}
%--------------------------------------
\hypertarget{CU12-1-6-2:TAE}{\textbf{Trayectoria alternativa E}}\\
\noindent \textbf{Condición:} El actor ingresó un dato con un número de caracteres fuera del rango permitido.
\begin{enumerate}
	\UCpaso[\UCsist] Muestra el mensaje \cdtIdRef{MSG6}{Longitud inválida} señalando el campo que presenta el error en la pantalla \IUref{IU6.1.4.2}{Modificar Punto de extensión}.
	\UCpaso Regresa al paso \ref{CU12.1.6.2-P4} de la trayectoria principal.
	\item[- -] - - {\em {Fin de la trayectoria}}.%
\end{enumerate}
%--------------------------------------
\hypertarget{CU12-1-6-2:TAF}{\textbf{Trayectoria alternativa F}}\\
\noindent \textbf{Condición:} El actor ingresó un dato con un formato incorrecto.
\begin{enumerate}
	\UCpaso[\UCsist] Muestra el mensaje \cdtIdRef{MSG5}{Formato incorrecto} señalando el campo que presenta el error en la pantalla \IUref{IU6.1.4.2}{Modificar Punto de extensión}.
	\UCpaso Regresa al paso \ref{CU12.1.6.2-P4} de la trayectoria principal.
	\item[- -] - - {\em {Fin de la trayectoria}}.
\end{enumerate}
%--------------------------------------
\hypertarget{CU12-1-6-2:TAG}{\textbf{Trayectoria alternativa G}}\\
\noindent \textbf{Condición:} El actor ingresó un punto de extensión que ya existe.
\begin{enumerate}
	\UCpaso[\UCsist] Muestra el mensaje \cdtIdRef{MSG7}{Registro repetido} señalando el campo que presenta la duplicidad en la pantalla \IUref{IU6.1.4.2}{Modificar Punto de extensión}.
	\UCpaso Regresa al paso \ref{CU12.1.6.2-P4} de la trayectoria principal.
	\item[- -] - - {\em {Fin de la trayectoria}}.
\end{enumerate}
%--------------------------------------
\hypertarget{CU12-1-6-2:TAH}{\textbf{Trayectoria alternativa H}}\\
\noindent \textbf{Condición:} El caso de uso no se encuentra en estado ''Edición'' o ''Pendiente de corrección''.
\begin{enumerate}
	\UCpaso[\UCsist]  Oculta el botón \raisebox{-1mm}{\includegraphics[height=11pt]{images/Iconos/talt}} del caso que no se encuentra en estado de ''Edición'' o ''Pendiente de corrección''.
	\item[- -] - - {\em {Fin del caso de uso}}.
\end{enumerate}
%--------------------------------------
\hypertarget{CU12-1-6-2:TAI}{\textbf{Trayectoria alternativa I}}\\
\noindent \textbf{Condición:} El texto contiene espacios.
\begin{enumerate}
	\UCpaso[\UCsist] Sustituye los espacios por guiones bajos.
	\UCpaso Continua en el paso \ref{CU12.1.6.2-TA6} de la trayectoria alternativa A.
	\item[- -] - - {\em {Fin de la trayectoria}}.
\end{enumerate}
