	\begin{UseCase}{CU6.4}{Consultar Término de Glosario}{
		Este caso de uso permite al colaborador (\hyperlink{jefe}{Líder} o \hyperlink{analista}{Analista}) consultar la descripción de un \hyperlink{terminoGLSEntidad}{Término de glosario}.
	}
		
		\UCitem{Actor}{\hyperlink{jefe}{Líder de Análisis}, \hyperlink{analista}{Analista}}
		\UCitem{Propósito}{Consultar la información de un término del glosario.}
		\UCitem{Entradas}{Ninguna}
		\UCitem{Salidas}{\begin{itemize}
				\item \cdtRef{proyectoEntidad:claveProyecto}{Clave del proyecto:} Lo obtiene el sistema.
				\item \cdtRef{proyectoEntidad:nombreProyecto}{Nombre del proyecto:} Lo obtiene el sistema.
				\item \cdtRef{terminoGLSEntidad:nombreTGLS}{Nombre del término:} Lo obtiene el sistema.
				\item \cdtRef{terminoGLSEntidad:descripcionTGLS}{Descripción:} Lo obtiene el sistema.
		\end{itemize}}
		
		\UCitem{Precondiciones}{Ninguna}
		\UCitem{Postcondiciones}{Ninguna}
		\UCitem{Errores}{
			\cdtIdRef{MSG12}{Ha ocurrido un error}: Se muestra en la pantalla \IUref{IU11}{Gestionar Términos del glosario} cuando el término que se desea consultar no existe.
		}
		\UCitem{Tipo}{Secundario, extiende del caso de uso \UCref{CU6}{Gestionar Términos del glosario}.}
	\end{UseCase}
%--------------------------------------
	\begin{UCtrayectoria}
		\UCpaso[\UCactor] Da clic en el icono \raisebox{-1mm}{\includegraphics[height=11pt]{images/Iconos/consultar}} de un registro de la pantalla \IUref{IU11}{Gestionar Términos de glosario} o la liga correspondiente a un término en la pantalla \IUref{IU6.3}{Consultar caso de uso}.
		\UCpaso[\UCsist] Obtiene la información del término seleccionado. \hyperlink{CU6-4:TAA}{[Trayectoria A]}
		\UCpaso[\UCsist] Muestra la pantalla \IUref{IU11.3}{Consultar Término}.
		\UCpaso[\UCactor] Consulta la información del Término.
		\UCpaso[\UCactor] Finaliza la consulta oprimiendo el botón \IUbutton{Regresar}.
		\UCpaso[\UCsist] Muestra la pantalla \IUref{IU11}{Gestionar Términos del glosario}.
	\end{UCtrayectoria}		
%--------------------------------------
\hypertarget{CU6-4:TAA}{\textbf{Trayectoria alternativa A}}\\
\noindent \textbf{Condición:} El término que se desea consultar no existe.
\begin{enumerate}
	\UCpaso[\UCsist] Muestra el mensaje \cdtIdRef{MSG12}{Ha ocurrido un error} en la pantalla \IUref{IU11}{Gestionar Términos del glosario}.
	\item[- -] - - {\em {Fin del caso de uso}}.
\end{enumerate}