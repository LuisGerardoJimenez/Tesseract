	\begin{UseCase}{CU7.1.1.3}{Eliminar Atributo}{
		Este caso de uso permite al actor eliminar un registro de la tabla de atributos de la pantalla \IUref{IU12.1}{Registrar Entidad} o \IUref{IU12.2}{Modificar Entidad}.
	}
		\UCitem{Actor}{\hyperlink{jefe}{Líder de Análisis}, \hyperlink{analista}{Analista}}
		\UCitem{Propósito}{Eliminar la información de un atributo de la tabla de atributos.}
		\UCitem{Entradas}{Ninguna}
		\UCitem{Salidas}{\begin{itemize}
				\item \cdtIdRef{MSG1}{Operación exitosa}: Se muestra en la pantalla \IUref{IU12}{Pendiente} para indicar que la entidad fue eliminada correctamente.
				\item \cdtIdRef{MSG10}{Confirmar eliminación}: Se muestra en la pantalla \IUref{IU12}{Pendiente} preguntando al actor si desea continuar con la eliminación de la entidad.
		\end{itemize}}
		
		\UCitem{Precondiciones}{\begin{itemize}
				\item Que el atributo no se encuentre asociado a un caso de uso.
				\item Que el atributo no se encuentre asociado a un caso de uso liberado.
		\end{itemize}}
		\UCitem{Postcondiciones}{Se eliminará un atributo de una entidad perteneciente a un proyecto}
		\UCitem{Errores}{\begin{itemize}
		\item \cdtIdRef{MSG13}{Eliminación no permitida}: Se muestra en la pantalla \IUref{IU12}{Pendiente} cuando no se pueda eliminar el atributo debido a que está siendo utilizado por otro elemento.
		\end{itemize}
		}
		\UCitem{Tipo}{Secundario, extiende de los casos de uso \UCref{CU7.1.1}{Gestionar Atributos}.}
	\end{UseCase}
%--------------------------------------
	\begin{UCtrayectoria}
		\UCpaso[\UCactor] Da clic en el icono \eliminar del registro que desea eliminar de la pantalla \IUref{IU12}{Pendiente}.
		\UCpaso[\UCsist] Muestra el mensaje emergente \cdtIdRef{MSG10}{Confirmar eliminación} con los botones \IUbutton{Aceptar} y \IUbutton{Cancelar} en la pantalla \IUref{IU12}{Gestionar entidades}.
		\UCpaso[\UCactor] Confirma la eliminación de la entidad oprimiendo el botón \IUbutton{Aceptar}. \hyperlink{CU7-1-1-3:TAA}{[Trayectoria A]}
		\UCpaso[\UCsist] Verifica que ningún elemento esté referenciando al atributo. \hyperlink{CU7-1-1-3:TAB}{[Trayectoria B}
		\UCpaso[\UCsist] Elimina la información referente al atributo.
		\UCpaso[\UCsist] Muestra el mensaje \cdtIdRef{MSG1}{Operación exitosa} en la pantalla \IUref{IU12}{Pendiente} para indicar al actor que el registro se ha eliminado exitosamente.
	\end{UCtrayectoria}	
%--------------------------------------
	\hypertarget{CU7-1-1-3:TAA}{\textbf{Trayectoria alternativa A}}\\
	\noindent \textbf{Condición:} El actor desea cancelar la operación.
	\begin{enumerate}
		\UCpaso[\UCactor] Oprime el botón \IUbutton{Cancelar} de la pantalla emergente.
		\UCpaso[\UCsist] Muestra la pantalla \IUref{IU12}{Pendiente}.
		\item[- -] - - {\em {Fin del caso de uso}}.%
	\end{enumerate}

	\hypertarget{CU7-1-1-3:TAB}{\textbf{Trayectoria alternativa B}}\\
	\noindent \textbf{Condición:} Algunos atributos de la entidad están siendo referenciados en algún caso de uso.
	\begin{enumerate}
		\UCpaso[\UCsist] Muestra el mensaje \cdtIdRef{MSG13}{Eliminación no permitida} en la pantalla \IUref{IU12}{Pendiente}.
		\item[- -] - - {\em {Fin del caso de uso}}.
	\end{enumerate}
