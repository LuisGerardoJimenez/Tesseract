	\begin{UseCase}{CU7}{Gestionar entidades}{
			
		Este caso de uso permite al colaborador (\hyperlink{jefe}{Líder de análisis} o \hyperlink{analista}{Analista}) visualizar en una tabla, el registro de las entidades que forman parte del proyecto, así como solicitar el registro de una nueva (en donde tambíen es posible gestionar los atributos que compondrán la entidad), modificar alguna existente (por medio de la cual también se podrán gestionar sus atributos), eliminarla (en caso de que la entidad por algún motivo ya no forme parte del proyecto) o simplemente consultar su información (descripción de la entidad y descripción de los atributos que componen la entidad). \\
     	Las acciones disponibles para cada \hyperlink{entidadEntidad}{Entidad} dependerán del estado en el que se encuentre el caso de uso donde son referenciadas. 
	}
	\UCitem{Actor}{\hyperlink{jefe}{Líder de análisis}, \hyperlink{analista}{Analista}}
	\UCitem{Propósito}{Proporcionar al actor un mecanismo para llevar el control de las entidades de un proyecto.}
	\UCitem{Entradas}{Ninguna}
	\UCitem{Salidas}{\begin{itemize}
			\item \cdtRef{proyectoEntidad:claveProyecto}{Clave del proyecto}: Lo obtiene el sistema.
			\item \cdtRef{proyectoEntidad:nombreProyecto}{Nombre del proyecto}: Lo obtiene el sistema.
			\item \cdtRef{entidadEntidad}{Entidad}: Tabla que muestra \cdtRef{entidadEntidad:nombreEntidad}{nombre} de todos las entidades registradas de un proyecto.
			\item \cdtIdRef{MSG2}{No existe información}: Se muestra en la pantalla \IUref{IU12A}{Gestionar Entidades} cuando no existen entidades registradas.
	\end{itemize}}
	
	\UCitem{Precondiciones}{Que exista al menos un proyecto registrado.}
	\UCitem{Postcondiciones}{Ninguna}
	\UCitem{Errores}{Ninguno}
	\UCitem{Tipo}{Primario}
\end{UseCase}
%--------------------------------------
\begin{UCtrayectoria}
	\UCpaso[\UCactor] Solicita gestionar las entidades seleccionando la opción ''Entidades'' del menú \IUref{MN3}{Menú de Proyecto}.
	\UCpaso[\UCsist] Obtiene la información de las entidades registradas del proyecto seleccionado. \hyperlink{CU7:TAA}{[Trayectoria A]}
	\UCpaso[\UCsist] Ordena las entidades alfabéticamente basándose en el nombres de los mismos.
	\UCpaso[\UCsist] Muestra la información de las entidades en la pantalla \IUref{IU12}{Gestionar Entidades} y las operaciones disponibles de acuerdo a la regla de negocio \BRref{RN15}{Operaciones disponibles}. \label{CU7-P4}
	\UCpaso[\UCactor] Gestiona las entidades a través de los botones: \IUbutton{Registrar}, \editar , \eliminar y \raisebox{-1mm}{\includegraphics[height=11pt]{images/Iconos/consultar}}. 
\end{UCtrayectoria}		
%--------------------------------------
\hypertarget{CU7:TAA}{\textbf{Trayectoria alternativa A}}\\
\noindent \textbf{Condición:} No existen registros de entidades.
\begin{enumerate}
	\UCpaso[\UCsist] Muestra el mensaje \cdtIdRef{MSG2}{No existe información} en la pantalla \IUref{IU12A}{Gestionar Entidades} para indicar que no hay registros de entidades para mostrar.  \label{CU7-TA1}
	\UCpaso[\UCactor] Gestiona las entidades a través del botón: \IUbutton{Registrar}. 
	\item[- -] - - {\em {Fin de la trayectoria}}.%
\end{enumerate}
%--------------------------------------

\subsubsection{Puntos de extensión}

\UCExtenssionPoint{El actor requiere registrar una entidad.}{Presionando el botón \IUbutton{Registrar} del paso \ref{CU7-P4} de la trayectoria principal o del paso \ref{CU7-TA1} de la trayectoria alternativa A.}{\UCref{CU7.1}{Registrar Entidad}}
\UCExtenssionPoint{El actor requiere modificar una entidad.}{Presionando el icono \editar del paso \ref{CU7-P4} de la trayectoria principal.}{\UCref{CU7.2}{Modificar Entidad}}
\UCExtenssionPoint{El actor requiere eliminar una entidad.}{Presionando el icono \eliminar del paso \ref{CU7-P4} de la trayectoria principal.}{\UCref{CU7.3}{Eliminar Entidad}}
\UCExtenssionPoint{El actor requiere consultar una entidad.}{Presionando el icono \raisebox{-1mm}{\includegraphics[height=11pt]{images/Iconos/consultar}} del paso \ref{CU7-P4} de la trayectoria principal.}{\UCref{CU7.4}{Consultar Entidad}}