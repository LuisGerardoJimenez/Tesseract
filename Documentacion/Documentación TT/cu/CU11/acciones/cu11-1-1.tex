	\begin{UseCase}{CU11.1.1}{Gestionar Acciones}{
     	Las acciones disponibles para cada \hyperlink{EntidadAccion}{acción} dependerán del estado en el que se encuentre el caso de uso donde son referenciadas. 
	}
	\UCitem{Actor}{\hyperlink{jefe}{Líder de análisis}, \hyperlink{analista}{Analista}}
	\UCitem{Propósito}{Proporcionar al actor un mecanismo para llevar el control de las acciones pertenecientes a una pantalla.}
	\UCitem{Entradas}{Ninguna}
	\UCitem{Salidas}{\begin{itemize}
			\item \cdtRef{proyectoEntidad:claveProyecto}{Clave del proyecto}: Lo obtiene el sistema.
			\item \cdtRef{proyectoEntidad:nombreProyecto}{Nombre del proyecto}: Lo obtiene el sistema.
			\item \cdtRef{EntidadPantalla:nombreIU}{Nombre de la pantalla:} Lo obtiene el sistema.
			\item \cdtRef{EntidadAccion}{Acción}: Tabla que muestra \cdtRef{EntidadAccion:imagenACC}{Imagen} y \cdtRef{EntidadAccion:nombreACC}{Número} de todas las acciones registradas en un pantalla.
			\item \cdtIdRef{MSG2}{No existe información}: Se muestra en la pantalla \IUref{IU12A}{Pendiente} cuando no existen acciones registradas.
	\end{itemize}}
	
	\UCitem{Precondiciones}{Que exista al menos una pantalla registrada.}
	\UCitem{Postcondiciones}{Ninguna}
	\UCitem{Errores}{Ninguno}
	\UCitem{Tipo}{Secundario, extiende del caso de uso \UCref{CU11}{Gestionar Pantallas}.}
\end{UseCase}
%--------------------------------------
\begin{UCtrayectoria}
	\UCpaso[\UCactor] Solicita gestionar las acciones de una pantalla seleccionando el icono (Pendiente) de la pantalla \IUref{IU7}{Gestionar Pantallas}.
	\UCpaso[\UCsist] Obtiene la información de las acciones registradas de la pantalla seleccionada. \hyperlink{CU11-1-1:TAA}{[Trayectoria A]}
	\UCpaso[\UCsist] Ordena las acciones alfabéticamente basándose en el nombres de los mismas.
	\UCpaso[\UCsist] Muestra la información de las acciones en la pantalla \IUref{IU12}{Pendiente} y las operaciones disponibles de acuerdo a la regla de negocio \BRref{RN15}{Operaciones disponibles}. \label{CU11-1-1-P4}
	\UCpaso[\UCactor] Gestiona las acciones a través de los botones: \IUbutton{Registrar}, \editar y \eliminar. 
\end{UCtrayectoria}		
%--------------------------------------
\hypertarget{CU11-1-1:TAA}{\textbf{Trayectoria alternativa A}}\\
\noindent \textbf{Condición:} No existen registros de acciones.
\begin{enumerate}
	\UCpaso[\UCsist] Muestra el mensaje \cdtIdRef{MSG2}{No existe información} en la pantalla \IUref{IU7A}{Pendiente} para indicar que no hay registros de acciones para mostrar.  \label{CU11-1-1-TA1}
	\UCpaso[\UCactor] Gestiona las acciones a través del botón: \IUbutton{Registrar}. 
	\item[- -] - - {\em {Fin de la trayectoria}}.%
\end{enumerate}
%--------------------------------------
\subsubsection{Puntos de extensión}

\UCExtenssionPoint{El actor requiere registrar una acción}{Presionando el botón \IUbutton{Registrar} del paso \ref{CU11-1-1-P4} de la trayectoria principal o del paso \ref{CU11-1-1-TA1} de la Trayectoria alternativa A.}{\UCref{CU11.1.1.1}{Registrar Acción}}
\UCExtenssionPoint{El actor requiere modificar una acción}{Presionando el icono \editar del paso \ref{CU11-1-1-P4} de la trayectoria principal.}{\UCref{CU11.1.1.2}{Modificar Acción}}
\UCExtenssionPoint{El actor requiere eliminar una acción}{Presionando el icono \eliminar del paso \ref{CU11-1-1-P4} de la trayectoria principal.}{\UCref{CU11.1.1.3}{Eliminar Acción}}
