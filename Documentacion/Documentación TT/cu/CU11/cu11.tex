	\begin{UseCase}{CU11}{Gestionar Pantallas}{
			
	El colaborador (\hyperlink{jefe}{Líder de Análisis} o \hyperlink{analista}{Analista}) podrá gestionar los \hyperlink{tElemento}{elementos} propios de cada módulo, uno de estos \hyperlink{tElemento}{elementos} son las pantallas.
	
	Una \hyperlink{pantalla}{Pantalla} es una maqueta que utiliza un conjunto de imágenes y objetos gráficos para representar la información y acciones disponibles en la interfaz. Su principal uso, consiste en proporcionar un entorno visual sencillo para permitir la comunicación con el sistema.\\
	
	Este caso de uso permite al colaborador visualizar en una tabla, el registro de las pantallas que componen un módulo en específico, así como solicitar el registro de una pantalla nueva  (en donde también es posible gestionar las acciones que se encuentra en pantalla), modificar alguna existente (teniendo la posibilidad de actualizar algún dato de registro), eliminarla (en caso de que la pantalla por algún motivo ya no forme parte del proyecto) o simplemente consultar su información (nombre, descripción, imagen y acciones) \\
	
	}
	\UCitem{Actor}{\hyperlink{jefe}{Líder de análisis}, \hyperlink{analista}{Analista}}
	\UCitem{Propósito}{Proporcionar al actor un mecanismo para llevar el control de las pantallas de un módulo de un proyecto.}
	\UCitem{Entradas}{Ninguna}
	\UCitem{Salidas}{\begin{itemize}
			\item \cdtRef{proyectoEntidad:claveProyecto}{Clave del proyecto}: Lo obtiene el sistema.
			\item \cdtRef{proyectoEntidad:nombreProyecto}{Nombre del proyecto}: Lo obtiene el sistema.
			\item \cdtRef{moduloEntidad:claveModulo}{Clave del Módulo}: Lo obtiene el sistema.
			\item \cdtRef{moduloEntidad:nombreModulo}{Nombre del Módulo}: Lo obtiene el sistema.
			\item \cdtRef{pantalla}{Pantalla}: Tabla que muestra \cdtRef{pantalla:claveIU}{Clave}, \cdtRef{pantalla:numeroIU}{Número} y el \cdtRef{pantalla:nombreIU}{Nombre} de todas las pantallas registradas en un módulo de un proyecto.
			\item \cdtIdRef{MSG2}{No existe información}: Se muestra en la pantalla \IUref{IU7}{Gestionar Pantallas} cuando no existen pantallas registradas.
	\end{itemize}}
	\UCitem{Precondiciones}{\begin{itemize}
			\item Que exista al menos un proyecto registrado.
			\item Que exista al menos un pantalla registrada.
	\end{itemize}}
	\UCitem{Postcondiciones}{Ninguna}
	\UCitem{Errores}{Ninguno}
	\UCitem{Tipo}{Secundario, extiende del caso de uso \UCref{CU5}{Gestionar Módulos}.}
\end{UseCase}
%--------------------------------------
\begin{UCtrayectoria}
	\UCpaso[\UCactor] Solicita gestionar las pantallas presionando el botón \raisebox{-1mm}{\includegraphics[height=11pt]{images/Iconos/pantalla}} de un módulo de la pantalla \IUref{IU4}{Gestionar Módulos}.
	\UCpaso[\UCsist] Obtiene la información de las pantallas registradas en el módulo seleccionado. \hyperlink{CU11:TAA}{[Trayectoria A]}
	\UCpaso[\UCsist] Ordena las pantallas alfabéticamente basándose en la clave de los mismos.
	\UCpaso[\UCsist] Muestra la información de las pantallas en la pantalla \IUref{IU7}{Gestionar Pantallas} y las operaciones disponibles de acuerdo a la regla de negocio \BRref{RN15}{Operaciones disponibles}.\label{CU11-P4}
	\UCpaso[\UCactor] Gestiona los proyectos a través de los botones: \IUbutton{Registrar}, \editar, \eliminar y \raisebox{-1mm}{\includegraphics[height=11pt]{images/Iconos/consultar}}. 
\end{UCtrayectoria}		
%--------------------------------------
\hypertarget{CU11:TAA}{\textbf{Trayectoria alternativa A}}\\
\noindent \textbf{Condición:} No existen registros de pantallas.
\begin{enumerate}
	\UCpaso[\UCsist] Muestra el mensaje \cdtIdRef{MSG2}{No existe información} en la pantalla \IUref{IU7}{Gestionar Pantallas} para indicar que no hay registros de pantallas para mostrar. \label{CU11-TA1}
	\UCpaso[\UCactor] Gestiona las pantallas a través del botón: \IUbutton{Registrar}. 
	\item[- -] - - {\em {Fin del caso de uso}}.%
\end{enumerate}


%--------------------------------------

\subsubsection{Puntos de extensión}

\UCExtenssionPoint{El actor requiere registrar una pantalla}{Presionando el botón \IUbutton{Registrar} del paso \ref{CU11-P4} de la trayectoria principal o del paso \ref{CU11-TA1} de la trayectoria alternativa A..}{\UCref{CU11.1}{Registrar Pantalla}}
\UCExtenssionPoint{El actor requiere modificar una pantalla}{Presionando el icono \editar del paso \ref{CU11-P4} de la trayectoria principal.}{\UCref{CU11.2}{Modificar Pantalla}}
\UCExtenssionPoint{El actor requiere eliminar una pantalla}{Presionando el icono \eliminar del paso \ref{CU11-P4} de la trayectoria principal.}{\UCref{CU11.3}{Eliminar Pantalla}}
\UCExtenssionPoint{El actor requiere consultar una pantalla}{Presionando el icono \raisebox{-1mm}{\includegraphics[height=11pt]{images/Iconos/consultar}} del paso \ref{CU11-P4} de la trayectoria principal.}{\UCref{CU11.4}{Consultar Pantalla}}