	\begin{UseCase}{CU11.4}{Consultar Pantalla}{
			
		Este caso de uso permite al colaborador (\hyperlink{jefe}{Líder} o \hyperlink{analista}{Analista}) consultar la información de una \hyperlink{pantalla}{Pantalla} previamente registrada en el hyperlink{moduloEntidad}{Módulo} del \hyperlink{proyectoEntidad}{Proyecto} sobre el cual se está operando.
			
		Además de la información general de la pantalla, el colaborador podrá consultar su maquetado y la lista de acciones que contiene.
	}
		\UCitem{Actor}{\hyperlink{jefe}{Líder de Análisis}, \hyperlink{analista}{Analista}}
		\UCitem{Propósito}{Consultar la información de una pantalla perteneciente a un módulo de un proyecto.}
		\UCitem{Entradas}{Ninguna}
		\UCitem{Salidas}{
			\begin{itemize}
				\item De la sección ''Información general de la pantalla'':
					\begin{itemize}
						\item \cdtRef{proyectoEntidad:claveProyecto}{Clave del proyecto:} Lo obtiene el sistema.
						\item \cdtRef{proyectoEntidad:nombreProyecto}{Nombre del proyecto:} Lo obtiene el sistema.
						\item \cdtRef{EntidadPantalla:claveIU}{Clave}: Lo obtiene el sistema.
						\item \cdtRef{EntidadPantalla:numeroIU}{Número}: Lo obtiene el sistema.
						\item \cdtRef{EntidadPantalla:nombreIU}{Nombre}: Lo obtiene el sistema.
						\item \cdtRef{EntidadPantalla:descripcionIU}{Descripción}: Lo obtiene el sistema.
						\item \cdtRef{EntidadPantalla:imagenIU}{Imagen}: Lo obtiene el sistema.
					\end{itemize}
				\item De la sección ''Acciones'':
					\begin{itemize}
						\item \cdtRef{EntidadAccion:nombreACC}{Nombre}: Lo obtiene el sistema.
						\item \cdtRef{EntidadAccion:descripcionACC}{Descripción}: Lo obtiene el sistema.
						\item \cdtRef{EntidadAccion:destinoACC}{Pantalla destino}: Lo obtiene el sistema.
						\item \cdtRef{EntidadAccion:tipoACC}{Tipo de acción}: Lo obtiene el sistema.
						\item \cdtRef{EntidadAccion:imagenACC}{Imagen de la acción}: Lo obtiene el sistema.
					\end{itemize}
		\end{itemize}}
		\UCitem{Precondiciones}{Ninguna}
		\UCitem{Postcondiciones}{Ninguna}
		\UCitem{Errores}{\begin{itemize}
		\item \cdtIdRef{MSG12}{Ha ocurrido un error}: Se muestra en la pantalla \IUref{IU7}{Gestionar Pantallas} cuando la pantalla que se desea consultar no existe.
		\end{itemize}
		}
		\UCitem{Tipo}{Secundario, extiende del caso de uso \UCref{CU11}{Gestionar Pantallas}.}
	\end{UseCase}
%--------------------------------------
	\begin{UCtrayectoria}
		\UCpaso[\UCactor] Da clic en el icono \raisebox{-1mm}{\includegraphics[height=11pt]{images/Iconos/consultar}} del registro que desea consultar de la pantalla \IUref{IU7}{Gestionar Pantalla} o la liga correspondiente a una pantalla en la pantalla \IUref{IU6.3}{Consultar caso de uso}.
		\UCpaso[\UCsist] Obtiene la información de la pantalla seleccionada. \hyperlink{CU11-4:TAA}{[Trayectoria A]}
		\UCpaso[\UCsist] Muestra la pantalla \IUref{IU7.3}{Consultar Pantalla}.
		\UCpaso[\UCactor] Consulta la información de la pantalla.
		\UCpaso[\UCactor] Finaliza la consulta oprimiendo el botón \IUbutton{Regresar}.
		\UCpaso[\UCsist] Muestra la pantalla \IUref{IU7}{Gestionar Pantallas}.
	\end{UCtrayectoria}		
%--------------------------------------
	\hypertarget{CU11-4:TAA}{\textbf{Trayectoria alternativa A}}\\
	\noindent \textbf{Condición:} La pantalla que se desea consultar no existe.
	\begin{enumerate}
		\UCpaso[\UCsist] Muestra la pantalla \IUref{IU7}{Gestionar Pantallas} con el mensaje \cdtIdRef{MSG12}{Ha ocurrido un error}.
		\item[- -] - - {\em {Fin del caso de uso}}.
	\end{enumerate}

	
