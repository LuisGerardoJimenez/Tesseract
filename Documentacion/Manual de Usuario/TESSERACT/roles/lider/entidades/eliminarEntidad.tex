%------------------------------------------------------------------------
%Editar Diplomado
\hypertarget{cv:eliminarEntidad}{\section{Eliminar Entidad}} \label{sec:eliminarEntidad}

	Esta funcionalidad le permitirá eliminar una entidad innecesaria o incorrecta. Para eliminar una entidad es necesario que no se encuentre asociado a casos de uso, al igual que sus atributos.

		\subsection{Procedimiento}

			%Pasos de procedimiento
			\begin{enumerate}
	
			\item Oprima el botón \IUBotonEliminar{} de un registro existente de la pantalla \ref{fig:GestionarEntidades} ''Gestionar Entidades''.
	
			\item Se mostrará el mensaje \ref{fig:confirmaEliminaEntidad} sobre la pantalla \ref{fig:GestionarEntidades} ''Gestionar Entidades''.
			
			%Pantalla
			\begin{figure}[htbp!]
				\begin{center}
					\includegraphics[scale=0.5]{roles/lider/entidades/pantallas/IU12-3MSG10}
					\caption{MSG de Confirmación}
					\label{fig:confirmaEliminaEntidad}
				\end{center}
			\end{figure}
						
			\item Oprima el botón \IUAceptar.
			
			\item Se mostrará el mensaje \ref{fig:entidadEliminada} en la pantalla \ref{fig:GestionarEntidades} ''Gestionar Entidades''.
			
			\begin{figure}[htbp!]
				\begin{center}
					\includegraphics[scale=0.5]{roles/lider/entidades/pantallas/IU12-3MSG1}
					\caption{MSG: Entidad Eliminada}
					\label{fig:entidadEliminada}
				\end{center}
			\end{figure}
			\end{enumerate}