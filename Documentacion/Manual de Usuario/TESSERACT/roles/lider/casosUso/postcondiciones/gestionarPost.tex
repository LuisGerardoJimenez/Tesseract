%------------------------------------------------------------------------
%Editar Diplomado
\hypertarget{cv:GestionarPostcondiciones}{\section{Gestionar Postcondiciones}} \label{sec:GestionarPostcondiciones}

	Esta funcionalidad le permitirá las acciones necesarias para controlar las postcondiciones pertenecientes a un caso de uso y visualizarlos en una tabla en el caso de uso sobre el que se está operando y solicitar el registro de uno nuevo.

		\subsection{Procedimiento}

			%Pasos de procedimiento
			\begin{enumerate}
				
			\item Ingrese a un proyecto existente desde la pantalla \ref{fig:GestionarProyectosColaborador}.
			
			\item Oprima el botón PENDIENTES de algún registro existente de la pantalla \ref{fig:GestionarCU} ''Gestionar Casos de Uso''.
	
			\item Se mostrará la pantalla \ref{fig:GestionarPostcondiciones} ''Gestionar Postcondiciones''.

			%Pantalla
			\begin{figure}[htbp!]
				\begin{center}
					\includegraphics[scale=0.6]{roles/lider/casosUso/pantallas/IU5gestionarModulos}
					\caption{Gestionar Postcondciones}
					\label{fig:GestionarPostcondiciones}
				\end{center}
			\end{figure}
		
				\item Seleccione la operación que desea realizar:
			
			Para (\hyperlink{cv:registrarPostcondicion}{Registrar}) dé clic en el botón \IURegistrar.
			
			Para (\hyperlink{cv:eliminarPostcondicion}{Eliminar}) dé clic en el icono \IUBotonEliminar{} de alguna postcondición ya registrada.
			
			\end{enumerate}