
\hypertarget{cv:registrarPrecondicion}{\section{Registrar Precondición}} \label{sec:registrarPrecondicion}

	Esta funcionalidad le permitirá registrar una precondición dentro del caso de uso que se esta operando.

		\subsection{Procedimiento}

			%Pasos de procedimiento
			\begin{enumerate}
	
			\item Oprima el botón \IURegistrar{} de la pantalla \ref{fig:GestionarPrecondiciones} ''Gestionar Precondiciones''.
			
			\item Se mostrará la pantalla \ref{fig:registrarPrecondicion} ''Registrar Precondición''.

			%Pantalla
			\begin{figure}[htbp!]
				\begin{center}
					\includegraphics[scale=0.5]{roles/lider/casosUso/pantallas/IU5-1registrarModulo}
					\caption{Registrar Precondición}
					\label{fig:registrarPrecondicion}
				\end{center}
			\end{figure}
		
			\item Ingrese la redacción de la precondición.
			
			\item Para la redacción podrá ingresar TOKENS para vincular diferentes elementos previamente registrados:
			
			\begin{itemize}
				\item Podrá referenciar elementos de tipo actor con el TOKEN: ''ACT·''.
				\item Podrá referenciar elementos de tipo entidad y/o atributos con los TOKEN: ''ENT·''y ''ATR·''.
				\item Podrá referenciar elementos de tipo término con el TOKEN: ''GLS·''.
			\end{itemize}
			
			\item Oprima el botón \IUAceptar.
			
			\item Se mostrará el mensaje \ref{fig:precondicionRegistrada} en la pantalla \ref{fig:GestionarPrecondiciones} ''Gestionar Precondiciones''.
			
			\begin{figure}[htbp!]
				\begin{center}
					\includegraphics[scale=0.6]{roles/lider/casosUso/pantallas/IU5-1MSG1}
					\caption{MSG: Precondición Registrada}
					\label{fig:precondicionRegistrada}
				\end{center}
			\end{figure}
			\end{enumerate}